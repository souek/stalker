\documentclass[../MAIN.tex]{subfiles}
\begin{document}
\ro{Podmoskowje, 16:56, 21.02.2013}
%
Poprawiłem płaszcz i spróbowałem usadowić się wygodniej na
tylnej kanapie szarego audi. Prowadził Andriej, pewnie, z
niewielką dozą nonszalancji, ignorując ograniczenia prędkości,
ale nie porzucając zdrowego rozsądku. Auto, choć niebędące
bynajmniej najnowszym modelem, było czyste i zadbane -
najwyraźniej właściciel bardzo o nie dbał, chuchając i
dmuchając na każdą, nawet najmniejszą, ryskę. Ponownie uniosłem
się na siedzeniu, wyrównałem zwijającą się na plecach warstwę
materiału, a potem spróbowałem znaleźć trochę więcej miejsca
dla nóg. Dobrze byłoby się teraz przespać z godzinkę, organizm
potrzebował odskoczni od emocji, jakie przyniosła ostatnia
doba, bolała mnie głowa, bolało stłuczone kolano, bolała i
dusza. W ustach czułem metaliczny posmak. Za stary już na to
jestem, pomyślałem. Za stary. Co mi przyszło\3k

Opuściłem głowę na pierś, przymknąłem oczy, spróbowałem
odepchnąć od siebie ból, skupić tylko na bodźcach zewnętrznych.
Z przodu urywana rozmowa Andrieja i Saszy, uspokajający dźwięk
równo pracującego silnika, w głośnikach ciche gitary, głos
któregoś ze starych rockmanów - czyżby Fish? - z tyłu mój
własny oddech, rytmiczne uderzenia serca. Ciepło, klimatyzacja
ogrzała wnętrze samochodu w ciągu kwadransa. Spać, rozkazałem
sobie. Spać, spać\3k
%
\ro{Moskwa - Metro Taganskaja, 18:12, 19.02.2013}
%
Na ruchomych schodach przeszył mnie lodowaty podmuch wiatru.
Boleśnie powoli wypełzałem na powierzchnię, ściśnięty w tłumie
anonimowych współtowarzyszy podróży wagonami moskiewskiego
metra. Dotarłem do szczytu schodów, opędziłem się od
szczerbatego łachmaniarza i ruszyłem nieśpiesznie w kierunku
domu. Na Tagance mieszkałem dopiero trzeci miesiąc, ale drogę
miałem wpisaną już, jak to mówią, w autopilota. Najpierw
kilkaset metrów wąskim chodniczkiem, między rzędami bloków
mieszkalnych. Później pasy, przejście przez ulicę, kolejna
ćwierć kilometra wzdłuż ogrodzenia mojego osiedla. Wreszcie,
szósta po lewej klatka, kodowe zamki przy furtce i wewnętrznych
drzwiach, kamery i uprzejmi, wąsaci ochroniarze. Wcześniej,
oczywiście, wizyta w markecie, kupno czegoś do jedzenia -
mogłem zjeść najgorsze gówno, byle dało się je szybko i łatwo
rozgrzać w mikrofalówce. Ciekawa sprawa, będąc jeszcze w Zonie,
zaklinałem się, że po powrocie będę jadał jak król. Tymczasem,
moja dieta składała się w głównej mierze z pizzy, chińszczyzny
i alkoholu.

Właśnie, przypomniałem sobie, muszę jeszcze kupić coś na
rozgrzewkę. Do kolacji trzy\3k może cztery piwka, potem
ćwiartka, na spokojny sen. Jak co wieczór.

Dwa pełne obroty klucza w zamku i drzwi do mieszkania stanęły
otworem. Przekroczyłem próg, zamknąłem za sobą, brzęczącą
szkłem reklamówkę postawiłem na podłodze, a płaszcz rzuciłem
niedbale do szafy. Nie zdejmując nawet butów, otworzyłem
pierwszą butelkę i chciwie przyssałem się do szyjki. Po kilku
łykach usiadłem ciężko na szafce w przedpokoju i zapatrzyłem
się w ścianę. ku*wa, jak mi to już wszystko zbrzydło\3k Niby
miałem wszystko, o czym tylko mogłem marzyć: służbowe
mieszkanie w dobrej dzielnicy, wysoką pensję, a na niezależnym
od moich mocodawców koncie, okrągłą sumkę, za którą mógłbym
uciec na drugi koniec świata i dostatnio się tam urządzić. A
jednak wciąż czegoś brakowało, z każdym kolejnym dniem czułem
coraz paskudniejszą pustkę i bezsens. Po raz pierwszy w życiu
czułem się samotny.

Pusta butelka uderzyła miękko o wykładzinę, odtoczyła się pod
ścianę. Wstałem z westchnieniem, wziąłem kolejną i poszedłem do
kuchni. W lodówce znalazłem jakieś marne resztki wczorajszego
obiadu, po odgrzaniu nie miały absolutnie żadnego smaku, ale
mimo wszystko zaspokajały głód. Zjadłem szybko, przepijając
kolejne kęsy piwem. Przeniosłem się do salonu, siadłem na
parapecie. Szczególnie polubiłem to właśnie miejsce - szeroki,
drewniany parapet, tuż przy oknie, z którego roztaczał się
widok na Moskwę, szarą, brudną i zimną. Całe mieszkanie było
urządzone z gustem, FSB nie żałowało pieniędzy na swoich
konsultantów, ale pozostałe pomieszczenia swoim przepychem
przypominały mi tylko o tym, że lokum nie należy do mnie.
Salon, przeciwnie, stał się moją enklawą, zaś ów parapet, jej
centralnym punktem. To na nim spędzałem większość wieczorów,
powoli nasączając się piwem i rozmyślając, tylko po to, by
później dobić się wódką i zasnąć ciężkim, niedającym
wytchnienia snem bez snów.

Najgorszym, co może cię spotkać w życiu, jest obudzenie się ze
świadomością, że twoje życie już dawno stanęło w miejscu, a ty
nie potrafisz wyrwać się z tego marazmu. To właśnie powiedział
mi mój ojciec, niedługo przed swoją śmiercią. Dopiero teraz, po
wielu latach, zaczynałem rozumieć, co chciał mi przekazać. Z
każdym kolejnym wieczorem sens jego słów stawał się dla mnie
coraz boleśniej wyraźny, a z każdym kolejnym łykiem coraz
bardziej chciało mi się wyć.

Od roku moim życiem rządziła pasywność. Dwa ostatnie miesiące w
Zonie spędziłem siedząc bezpiecznie na zapleczu Baru,
nadzorując transporty artefaktów za kordon. Któregoś dnia
otworzyłem oczy z przekonaniem, że już nie dam rady wyjść w
Strefę. Nie dam - i już. Nie wiem, co się stało, wysiadły mi
nerwy, czy też dostałem znak od matuszki Zony, nie wiem. Jako
bliski współpracownik Barmana, mogłem liczyć na jego pomoc,
facet był bardziej wpływowy, niż ktokolwiek mógł przypuszczać.
Zgodził się zorganizować transport przez granicę, jednak pod
jednym warunkiem: zgodzę się na współpracę z “wiadomymi
organami”. Miało to przynieść korzyści wszystkim, a mnie
szczególnie. Być może wcześniej bym odmówił, teraz tylko
wzruszyłem ramionami. Wszystko, byle uciec z tego przeklętego
miejsca.

Lew towarzyszył mi od pierwszych minut spędzonych w Zonie.
Razem zostaliśmy przeszmuglowani przez wojskowy kordon, w
milczeniu przeciskaliśmy się przez dziurę w zasiekach,
popędzani przez śmierdzącego potem i nikotyną przewodnika.
Mężczyzna zniknął, gdy oddaliliśmy się o kilkaset metrów od
granicy, zostawiając nas zdanych tylko na siebie. Popatrzyliśmy
po sobie, to ja pierwszy wyciągnąłem do niego dłoń.\\
-- Foma -- powiedziałem tylko.\\
-- Lew -- odrzekł, ściskając moją prawicę.\\
Ruszyłem przodem, przedzierając się przez gęste krzaki, i wcale
nie czułem się zagrożony. No, może przez pierwsze kilka godzin
- tak. Ale potem już nie. Przez następne, długie trzynaście
miesięcy. Lew stał mi się niczym brat. Razem zaczynaliśmy,
uczyliśmy się przeżycia w Strefie, razem głodowaliśmy, gdy
zabrakło pieniędzy. Przeżyliśmy tę zabawną przygodę, gdy
nieopodal Skadowska znaleźliśmy trupa w naukowym kombinezonie.
Wreszcie, płacąc za to własną krwią i potem, wspólnie
wypracowaliśmy sobie w Zonie pozycję, zdobyliśmy uznanie i
szacunek, zaczęliśmy zarabiać prawdziwe pieniądze. A potem
wszystko trafił szlag. Te dwa miesiące, które spędziłem w
grubych murach baru 100 Radów bardzo nas od siebie oddaliły.
Lew nadal chodził w Zonę, teraz już sam. Jako doświadczony
łowca mógł sobie pozwolić na samodzielne wypady, świetnie
wyposażony, opanowany i śmiertelnie niebezpieczny. Zawsze był
lepszym stalkerem ode mnie, musiałem to przyznać. A jednak to
właśnie ja przewodziłem naszej parze. Sam nie wiem, co o tym
zadecydowało, może fakt, że wtedy, pierwszego dnia, to ja
ruszyłem przodem. Lew milcząco zaaprobował moje przywództwo,
choć nie wahał się oponować, gdy któryś z moich pomysłów mu się
nie spodobał.

O swoim wyjeździe poinformowałem go już po rozmowie z Barmanem.
Skinął tylko głową - nigdy nie mówił dużo - i wyciągnął do mnie
dłoń. Uścisnąłem ją, mocno, w milczeniu, bo i co miałem
powiedzieć. Odwróciłem się i odszedłem, by następnego wieczora
opuścić Strefę na pace wojskowej ciężarówki. Nie widziałem go
już więcej.

Następne miesiące minęły mi w kolejnych, coraz to bardziej
tajnych ośrodkach Federalnej Służby Bezpieczeństwa. Z początku
byłem traktowany niczym więzień, przesłuchiwano mnie,
wielokrotnie zadając te same pytania o moje życie w Zonie.
Później nadeszła kolej na szkolenia, najwyraźniej wierchuszka
uznała, że doświadczenia stalkera Fomy Pietrowicza mogą być
wartościowe dla interesów Federacji. W końcu wręczono mi klucze
do mieszkania na Tagance i kopertę z pokaźną premią
motywacyjną, zaznaczając mimochodem, że nie wolno mi opuszczać
granic miasta Moskwa. Oficjalnie byłem właścicielem niewielkiej
firmy, zajmującej się sprzedażą i montażem urządzeń
monitorujących. Nieoficjalnie, przez pięć dni w tygodniu
ślęczałem w maleńkim gabinecie, w ponurym budynku, znajdującym
się przy ulicy o nazwie znanej w całej Europie Wschodniej:
Bolszaja Łubianka. Przekopywałem się przez sterty wojskowych
raportów z Zony, uzupełniając je o swoje komentarze. Jako
człowiek, który przeżył w tym piekle ponad rok, mogłem wnieść
naprawdę wiele w schematy operujących w Strefie Wykluczenia
rosyjskich sił specjalnych. Traktowano mnie z szacunkiem, a
moje uwagi zawsze brano pod uwagę przy planowaniu kolejnych
zadań. Mimo tego, że robiłem coś bez wątpienia pożytecznego,
czułem tę cholerną pustkę. Tak, jak źle było mi w Zonie, tak i
czułem się poza jej niewidzialnymi granicami.

Moje rozmyślania przerwał dźwięk dzwoniącego telefonu.
Niewiele, bardzo niewiele osób miało mój numer, dlatego od razu
stałem się czujny. Zeskoczyłem z parapetu, odstawiłem butelkę
na szafkę i szybkim krokiem ruszyłem do przedpokoju; telefon
musiał zostać w kieszeni płaszcza. Spojrzałem na wyświetlacz,
nie znałem numeru. Odebrałem, rzucając do słuchawki szorstkie
“Tak?”. Rozmówca miał lekko zachrypnięty, niski głos. Nie
przedstawił się, od razu przeszedł do \mbox{rzeczy}.

-- Od Barmana. Lew nie żyje. Rozstrzelany przez pluton
egzekucyjny Powinności, ciało powieszone na drzewie przy
południowym wejściu do Baru. Przesyłam zdjęcie. Przyczyna
aresztowania i egzekucji nieznana, Barman przypuszcza że ma to
związek z twoją współpracą z FSB. Możliwe że w grę wchodzi
ingerencja SBU. Barman prosi o kontakt.

Klik. Przerwane połączenie. W chwilę później pojedynczy sygnał,
odebrano wiadomość obrazkową, czy chcesz wyświetlić? Ciało Lwa,
zawieszone na linie, okropnie skrwawione, klatka piersiowa w
strzępach. Szyja okolona grubym sznurem, ciągnącym się gdzieś w
górę, poza kadr. Głowa pod nienaturalnym kątem, twarz
wykrzywiona w grymasie, oczy otwarte, usta też. Wybita ze stawu
żuchwa, spłaszczony od ciosów nos. Krew.

Wolno, bardzo wolno odkładam telefon na półkę. Czuję jak wali
mi serce. Niezgrabnym, sztywnym krokiem kieruję się do kuchni.
Otwieram zamrażarkę, wyjmuję butelkę wódki, zrywam korek,
ciągnę kilka długich łyków.

Lew nie żyje.

Lodowata wódka mrozi zęby i pali przełyk, gorącą falą uderza w
ściany żołądka, gorzka gula podchodzi do gardła.

Rozstrzelany.

W oczach mrocznieje, padam na ziemię, boleśnie tłukę kolanem o
płytki posadzki. Zaraz odchylam głowę, leję w siebie kolejną
porcję alkoholu.

Widzę jego mętne, martwe spojrzenie. Puste oczy patrzą na mnie
z dna butelki, rozpaczliwie łykam palący płyn, chcąc uciec od
czającego się w nich wyrzutu. Czekam na wódczane zapomnienie,
czekam, aż odpłynę w alkoholowy mrok, czekam na niego, jak na
zbawienie. \mbox{Czekam\3k}
%
\ro{Tanganka, 09:24, 20.02.2013}
%
Obudziły mnie mdłości. Leżałem na podłodze w kuchni, w kałuży
wody, wyciekającej z otwartych drzwiczek zamrażarki. Głowa
pękała, gdy tylko spróbowałem się poruszyć, kolano zakłuło
ostrym, przenikliwym bólem. Zawlokłem się do łazienki,
klęknąłem przy sedesie i zwymiotowałem. Gdy już przestały mną
wstrząsać spazmy, wszedłem pod prysznic, tak jak stałem, w
ubraniu. Strugi zimnej wody spłynęły po moim ciele, zacząłem
drżeć, ale nie zakręciłem kurka. Stałem tak kilka minut,
skupiając się na fizycznym dyskomforcie, odwlekając moment, w
którym będę musiał zebrać się w sobie i wrócić myślami do
wczorajszego wieczoru. W końcu wyszedłem z kabiny i z trudem
zdjąłem przemoczone ubranie. Wycierając się szorstkim
ręcznikiem zacząłem układać w głowie plan działania.
Wiedziałem, że tylko w ten sposób mogę pokonać pragnienie
odpłynięcia po raz kolejny w alkoholowy mrok.

Po pierwsze, co powiedział mi kontakt Barmana. SBU śledzące
moją współpracę z FSB, tak, to możliwe. Pojawienie się Zony
znacząco wzmocniło pozycję Ukrainy na arenie międzynarodowej,
wzrok całego świata skierował się na to niebogate państewko we
wschodniej Europie. Decydenci uznali, że oto nadeszła wspaniała
okazja, by całkowicie uniezależnić się od potężnego sąsiada ze
wschodu. Więzy z Federacją Rosyjską gwałtownie zerwano,
odrzucono propozycje pomocy w okiełznaniu nieznanego nauce
zjawiska, skierowano się za to ku zachodowi, który chętnie
pompował w podupadły kraj pieniądze, w zamian za możliwość
badania cudów Strefy. Jednocześnie, Ukraina postawiła
ultimatum: Zona leży na naszym terytorium, więc na jej terenie
mogą przebywać tylko nasze wojska. W ten sposób zapewniła sobie
pełną kontrolę nad dostępem do niej, co było posunięciem tyleż
mądrym, co ryzykownym, biorąc pod uwagę śmiertelnie
niebezpieczną naturę Strefy. Utrzymanie jej w ryzach własnymi
siłami mogło okazać się zbyt trudnym zadaniem dla ukraińskich
sił zbrojnych, jednak póki co odnosiły one niemałe sukcesy w
walce z szalejącymi tam wynaturzeniami.

Oczywiście, do Zony przenikali wszelkiej maści najemnicy na
usługach zainteresowanych nią państw, a także zorganizowane
grupy, między innymi te, dowodzone przez Centrum Specjalnego
Przeznaczenia FSB, słynne oddziały specnazu: Alfa i Wympieł.
Bardzo możliwe, że próbowała je powstrzymać Służba
Bezpieczeństwa Ukrainy, pytanie tylko, jakim sposobem na
celowniku znaleźliśmy się ja i Lew. Czyżby ktoś przypuszczał,
że mój przyjaciel także współpracuje z rosyjskimi siłami
specjalnymi? Potrafiłem zrozumieć, dlaczego to on stał się
pierwszym celem, wiedziałem, że znajduję się pod dyskretną, ale
nieustanną ochroną FSB, z jednej strony dbającej o moje
bezpieczeństwo, z drugiej - mającej mnie zatrzymać, gdybym
jednak postanowił zniknąć z Moskwy. Zona była miejscem, w
którym ludzie codziennie tracili życie, cóż prostszego, niż
zabić jednego stalkera. Z drugiej strony, może Barman się mylił
i tajne służby nie stały za śmiercią Lwa?

Nie wiem. Tutaj na pewno się tego nie dowiem, pomyślałem. Muszę
wydostać się z miasta, wrócić do Zony. Poza chęcią odkrycia
prawdy, kierowała mną żądza zemsty. Znajdę ludzi, którzy go
zabili i własnoręcznie poślę do grobu. Inaczej nigdy nie
odnajdę spokoju.

Tuż przed moim wyjazdem ze Strefy, Barman wyjaśnił mi, czemu
chce, bym podjął współpracę ze Federalną Służbą Bezpieczeństwa.
Okazało się, że jest ona jednym z głównych kupców artefaktów,
przez co pozostaje w ścisłej współpracy z siatką handlarzy
działających w Zonie. Handlarze chcieli mieć wewnątrz
organizacji swojego człowieka, który będzie miał oczy i uszy
otwarte na wszystkie wartościowe informacje. Jako że przez
kilka miesięcy pracy u Barmana zdobyłem jego zaufanie, jawiłem
się jako doskonały kandydat na to stanowisko. Kusiła także
pokaźna suma, jaką mi zaoferowano za podjęcie się tego zadania.
Zgodziłem się, ryzykując niebezpieczną zabawę w kotka i myszkę
z ludźmi nieskończenie bardziej ode mnie doświadczonymi w
takich rozgrywkach. Ustaliliśmy bezpieczny kanał, za pomocą
którego miałem się z nim kontaktować, oczywiście tylko w razie
absolutnej konieczności. Nadeszła pora, by sprawdzić, czy
ludzie Barmana będą w stanie porwać mnie sprzed nosa agentów
FSB.

Z szafki w salonie wyciągnąłem komórkę, tani, pozbawiony
jakichkolwiek gadżetów model samsunga. W portfelu nosiłem kartę
sim, jak dotąd ani razu nieużytą. Wyłamałem ją z ramki,
włożyłem do telefonu, uruchomiłem urządzenie. Wstukałem
wyuczony na pamięć numer, nacisnąłem zieloną słuchawkę.
Odczekałem trzy sygnały, przerwałem połączenie. Kartę
połamałem, telefon zaś cisnąłem przez okno. Mieszkanie
znajdowało się na dziesiątym piętrze, nie było szansy, by
aparat wytrzymał spotkanie z ziemią.

Samo zalogowanie się telefonu do systemu operatora było
sygnałem, uruchamiającym szereg akcji, w efekcie których mieli
do mnie dotrzeć współpracownicy handlarzy. Cóż, niedługo się
przekonam, co z tego wyjdzie, pomyślałem.

Ubrałem się, wybierając jak najwygodniejsze i
najpraktyczniejsze odzienie. Przez chwilę wahałem się nad
zabraniem ze sobą broni, leżący w szufladzie Heckler{\&}Koch
kusił swoją zabójczą skutecznością. W końcu odrzuciłem ten
pomysł, więcej byłoby z nim zmartwień niż pożytku. Do kieszeni
spodni wsunąłem zawczasu przygotowany zwitek pieniędzy, głównie
hrywien i dolarów. W końcu, zadowolony z poczynionych
przygotowań, usiadłem na parapecie i spojrzałem w dół, na
buro-śnieżną Moskwę.

Dwie godziny później, jak gdyby nigdy nic, do moich drzwi
zapukało dwóch postawnych mężczyzn, którzy przedstawili się
jako Andriej i Sasza, po czym zaprosili mnie do zaparkowanego
pod klatką szarego audi.
%
\ro{Kijów, Ukraina, 07:12, 22.02.2013}
%
-- E, stalker, obudź się. Jesteśmy na miejscu.
\\
To Andriej potrząsnął mnie lekko za ramię. Otworzyłem oczy,
spojrzałem na zegarek. Spałem prawie dziesięć godzin, co
ciekawe, nie obudzono mnie nawet przy przekraczaniu
rosyjsko-ukraińskiej granicy. Wyglądało na to, że Barman był
człowiekiem o wiele bardziej wpływowym, niż przypuszczałem.
\\
-- Jesteśmy - powtórzył kierowca. - Kijów. Wysiadka.
\\
Wysiadłem z samochodu, poprawiłem płaszcz i zlustrowałem
spojrzeniem okolicę, w której się znaleźliśmy. Osiedle niskich,
szarych domków jednorodzinnych, zatrzymaliśmy się na podjeździe
przed jednym z nich.
\\
-- W środku czekają ludzie, którzy przetransportują cię do
Strefy - poinformował Sasza, opierający się o maskę samochodu.
\\
-- Nasza robota jest już skończona.
\\
-- Nic nie pozostaje, tylko życzyć ci powodzenia -- dodał
Andriej.
Podziękowałem skinieniem głowy.
\\
Mężczyźni wsiedli do wozu,
Andriej uruchomił silnik i ruszył ostro, bluzgając spod kół
mieszaniną błota i śniegu. Odwróciłem się i ruszyłem w stronę
domu.
W głowie tłukła mi się tylko jedna myśl: wracam do Zony.\\
Wracam.\\
Wracam.
\end{document}