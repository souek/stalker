\documentclass[../MAIN.tex]{subfiles}
\begin{document}
\ro{Rozdział I}
Szedłem. Szedłem wycieńczony już kilkadziesiąt minut. Jeszcze
trochę i~dojdę w~końcu na Wysypisko. Była dopiero połowa
grudnia, a już cała Zona przykryta została warstwą śniegu grubą
na jakieś czterdzieści centymetrów. Zima tutaj stanowiła bardzo
ciekawe zjawisko. Wszyscy żyjący tutaj stalkerzy~--~od kota po
weterana ograniczali swój zapał do walki. Jeśli nie było
potrzeby to nikt nie walczył, coraz mniej było słychać odgłosy
wystrzałów. Niektórzy na okres zimowy wracali do swych domów i
wyjeżdżali z~Zony, jednak duża część tu żyjących ludzi uważała
ten skrawek skażonej ziemi za swój dom. Wielu ludzi lubiło tę
mroźną porę roku w~strefie, ponieważ mutanty nie zachowywały
się tak agresywnie jak dotychczas, były znacznie łatwiejsze do
zabicia. Zima ograniczała także owe zabójcze niegdyś anomalie.
Wiele z~nich nie stanowiła już takiego niebezpieczeństwa, co na
przykład latem. A działo się to najpewniej przez mróz, który w
jakiś sposób je rozbraja z~tych co bardziej niebezpiecznych
możliwości.

Dzisiejszy dzień należał do dni łagodnych podczas ukraińskiej
zimy, podczas której mrozy sięgały nawet minus czterdziestu
stopni Celsjusza. Temperatura aktualnie wynosiła jedynie pięć
stopni na minusie, nie było także zbyt wielkiego wiatru. W
pewien sposób gruba warstwa śniegu, pokrywająca wszystko
łącznie z~powykręcanymi gałęziami rachitycznych drzew
sprawiała, że zimno nie było zbyt mocno \mbox{odczuwalne}.

Wracałem do Baru 100 Radów z~małej, nieudanej wyprawy. Razem z
dwoma przyjaciółmi: Lorekinem i~Bolem, byłem w~Dolinie Mroku i
próbowałem wraz z~towarzyszami przejąć posterunek bandytów.
Stacjonowali w~budowlach, w~których jak podejrzewano mieści się
tajne, radzieckie laboratorium. Zamierzaliśmy zaatakować ich z
zaskoczenia, jednak coś nam to nie wyszło. Strażnik zobaczył
nas, kiedy się skradaliśmy, zanim jeszcze podeszliśmy do
bramy~--~od razu zaczął bić na alarm. Wycofaliśmy się
przestraszeni za zdezelowany autobus, który stał już jakieś
dwadzieścia pięć lat
nieopodal na wpół opuszczonego zakładu. Sądziliśmy,
że bandziorów będzie góra sześciu, a było ich jakieś
trzy, albo nawet cztery razy więcej. Rozpoczęła się
nierówna wymiana ognia.
Spychali nas coraz bardziej w~stronę zamarzniętych bagien i~na
szczęście nikt nas nie dostał pestką, również my nikogo nie
trafiliśmy. Odpuściliśmy, więc sobie strzelanie, już i~tak
sporo pocisków wystrzeliliśmy w~kierunku wroga. Rzuciliśmy się
w stronę odległego Wysypiska, a nasi prześladowcy nie
przerywali ognia, lecz nie usiłowali nas gonić. Dopadliśmy po
chwili porzuconej koparki i~po krótkiej wymianie zdań
postanowiliśmy się rozdzielić. Lorekin miał pewną sprawę u
Sidorowicza, a Bolo postanowił jemu towarzyszyć. Ja za to
udałem się z~powrotem do Baru.
\\
Trzymałem luźno w~dłoni mój wysłużony, brytyjski
karabin~--~IL-86. Nie zostało mi niestety do niego sporo
amunicji, a
miałem jej wcześniej\3k dużo, naprawdę dużo. Na sobie miałem
standardowy, stalkerski kombinezon Świt. Różnił się on jednak
od wielu innych~--~miał wszyty w~środku materiał termo aktywny,
dzięki czemu nie odczuwałem za bardzo chłodu. Było zrobione to
tak praktycznie, że kiedy nadejdzie cieplejsza pora roku mogę
zdjąć tę warstwę ubioru. Także kaptur został nieznacznie
zmodyfikowany, nie musiałem zakładać i~zdejmować maski przeciw
gazowej, bo wystarczyło, że wysunę ją z~kaptura okrywającego
głowę. W tej chwili akurat maska była schowana, bo nic mi nie
groziło.

W Zonie jestem już jakiś czas, około półtora roku. To czemu
tutaj się zjawiłem to bardzo skomplikowana sprawa. Pochodzę z
Polski, z~Wrocławia, a mój ojciec urodził się we Lwowie i~uczył
mnie języka ukraińskiego, więc potrafię porozumiewać się w
trzech językach: polskim, angielskim i~ukraińskim. Na studiach
poznałem Adę, zakochaliśmy się w~sobie, a zaraz po ukończeniu
nauki wzięliśmy ślub. Rok później urodziła się nasza córeczka-
Julka. Wszystko byłoby dobrze, gdyby nie to, że Jula urodziła
się z~nowotworem mózgu~--~była niepełnosprawna.
\\
Zgłaszałem się
wraz z~żoną do różnych fundacji z~prośbą dofinansowania
operacji, dzięki której Julka będzie mogła żyć bez choroby. W
ciągu tych trzech lat czepiałem się wszystkiego co mogłoby
mojej rodzinie zapewnić pieniądze, aż w~końcu postanowiłem dla
zarobku wyjechać tutaj, do Zony. To jednak tylko pogorszyło
sytuację. Ada porzuciła mnie i~musiałem wrócić do Polski, aby
wziąć rozwód. Sąd odebrał mi prawo opieki nad córeczką i~nie
mogłem nawet się z~nią spotykać. Więc zdołowany postanowiłem
wrócić do Zony, wzbogacić się albo stracić wszystko.

Minąłem nareszcie pagórki dzielące Wysypisko od Doliny Mroku.
Na posterunku broniącym przejście nikt nie stał, najwidoczniej
Wolnościowcy mają ważniejsze sprawy od tego. Szedłem w~stronę
masywu zrujnowanej budowli, w~której mieści się Pchli Targ.
Stalkerzy handlowali tam zdobytym przez siebie różnorakim
towarem, począwszy od kiełbasy, a skończywszy na pociskach do
wyrzutni rakiet. Oczywiście poza miejscem handlu, było to także
miejsce spotkań. To bardzo ważny punkt strategiczny na
Samotników i~nie tylko, często atakowany przez łatwo
odpieranych bandytów. Bandyta to nie człowiek, bandyta to wrzód
na dupie narodu. Stwierdził kiedyś tak Dywan, mój dawny kumpel
rozerwany niedawno przez przeciw pancerną minę.

Przechodziłem przez niezbyt gęsty, pełen niskich drzewek
zagajnik. Wiedziałem, że są tam anomalie i~znałem to miejsce na
tyle dobrze, iż potrafiłem je ominąć. Nie orientowałem się
zbytnio co to za anomalia, obchodziło mnie bardziej to, bym w
żadną z~nich nie wszedł. Podczas zimy poza tym, że anomalie nie
były takie niebezpieczne to także artefakty rzadko powstawały,
więc szansa na ich znalezienie była mizerna. Rzucałem co chwilę
przed siebie śrubki, akurat w~te miejsce, w~które miałem zamiar
pójść. Moja intuicja zawiodła mnie tylko raz, czyli tylko jedna
śrubka wpadła w~jedną z~tych dziwnych anomalii. Przed sobą
miałem zamarznięta bajoro i~Pchli Targ, całe szczęście, że nie
było nigdzie widać bandytów. Pospiesznie ruszyłem w~stronę
masywu budowli. Moje nogi, aż do kolan zagłębiały się w~grubej
warstwie śniegu. Zwolniłem jednak po chwili tempa, ponieważ
byłem strasznie zmęczony, a plecak w~którym miałem jedynie
termos, trochę amunicji i~chleb ciążył mi na plecach, jakbym
niósł w~nim kamienie. Jeszcze chwila i~będę na miejscu, może
nie spotka mnie pech i~zobaczę się z~Bergiem.
Doszedłem jakoś do celu i~powoli wdrapałem się po schodach.
Rzuciłem plecak i~karabin pod ścianę, a sam padłem zmęczony na
jakiś stary materac. Wiedziałem, że zaraz ktoś mnie zauważy i
radośnie się ze mną przywita. No i~zgadłem.
%
\sx Dobrze ciebie widzieć, Igi! – poznałem po głosie Makarona.-
Gdzie jest Lorekin? Wisi mi czterysta dwadzieścia rubli za
amunicję.
\xx Też dobrze mi ciebie widzieć, po nieudanym wypadzie Lorek
poleciał z~Bolkiem do Sidorowicza.
\xx Nieudany? Co się stało? A po co poleciał do naszego
drogiego
zgreda, Sidorowicza?~--~jak zawsze pytania i~pytania.
\xx Później powiem co się stało, daj mi odpocząć. Nie mój
zasrany
interes po co miał się spotkać z~Handlarzem.~--~to był sygnał
dla
Makarona, żeby się odczepił, teraz go jeszcze tylko spytałem na
odchodne: ~--~Widziałeś gdzieś Cięgło?
\xx No, jest na górze u naszego ,,doświadczonego technika''.
\qd
Uznałem, że skoro jest na wyższym poziomie u technika to zaraz
zejdzie tu na dół, do mnie. A co do tego technika\3k raczej
można go nazwać kimś kto czyści i~najwyżej trochę podreperuje
broń.
Teraz to on tylko jest na jakiś czas, oby jak najkrótszy. Wielu
stalkerów teraz musi biegać, aż do Baru, albo do Wolności,
jednak Bar jest bardziej lubiany od Hangaru. To zapewne za
sprawą lepszego położenia, stamtąd niedaleko do Dziczy,
Magazynów Wojskowych i~także kilku innych istotnych miejsc.
\\
W tym momencie jeden z~odpoczywających bywalców Pchlego Targu
podniósł swą starą, akustyczną gitarę i~zaczął grać smętnie
jakąś melodię, a w~tym samym momencie pojawił się Cięgło, więc
wstałem i~się z~nim przywitałem:
%
\sx Dobry.
\xx Dobry, dobry. Ten technik, idiota zarysował moje SGI!
Rozumiesz?! A jak próbowałem jemu poprawić trochę jego łuk
brwiowy to wachman mnie tutaj wywalił.~--~Cięgło powiedział
oburzony. Zawsze miał bzika na punkcie broni palnej, a jakiego
karabin szturmowy był jego oczkiem w~głowie.
\xx Nie przesadzaj, broń już ogarnąłeś to się zajmij pancerzami
haha. Mam dla ciebie pewną propozycję~--~możesz się ze mną
wybrać
do Baru, a później do Jantaru?
\xx Jasne, muszę pójść do technika z~Powinności, a co do
Jantaru
to zombiaczków chciało się pozabijać, co?
\xx Taa, tak dla zabawy. Przy okazji możemy przecież wziąć to
co
mają te upiory przy sobie. Czasem trafiają się prawdziwie dobre
fanty.~--~uzasadniłem swój wybór.
\xx No dobrze, przyda mi się trochę gotówki.
Idziemy?~--~zapytał
Cięgło.
\qd
Skinąłem głową, narzuciłem swój plecak na plecy,
wziąłem karabin w~dwie ręce i~powoli skierowałem się razem z
kumplem w~stronę Baru. Poza karabinem na amunicję NATO 5.56x45
miałem schowanego w~kaburze PBS’a. Miał już swoje lata, lecz
mimo tego, iż był używany przez wielu ludzi nie zawodził, ani
trochę. Wszystko działało sprawnie. Ktoś kiedyś powiedział:'' Z
radzieckimi rzeczami jest tak, że albo od razu się zniszczą,
albo będą niezniszczalne''. Ten kto wypowiedział te słowa miał
rację. Na samym początku pobytu w~Zonie używałem automatu
Kalasznikova z~lat siedemdziesiątych i~praktycznie co dziennie
musiałem wymieniać jakąś część.
\\
I w~tym momencie, gdy już byłem wraz z~Cięgłem niedaleko
posterunku Powinności rozległy się głośno strzały.
%
\ro{Rozdział II}
%
Odwróciłem się i~ujrzałem trzy postacie w~kominiarkach. Bandyci
znów podjęli próbę zniszczenia Pchlego targu, tylko że atakując
tak ,,licznym'' oddziałem dostaną jedynie tęgie lanie. Jeden z
nich był ubrany w~dosyć sponiewierany kombinezon Wolności,
pewnie chuj zdjął go z~trupa. Żaden z~przeciwników nie
wyglądał na groźnego i~nikt nie był zbyt dobrze uzbrojony, więc
broniący się stalkerzy z~łatwością ich pokonają.
\sd
\xx Słuchaj Cięgło, poczekajmy tu chwilę i~kiedy się uspokoi to
podejdziemy i~zarekwirujemy co nieco.~--~powiedziałem, na co
mój towarzysz w~zamyśleniu kiwnął głową.
\qd
Obydwaj oglądaliśmy nierówne starcie.\\
Bandzior w~kombinezonie Wolności, najwidoczniej dowodzący
podszedł najbliżej i~schował się za transformatorem i~darł się
na dwóch swoich kompanów, aby do niego dołączyli. Coś im nie
wyszło. Jeden z~samotników skutecznie ostrzelał z~Vintara
przeciwnika i~dwóch beznadziejnych pomagierów ,,Wolnościowca''
kopnęło w~kalendarz. Jeden z~nich dostał kulkę w~szyję, a drugi
w klatkę piersiową. Na śnieg trysnęła fontanna, ciemnej,
gorącej krwi. Z gardła ostatniego bandziora wydobył się
zduszony krzyk i~zostawiając swój stary, radziecki karabin
szturmowy rzucił się do ucieczki.
\\
Wyczułem okazję i~zaraz po
wycelowaniu wystrzeliłem w~stronę uciekiniera. Karabin zaciął
się, a tylko jeden z~wystrzelonych kilkunastu pocisków
trafiając cel w~udo, bandyta krzyknął i~runął w~zaspę przy
drodze.
\\
Ruszyłem truchtem w~stronę miejsca krótkiej wymiany ognia,
Cięgło ruszył za mną. Nie tylko my poszliśmy, aby przeszukać
trupy. Z Pchlego Targu po zasypanej drodze biegł zabójca dwóch
bandytów i~jeszcze jakiś samotnik. Szczerze mówiąc miałem ich w
dupie, biegłem tam tylko po to, by zająć się ,,Wolnościowcem''.
Przeładowałem pospiesznie swój karabin i~zwolniłem tempa,
przeciwnik pewnie ma pistolet i~nie zamierza umrzeć bez
honorowej, ostatniej walki. Ale moje przypuszczenia nie były
trafne. Krwawiący bandyta czołgał się zostawiając po sobie
ślad, starał się uciec. Z łatwością dopadłem go i~celnym
strzałem w~głowę dobiłem. Kolejny zabity przeze mnie człowiek,
pewnie nie ostatni. Jakbym służył jeszcze w~Wolności z~chęcią
bym go przesłuchał.~--~rozmyślałem sobie, kiedy dobiegł mnie
gniewny, lecz opanowany głos:\\
--~Zostaw ten pistolet kocie, on jest mój.~--~odwróciłem się i
zobaczyłem stalkera z~Vintarem celującego do mężczyzny z
Pchlego Targu. Ten wystraszył się i~porzucił łatwy łup.
Spróbował szczęścia ponownie i~udał się w~moją stronę. A niech
sobie bierze co chce, tylko wezmę naszywkę. Kucnąłem przy
ciepłych zwłokach i~zerwałem z~ramienia naszywkę Wolności,
sprzeda się w~Barze.

Wejście do bazy Powinności i~Baru nie zaskakiwało. Niezależnie
od pory roku zawsze na asfaltowej drodze i~na jej poboczu
leżały truchła zmutowanych psów lub wilków. Dziwnym trafem
mutanty nigdy nie miały ogonów. Zapewne ich krew nigdy nie
znikała i~cały czas pokrywała się nową warstwą, nikt nie miał
ochoty tego sprzątać, jedynie co jakiś czas martwe mutanty
zrzucano do głębokiego dołu wykopanego niedaleko drogi.
Kolejnym nieodłącznym elementem tego widoku byli, są i~będą
Powinnościowcy strzegący bazy za blaszanymi tarczami, a tuż
przed nimi rozpościerały się prowizoryczne zasieki
ograniczające bojowe możliwości mutantów. Gdy potwory
napotykały zaostrzone końcówki kijów, stalkerzy w~prezencie
wysyłali przeciwnikom naboje, za darmo.

Szedłem z~Cięgłem prosto do Baru. Przy wejściu przywitały mnie
zaciekłe, gniewne spojrzenia strażników. Wiedzieli, że
przyczyniłem się do śmierci kilku ich towarzyszy, lecz nie
zamierzali z~jakiegoś powodu nic mi zrobić. Nienawidziłem
przechodzić przez ,,bramę'' i~przez hangar kierujący wędrowców
do
Baru i~na Arenę. Czułem na sobie nienawistne spojrzenia moich
dawnych wrogów. Mogliby mnie zabić, lecz Łukasz, ich zasrany
dowódca nie pozwalał strzelać do spokojnych samotników. Mało
kto wznieca tutaj bójki czy strzelaniny, a jak do tego dojdzie
to Powinność szybko wyciąga konsekwencje.
%
\sx Nullus, ja idę na Arenę, żeby pogadać, pooglądać co się
dzieje.~--~poinformował mnie Cięgło.
\xx Dobra, zobaczymy się później. Tylko uważaj, żeby zamiast na
widownię nie trafić na arenę, hahaha.
\qd
Na Arenę przychodzili stalkerzy w~wielu celach. Jedni zakładali
się o to kto wygra w~walce w~specjalnie przystosowanym do tego
budynku, inni nie ryzykowali rubli i~tylko oglądali walczących,
a jeszcze inni przybywali tu by zaryzykować własne życie w
walce. Istniały trzy możliwości walk. Pierwsza to kiedy
stalkerzy walczyli ubierając własny kombinezon i~używając
własnej broni, dostawali wtedy jednak mniej pieniędzy. Kolejną
możliwością jest walczyć z~tym co się otrzyma od zarządców
Areny. Nie zawsze przeciwnicy są uzbrojeni tak samo, lecz to
zależy od stopnia zaawansowania walczących. Ostatnią natomiast
możliwością jest taka walka jak opisana powyżej, jednak kiedy
trwają zawody. Odbywają się one co kwartał i~są naprawdę
ciężkie do wygrania. Wielu stalkerów ginie w~początkowych
rundach przegrywając z~silniejszymi przeciwnikami.

Ostatnio mistrzem areny został Tancerz. Bardzo dziwny stalker.
Małomówny, spokojny, opanowany, mający zaledwie 170cm wzrostu i
ważący prawie 70kg wbrew pozorom jest śmiertelnie
niebezpiecznym przeciwnikiem. Najdziwniejszą rzeczą w~jego
wyglądzie były ciemno zielone włosy i~kredowo biała cera, a
poza zielonymi włosami miał także zielone oczy. Rzadko opuszcza
teren chronionej bazy i~wiele czasu spędza na arenie. Zawsze
nosi czarno-czerwony kombinezon Powinności, lecz nie należy do
żadnej frakcji. Kombinezon jest pozbawiony wielu elementów,
więc zapewnia mniejszą ochronę, a zarazem jest lżejszy i
wygodniejszy. A co do używanej przez niego broni\3k Tancerz na
arenie używał wyłącznie noża, bądź dwóch. Broń wybierali zawsze
zarządcy, lecz jeśli walczący prosił o ,,najsłabszą''
broń~--~nóż~--~to zostawał on jemu wręczany. Tancerz jednak
wyjątkowo dobrze potrafił posługiwać się tym orężem.
Przemieszczając się niemożliwie wręcz szybko i~zręcznie
potrafił stawić czoła
uzbrojonemu po zęby stalkerowi. Walka wręcz w~jego wykonaniu
była bardzo szybka i~dynamiczna, przypominała taniec, bardzo
szybki taniec. Był tak szybki i~zręczny, że jeszcze nigdy nie
został poważnie ranny. Tym, że pokonał wielu na arenie narobił
sobie sporo wrogów, którzy chcieli jego śmierci, ponieważ zabił
wielu czyichś przyjaciół, kolegów czy też ojców. Zdawał sobie z
tego doskonale sprawę, więc nie opuszczał bezpiecznej bazy
stalkerów.

Schodziłem już betonowymi schodami do Baru. Tutaj wszyscy, no
prawie wszyscy nie mogą do siebie strzelać. Pierwsze co można
poczuć wchodząc do tego miejsca to mocny zapach pieczonego
dzika, wysoka temperatura zabijała szkodliwe bakterie i~wirusy,
dzięki czemu mięso nadawało się do zjedzenia. Panowała tutaj
magiczna, wyjątkowa atmosfera. Stalkerzy prowadzili ożywione
dyskusje, przechwalali się, śmiali, stawali się pracownikami i
zleceniodawcami. Ja jednak dawno już skończyłem z~przyjmowaniem
zadań od stalkerów i~bezcelowymi rozmowami. Przychodziłem tutaj
w celu pohandlowania bądź porozmawiania z~dobrymi znajomymi.
Zobaczyłem tylko Kruka, czy też Wronę, nie wiedziałem jak go
tutaj nazywają. Ja znałem go wcześniej, w~dzieciństwie. Od
jakiegoś czasu jest handlarzem informacji w~Zonie, ja jednak
miałem wiele dosyć ogólnych wiadomości za darmo.
\sd
\xx Jak mija dzień, Andriej?
\xx Dobrze, panie Nullusie, bardzo dobrze.~--~w~Zonie mówili na
mnie Nullus, jednak tylko na Pchlim Targu jego bywalcy zwracali
się do mnie Igi.
\xx Zrobiłeś jakiś biznes, że taki ten dzień
dobry?~--~zapytałem.
\xx Tak, jakiś idiota pytał o plany Agropromu. Akurat miałem
jedne, więc opchnąłem je jemu za marne siedem tysięcy dwieście
rubli. A wiszą na drzwiach od Baru, ślepy jakiś. Hahahaha.-
Kruk (czy tam Wrona) uwielbiał się nie do końca uczciwie
wzbogacać. Wkręcił już wielu naiwniaków zarabiając przy tym
sporo kasy.
\xx Wiesz, że tego nie popieram. Planuję małą wyprawę do
Jantaru.
Coś ma się tam dziać ciekawego?
\xx A więc to ciebie dzisiaj ciekawi? Powinnościowcy coś tam
mówili o przejęciu fabryki w~Jantarze. Ale to raczej celowo, by
zmylić przeciwników i~wprowadzić ich w~błąd. Przecież to wręcz
nie możliwe do cholery, żeby zrobić coś takiego! Fale
psioniczne zamieniłyby ich w~bezmyślne zombii, a nawet jeśliby
wyłączyli te fale to wybicie zombii i~snorków byłoby wręcz
niewykonalne. Musieliby odesłać do tego setki swoich żołnierzy,
których nie mają haha. Ahh\3k nie mam ochoty rozgadywać się o
tym
co musieliby niewykonalnego zrobić haha. Jeśli to wszystko to
możesz już iść, panie Nullus.~--~powiedział Kruk (czy tam Wrona
czy jak tam jemu było\3k) po czym wstałem pożegnałem się z~nim
skinieniem głowy i~po woli skierowałem się do wyjścia.
\qd
Musiałem jeszcze tylko pójść do Szczura opylić naszywkę
Wolności i~udać się do technika Powinności i~mogę już szukać
Cięgła, a następnie udać się do Jantaru. Na początku poszedłem
do jednego z~mniejszych hangarów i~poszedłem do handlarza
drobiazgami~--~Szczura. Niski, koło trzydziestki, wyglądem
przypominający szczura mężczyzna wykupił od Powinności
najmniejszy hangar, albo raczej budyneczek i~zrobił w~nim coś w
rodzaju sklepiku. Skupował i~sprzedawał naszywki, kombinezony
różnych frakcji oraz informację, jednak nie był jakoś
szczególnie lubianym człowiekiem w~Zonie. Irytował swoimi
wrednymi odzywkami i~złośliwym zachowaniem, toteż starałem się
do niego nie przychodzić zbyt często.
\sd
\xx O przyszedł pan, panie Nullus. Nazbierało się troszkę
naszywek i~chce się je sprzedać, żeby mieć tyle pieniędzy co
ja?~--~powiedział na mój widok.
\xx Milcz, bo jeszcze coś ci się stanie gnoju.~--~nienawidziłem
gościa i~to jeszcze jak. Na sam jego widok miałem ochotę
chwycić go za jego chudą szyję i~cisnąć nim w~ścianę z~całej
ściany, tak żeby się już nie podniósł. Niestety nie mogę tego
zrobić, bo chuj zatrudnił dwóch najemników. Jeden z~nich miał
egzoszkielet i~trzymał w~dłoniach gietę, czyli GP37, a drugi
natomiast miał jedynie lżejszy pancerz~--~kombinezon Bułat.
\xx Spokojnie panie Nullus, pieniądze się przydadzą, a w~tej
chwili to pan na pewno takowych potrzebuje. Khy, khy
potrzebował\3k na córeczkę.
\xx Odwal się, bo coś tobie zrobię! Nie twoje sprawy.-
powiedziałem gniewnie.
\qd
Coś złowrogiego błysnęło w~moich oczach.
Nienawidziłem kiedy ktoś pcha się w~moje prywatne, a ten tutaj
przesadzał. Musiałem się opanować i~przejść do rzeczy. Rzuciłem
na ladę kilka naszywek i~powiedziałem:
\sd
\xx Masz i~się zamknij. Pięć naszywek renegatów, dwie bandytów
i
jedna z~Wolności.
\xx Dwadzieścia rubli za jedną renegatów, dwadzieścia pięć za
jedną bandytów i~pięćdziesiąt za tą z~Wolności.
\xx Łżesz! Przecież naszywki Wolności chodziły za osiemdziesiąt
rubli! Dawaj tyle ile powinno być.
\xx Panie Nullus, niech się pan uspokoi. Na ,,rynku'' są nowe
ceny,
trzeba się do nich przystosować panie Nullus. To co dostałeś
już tobie daję.~--~po tym położył na ladę dwieście rubli.
\xx Tym razem tobie odpuszczę, sukinsynu.~--~wziąłem pieniądze
i
wściekły wyszedłem z~budynku.
\qd
Skierowałem się powoli do technika, żeby zrobił coś z
zacinającym się mechanizmem karabinu, gdy minąłem jakiegoś
stalkera, który ożywiony rozmawiał ze swoim kumplem
\sd
\xx \3k a widziałeś jak Róg tak sprawnie uniknął jego strzałów
i
trafił go ze swojej dwururki? Wygrał raniąc tego\3k jak jemu
było? A no tak! Cięgło! Ale krwawił\3k
\qd
Przeraziłem się i~aż nie mogłem uwierzyć, że Cięgło poszedł na
Arenę walczyć. Co za idiota\3k Nie pójdzie ze mną do Jantaru i
zapewne do końca zimy nie opuści bazy. Cholera, jego problem.
Udam się do niego za jakiś czas, teraz jestem
zajęty~--~rozmyślałem sobie.

Teraz podążałem niespiesznie przez zasypaną śniegiem bazę do
warsztatu Sosny. Już od jakichś dobrych dwóch godzin panował
mrok, w~zimowy czas w~Zonie ciemność zapadała szybciej niż
gdzie indziej. Szedłem na wpół odśnieżoną uliczką, którą
otaczały odrapane, zaniedbane budynki, a na poboczu leżały
różnorakie pudła, skrzynie i~inne nie potrzebne już nikomu
śmieci. Po chwili marszu doszedłem do zniszczonych drzwi, z
których łuszcząca się brązowa farba odpadała płatami.
\\
Otworzyłem drzwi i~wszedłem do długiego korytarza, a brudne,
okratowane okna sprawiały niemiłe wrażenie zaniedbania. Brudne
były nie tylko szyby ale i~ściany, a także pokryta kurzem
podłoga.
\\
Na końcu korytarza były ciężkie metalowe drzwi, które
otworzyłem pchnięciem dłoni. Sosna poza zajmowaniem się naprawą
i modyfikacji broni sprzedawał do niej amunicję, granaty, a
nawet czasami miał na składzie jakąś broń snajperską. Miejsce
pracy opuszczał w~soboty i~wyruszał w~teren polować na mutanty
oraz bandytów. Miał około dwadzieścia pięć lat, lecz groźnie
wyglądająca blizna i~gęsta broda postarzała go o kilka lat.
\sd
\xx Chcę kupić amunicję NATO i~dobrze, by było jakbyś trochę
podreperował mój karabin.~--~krótko i~zwięźle od drzwi
powiedziałem po co przyszedłem.
\xx Hmm\3k mało mam teraz amunicji, dwa magazynki wystarczą?-
skinąłem niechętnie głową, pieniądze mi się kończyły i~nie
miałem ich zbyt wiele, lecz wolałbym mieć sporo naboi.~--~A co
do
twojego karabinu to sprawdzę co się z~nim stało dopiero jutro,
bo dzisiaj muszę jeszcze zmodyfikować TRs’a.
\qd
Kiwnąłem głową, położyłem na stoliku należne dwieście
trzydzieści rubli, wziąłem zakupioną amunicję i~wyszedłem bez
słowa z~budynku. Wzięła mnie ochota na jakieś małe piwko, już
dawno nie piłem niczego związanego z~alkoholem. Udałem się
powolutku do Baru i~byłem w~nim już po dwóch minutach krótkiej
drogi. Wieczorem odpoczywało tutaj najwięcej bywalców.
\\
Niektórzy postanowili utopić swoje smutku w~kuflu piwa, bądź
też w~kieliszku wódki, albo raczej w~kilku kuflach czy
kieliszkach. Wielu stalkerów grało w~karty i~nie zwracali uwagi
na ściany pozbawione tynku, zniszczoną podłogę czy odrapane
ściany. To nie było dla nich ważne, mało to ich obchodziło, bo
w końcu przyszli tutaj odetchnąć, a nie podziwiać zniszczony
lokal.

Podszedłem do tłustego Barmana i~poprosiłem o małe piwo, które
dostałem chwilę później. Postanowiłem podejść do grupki
grających w~karty mężczyzn i~podejrzeć jak grają. Nigdy mnie to
nie kręciło to też po chwili odszedłem od nich z~pustym kuflem,
odstawiłem go na ladę i~znużony wyszedłem znowu na
powierzchnię. Musiałem znaleźć miejsce do odpoczynku, więc
wszedłem do jednego z~hangarów zajętych przez samotników i
usiadłem przy ognisku. Akurat jeden ze stalkerów opowiadał
jakaś na wpół wymyśloną historyjkę:
\sd
\xx \3k i~jak mówiłem skradałem się do nich powoli, powoli, a
ci
durnie niczego się nie spodziewali! Już miałem zaatakować ich
pod osłoną ciemności kiedy nagle rzuciła się na nich pijawa!
Takiej to ja jeszcze nie widziałem, miała prawie trzy metry, a
jak groźnie wyglądała! Gdy już zręcznie uśmierciła dwóch
najemników to wiedziałem, że wyczuła moją obecność, więc
wystrzeliłem do niej z~granatnika, ale się rozjebała na setki
kawałków!~--~opowiadał podniecony stalker, wyglądający jeszcze
na
kota.
\qd
Pewnie na widok nibypsa spieprzał ile sił w~nogach,
idiota. Nie miałem ochoty słuchać tych bredni, więc zająłem
miejsce w~kącie hangaru i~zmęczony czekałem na nadejście snu.
%
\ro{Rozdział III}
%
Wstałem dosyć wcześnie, tuż przed siódmą rano. Byłem głodny,
bardzo głodny, więc wyjąłem resztki prowiantu z~plecaka i
rozpocząłem poranny posiłek. Do zjedzenia miałem naprawdę
niewiele~--~trzy kromki chleba, połowa jakieś konserwy w
blaszanej puszce i~termos herbaty do popicia. Nie smakowało to
jakoś wytwornie, ot zwykły, pożywny posiłek typowego stalkera.
\\
O tej porze roku słońce dopiero miało wstać, więc w~hangarze
panował mrok, z~którym walczyły płomienie ogniska. Drzwi były
zamknięte. Przecież nikt nie chciał, by w~hangarze panował
chłód. W budynku poza mną siedziało przy ognisku jakichś dwóch
stalkerów. Jeden z~nich spojrzał na mnie i~wrócił do swoich
wcześniejszych zajęć. Nie byli jakoś specjalnie dobrze
wyposażeni. Oceniałem ich na kotów, nowicjuszy, a może i~trochę
zwykłych stalkerów. Pierwszy z~nich, wyższy miał na sobie
kombinezon najemnika, lecz nic nie świadczyło, o tym że można
go nająć na przykład do ochrony. Drugi jednak miał na sobie
tylko poplamiony krwią, ciemnobrązowy sweter, a jego
standardowy kombinezon ,,Świt'' leżał obok. Był wyraźnie
podniszczony i~splamiony krwią, ciekawe co ci dwaj przeżyli
jakiś czas temu.

Przerwałem w~pewnym momencie swój posiłek i~to co zostało z
powrotem włożyłem do plecaka, nie było tego wiele. Poza tym co
miałem z~sobą wziąłem jeszcze stary, duży koc. Później się mi z
pewnością przyda. Planowałem czym prędzej wyruszyć do Jantaru,
by zapolować na zombii, wyszedłem z~hangaru i~opuściłem obóz.

Do Jantaru szedłem dłuższą trasą przez Wysypisko i~okolice
Agropromu. Nie zamierzałem walczyć z~najemnikami oraz mutantami
z Dziczy. Droga minęła mi spokojnie. Mutanty, które napotykałem
na swojej drodze nie miały najmniejszej ochoty ze mną walczyć,
więc uciekały spłoszone przede mną. Wszystkie anomalie jakie
czekały na nieuważnych stalkerów minąłem używając metody z
śrubkami. Nie spotkałem nawet żadnego bandyty czy innego
zawadiaki. Dzisiaj najwidoczniej miałem szczęście, chociaż nie
do końca. Cały czas, w~odległości co najmniej stu metrów szedł
za mną jakiś stalker. Musiałem przyznać, że zaniepokoił mnie
jego widok. Miał zapewne na sobie ciężki kombinezon ,,Bułat''.
Nie widziałem z~takiej odległości dokładnie. Ale co tam, nie
ważne, najlepiej jest się nie przejmować.

Stałem na pagórku, przeze mną rozciągały się bagna, dawniej
będące jeziorem oraz niedaleko trujących moczar stał bunkier
jajogłowych. Daleko było widać masyw budynków starej,
opuszczonej fabryki, na terenie którego teraz grasowały zombii
oraz snorki.

Jajogłowi czyli naukowcy stacjonujący w~swoim
bunkrze akurat tutaj zajmowali się przede wszystkim badaniem
fal psionicznych, które zamieniał stalkerów w~bezmyślne zombii.
Bunkier broniony był przez jeden, liczny oddział najemników
odpierający przeważnie ataki stalkerów, którzy już nie byli
ludźmi. Snorki to natomiast przeważnie wojskowi w
charakterystycznym maskach przeciw gazowych ,,Słoń'' wystawieni
na zbyt długi czas na radiację. Żołnierze przeszli mutację i
przeistoczyli się w~krwiożercze potwory poruszające się na
czterech, mocno umięśnionych kończynach. Najczęściej kombinezon
jest rozerwany w~wielu miejscach i~można zobaczyć okaleczony,
zakrwawiony tułów.

Udałem się w~stronę drogi prowadzącej do fabryki, dzieliło mnie
od niej około tysiąc metrów. Przedzierałem się przez krzaki
leżące u podnóży stromych pagórków. Nie chciałem, żeby
ktokolwiek mnie zauważył. Nie lubiłem tego miejsca, zawsze było
tutaj brzydko i~nie poprawiał tego wrażenia biały śnieg. Po
kilkunastu minutach marszu dotarłem do zwykłej, kiepsko
utwardzonej drogi, po której bokach stały krzywo wbite słupki.
Szedłem wzdłuż drogi do fabryki i~mijałem co jakiś czas
zdezelowany pojazd, czy to furgonetkę, czy to autobus. Miałem
wrażenie, iż ktoś mnie cały obserwuje.

Odwróciłem się tknięty
przeczuciem i~wzrokiem szukałem potencjalnego prześladowcy.
Nikogo nie zauważyłem, jeśli ktoś w~ogóle za mną podążał musiał
umieć dobrze się ukrywać. Dodatkowo łatwiej było jemu mnie
śledzić, ponieważ zostawiałem w~śniegu ślady wojskowych butów,
przez niektórych potocznie nazywanych glanami. Zamiast dalej
rozmyślać nad niewidzialnym wrogiem przyśpieszyłem tempa.
Zwolniłem dopiero niedaleko murów okalających dawny zakład
pracy. Dopiero tutaj zwykła ,,wiejska'' droga zmieniała
nareszcie
swoją powierzchnię na asfaltową. Na pokrywie śniegu przed
główną bramą na szczęście nie było widać żadnych, nawet
najmniejszych czyichś śladów.

Zbliżyłem się powoli do skrzyń
leżących i~tym samym prawie tarasujących wejście na teren
fabryki i~ostrożnie spoza z~nich wyjrzałem. Na widoku były
tylko trzy zombii. Przyjąłem moją taktykę polegającą na
wywabieniu stworów poza bramę, a następnie zlikwidowanie ich z
ukrycia.

Jak znam życie kryje się ich tu więcej i~zamiast
trzech nadejdzie ich tutaj niewiele więcej. Sprawdziłem czy w
magazynku mojego karabinu znajduje się amunicja,
odbezpieczyłem, oparłem broń na swoim ramieniu, wycelowałem i
oddałem dwa strzały w~stronę najgroźniej wyglądającego zombii.
Nim zrozumiał co się stało zdążyłem już wystrzelić do jego
dwóch pozostałych kompanów. Takiego stwora ciężej było pod
pewnym względem zabić. Skóra podczas mutacji stwardniała, a
wiele komórek odpowiedzialnych za odczuwanie bólu obumarło,
przez co zombii miał jakby drugi pancerz. Oczywiście wraki
dawnych stalkerów miały na sobie kombinezony. Czasami, lecz
bardzo rzadko trafiają się mutanty w~egzoszkieletach, lecz
jeszcze nie przytrafił mi się taki pech. Odszedłem jakieś
dwadzieścia metrów od bramy i~kucnąłem za autobusem, zerkałem
co chwilę czy nie nadchodzą przeciwnicy. I w~końcu pojawił się
jeden zombii, a za nim kolejne cztery. Stały w~tym miejscu skąd
oddałem strzały i~nie miały pojęcia co robić. Odczekałem
chwilę, by mieć pewność, że żaden kolejny mutant nie nadejdzie.

W końcu po kilku minutach czekania otworzyłem ogień. Strzelałem
seriami do pięciu pocisków i~starałem się, by pestki trafiały w
głowę, bądź jej okolice. Już po chwili śnieg zabarwił się na
czerwono i~tylko jeden z~ocalałych potworów otworzył do mnie
ogień jednocześnie idąc w~moim kierunku niemrawo. Wszystkie
kule wystrzelone z~jego karabinku Viper 5 świstały mi nad głową
lub trafiały niewinny autobus. Spokojnie, nie zwracając uwagi
na ostrzał przeciwnika sam odpowiedziałem jemu ogniem kończąc
tym jego wędrówkę po ziemi. Podszedłem powoli do jednego z
zombii i~mocnym kopniakiem zakończyłem jego przedśmiertne
konwulsje.

Jak każdy stalker nie zostawiałem łupu przy o
ofiarach to też zacząłem przeszukiwać trupy. Przy pierwszym
najbardziej wojowniczym zombii znalazłem użyty zresztą
karabinek Viper. Niechętnie wziąłem się przeszukiwania reszty
ciał.

Szczególnie cennych łupów nie znalazłem, jednak trochę tego
było. Wziąłem sporo egzemplarzy broni palnej, która była w
różnym stanie. Poza Viperem znalazłem starą dubeltówkę, AK
74/u, dwa karabiny szturmowe AK 74/2 oraz mocno zniszczony
IL86. No i~oczywiście znalazłem broń mniejszego kalibru, czyli
USP i~dwie PMm’ki. Cały ten skromny arsenał poza pistoletami
położyłem na wziętym wcześniej przeze mnie kocu i~zacząłem
ciągnąć go za sobą. Całą amunicję oraz trochę przydatnego
prowiantu włożyłem do plecaka, który nie ciążył na plecach zbyt
mocno, lecz odczuwalnie. Zdobytą broń mogę opchnąć taniej
najemnikom, którzy później sami ją sprzedadzą, albo pomęczyć
się i~sprzedać ją na Pchlim Targu lub w~Barze. Potrzebowałem
jednak pieniędzy, więc postanowiłem zaryzykować i~sprzedać
drożej swój pryz.

Ehh\3k znów trzeba wracać tą, nudną drogą. To
był właśnie ten minus dla samotnych samotników. Długie
nieciekawe podróże, choć lepsze gdy są nudne, niż kiedy
zaatakuje ciebie jakiś wróg. Nie można było powiedzieć, że nie
lubię chodzić. Robiłem to nawet bardzo często. Tutaj jednak
doskwierała mi samotność, teraz zapewne wracałbym z~Cięgłem do
Baru, jednak ten idiota musiał pójść na Arenę, ehhh. W innej
porze roku takie samotne przemierzanie byłoby niebezpieczne,
cholernie niebezpieczne. O tej porze jednak nie groziło mi
wiele, szczęście i~pech dla niektórych. A teraz wędrowałem
przez te pustkowia z~łupem i~miałem obawy co do podróży. Mimo
tego, że była zima to ktoś i~tak może polować na nieuważnych
\mbox{wędrowców}.

Rozważałem nad udaniem się do siedziby Samotników
mieszczącej się niedaleko. Ostatnio w~Zonie jest mniej
stalkerów, więc ciężko znaleźć kupca na broń, nawet tą w
lepszym stanie. Idę cały czas drogą powrotną, czyli taką samą
jak wcześniej jednak coś nie mi tutaj nie pasuje. Nie mam
pojęcia co, może kruki i~wrony są cicho?
\\
--~Musisz być czujny,
Nullus.~--~podpowiadał mi mój cichy głosik w~głowie. Co do
cholery się dzieje? Można powiedzieć, że wszystko ucichło.
Nawet moje kroki na wilgotnym śniegu nie są takie\3k głośne. I
nagle zobaczyłem to.
Nie było to nic niezwykłego, lecz czułem
się na ten widok dziwnie, bardzo dziwnie.

ON stał tam cały czas
bez ruchomo. Niby to stalker, stalker w~pancerzu ,,Bułat'', ale
coś tutaj nie pasowało. Stałem dwadzieścia metrów od tego
człowieka i~nie mogłem uciekać. Do dwóch metrów brakowało jemu
pewnie dziesięć centymetrów, albo ewentualnie trochę więcej.
Nie widziałem jego twarzy, którą zasłaniała maska. Nic
szczególnego się nie stało, ale się strasznie bałem. Panowała
nienaturalna cisza i~przeważnie tylko dlatego się bałem, jednak
w tej złowrogiej postaci coś było\3k

W pewnym momencie
dostrzegłem coś na jego ciemnym niczym atrament pancerzu.
Czerwone, jaskrawe słowa w~jakimś dziwnym, nieznanym alfabecie
pojawiły się z~nikąd. Jednak mimo, iż nie miałem pojęcia co to
za dziwaczne znaki układające się w~wyrazy, to znałem ich
znaczenie. Mówiły wręcz do mnie – Zgiń! – Zgiń! Wtedy też
odezwał się cichy głos – Nullus, uciekaj póki możesz, albo
zgiń! Ledwo oderwałem ociężałe nogi od ziemi i~rzuciłem się do
szaleńczego biegu w~stronę bunkra naukowców. Nigdy nie biegłem
tak szybko jak teraz, pewnie biję rekordy prędkości. Przerażony
bałem się odwrócić, ale jednocześnie tego chciałem. Walczyłem
sam ze sobą i~jedna moja część wygrała. Spojrzałem za siebie i
zauważyłem\3k
%
\ro{Rozdział IV}
%
Spojrzałem za siebie zatrwożony i~zauważyłem to, że nie było
widać złowrogiej postaci. Odetchnąłem z~ulgą. Strasznie bolała
mnie głowa i~nie tylko. W jednej chwili poczułem się bardzo
zmęczony i~ospały. Incydent wybił mnie z~równowagi. Co się w
ogóle stało? Jakiś stalker w~pancerzu stał w~miejscu i~pewnie w
masce przeciwgazowej miał zainstalowane dwie diody świecące na
czerwono. Taki żart dowcipnisia, czego ja się wystraszyłem?

Przypomniałem jednak sobie tę chwilę, nienaturalną ciszę,
złowrogą atmosferę i~nie do końca wierzyłem w~opcję, że ktoś
postanowił zażartować sobie kosztem innych. Szczerze mówiąc nie
mam pojęcia co się stało chwilę temu, muszę odpocząć. Wciąż
zszokowany tym jakże dziwnym wydarzeniem, szedłem powoli w
stronę bunkra naukowców. A może oni coś będą wiedzieć na ten
temat? Szybko odrzuciłem tą myśl, pewnie powiedzą, iż przez
radioaktywność i~fale psioniczne mam jakieś pieprzone majaki.
Durnie, nie wyjrzą z~tej swojej ,,twierdzy'' i~się wymądrzają.
Muszę iść z~powrotem do Baru, coś zjeść, pogadać, sprzedać i
może pójdę do Cięgła\3k Aj, to nie ważne, jestem ciekaw co się
stało chwilę temu. Kto może to wiedzieć? Medyk? Jajogłowi? Nie
mam pojęcia. Dobra, później pomyślę, teraz muszę wrócić z~tym
co mam do 100Radów.

Droga minęła spokojnie, absolutnie nic się nie działo. I
dobrze, pokonanie tej samej trasy samemu w~lato, a nie w~zimę
jest naprawdę dużym osiągnięciem. Znużony zszedłem do zatęchłej
piwniczki nazywanej barem. Ostatni wypad poza małym łupem
przyniósł mi tylko nie potrzebne zmartwienie, a zarazem
tajemnicę. Podszedłem do Kruka i~zagadałem:
\sd
\xx Widzisz, mam problem wiążący się jakby z~mitami Zony. Czymś
dziwnym\3k rzadkim. Kto może coś o tym wiedzieć?
\xx Jak to kto? Leśnik, chyba wiesz kto to.~--~Kruk
odpowiedział z
serdecznym, słowiańskim uśmiechem na twarzy.
\qd
Przyszedłem do Baru tylko po to i~nic więcej, nie mówiąc nawet
‘dzięki’ wybiegłem z~piwniczki na powierzchnię. Muszę sprzedać
handlarzowi broń, która pozostawia wiele do życzenia. Od
wczoraj w~bazie praktycznie nic się nie zmieniło, tylko śniegu
jeszcze napadało. Wszyscy sprawiali wrażenie ospałych i
znużonych brakiem zajęcia. Nareszcie doszedłem do ośnieżonej
kanciapy handlarza bronią. Zmierzył mnie swoimi oczami i~czekał
co zrobię. Położyłem na ladę znalezione pisi bez słowa czekałem
na wycenę. Stalker wziął powoli do ręki Makarova i~dokładnie go
obejrzał, sprawdził czy się nie zacina, ani czy nie ma żadnych
usterek. To samo zrobił z~drugim pistoletem i~wycenił:
\sd
\xx Pierwszy nawet nieźle się trzyma, wszystko chodzi jak
należy,
choć za jakiś czas ktoś musiałby go naprawić, trzysta
rubli?~--~
zapytał i~skinąłem w~zamyśleniu głową~--~Drugi już taki świetny
nie jest, jeśli w~tydzień go nie sprzedam to będę musiał oddać
go technikowi. Nie wygląda najlepiej, dam tylko sto
pięćdziesiąt i~ani rubla więcej. A za ten ostatni, USP dam
tobie marne trzysta pięćdziesiąt, podniszczony mocno jest.
Pasuje?~--~przytaknąłem głową i~mężczyzna wręczył mi plik
banknotów, wiele tego nie było.
\qd
Wyszedłem już bez skromnego łupu. Nie pozostało mi nic do
roboty, więc znowu udałem się do Baru. Może ktoś będzie miał
coś ciekawego do powiedzenia. Kątem oka zobaczyłem dwóch
stalkerów idących za mną. Odniosłem wrażenie, że skądś ich
znam. Odrzuciłem szybko tą myśl, przecież w~Zonie jest wiele
ludzi. Przyśpieszyli kroku wyraźnie idąc w~moją stronę.
Pierwszą myślą była ucieczka, tylko po co? Jestem zbyt nerwowy,
opanuj się Nullus. Wyższy z~stalkerów powiedział do mnie:
\sd
\xx Pójdziesz z~nami, tylko niczego nie próbuj.
\xx Spieprzajcie, posłańcy. Nigdzie z~wami nie
idę.~--~odpowiedziałem na zaczepkę.
\qd
Zdawałem sobie sprawę, że mogą mi
coś zrobić, a strażnicy nawet nie ruszą palcem, by mi uratować.
Ruszyłem szybkim krokiem w~stronę Baru mijając dwóch\3k
bandytów?
Usłyszałem wtedy zirytowany głos jednego z~nich:
\sx Nabój!
\qd
Ból rozsadzał moją czaszkę, widziałem jakieś niewyraźne
kształty przed sobą. Spróbowałem dotknąć ręką swoją, ciężką,
obolałą głowę. Nie mogłem. Coś krępowało skutecznie ruchy moich
rąk i~nóg. Siedziałem na stary krześle, przywiązany do niego.
Gdzie ja jestem? Nagle ktoś wylał na mnie wiadro zimnej wody.
Rozbudziłem się na dobre i~zacząłem kaszleć ochryple.
Podniosłem głowę i~ujrzałem stare, masywne biurko. Nie sam
mebel mnie zaintrygował, lecz postać przy nim siedząca.
\sd
\xx Widzisz Nullus, nie chciałeś po dobroci, to masz na
przymus.
Byśmy spokojnie pogadali, a tak to musisz siedzieć związany.
Szkoda, że muszę ciebie zmuszać.
\xx Ty gnoju!
\xx Spokojnie, powinieneś dziękować. Chłopcy ciebie nieśli z
Baru, aż do naszych skromnych magazynów wojskowych. Musieli się
zmęczyć.~--~drwiny przesiąkały słowa stalkera, a raczej dowódcy
stalkerów.
\xx Ale po co ja tobie?! Spłaciłem dług, zostałem wolnym
stalkerem i~jesteśmy kwita!
\xx Nullus, szukam ludzi. W tym ciebie.
\xx Nie ma mowy, Łukasz. Od czarnej roboty to masz swoich
durnych
anarchistów! Chcesz, żebym dołączył do masakry pod Barierą?
Myślisz, że nikt nie wie co tam się dzieje?! Wysyłasz tam
dziesiątki stalkerów na bezsensowną rzeź z~Monolitem!
\xx Czy ktoś mówił, że jesteś mi potrzebny akurat do tego? A ta
cała twoja rzeź to próba powstrzymania fanatyków i~ich
zabójczych ideologii. Durna Powinność myśli, że to jedyne co
robimy. Nie spodziewają się niczego.
\xx Łukasz, co ty do cholery kombinujesz?!~--~nic już nie
rozumiałem.
\xx Chyba ciebie moi dzielni chłopcy za mocno zdzielili po
łbie,
haha. Mam mało ludzi, coraz mniej. Sam chyba wiesz dlaczego. W
zimę tylko Monolit jest aktywny i~ktoś musi go powstrzymać.
Zostało mi tylko kilku naprawdę dobrych stalkerów. Ty też do
nich należysz.
\xx Tylko, że już nie jestem w~Wolności i~nie zmusisz mnie bym
do
niej wrócił.
\xx Czy ja mówię, że masz być w~Wolności? Ja chcę tylko, żebyś
mi
pomógł w~pewnej sprawie. Potrzebuję doświadczonych ludzi, a nie
kotów. Zrobisz co miałeś zrobić i~jesteś wolny. Oczywiście
dostaniesz jakąś sumkę w~nagrodę za służbę\3k~--~Łukasz
przeszedł
do konkretów.
\xx Naprawdę myślisz, że będę dla ciebie pracował? Nie żartuj
sobie. Rozwiąż mnie i~wypuść cholero.
\xx Mało kto teraz odwiedza wieżę ciśnień i~pijawki są
spragnione
krwi, ktoś musi je nakarmić.~--~ta groźba do mnie trafiła. Czy
w
taki sposób chcę skończyć swoje życie?
\xx Mów o co chodzi.
\xx Dwie grupki naszych mają przejąć Agroprom. Znaleźliśmy
ścieżkę, dzięki której podejdziemy prawie pod sam Agroprom
mijając wszystkie posterunki Powinności. Pierwsza,
czteroosobowa grupa ma przejść pod południową bramę po drodze
eliminując wszystkie oddziały Powinności. Druga grupa, co
najmniej dwa razy większa dojdzie później i~na sygnał zaatakuje
od bramy wschodniej.
\xx Szaleńczy plan\3k jak niby w~czterech mamy przejść przez
wszystkie posterunki?
\xx O ludzi to się nie martw, lepszych ze świecą szukać. A
sprzęt
dostaniesz od Maksa, tylko później musisz go z~powrotem oddać i
weźmiesz to co miałeś, umowa stoi?
\xx No, nie mam wyjścia.~--~odpowiedziałem niechętnie.
\xx Dobra, to może się przebierzesz, ogarniesz i~idź do Maksa.
\qd
Poszedłem, świeży, umyty do jednego z~baraków. Tam podobno
przebywał Maks. Tutaj było całkiem inaczej niż u Powinności.
Panowała samowola. Tu i~ówdzie ktoś jarał skręta, wielu
stalkerów grało w~karty z~pieniądze, wszędzie było słychać
radosne okrzyki i~rozmowy. Mimo luźnej atmosfery główna
siedziba Wolności była dobrze chroniona przez przede wszystkim
snajperów. W końcu znalazłem Maksa, leżał na jednym z~polowych
łóżek i~czyścił rewolwer.
\sd
\xx Łukasz mówił, żebym poszedł do ciebie po wyposażenie.
\xx O, Nullus! Brakowało tobie starych, dobrych kompanów, co?
\xx Można tak powiedzieć, gdzie ten sprzęt?~--~nie miałem
ochoty na
rozmowę.
\xx Czekaj, już daję.~--~mówiąc to podszedł do wielkiej skrzyni
leżącej pod oknem i~wręczył mi to czego potrzebowałem.
\xx Dzięki.~--~dźwigając sprzęt usiadłem na pryczy
i~przejrzałem to
co otrzymałem. Kombinezon ,,Strażnik Wolności'' miał w~jednej z
dodatkowych kabur pistolet Kora z~tłumikiem. Z broni długiej
dostałem wyciszony karabin snajperski L96A1. Najbardziej
cieszył mnie jednak mały karabinek typu SMG Scorpion.
Spojrzałem na sprzęt w~uznaniem i~odłożyłem go na bok.
\xx Maks, kiedy wyruszamy na Agroprom?
\xx Jutro rano, lepiej odpocznij.
Skinąłem głową, a że było ciemno, położyłem się czekając na
sen. W oknie zobaczyłem ciemną sylwetkę, której oczy się jakby
świeciły na czerwono. Podskoczyłem w~pierwszej chwili na łóżku
i nie mogłem oderwać wzroku od złowrogiej postaci. Było w~niej
coś hipnotyzującego. Zrobiłem jednak coś co przerwało ten
nienaturalny trans. Mrugnąłem. Postaci za oknem już nie było.
\xx Nullus, co się dzieje?~--~Maks zapytał się mnie troskilwie.
\xx Nic, nic. Masz szluga? Muszę zapalić.
\xx Tak, mam. Łap.~--~mówiąc to towarzysz podał mi papierosa.
\qd
Nie paliłem nigdy nałogowo. W ogóle paliłem tylko w~momentach,
gdy byłem naprawdę zdenerwowany, zatrwożony. A szczególnie, gdy
wiele negatywnych emocji się skumuluje. Ten kawałek bibuły z
mieszanką tytoniu uspokajał moje psychiczne problemy, dzięki
papierosom w~trudnych sytuacjach umiałem spokojnie pomyśleć,
zapanować na sobą. Narzuciłem na siebie kurtkę i~wyszedłem z
baraku zapalić. Jakieś dziwaczne monstrum z~Zony postanowiło
mnie śledzić czy co? Chyba tak. Na dodatek, w~końcu kiedy
postanowiłem udać się do Leśnika, to Wolnościowcy musieli mnie
porwać. Zajebiście, no Nullus, po prostu zajebiście\3k Jeszcze
szanse na przeżycie tego durnego napadu na Agroprom są naprawdę
małe. Taktyka do najlepszych nie należy\3k Ale trzeba w~końcu
spojrzeć też pozytywnie. Jeśli wszystko się uda to zgarnę
trochę pieniędzy i~wyjdę na prostą, może. Wtedy trzeba będzie
pójść do tego dziadka, Leśnika i~co nieco się dowiedzieć o tym
czymś co mnie\3k śledzi? Może nawet bym opuścił Zonę? Chociaż
to
i tak nie ma sensu, bo co bym tam robił?

Teraz już tylko strażnicy stali na swoich wartach na zimnym
powietrzu. Nie było ich i~tak wielu, nikt nie spodziewał się
jakiegokolwiek ataku. Wszyscy inni Wolnościowcy odpoczywali w
starych, wojskowych barakach. Kiedyś było tu inaczej\3k
Papieros
wypalił się już prawie całkowicie, rzuciłem więc niedopałek w
głęboki śnieg i~wróciłem z~powrotem do schronienia. Trzeba już
iść spać, ciekawe co przyniesie kolejny dzień.
\end{document}