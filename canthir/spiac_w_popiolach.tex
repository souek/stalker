\documentclass[../MAIN.tex]{subfiles}
\begin{document}
\ro{Śpiąc w popiołach - I}  
%
Zimno\3k Twardo\3k Źle\3k Kurna, ileż można leżeć na gołym betonie przykryty tylko wojskową kurtką? Wiedziałem, że ta wycieczka do Chabna źle się skończy\3k Mówili przecież, że miasto puste, że żywej duszy nie ma, żeby sobie nim głowy nie zawracać, bo to strata czasu. Ale nie, mi coś nie pasowało i musiałem pójść i samemu sprawdzić. No i polazłem, jak inni przede mną. Którzy leźli i nie wracali. Dlaczego? Niby płot zaraz dalej, ale przecież nie musieli tak daleko leźć, żeby z Zony czmychnąć. To nie ma sensu\3k

Wstałem z podłogi, rozprostowałem kości i oparłem się o ścianę. Spękany tynk posypał się z cichym chrzęstem pod nogi. Otrzepałem pył i kurz z ubrania, zebrałem plecak i wyjrzałem przez okno. Szaro. Niebo zasnute nieprzeniknionym ponurym stropem chmur, leniwy obłok mgły wypełzający spomiędzy drzew, pękający beton budynków i dziurawy asfalt zarastających dróg. Pustka. Pustka tak namacalna, że na wskroś przenikająca człowieka jak zimny wiatr. Brrr, nie lubię tego uczucia. Mrozi krew w żyłach, pozbawia oddechu, mąci w głowie. To chyba nie jest normalne, nie czułem czegoś podobnego poza Zoną. A może tylko nie zwracałem uwagi? Z resztą czy to ważne co jest poza Zoną?

Na drodze pojawił się pies. Gdzieś dalej zaszczekał jeszcze jeden, po chwili kolejny. Dobry znak, jest jakieś życie. Mutasów nie spodziewałbym się, to nie ich region, tutaj tylko kundle\3k O ile nie są wygłodzone nie stanowią problemu, płoszy je strzał z śrutówki. Tylko u mnie z śrutem krucho, zostawiam na poważniejsze okazje. Psy, w sumie cztery, przeszły skrajem drogi pod wejście do budynku w którym byłem i siadły. Szybko jednak poderwały się i odbiegły wzdłuż bloku podkulając ogony. Kie licho je spłoszyło? No nic, trzeba iść. Przewiesiłem kałacha przez ramię na wszelki wypadek i wyszedłem.

Przed budynkiem z chodnika wyrastała brzoza. Trochę powykręcana i dziwnie szara. Wszytko tu szare, aż się zaczynam bać, czy od wdychania tej szarości sam nie zszarzeję. A może to od radiacji coś mi się psuje w oczach? Cholera, lepiej o tym nie myśleć\3k No nic, idę przed siebie. Obłok mgły otwiera się przede mną i bezgłośnie zapieczętowuje tuż za mną. Kąsa chłodem i wilgocią. Wiatr zawodzi na liniach energetycznych, jęczą kołyszące się sosny. Nawet nieruchomy wrak przerdzewiałego Zaporożca bez kół zdaje się skrzypieć. A w tle, jakby z dala, spod ziemi, jakieś dziwne i niepokojące dudnienie. A może to krew w uszach? Serce się tłucze pod żebrami jak szalone, oddech mam płytki, jakbym się bał wziąć tej mgły do płuc. Zakładać maskę gazową? Nonsens\3k Radiacja ledwie co, żrących wyziewów ani śladu, to nie co te cholerne świecące bagna. Mgły się boję? Dureń ze mnie, może czas skończyć z tą zabawą\3k

Idąc tak krok za krokiem doszedłem do skrzyżowania gdzie skręciłem w lewo. Przechodząc wzdłuż szpaleru słupów lamp ulicznych poczułem się obserwowany. Ciężar czyjegoś spojrzenia leżący na barkach zdał się równie namacalny jak ten kałach. Nie, nawet bardziej, karabin wydawał się rozpływać we mgle, a wzrok wgryzał się w plecy. Ale się nie odwróciłem. Bo po co? Zobaczyć obdrapany budynek z betonowej płyty z powybijanymi szybami, wrak Zaporożca i pokręconą brzózkę? Nie, gdybym za każdym razem się tak rozglądał odkąd jestem w Zonie to niechybnie bym sobie już dawno łeb ukręcił. Człowiek jest odcięty od źródeł bodźców, które to doznaje zewsząd przebywając między ludźmi, więc zaczyna głupieć. Od jednej strąconej wiatrem gałązki można zawału dostać, kurna, szaleństwo. W dupie mam, nikt na mnie się nie gapi, przylazłem tutaj i lezę dalej. Pewnie to tamte pieski wróciły jak mnie zwietrzyły, jak tylko usłyszę, ze podchodzą to im wygarnę serią, a co\3k

Poszedłem dalej, po prawej za jakimiś chaszczami stał budynek szkoły. Gdyby nie wyasfaltowany plac apelowy nie byłoby jej widać zupełnie. Spomiędzy popękanych płytek chodnikowych wyrastała trawa do pasa, podjazd zarastały w poprzek jakieś kolczaste krzewy. Las normalnie, nie miasto\3k Placówka edukacyjna mieściła się w czterech budynkach, z jednej strony sala gimnastyczna, dalej parterowy pawilon stołówki, za nim czteropiętrowy dydaktyczny i przyrośnięty do niego niższy z aulą na parterze a biurami na piętrze. Ten sam projekt co we wszystkich podobnych miasteczkach. Byłeś w jednej z tych szkół to jakbyś był we wszystkich. Większość zaopatrzona w schrony w piwnicach, na wypadek wojny, która nie nadeszła. Mnie zaś interesowały zapasy tam zgromadzone, środki opatrunkowe, zestawy do uzdatniania wody, filtry do masek, jakieś łachy, może dokumenty\3k

Wdrapałem się po skruszonych stopniach do kuchni i zapaliłem latarkę. Podłogę wyścielała bodaj centymetrowa warstwa pyłu, blaty tonęły w kurzu, wszędzie walały się poprzewracane garnki i talerze. Z sufitu rozpinały się pajęczyny rzucające na ściany długie cienie tańczące przy każdym poruszeniu źródłem światła. Nieco dalej od wejścia, na środku pomieszczenia, czernił się ślad po ognisku. Połamane krzesła leżały pod ścianą wręcz zapraszając do użycia ich w roli opału. Ale jeszcze nie czas na przerwę, najpierw muszę znaleźć tu coś ciekawego. W sali jadalnej stoły zepchnięte były pod ścianę w imponującą rozmiarami hałdę. Ku mojemu zdziwieniu połowa okien przeszklonej ściany nie była wybita. Bezmyślnie schyliłem się po kawałek gruzu nieco mniejszy od pięści, wziąłem zamach i zatrzymałem się w połowie ruchu. Co mi ze stłuczonej szyby? Cała przynajmniej cieszy oko, wybitych w Zonie pełno. Do dziś myślałem, że wszystkie\3k Szkło było brudne od lepkiego kurzu i marnie przepuszczało światło, potęgowało tylko wrażenie 
wszechobecnej szarości. Jednak swoją obecnością odgradzało wnętrze sali od zewnętrza dając coś w rodzaju poczucia bezpieczeństwa. W każdym radzie uczucie czyjegoś wzroku ustało. Odetchnąłem z ulgą, cisnąłem kawałkiem betonu w podłogę obserwując jak odbija się i toczy gdzieś dalej, w cień.

Korytarze między salami lekcyjnymi były co najmniej przygnębiające. Tak samo tonące w kurzu, z łuszczącą się farbą olejną ze ścian. Ale na tych ścianach wisiały wyblakłe obrazki, z uśmiechniętym słońcem świecącym nad trójką postaci grających w piłkę, z rodziną siedzącą przy stole, z jakimiś zwierzętami w lesie. Prawie jak mutki, które można spotkać nieopodal, zaśmiałem się cicho. Dalej, w ramce, za stłuczoną szybką, przebarwione zdjęcie trzech rzędów dzieci siedzących obok wychowawczyni z podpisem: klasa IIIc, rok 1985. Z rozmyślań wyrwał mnie trupi odór, zaskakując na tyle skutecznie, że mało nie spadłem poślizgnąwszy się na schodach. Stalkerska ciekawość została odsunięta przez stalkerską ostrożność, przynajmniej tak starałem sobie to wytłumaczyć. Trup w Zonie to nie nowina, co innego trup w szkole w mieście poza uczęszczanymi szlakami. Rozumiem na środku drogi, w rowie, przy wejściu, ale na piętrze? Rozejrzę się po budynku to do niego wrócę, przynajmniej mi nie ucieknie\3k

W części administracyjnej bez niespodzianek. Ślady ogniska z papierów i zacieki na ścianach od wody. Stojący w gabinecie dyrektora sejf otworzył ktoś przede mną, może i lepiej, bo nie mam środków gdybym miał z nim walczyć. Do plecaka wrzuciłem kilka pożółkłych zeszytów i jakieś ołówki, niby zabawki, ale mogą się przydać. Z ocalałych dokumentów moją uwagę przykuło parę z nich: rozkaz opuszczenia budynku z początku maja '86 i fragment korespondencji dotyczącej jakiś zamówień z końca '85. Widocznie coś za plecami szerszej uwagi chcieli robić, bo tylko mowa o 'obiekcie', 'urządzeniach' czy też 'instalacji'. Zabrałem cały plik w nadziei, że znajdzie się ktoś, kto będzie wiedział o co chodzi. Po wyłamaniu paru szuflad wszedłem w posiadanie obiecująco wyglądającego pęku kluczy. Wyglądały na tyle solidnie, żeby sugerować, że nie sale lekcyjne im pisane czy składzik sprzętu sportowego. Humor mi się poprawił, już miałem wracać, kiedy z niższego piętra wyraźnie dobiegło mnie skrzypienie drzwi. Zamarłem. Wiatr ponuro 
szumiał za oknami, pogwizdywał w pustych korytarzach, ale nie był na tyle mocny, żeby ruszyć zastanymi od dziesięcioleci drzwiami. Czy mogłem usłyszeć jeszcze czyjeś kroki? Nie, krew szumiała w uszach, serce waliło w piersi, chyba nawet w oczach pociemniało\3k Czego ja się, kurna, boję? Jest dzień, mam w garści kałacha gotowego do strzału, powinienem czuć się panem sytuacji, biegać raźno jak po Uniwermagu, a nie trząść ze strachu galotami na jakieś odgłosy, które mogły równie dobrze mi się przyśnić. Dupa a nie Stalker ze mnie. Dzieci bawić a nie artefakta zbierać, o.

Zbiegłem po schodach na tyle szybko na ile pozwalały obkruszone stopnie i sypki gruz uciekający spod butów. Lewą ręką trzymałem się poręczy, prawą karabinu. Ja tu zaraz, kurna, pokażę\3k Piętro niżej pusto, na parterze pusto, w auli pusto. Kie licho? Z satysfakcją wybiłem butem drzwi wyjścia ewakuacyjnego z auli. Przeżarte przez mole drewno trzymane grubą warstwą olejnicy rozsypało się w drzazgi. Po mgle nie było już śladu, nawet słońce jakby skuteczniej przedzierało się przez chmury. Zmrużyłem oczy i rozejrzałem się po boisku. Ani żywej duszy. Tylko zarośla i sterta połamanych ławek i krzeseł. Noż do diaska\3k

Poszedłem przed główne wejście do szkoły z zamiarem odwiedzenia nieboszczyka, a później sprawdzenia piwnic. Bez ostrzeżenia znowu poczułem ten ciężar czyjegoś wzroku na sobie, zadrżałem, ale szybko odepchnąłem strach nie zatrzymując się.

- Zaczekaj!

Czyjś głos dobiegł mnie zza pleców, obróciłem się na pięcie i zrozumiałem jakie uczucie mają na myśli Stalkerzy mówiąc, że mało się nie posrali ze strachu. Kilkanaście metrów ode mnie stała dziewczyna. W czarnych spodniach i skórzanej kurtce, miała ciemne proste włosy do ramion, wiatr nimi poruszał leniwie.

- Zaczekaj! - powtórzyła wyciągając rękę w moim kierunku. Serce podskoczyło do gardła, nie mogłem złapać oddechu, nogi mało się nie ugięły w kolanach.

- Stój! Nie ruszaj się! - krzyknąłem mając nikłą nadzieję, że cokolwiek tymi słowami zdziałam. Dziewczyna zrobiła krok w moją stronę. Robiła wrażenie nienaturalnie bladej i chudej, podświadomie zacząłem porównywać ją z czymś z katalogu spotkanych mutantów. Wystrzeliłem parę razy w powietrze.

- Stój mówię! Nie zbliżaj się! - głos mi drżał, nogi mi drżały, ręce bez kałacha pewnie latałyby jak w epilepsji.

- Nie! Poczekaj! Nie strzelaj! - szła co raz szybciej w moim kierunku. Utkwiła kamienny wzrok we mnie tak, że czułem jak mnie nim przewierca. Spanikowałem\3k

- Odejdź! Nie zbliżaj się! - wystrzeliłem w ziemię, gdzieś między nami. Kula świsnęła w powietrzu, dziewczyna zaczęła biec w moją stronę, przycisnąłem spust, broń szarpnęła. Pierwszy strzał poleciał gdzieś w krzaki, później broń się zacięła. Zamek się na łusce zaciął? Czort wie, zrobiłem dwa kroki do tyłu, a potem świat zawirował. Zobaczyłem niebo obramowane gałęziami pokrzywionych brzózek a następnie ciemność. Ból\3k

\ro{Śpiąc w popiołach - II}

Obudził mnie ból, promieniujący gdzieś z tyłu głowy i zalewający całe ciało. Sięgnąłem nie bez trudu ręką do potylicy, włosy lepiły się od krwi, pod nimi guz wielkości piłki golfowej. Zdrowo musiałem przypierniczyć\3k Otworzyłem oczy i rozejrzałem się z trudem łapiąc ostrość. Leżałem na podłodze sali gimnastycznej z głowa opartą na stopniu trybuny. Plecak leżał gdzieś koło lewego kolana, broń trochę dalej, poza zasięgiem rąk. Spróbowałem wstać, ale bez powodzenia. Po chwili szamotania dotarło do mnie, że nie mam czucia w nogach. Leżałem tak przez chwilę myśląc nad tym w jak beznadziejnym położeniu się znajduję. Nawet jeżeli dopełzłbym jakoś do kałacha to nabojów mi na długo nie staczy, później skończę jako jadło tutejszej fauny, albo po prostu padnę z głodu.

- Ku*wa\3k To nie tak miało się skończyć\3k - Ogarnęła mnie bezsilność. Najgorszemu wrogowi bym tego nie życzył, a tu masz ci los. Zona nie przepuściła\3k Chwilę gapiłem się w sufit jakbym mógł tam znaleźć odpowiedź na dręczące mnie pytania jak wyjść z tej beznadziejnej sytuacji. Cholernie beznadziejnej\3k Najbliższe obozy są parę kilometrów poza zasięgiem nadajnika w moim PDA, nawet nie mam jak wzywać pomocy, nie mam kogo wzywać. Trzęsąc się z przerażenia straciłem przytomność nawet nie wiedząc kiedy.
\\
\centerline{$\sim(@)\sim$}
\\
Ocknąłem się wyspany i jakby mniej obolały. W stłuczonej głowie nadal się kręciło, ale póki leżałem nie sprawiało to dużego problemu. Otworzyłem oczy i mało nie wyskoczyłem z własnego ciała ze strachu. Nade mną pochylała się ta dziewczyna patrząc się we mnie. Nie miałem siły nawet krzyczeć.

- Ty\3k Czego chcesz? – Wysapałem przez ściśnięte gardło. Zamknąłbym oczy, gdyby dziewczyna zniknęła gdy znowu je otworzę, ale nie miałem złudzeń, że to zadziała. Była realna, dość realna bym poczuł chłód jej dłoni na swoim czole.

- Nie bój się, nic ci nie zrobię\3k Myślałam, że się nie obudzisz. Długo spałeś. – Powiedziała siadając koło mnie na stopniu.

Kiedy już otrząsnąłem się z przerażenia zacząłem składać w myślach zdarzenia, które mnie tu przywiodły. Wycieczka do Chabna, szkoła, dziewczyna, strzały i gleba. Ale spotkałem ją przed szkołą, dlaczego więc budzę się oparty o trybuny?

- Gdzie\3k Gdzie ja jestem? – Zapytałem rozcierając obolałą głowę.

- To sala gimnastyczna szkoły. Przyciągnęłam cię tutaj z dworu, tam psy się kręciły jakieś i nie wiedziałam kiedy wróci ci przytomność, a gdyby zaczęło padać\3k - Powiedziała przepraszającym tonem. Patrzyła na mnie jakimś dziwnym wzrokiem, którego nie widziałem od miesięcy w Zonie. Co to mogło być, troska? Przynajmniej zrozumiałem od czego mnie ręce tak bolą, musiała mnie za nie zawlec aż tutaj. Gdy kątem oka dojrzałem dwie koleiny wyciągnięte w pyle kończące się u moich stóp zrobiło mi się jej żal. Ja prawie jej nie zastrzeliłem, a ona jeszcze mi pomaga, za co, dlaczego? Nie wiem\3k

Pierwszy raz miałem okazję się jej przyjrzeć od czasu kiedy mało jej nie zastrzeliłem na wszelki wypadek. Była młodsza ode mnie, mogła mieć dwadzieścia lat, może mniej. Ciemnymi oczami wpatrywała się w podłogę gdzieś koło moich nóg, ciemne proste włosy spływały wzdłuż ramion. Była śliczna i to nie dlatego, że była pierwszą dziewczyną jaką widziałem od dobrych kilku miesięcy. Jak z obrazka. A ja durny nie dość, że wystraszyłem się jej na śmierć to mało jej ołowiem nie poczęstowałem. Ale głupio\3k

Odepchnąłem się rękoma od podłogi doprowadzając się do pozycji bardziej siedzącej. Chwyciłem za nogawki i przyciągnąłem nogi gdy nagle poczułem w nich ból. Spróbowałem nimi poruszyć i ból się nasilił, ale nogi drgnęły. Mrowienie było trudne do zniesienia, ale świadczyło, że paraliż mnie nie dotknął. Odetchnąłem z ulgą.

- Jestem Iwo, a ty? Tak w ogóle co to ty tu robisz? – Przerwałem niezręczną ciszę.

- Ja\3k Ja nie wiem, nie pamiętam\3k - Odpowiedziała opuszczając głowę i zalewając się rumieńcem.

- Ale jak to nie pamiętasz?

- No nie pamiętam\3k Obudziłam się w piwnicy tej szkoły trzy dni temu i nie pamiętam nic co działo się ze mną wcześniej. Nie wiem kim jestem, co to za miejsce, co ja tu robię. Nie wiem i boję się\3k

- Możemy tam pójść i rozejrzeć się, może znajdziemy coś co ci pomoże. Zaprowadzisz mnie tam?

Kiwnęła potakująco głową w milczeniu. Za oknami robiło się powoli coraz ciemniej, trzeba by miejsce na obóz znaleźć albo przynajmniej nocleg jakiś.

- Chodź, trzeba nazbierać chrustu na opał, jutro rano ruszymy na poszukiwania.

Z trudem stanąłem na własnych nogach i jeszcze zataczając się zebrałem plecak i broń. Bez słowa wyszliśmy ze szkoły i zaprowadziłem ją do bloku, gdzie spędziłem poprzednią noc. Tam nazbieraliśmy gałęzi z rosnących pod budynkiem krzaków. Dziewczyna szybko zasnęła owinięta w moją kurtkę, ja siedziałem oparty o ścianę wpatrując się w ogień i zastanawiając co z tego wszystkiego wyjdzie. Szukałem artefaktów a znalazłem dziewczynę. Artefakty można sprzedać u Sidorowicza i mieć spokój, a dziewczynę\3k Nie, ja nie z takich, tak nisko jeszcze nie upadłem. Trzeba więc będzie się nią zaopiekować\3k No nic, czas pokaże.

~(@)~

Znowu obudził mnie chłód. Długą chwilę stałem w oknie próbując rozgrzać skostniałe z zimna dłonie. Miasteczko pogrążone było jeszcze w porannych szarościach i otulone gęstą jak wata mgłą. Na ulicach oczywiście ani śladu żywej duszy, ani psa, nic. Nawet wiatr jakoś tak leniwie porusza gałęziami zdziczałych krzaków. Dziwne miejsce to Chabno, niby zwyczajna opuszczona mieścina, za spokojna na Zonę, a z drugiej strony\3k Diabli wiedzą.

Skończywszy rozmyślania zabrałem się za rozniecenie ogniska z pozostałego chrustu. Poszło szybko i sprawnie i nawet dziewczyna się nie obudziła. Ogień przyjemnie grzał i oświetlał odpędzając lęki minionej nocy, mój cień tańczył niezmordowanie nad naszymi głowami. Zajrzałem do plecaka, wygrzebałem dwie konserwy turystyczne, paczkę nabojów do kałacha, butelkę wody i scyzoryk. Doładowałem magazynek nabojami i zabrałem się za konserwy kiedy dziewczyna się obudziła. Ucieszyłem się w duchu, że nie zobaczyła mnie z bronią w ręku, głupio by to musiało wyglądać\3k

- Dzień dobry, jak ci minęła noc? Przygotowuję śniadanie. Może niezbyt okazałe, ale w Zonie trudno o lepsze.

- Dziękuję, dobrze\3k Gdzie, w Zonie? – Zapytała przecierając dłonią zaspane oczy.

- W Zonie, czyli inaczej Czarnobylskiej Strefie Wykluczenia. To miasteczko to Chabno a jesteśmy jakieś sto kilometrów na północ od Kijowa. Mówi ci to coś?

- Nie\3k To znaczy słyszałam o Zonie, ale nie mam pojęcia co mogę tutaj robić. A co ty tu robisz?

- No widzisz, kilka lat temu coś walnęło w elektrowni albo jej pobliżu i od tego czasu w Strefie zaczęły się dziać dziwne rzeczy. Między tymi dziwnymi rzeczami pojawiały się fragmenty odmienionej materii nazywane artefaktami. Naukowcy z całego świata gotowi byli płacić każde pieniądze za artefakty z Zony, tylko ktoś musiał im je przynieść. Wdzierających się do Strefy nazywa się Stalkerami. Jednego z nich masz przed sobą. Masz konserwę, pewnie głodna jesteś\3k

Tym sposobem uniknąłem nieuchronnie wiszącej w powietrzu lawiny pytań na mój temat. No, nie uniknąłem a raczej odroczyłem, ale dobre i to. Zresztą należy się jej trochę wiedzy o mnie, ale to nie dziś\3k
%
\ro{Śpiąc w popiołach - III}  
%
Zjedliśmy śniadanie i wyszliśmy z budynku zostawiając za sobą zgaszone ognisko, ostatnie źródło ciepła i światła na tym pustkowiu. Słońce było co prawda już nad horyzontem, ale niemożliwością było odgadnięcie jego położenia. Mgły podniosły się na tyle, że tworzyły teraz kożuch tuż nad naszymi głowami. Oblepiały swoją wilgocią drzewa, płoty, mury i rozpraszały światło słoneczne odbierając mu ostatnią cząstkę ciepła. Słyszałem tylko oddech swój i dziewczyny, niosące się echem nasze kroki i trzeszczący plecak pełen dobytku. Trochę mnie martwił brak psów, które widziałem wczoraj, niemniej na wszelki wypadek broń trzymałem w ręku.

Doszliśmy do szkoły nie zamieniając słowa po drodze. Dziewczyna była chyba głęboko zamyślona, bo z lekko opuszczona głową wpatrywała się nieobecnym wzrokiem, gdzieś w nieskończoność poza kurtyną gęstego oparu. Powrót do miejsca przebudzenia pewnie nie jest dla niej miłym wspomnieniem, ale nie widzę innego sposobu, żeby dowiedzieć się czegoś o niej niż udając się tam.
\sd
\xx Dalej ty mnie musisz poprowadzić, nie zdążyłem wczoraj zejść do piwnic, bo\3k Zresztą\3k Mam nadzieję, że to nie problem dla ciebie tam wrócić? – Zagadnąłem specjalnie trochę głośniej, jakbym obawiał się, że w przeciwnym wypadku mnie nie usłyszy.
\xx Tak, znaczy\3k Nie, nie ma problemu, chodźmy. Wejście jest obok tamtego budynku. – Wskazała ręką na część administracyjną.
\qd
Obeszliśmy szkołę szybkim krokiem. Dziewczyna prowadziła nie oglądając się za siebie, ja przeciwnie, nieustannie rozglądałem się na boki wypatrując jakiegoś urojonego zagrożenia. Czułem dziwny, nieopisany niepokój, jakby coś zaczajało się na nasze głowy i czekało tylko dogodnego momentu, żeby wyskoczyć z ukrycia. Ale nic nie wskazywało, żeby był tu ktokolwiek poza nami. Zatrzymaliśmy się przed parą blaszanych drzwi umiejscowionych w asfalcie przy ścianie budynku, jedne ich skrzydło otwarte odsłaniało prowadzące w mrok strome betonowe schody.

- Tędy\3k – powiedziała zatrzymując się i widocznie czekając aż pójdę przodem. Przełknąłem ślinę próbując ukryć zdenerwowanie. Założyłem i włączyłem latarkę czołówkę i otworzyłem drugie skrzydło drzwi. Odpowiedziało okropnym piskiem i skrzypieniem zawiasów po czym z łomotem opadły na asfalt rozsypując wokół rdzawy pył. Spojrzałem w ciemną otchłań i zwątpiłem w to, czy na pewno chcę tam schodzić\3k

Schody były śliskie od wilgoci, na betonowych ścianach spływały kolorowe wzory zacieków, rude od rdzy, zielone od glonów, białe od soli. Strop zdobiły ciągnące się, włochate girlandy pajęczyn. Pod nogami cmokała brunatna warstwa cuchnącej ziemi z gnijącą materią organiczną ze sterczącymi gdzieniegdzie ostrymi kawałkami bladego gruzu, przypominającego połamane kości. Światło latarki niechętnie oświetlało ten ponury widok, jakby chciało oszczędzić mi oglądania go. 

Schody prowadziły do krótkiego niskiego korytarza, a ten kończył się ciężkimi drzwiami jak od schronu przeciw lotniczego. Zatrzymałem się, żeby się im lepiej przyjrzeć. Były uchylone. Zgarnięte błoto na podłodze wskazywało, że były niedawno otwierane, w przejściu zobaczyłem ślad buta dziewczyny i\3k Jakiś inny? Złapałem się za głowę i westchnąłem nad swoją głupotą. Teraz, jak już wszystkie ślady zadeptaliśmy, to mogę zgadywać kto to mógł być. Trzeba było od razu zwrócić na to uwagę. No nic, może nikogo nie spotkamy. Oby\3k

Weszliśmy do przestronnego pomieszczenia, którego podłoga i ściany wyłożone były białymi kafelkami przywodzącymi na myśl łazienkę, szpital albo rzeźnię. Na wprost prowadził korytarz ginący gdzieś w ciemności, przed nim od ściany odchodził kontuar przypominający portiernię czy recepcję. Na prawej i lewej ścianie zauważyłem parę zamkniętych drzwi.

- Zaczekaj. – Poleciłem. Dałem dziewczynie zapasową czołówkę i chwilę się zawahałem. Gdy sięgnąłem po pistolet wyraźnie się przestraszyła, bo cofnęła się o pół kroku i wbiła wzrok w broń.

- Umiesz się tym posługiwać? – Zapytałem retorycznie i od razu przeszedłem do krótkiego szkolenia. – Tu masz spust, tu masz bezpiecznik, tym przyciskiem zwalniasz magazynek jak chcesz przeładować broń. To jest zamek, po ostatnim strzale zostaje w tylnym położeniu, po włożeniu pełnego magazynka automatycznie wraca. Celujesz zgrywając to, z tym\3k - Kontynuowałem pokazując na muszkę i szczerbinkę. Trochę się bałem, dając jej broń, ale gdyby coś mi się stało, to może uratować jej życie. Zresztą do czego mielibyśmy tutaj strzelać\3k

- Broń nosisz zawsze zabezpieczoną, chyba, że powiem inaczej, strzelasz z wyprostowanej ręki, najlepiej przytrzymaj sobie drugą, o tak. Niby takie gówno, ale kopie, jak źle złapiesz to możesz sobie łokieć albo bark uszkodzić. – Zademonstrowałem pozycję strzelecką przymierzając się gdzieś w kierunku końca korytarza. – Nie strzelaj, jeżeli jestem na linii strzału, jak się skończy magazynek to najpierw szukasz jakiejś osłony i chowasz się za nią, później przeładowujesz. Jasno się wyraziłem? Trzymaj\3k

Kiwnęła potakująco głową i przyjęła Makarowa. Dałem jej jeszcze jeden pełen magazynek, ot tak, na wszelki wypadek. Trzymała broń z przejęciem oglądając ja uważnie, ważąc w dłoni.

Podszedłem do pierwszych drzwi po prawej. Namalowany na nich symbol czerwonej błyskawicy sugerował, że znajdę za nimi jakąś rozdzielnię elektryczną. I faktycznie, kiedy już z pomocą solidnego kopniaka pomogłem sobie otworzyć zaklinowane drzwi, miałem przed sobą dużą wajchę, zapewne wyłącznik główny oraz kilkanaście przełączników i kontrolek. Kolejne grupy były opisane: ośw. 1, ośw. 2, went. 1, went. 2, mag, apar, chłod, lab. Bez głębszego namysłu chwyciłem dźwignę włącznika i podniosłem ją do pozycji załącz. Mechanizm zazgrzytał metalicznie, syknęła iskra i kontrolki rozświetliły się na czerwono. Nie spodziewałem się tego. Stałem tak chwilę oniemiały wpatrując się w delikatnie mrugające kontrolki. Stalkerze, wiesz co robisz? Coś mnie znowu podkusiło i sięgnąłem do pierwszego przełącznika od oświetlenia. Chwilę go trzymałem nie będąc pewnym czy na pewno chcę przekręcić, ale przekręciłem. Hol napełnił charakterystyczny klekot zapalających się jarzeniówek, a ich zimny blask odbijający się w kafelkach oślepił 
mnie na chwilę.

- I stała się światłość! – Zaśmiałem się głośno. Jakkolwiek obecność elektryczności była dość dziwnym znakiem, nasuwającym całą masę pytań, to jednak zdecydowanie wolę zwiedzać te lochy w świetle lamp, a nie tylko skromnej latareczki. Machnąłem ręką na cicho buczącą plątaninę kabli i wróciłem do dziewczyny przymykając za sobą drzwi.

- Którędy dalej? – Zapytałem.

- Drugie drzwi po lewej.

Poszliśmy tam. Pokój przypominał salę szpitalną, izolatkę. Tak jak hol i korytarz, całe pomieszczenie było wyłożone białymi kafelkami. Przy jednej ścianie stało samotnie łóżko szpitalne z wygniecionym materacem i zwiniętą w nogach pościelą, na przeciwległej ścianie, na wprost łóżka zobaczyłem okno z zasłoniętą żaluzją, do dyżurki? W kącie, obok pustego regału, leżały przewrócone dwa krzesła i jakiś statyw albo pulpit, w drugim rogu umywalka. Poza tym pusto. Powietrze tu miało jakiś dziwny zapach, trochę duchoty i wilgoci, trochę jakiś środków czyszczących. Niby nic groźnego, ale zatęskniłem za powierzchnią\3k

Przeszliśmy do dyżurki, nieco węższej od sali. Na drugiej ścianie było okno do kolejnej sali. Było\3k Przez stłuczoną szybę zobaczyłem zakrzepłe ślady krwi na ścianach, rozmazane kałuże na podłodze, porozrzucane meble, krajobraz jak po bitwie.

- O ku*wa\3k - Wyrwało mi się. Początkowo rutynowa wyprawa nabierała rumieńców, a jeszcze nie znaleźliśmy tego, czego szukaliśmy. Przełknąłem głośno ślinę i wróciłem do przeszukiwania pokoju. Dziewczyna stała tuż koło mnie, patrząc co robię.

W dyżurce było biurko, dwa krzesła i dwie szafki, z czego jedna przeszklona z lekami. Ta, jako pierwsza zwróciła moją uwagę. Nazwy leków stojących na górnych półkach brzmiały dla mnie obco, połowy pewnie nie umiałbym nawet wymówić\3k Niżej stały środki odkażające, opatrunki, przeciwbólowe. Cenny łup, bandaże mi mogą się przydać, resztę opchnę jajogłowym za jakąś okrągłą sumkę, oni już będą wiedzieli co z tym zrobić. Drzwiczki były zamknięte, więc bezceremonialnie stłukłem je kolbą od kałacha. Kiedy pakowałem leki do plecaka, dziewczyna mnie zawołała.

- Zobacz tutaj, teczka z moim zdjęciem!

Faktycznie, w otwartej przez nią szufladzie na wierzchu leżała papierowa teczka na dokumenty. Napis w rogu głosił: AKTA 0/X-11/03-12/031, a obok przyklejone było jej czarno-białe zdjęcie. W sumie to nie domyśliłbym się, że to ona jest na tej fotografii, wyglądała na nim jak półtora nieszczęścia, z przymkniętymi oczami i twarzą bez wyrazu, jak na prochach jakich. Brak kolorów i mdłe oświetlenie tylko potęgowały ten efekt. Sięgnąłem po teczkę i zajrzałem do środka. Tam kolejne jej zdjęcie, trochę wyblakłe legitymacyjne ze szkoły średniej chyba\3k W nagłówku strony znowu numer akt, niżej: Obiekt 031, a niżej już dane biometryczne jak wiek, wzrost, masa ciała, kolor oczu, grupa krwi\3k Na następnej stronie wyniki badania krwi chyba, na kolejnej też coś podobnego, bo tabelki z różnymi liczbami. Dalej złożony w harmonijkę elektrokardiogram z długim zygzakiem wykreślonym niebieskim flamastrem na papierze. Na kolejnej harmonijce też były wyrysowane zygzaki, ale miały inny kształt, piki były szersze i nieregularne, było 
też więcej wykresów. To samo na następnej, następnej, kolejnej i chyba wszystkich pozostałych\3k Przyjrzałem się opisom: płat czołowy, płat skroniowy, płat ciemieniowy, \3k No jasne, to EEG!

- Świetnie, pokażemy jajogłowym, to pewnie coś wymyślą. Mają tam profesorka, co jest dobry w te klocki, jak go ładnie poprosimy, powinien ci pomóc. Co tam masz jeszcze? – Zadowolony pakowałem teczkę do plecaka.

- Teczki pięciu kolejnych osób\3k

- Widziałaś kogoś z tych ludzi wcześniej? – Zapytałem zaciekawiony. Pokręciła przecząco głową. Pozostałe teczki rzuciłem do stojącego w kącie kubła i wróciłem do przeszukiwania biurka. Pełno było w nim różnorakich papierów. Już miałem się skierować do drugiej szafy, kiedy dziewczyna szarpnęła mnie za rękaw w kierunku drzwi.

- Chodźmy stąd! Tam ktoś jest! – Powiedziała z szczerym przerażeniem w oczach, nie przestając ciągnąc mnie za ramię.

- Nie bój się, nikogo tam nie ma, usłyszałbym. Poza tym mamy broń, hmm? – Próbowałem ją przekonać. Poza buczeniem świetlówek i szybkim oddechem dziewczyny zasadniczo nic nie było słychać.

- Jest, czuję go, idzie tutaj\3k

- Ech\3k Ale jeszcze\3k A, zresztą, niech ci będzie\3k - Dałem się jej przekonać, ale nie byłem pewien czy postępuję właściwie. Może ona chce mnie odciągnąć od znalezienia czegoś, albo prowadzi prosto w pułapkę? Bezsens, mogła mnie wcześniej rozwalić, ma przecież mój pistolet\3k

- Chodź, szybciej! – Wołała mnie już z korytarza.

Kiedy wyszedłem z dyżurki zobaczyłem w głębi korytarza czyjąś sylwetkę. Zatrzymałem się, żeby się przyjrzeć. To nie mogło być przewidzenie, faktycznie ktoś szedł w naszym kierunku. Nie słyszałem jego kroków, bo był boso. Co to za dziwoląg, boso w Zonie? Na wszelki wypadek odbezpieczyłem kałacha i odciągnąłem zamek. Metaliczny szczęk broni poniósł się echem po korytarzu. Już przymierzałem się do celowania w nieznajomego, kiedy sprawy przybrały bardzo zły obrót.

Nagle usłyszałem głuchy huk i okropnie zakręciło mi się w głowie. Świadomość wróciła mi sekundy później, kiedy kałach leżał na podłodze, a ja klęcząc próbowałem się z trudem podnieść. Nastąpiło kolejne uderzenie, jakby ktoś ściskał mi czaszkę w wielkim imadle. Zrobiło mi się ciemno przed oczami. Myślenie w takim stanie zadawało fizyczny ból, ale zdałem sobie sprawę z beznadziejności sytuacji. Wyskoczył na mnie kontroler, a ja nawet nie mogę broni podnieść, a co dopiero z niej strzelać\3k Przypomniałem sobie o granacie zaczepnym przyczepionym do pasa, mojej ostatniej desce ratunku w tych okolicznościach. Kiedy sięgałem po niego nastąpił kolejny cios, jakby ktoś mózg wyżymał mi jak mokrą gąbkę. Leżałem już całym ciałem na zimnych kafelkach. W uszach dzwoniło niemiłosiernie. Co tak szumi? Krew w żyłach, czy to dziewczyna krzyczy? Chwyciłem zębami zawleczkę i wyszarpnąłem. Ścisnąłem cytrynkę w dłoni, spodziewając się następnego uderzenia. Zamiast tego, jakby z otchłani, niósł się huk wystrzałów. Otworzyłem oczy i 
podniosłem się na wolnej ręce. Kontroler trafiony w ramię zatrzymał się dobre kilkanaście metrów ode mnie trzymając się za ranę.

- Schowaj się! – Krzyknąłem ostatkiem sił do dziewczyny i rzuciłem granat. W połowie zamachu przeciwnik jednak ugodził jeszcze raz, granat wypuściłem za wcześnie\3k Zasłoniłem głowę rękami i otworzyłem usta oczekując wybuchu.

Huk był okrutny. Jeszcze czaszka nie przestała mnie boleć od ataków kontrolera a posypał się na mnie grad pokruszonych kafelków i gruzu. Zawyłem z bólu, kiedy odłamki rozcinały mi skórę ramion. Pył jeszcze nie opadł, w uszach jeszcze mi dzwoniło, kiedy dziewczyna podbiegła do mnie próbując pomóc mi wstać.

- Spokojnie, dam radę\3k - Skłamałem. Stojąc już na własnych nogach zatoczyłem się, oparłem o ścianę i zwróciłem zawartość żołądka. Rany na rękach piekły jak ogień, ale jakoś nie zwracałem na to uwagi. Otarłem usta o ciemniejący od krwi rękaw.

- Dobra, wynosimy się stąd\3k - Wystękałem przyciągając dziewczynę do siebie. – Uratowałaś mi życie, wiesz? – Trzęsła się nie gorzej ode mnie.

Zabrałem kałacha i wróciliśmy do obozu. W drodze powrotnej nie było już mgieł, świat jakby nabrał trochę kolorów, nawet parę ulic dalej szczekały psy. Otarłem się o śmierć, ale żyłem. Żyłem i czułem, że żyję.

\end{document}