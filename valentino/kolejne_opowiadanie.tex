\documentclass[../MAIN.tex]{subfiles}
\begin{document}
\ro{1}
%
\sx Daj no jednego. -- rzucił Mikołaj
\qd
Leon sięgnął do kieszeni kombinezonu, wyjął paczkę LM’ów, wyciągnął jednego papierosa i podał go Mikołajowi. Sam wsadził sobie jednego w usta, wyjął zapalniczkę, zapalił ją, po czym zaciągnął się i wypuścił dym z ust. \\
Siedzieli we dwójkę na przystanku autobusowym, niedaleko złomowiska. Zachował się całkiem nieźle -- poza tym, że cały zardzewiał i miał sporą dziurę w dachu. Mikołaj siedział na starym fotelu, opierając nogi o głaz, zaś Leon leżał na ławce z założonymi na głowie rękoma.\\
Obaj, paląc, myśleli jak wrócić do obozu. Na zewnątrz padał ulewny deszcz, którego krople głośno bębniły o blaszany dach przystanku.
\sx Idę sprawdzić, czy nikt się nie zbliża -- rzekł Leon, podnosząc się z ławki.
\qd
Wyjął lornetkę, narzucił na głowę kaptur zielonego kombinezonu i wychylił się spod dachu. Zmrużył oczy, po czym przyłożył lornetkę do oczu i zaczął się rozglądać.
%
\sx Może zadzwonimy po wojsko? -- zapytał Mikołaj.
\xx Nie ma mowy -- odparł Leon, dalej obserwując. -- Wybij sobie wojsko z głowy -- dodał.
\qd
Mikołaj westchnął, zaciągnął się jeszcze raz papierosem i wyrzucił jego niedopałek do pobliskiej kałuży. Usiadł na ławce i zamknął oczy. Był strasznie zmęczony, marzył o długim, twardym śnie w wygodnym łóżku. Po chwili już chrapał.
\sx Wstawaj do cholery! -- krzyknął Leon, potrząsając Mikołajem -- Obudź się!
\qd
W końcu obudził przyjaciela -- przetarł on oczy i wybałuszył oczy.
%
\sx Czego?! -- spytał głośno.
\xx John F. tu idzie! Spadamy stąd!
\qd
John F. -- Finn. ,,Rambo'' -- nazywał go czasem Leon. Brzmiało to śmiesznie, ale żaden inny pseudonim nie pasował do osoby, która zbliżała się w kierunku przystanku. Jeden z najsławniejszych stalkerów w Zonie, znany z rozgramiania sporych oddziałów, według niektórych, armii, w pojedynkę. Kilka opowieści mówiło, że posiada on pewien artefakt, który czyni go nieśmiertelnym.\\
Prawie nikt w to nie wierzył, ale jak wytłumaczyć niejedną wojskową eskapadę zniszczoną przez samego Johna? \\
Mikołaj nieomal zemdlał ze strachu. John posiadał opinię najlepszego płatnego mordercy w Zonie.\\
A wielu ludzi miało by w interesie śmierć Leona, jak i jego przyjaciela. Mikołaj stanął na równe nogi, niczym wystrzelony, wyciągnął z plecaka AK-74SU i załadował go.
%
\sx Zwiewajmy. Nie mamy z nim szans -- rzekł.
\xx Święte słowa -- Leon podniósł z ziemi swego M14 i wychylił się zza przystanku.
\xx I co? -- spytał Mikołaj.
\qd
Leon uciszył go machnięciem ręki i z karabinem gotowym do strzału, rozglądał się dalej. Nagle w blaszanej ściance rozkwitła mała dziurka -- zapewne od pocisku. Kawałek blachy wbił się Leonowi w rękę. Zawył i równocześnie z Mikołajem padł na ziemię.
%
\sx Kurwa! -- zaklął, wpatrując się przerażonym spojrzeniem w wystającą z nadgarstka blaszkę.
\qd
Chwilę później dobiegło ich echo wystrzału.
%
\sx Uciekamy! -- wrzasnął Leon i podniósł się z ziemi.
\qd
Deszcz przybrał na sile, po chwili rozległ się głośny grzmot, a okolicę rozświetliła błyskawica. -- Wynośmy się stąd w cholerę! \\
Obydwaj wybiegli spod daszku i od razu poczuli ciężkie krople, uderzające w kombinezony. Pierwszy biegł Leon, (miał lepszą kondycję) zaś jego kolega trzymał się tuż za nim. Przeskoczyli wyłamane drzwi samochodu, oparte o kilka brązowych rur, po czym przyśpieszyli. Z metru na metr biegło im się coraz gorzej -- głównie za sprawą coraz nieprzyjemniejszego deszczu. \\
Mikołajowi zakręciło się w głowie, zawahał się po czym z łomotem upadł na ziemię -- Leon usłyszawszy to, odwrócił się przez plecy, kucnął przy stalkerze i próbował go ocucić.
%
\sx Dalej! -- krzyknął z rozpaczą, po czym spoliczkował nieprzytomnego Mikołaja. -- Wstawaj!
\qd
Nad Leonem z charakterystycznym świstem przeleciał kolejny pocisk. Uderzył Mikołaja w twarz raz jeszcze i zniżył głowę, co uratowało ją przed trafieniem.
%
\sx Obudź się! -- ryknął.\qd
%
\ro{2}
%
Wychodząc z samochodu nałożyłem kaptur, gdyż ciągle padało. Zatrzasnąłem drzwiczki i zamknąłem je na klucz. Deszcz był tak gęsty, że ograniczał widoczność, przez co poruszałem się niemal po omacku. Z kapturem na głowie i rękoma w kieszeni, ruszyłem w stronę budynku. \\
Znajdowałem się na wschodnie Prypeci i miałem zamiar dostać się do starej, opuszczonej szkoły podstawowej. \\
Przez strugi deszczu widziałem niewiele, jedynie zardzewiałe drabinki i karuzele -- pozostałości po okolicznym placu zabaw. Natknąłem się na wysoki mur -- przez chwilę poruszałem się wzdłuż niego. Był on pokryty spłowiałym graffiti, przedstawiającym gitarę w czyichś rękach, lecz twarz i tułów gitarzysty były nie do rozpoznania -- znikły, pozostawiając jedynie wypłowiały cień niegdyś pięknego dzieła. Idąc wzdłuż muru, co chwilę nerwowo oglądałem się na wszystkie strony -- dostrzegałem kontury okolicznych bloków mieszkalnych oraz kilku wieżowców. \\
Zatrzymałem się przy poddaszu szkoły -- cały budynek miał kształt prostokąta i był długi na około 20 metrów. Drzwi, niegdyś zielone, teraz miały szary kolor i obłaziło z resztek farby. Dzwonek do drzwi rozmontowano, wraz ze wszystkimi oknami szkoły -- zostały po nich jedynie dziury nieprzeniknionej ciemności. Po dwie z nich szpeciły lewą i prawą stronę budynku.\\
Wyjąłem pistolet -- przerobionego Colta M1911A1, załadowałem go i przełożyłem do prawej ręki. Lewą sformowałem w pięść i załomotałem w drzwi trzy razy.
%
\sx Otwierać! -- krzyknąłem -- Służby specjalne!
\qd
Usłyszałem kroki. Szybkie, acz krótkie, których echo rozbrzmiewało w wąskim korytarzu przede mną.
Przystawiłem ucho do drzwi i nasłuchiwałem -- kroki stały się głośniejsze. Na wszelki wypadek odsunąłem się na bok, jednak wciąż przytrzymując ucho przy drzwiach. Miałem szczęście, że dźwięk deszczu nie był tu tak uciążliwy, inaczej niczego bym nie usłyszał. Po chwili nieznajomy za drzwiami się zatrzymał.
%
\sx Kto tam?! -- krzyknął.
\qd
Nabrałem powietrza w płuca i odrzekłem:
%
\sx Służby specjalne! Ot\3k
\qd
Nie dokończyłem, bo drzwi rozłamały się w pół i ze stukotem opadły o betonowy ganek. Przez dźwięk latających drzazg dobiegło mnie echo strzału -- wydawało mi się, że była to strzelba. Upadając na ziemię, przekląłem szpetnie, a kiedy w uszach przestało mi dzwonić wstałem, wychyliłem Colta zza zniszczonych drzwi i strzeliłem pięć razy na oślep.\\
Nacisnąłem przycisk przy spuście i magazynek wyskoczył z broni, odbijając się kilkakrotnie od podłoża. \\
Z kieszeni wyjąłem kolejny i wsunąłem go do pistoletu, po czym nacisnąłem przycisk, który spowodował, że zamek broni wrócił na miejsce -- kiedy to nastąpiło, naciągnąłem kurek. \\
Wszedłem szybko po środka szkoły, trzymając broń wysoko, gotową do strzału. \\
Przede mną ciągnął się wąski, czterometrowy korytarz, w którym unosił się spalony proch, jak i jego duszący zapach. Każdy mój krok wzniecał tumany kurzu i ocierał porozbijane szła z szarą podłogą. Ściany i sufit korytarza były równie szare, lecz gdzieniegdzie można było dostrzec resztki beżowej farby. Podłoże ścieliły pozostałości po wyposażeniu szkoły -- probówki, książki, kawałki tablic i desek. \\
Metr przede mną, na lewo znajdowało się przejście do pomieszczenia -- pośrodku framugi leżała klamka. Przyległem do lewej ściany i wzdłuż niej, powoli, zbliżałem się do najbliższego pokoju. Kiedy od wejścia dzieliło mnie około pół metra, usłyszałem wystrzał i dźwięk kuli wbijającej się w gips. Obok mojej głowy rozprysła chmura tynku i białego pyłu -- przez całe to zamieszanie broń wypadła mi z ręki. \\
Jęcząc, zacząłem panicznie przecierać oczy, próbując pozbyć się drażniących białych drobinek.\\
Kiedy przestały mnie piekielnie piec i odzyskałem ostrość widzenia, schyliłem się po broń. W tym momencie padł kolejny strzał, który przewiercił się przez ścianę i utkwił w moim lewym łokciu. Wrzasnąłem głośno, zalewając mym wrzaskiem całą szkołę i osunąłem się na kolana.\\
Usłyszałem, że ze strony pokoju ktoś się zbliża. \\
Zza rogu wychylił się mężczyzna, około trzydziestki, łysy, o smukłej głowie i łagodnych rysach. Zadziwiły mnie jego oczy -- całkowicie białe, bez siatkówki i źrenicy -- samo białko, niczym jajko obrane ze skorupy.
%
\sx Witaj -- rzekł mężczyzna ochrypłym głosem. -- Witaj, Alex.
\qd
Skąd on znał moje imię? \\
Mężczyzna wyszedł zza framugi, ukazując swój strój. \\
Był ubrany w garnitur. \\
Beżowy garnitur i spodnie, bez krawatu. Garnitur był rozpięty, luźno narzucony i odsłaniał niebieską koszulę, na którą go nałożono. Nosił czarne, wypucowane mokasyny. Nieznany człowiek ruszył w moją stronę -- o dziwo, poruszając się, nie czynił żadnego hałasu -- nawet drobina kurzu nie drgnęła, kiedy podeszwy jego obuwia stąpały o podłogę.
%
\sx Myślałeś, że pójdziesz dalej i poznasz tajemnice tego miejsca? Że Nasz trud pójdzie na marne? Że rozbijesz naszą grupę? Nic o niej nie wiesz -- i tym lepiej dla Ciebie. Może, kiedyś\3k trzeba dokonać czegoś naprawdę wielkiego, aby się tutaj dostać. Na razie, jeśli będziesz dalej próbował, czeka Cię to\3k -- rzekł, po czym podniósł lewą rękę.
\qd
Mój Colt wzniósł się w powietrze. Jego lufa skierowała się w stronę sufitu, po czym broń wypaliła siedem razy, opróżniając cały magazynek. Huk strzału mnie zdezorientował i po chwili ujrzałem najbardziej niesamowitą rzecz, jaką było mi kiedykolwiek widzieć. \\
Wszystkie siedem pocisków unosiło się bez ruchu w powietrzu.
%
\sx Wyciągnij dłoń, przyjacielu -- poprosił nieznajomy.
\qd
Mimo woli, oszołomiony całą sytuacją, usłuchałem. Jeden z pocisków pofrunął w moją wprost na mnie i dotknął serdecznego palca. Pocisk był gorący i instynktownie cofnąłem rękę.
%
\sx Wynoś się i nigdy nie wracaj! -- wrzasnął mężczyzna.
\qd
Nagle jego oczy jakby wywróciły się do góry nogami, powoli odsłaniając czerwone okręgi na białkach gałek ocznych. Były dosłownie krwiste i widok ten przeraził mnie do szpiku kości.
Pozostałe sześć pocisków z niesamowitą prędkością wbiły się w ścianę, o którą byłem oparty i każdy z nich, z hukiem przebiły ją na wylot. Cały byłem obsypany tynkiem.
%
\sx Wynocha! -- powtórzył -- Następnym razem zabiję Ciebie i wszystkich twoich przyjaciół!
\qd
Poczułem ból w klatce piersiowej i mimowolnie zamknąłem oczy.\\
Zemdlałem\3k

Kiedy odzyskałem przytomność, siedziałem w samochodzie. Deszcz padał zauważalnie mocniej, a wraz z nim grad.\\
Podniosłem wzrok w kierunku budynku.
Szkoła zniknęła, a na jej miejscu znajdowało się 5-cio metrowe drzewo.\\
Mikołaj otworzył oczy.\\
Leon, przemoknięty do suchej nitki, wrzeszczał ze wszystkich sił. Półprzytomny Mikołaj prawie w ogóle go nie słyszał, lecz z uchu ust wyczytał ,,Wstawaj!''. \\
Przed omdleniem (chociaż nawet tego nie był pewien) nie poczuł niczego, co by to zapowiadało -- zawirowania w głowie, charakterystycznego szumu w uszach ani czarnych plam przed oczyma. Po prostu padł jak trup, chociaż chwilę przed upadkiem wydawało mu się, że na 2 sekundy stracił równowagę, zataczając się na lewo i prawo. \\
W jego głowie pojawił się obraz -- para ludzkich oczu z czerwonymi nań okręgami. \\
Widok ten przeraził go do głębi, ale sprawił też, że Mikołaj poderwał się na równe nogi niczym wystrzelony z procy. Leona zamurowało -- wpatrywał się w przyjaciela z szeroko rozwartymi ustami, a zaraz potem krzyknął :
%
\sx Szybciej!
\qd
Znowu poderwał się do biegu, Mikołaj zawtórował mu i po chwili we dwoje, metr od siebie, biegli na południe. \\
Wciąż znajdowali się na wysypisku --  wbiegli w jego część, gdzie wszędzie rosła niska trawa, która tonęła w kolorze brązowej ziemi, szczególnie widocznej o tej porze roku. Deszcz nieco zelżał, ale wciąż utrudniał poruszanie się. \\
John F. nie odpuszczał -- gonił dwójkę stalkerów i był pewny, że ich zaraz dopadnie.\\
Dawno zarzucił swego SVD na plecy i biegł teraz z rewolwerem Colt Pyton -- nie miał zamiaru marnować amunicji strzałami z daleka, lecz z pewnej odległości wykończyć uciekinierów.\\
Wygrzebał z kieszeni artefakt, który ,,poprawiał kondycję'' -- tak skromnie opisano mu go w momencie zakupu -- John chciał oszczędzić sobie specjalistycznego, medycznego żargonu. \\
Artefakt zadziałał od razu -- dzięki czemu Finn przyśpieszył gwałtownie i był coraz bliżej swych przyszłych ofiar.
%
\ro{3}
%
Mikołaj z Leonem zobaczyli zardzewiały, stary szkolny autobus -- nie miał okien, kół ani dachu -- wśród deszczu dało się spostrzec kontury siedzeń i wypłowiały napis ,,Autobus Szkolny'' na boku pojazdu, napisany w języku ukraińskim. Z obu stron -- Leon z lewej a Mikołaj z prawej schowali się za stary wehikuł, ślizgając butami po błocie. Leon upadł na plecy, niemal słysząc jakiś trzask w okolicy lewego przedramienia, przy okazji przypominając sobie o ranie w nadgarstku. Poczuł silny ból, jednak przemógł go i szybko oparł się na autobusie, tuż przy Mikołaju.
%
\sx Już po nas. -- rzekł bezradnie. -- Nawet siedząc w czołgu nie mielibyśmy z nim szans.
\xx Przestań pier\3k -- urwał, bo nad uchem jego i Leona ktoś wystrzelił z karabinu. Obaj skulili się ze strachu, zaciskając uszy z całej siły.
\qd
John ujrzał Mikołaja i Leona wbiegającego za autobus.\\
,,No to po was, koledzy'' -- pomyślał z ironicznym uśmieszkiem. \\
Na wszelki wypadek sprawdził bębenek Colta -- po upewnieniu się, że jest pełny, zaczął szybko zbliżać się do autobusu.
Deszcz nasilił się niespodziewanie, szybko i gwałtownie -- krople zgrubiały dwukrotnie, a wraz z nimi zaczęło padać coś, co dla Johna było mniej prawdopodobne niż deszcz żab. \\
Grad, chociaż małej wielkości, był widoczny, tworząc coś w rodzaju białego potoku. Mimo ogromnego zdziwienia, John ruszył dalej, nie chlupocząc już w błocie, lecz krusząc białe lodowe kulki. Będąc co najwyżej 10 metrów od autobusu, ujrzał znad jego dachu jasny błysk.\\
Przez sekundę w głowie przemknęła mu myśl ,,Jak mogłem go nie zauważyć?!''\\
Po tej krótkiej chwili John poczuł silny ból okolicy brzucha, chwilę później upadł na ziemię. Nadeszła kolejna myśl.\\
,,Nie wiem skąd urwał się ten kretyn, ale najwyraźniej nie wie, kim jestem.'' \\
Postanowił zaczekać na odejście człowieka, który właśnie dokonał próżnej próby zabicia go. Obawiał się, że jeśli wstanie, to wybawca Mikołaja i Leona nafaszeruje go taką ilością ołowiu, że nie będzie w stanie podnieść się na równe nogi. John słyszał głos trzech mężczyzn -- rozpoznał z nich Leona wraz z przyjacielem, lecz głosu trzeciej osoby, tajemniczego wybawcy, nie znał. \\
Odczekał pięć minut. \\
Grad ustąpił miejsca deszczu, który wyraźnie zelżał. \\
Zaczęło się przejaśniać, wiatr ustał, a czarne chmury zniknęły. \\
John był pewien, że póki co jest bezpieczny. Podniósł się z pokrytej błotem trawy i usiadł. Westchnął ciężko po czym sięgnął prawą ręką do plecaka.
%
\sx Ci idioci nie wiedzą, z kim mają do czynienia. Nie doceniają mnie. Ten pajac chciał mi chyba tylko zrobić dziurę w kombinezonie. -- Finn był wyraźnie zdenerwowany, niemal wściekły.
\qd
Wyjął artefakt zwany ,,Duszą'' -- położył go obok lewej nogi i rozpiął kombinezon, aby obejrzeć ranę. Będąc w białym (nie licząc czerwonej plamy na lewej piersi) podkoszulku, podwinął go i zobaczył miejsce trafienia w całej ‘okazałości’ , mimo że dla Johna była mniej uciążliwa niż łaskotki -- wtedy chociaż chichocze, a teraz jedyną szkodą będzie dziurawe odzienie. \\
Rana nie odbiegała zbytnio swym wyglądem od powszechnie znanych opisów -- dziura, a raczej dziurka była małej wielkości, z której środka powoli sączyła się krew. \\
,,Żałosne!'' -- pomyślał John. Podniósł Duszę lewą ręką i zacisnął ją, aby zadziałała jak najszybciej. Stało się tak od razu -- rana zmniejszała się błyskawicznie, z sekundy na sekundę, dosłownie nikła w oczach. \\
Chwilę później, po trafieniu będącym dla normalnego człowieka śmiertelnym, została tylko plama krzepnącej krwi.
%
\sx Jeszcze was dorwę\3k -- sapnął -- Mieliście szczęście. Ale nie na długo.
\qd
John schował Duszę do plecaka, zarzucił kombinezon z powrotem na siebie i lekkim truchtem ruszył do swego mieszkania w Prypeci. Wbrew pozorom, można było się tam nieźle urządzić.

Ile czasu gapiłem się na do drzewo? Pięć minut? Dwadzieścia? Pewnie o wiele dłużej, ale trudno było mi się dziwić -- do cholery, właśnie byłem świadkiem zniknięcia budynku szkoły i zastąpienia go przez drzewo. Może wyobraziłem sobie to wszystko? Ale na podstawie czego miałbym sobie wyobrażać mężczyznę w garniturze, do tego stuprocentowo białymi oczyma, na które nasuwały się czerwone, świecące okręgi? Podobno sny i różne wizje wybiera się mimo woli, ale na podstawie tego, co się widziało. A mogłem się założyć się o głowę, że do tej chwili nie miałem o tym miejscu ani jednego wspomnienia. \\
Po tych rozmyślaniach dotarł do mnie fakt, że właśnie zaczął padać grad -- w tym miesiącu prędzej spodziewałbym się czterdziesto stopniowego upału. Najpierw jakiś facet z ,,nietypowymi'' oczyma i ubiorem, potem zamiana szkoły w drzewo, a teraz grad. Zdecydowanie najdziwniejszy dzień w moim życiu.\\
Najwyższy czas pojechać do obozu. \\
Spojrzałem na deskę rozdzielczą -- elektroniczny zegarek wskazywał 18:24, zatem niedługo się ściemni, a nie miałem zamiaru nocować w Prypeci. \\
Kluczyki były w stacyjce -- przekręciłem je dwa razy, zanim silnik zapalił. Zwolniłem hamulec ręczny, wrzuciłem wsteczny bieg i wyjechałem spod drzewa. Jeszcze trochę kręcenia kierownicą i ruszyłem na drogę prowadzącą bezpośrednio do obozu.
Komu opowiem całą tą historię? Na pewno nie dowódcy, który wyśmiałby mnie i wywalił z roboty. Chyba jedynym człowiekiem, jakiemu mogłem zaufać, był Leon -- stalker, który nieoficjalnie należał do tutejszych służb -- miał w nich wielu znajomych, a z racji niesionej pomocy był w obozie zawsze mile widzianym gościem. Praktycznie, przesiadywał w nim dzień w dzień. Miałem przeczucie, że i dziś go w nim zastanę -- opowiem mu o wszystkim w barze, przy dobrym, zimnym piwie. \\
Deszcz wraz z gradem dudniły w dach samochodu, przy okazji pogarszając widoczność. \\
Gdyby nie wycieraczki, jechałbym z głową wystawioną przez okno. \\
Mając pięćdziesiątkę na liczniku skręciłem w lewo, przy supermarkecie.\\
Gdyby nie szczęśliwy fart, potrąciłbym człowieka. \\
Wysoki mężczyzna w zielonym kombinezonie, z nałożonym kapturem i małym plecakiem przebiegł tuż przed maską Hondy -- reflektory oświetliły go na chwilę, lecz nie dostrzegłem niczego niezwykłego. \\
Niemal czułem, jak otarł się o zderzak auta. \\
Jakimś cudem nie postąpiłem w tej sytuacji jak wielu ludzi i nie zjechałem gwałtownie z drogi, przy okazji unikając czołowego zderzenia z blokiem. Wykonałem skręt jak gdyby nigdy nic, jednak omal nie dostałem zawału i musiałem się zatrzymać, aby odsapnąć. \\
Dysząc ciężko, myślałem, co samotny (prawdopodobnie stalker) robił w centrum Prypeci. Każdy, kto jest normalny, powinien o tej porze z niej uciekać, a nie biec w samo centrum. \\
Nie mogłem się doczekać powrotu do bazy -- musiałem w końcu odpocząć -- na dziś miałem dość tych wszystkich, delikatnie powiedziawszy, nietypowych zdarzeń.
%
\ro{4}
%
Leon i Mikołaj jechali wojskowym dżipem prostu do bazy -- nieoficjalnie była to siedziba wszystkich ważniejszych pracowników urzędu bezpieczeństwa w całej Zonie. W wehikule oprócz dwóch stalkerów znajdował się ich wybawca -- Mick, który prowadził.
%
\sx Jeszcze raz dzięki. -- Mikołaj poklepał Micka po ramieniu.
-Nie ma za co. -- odpowiedział z uśmiechem -- A tak w ogóle, to kim on był?
\qd
Leon i Mikołaj wybuchli śmiechem.
\sx A wy z czego? -- zapytał kierowca. \qd
Dialog podjął Leon.
\sx Nie ,,był'' lecz ,,jest'' -- powiedział. -- Właśnie zostałeś kolejną osobą, która myślała, że uśmierciła Johna F.
\qd
Adam, jeden z bardziej znanych miejscowych handlarzy, siedział w swojej przyczepie postawionej aktualnie w tunelu kolejowym na terenie opuszczonej bazy wojskowej. \\
Przyczepa miała 10 metrów długości i 5 szerokości i z zewnątrz wyglądała niezbyt zachęcająco. \\
Była białego koloru, jednak większość farby odpadła i wypłowiała. Po jej lewej stronie wbudowane zostały dwa małe okienka, pomiędzy którymi znajdowały się drzwiczki wejściowe. Adam zamykał je z obu stron na stalową, antywłamaniową kłódkę -- fabryczny zamek został zniszczony przez złodziei 3 dni po zakupie. Koła zdemontował znajomy Adama -- przyczepa była teraz oparta o kilka cegieł, aby nie przechylała się w pionie. \\
Wnętrze prezentowało się o wiele okazalej -- można było o nim powiedzieć, że jest wręcz eleganckie. \\
Sufit i ścianę obito miękką, jasnobeżową tkaniną, a całą podłogę pokrywał czysty i zadbany czerwony dywan -- przy drzwiach położono na nim czarną wycieraczkę. \\
Tuż przy drzwiczkach wejściowych stał drewniany wieszak w kolorze ciemnego dębu, na którym wisiał czarny, sięgający butów płaszcz oraz ciemnozielona, wełniana czapka.\\
Dwa metry na lewo od wyjścia postawiono mały aneks kuchenny, którego powierzchnię stanowił granit -- nad aneksem wisiała blaszana szafka z talerzami i szklankami -- pół metra na prawo zamontowano błyszczący zlew, pobierający wodę ze zbiornika pod przyczepą. Wszystkie sztućce Adam trzymał w małej szufladce na skraju przyczepy. Na jej końcu, na całej jej szerokości postawiono rozsuwaną, lekko wytartą, lecz ciągle dobrze utrzymaną kanapę w kolorze niebieskim. \\
Na drugim skraju obecnego mieszkania Handlarza przytwierdzono do podłogi ubikację oraz wannę -- nad nimi, w suficie, powbijano haki, z których zwisała syntetyczna zasłona. \\
Adam siedział na kanapie, ubrany w biały podkoszulek, lekkie białe spodnie i czarne skarpetki. 20-to calowy telewizor stał na podłodze, gdyż stół był w naprawie. \\
Telewizor był włączony na ukraiński kanał informacyjny nadający 24 godziny na dobę. Prezenter -- wysoki i młody Latynos o czarnych włosach -- opowiadał o wyniku jakiegoś meczu. Kiedy skończył, zapowiedział najbliższą relację na żywo.
%
\sx Za chwilę przeniesiemy się do Mryńska, małej miejscowości na wchodzie Ukrainy, gdzie nastąpi otwarcie nowej elektrowni atomowej -- pierwszej wybudowanej po pamiętanej przez nas wszystkich katastrofie w Czarnobylu. Tym razem nie ma możliwości żadnych usterek -- kompleks wyposażono w najlepsze sprzęty na świecie, sprawdzone przez wiele krajów czerpiących energię w ten sposób. Oddaję głos naszemu wysłannikowi, Arturowi Boruckiemu.
\qd
Ekran zmienił się i ukazywał teraz twarz pewnego białego mężczyzny z mikrofonem w ręku. Za nim można było dostrzec beżową ścianę jakiegoś budynku i reporterów innych stacji telewizyjnych -- także furgonetki, w których przyjechali. Studio telewizyjne ukazywane było teraz w małym okienku w lewym górnym rogu.
%
\sx To niewątpliwie wielkie wydarzenie\3k -- zaczął reporter.
\qd
Streścił krótki historię katastrofy w Czarnobylu -- jej czas, przyczynę i skutki. Porównał wyposażenie dwóch tamtej i Mryńskiej elektrowni, aby jeszcze bardziej zapewnić wszystkich, że kolejny tego typu wypadek jest niemożliwy. Cała opowieść trwała ponad dwie minuty, a kiedy Artur skończył, kamera przesunęła się w lewo, oddaliła widok, co pozwoliło sfilmować elektrownię w całej okazałości. \\
Wielkością była dość podobna do tej z Czarnobyla -- różnice stanowiły -- niższy komin bez dodatkowych obudowań wokół oraz nowoczesne zaprojektowanie -- estetyczne, kremowe ściany i duże okna. \\
Znowu pokazywano studio -- wysoki Latynos obwieścił, że za 10 minut elektrownia wykona swój pierwszy test bezpieczeństwa i że będzie to dowód na to, iż okoliczni mieszkańcy nie muszą się niczego obawiać. \\
Nagle na twarzy spikera pojawił się strach -- rozszerzył on usta, wytrzeszczył oczy i opadły mu ręce. Lewą poderwał do prawego ucha -- nacisnął znajdującą się w nim słuchawkę. Wymamrotał coś pod nosem po czym powiedział zdesperowanym tonem.
-W elektrowni nastąpił wybuch. Artur? Artur, słyszysz mnie? -- zaczął kręcić ręką do kamery, na znak aby przełączono ją na operatora wysłannika. \\
Nastąpiło to po dwóch sekundach i zanim Adam ujrzał obraz, usłyszał krzyki i dźwięk walącej się ściany oraz łomocącego gruzu. \\
Operator biegł sprintem do ciężarówki dysząc ciężko. Uderzył całym swoim ciałem w drzwi furgonetki, otwierając niemal nie wyważył ich z nawiasów, niedbale rzucił kamerę na tylne siedzenie -- ukazywała ona teraz siedzenia kierowcy i pierwszego pasażera.
Artur obejrzał się przez ramię o gorączkowo łypał oczyma -- szukał zapewne swojego asystenta Artura. Kiedy spojrzał w stronę lewego okna, jego asystent wskoczył do środka przez boczne drzwi. Zaraz po ich zatrzaśnięciu krzyknął:
\sx Jedź! \qd
Operator i zarazem kierowca zapalił silnik i furgonetka ruszyła naprzód. Słychać było trzeszczące pod kołami liście, a całym wehikułem lekko potrząsało. Kiedy dwoje dziennikarzy wyjechało z wybojów, Artur obrócił się, chwycił kamerę i postawił ją na desce rozdzielczej. Przed nimi jechało jeszcze kilka furgonetek przyozdobionych logami różnych stacji.
Adam poderwał się z kanapy, wybiegł z przyczepy i powiadomił wszystkich, co się stało.
%
\ro{5}
%
Mick, Leon i Mikołaj podjechali pod bramę obozu.\\
Grad przestał padać i nad siedziba Bezpieki została oświetlona łagodnym światłem.\\
Znajdowała się ona w lesie -- daleko na północ od Kordonu w samym jego środku. W obrębie 10-ciu metrów od murów wycięto drzewa i postawiono w ich miejsca 4 wieże strażnicze -- 2 obok siebie, przy bramie i dwie z tyłu obozu. Na każdej znajdował się strażnik. \\
Bramę wjazdową stanowiła rozsuwana, wzmocniona blacha dodatkowo poprzedzona szlabanem pomalowanym w czarno-białe, pionowe paski. Tuż przy niej znajdowała się mała budka. \\
Wychylił się z niej młody mężczyzna w czarnej skórze i sprawdził pasażerów dżipa. Gdy spostrzegł salutującego Micka, schował głowę z powrotem za okno, podniósł szlaban i rozkazał strażnikowi za bramą przesunąć ją w lewo. \\
Dżip wjechał do środka, a jego pasażerowie ujrzeli dawno wyczekiwany obóz. \\
Składał się z 4 baraków umieszczonych po lewej stronie, przy murze, 2 dwupiętrowych budynków (znajdowały się w nich biura) naprzeciwko bramy, baru położonego mniej więcej pośrodku (był to budynek o wielkości przeciętnego sklepu monopolowego) i małego, naprędce zbudowanego, drewnianego domku w północno-zachodnim rogu. Na lewo od niego było podziemne wejście do bunkra -- zwyczajna, zamykana na kłódkę klapa -- w którym znajdowali się najważniejsi ludzie tutejszego Urzędu.
Trójka stalkerów wysiadła z pojazdu.
%
\sx Dokąd teraz? -- spytał Mick zamykając drzwi kierowcy.
\xx Do baru. Trzeba pogadać z Radkiem.
\xx O ile wiem to\3k -- urwał, kiedy do obozu wjechał zielony Mercedes. -- To on. A więc wy sobie tu pogadajcie, a ja czekam na was w barze. -- Schował kluczyki do kieszeni i ruszył w stronę ,,gospody''.
\qd
Mercedes skierował się w lewo i stanął pod drewnianym domkiem. \\
Wysiadł z niego Radek -- najlepszy i najbardziej znany agent w całej Zonie. Zasłynął ponad 2-wu letnią inwigilacją miejscowej grupy płatnych zabójców, w końcu doprowadzając ją do rozbicia, a jej członków do więzienia. \\
Miał trzydzieści lat, sto dziewięćdziesiąt centymetrów wzrostu, czarne, krótkie włosy i bardzo błękitne oczy. Na brodzie nosił lekki, ciemny zarost. \\
Ubrał się w czarny, sięgający butów płaszcz, z którego lewej kieszeni wystawał kaptur ubrania przeciwdeszczowego.
Zamknął drzwi, podszedł do Leona i Mikołaja, po czym zaprosił ich do baru. \\
Z głębi budynku dobiegały odgłosy muzyki -- śpiew kobiety na tle harf i gitary. Bardzo podobała się Leonowi. \\
Muszę się dowiedzieć, co to za piosenkarka. \\
Ciekawe, jak spisał się Radek. \\
Podszedłem do barmana.
\sx Trzy piwa. -- zapłaciłem, zanim gospodarz wyjął butelki spod lady.
\qd
Otworzył je i przesunął w moją stronę. Podziękowałem, chwyciłem dwie z nich do lewej ręki, zaś z trzeciej pociągnąłem łyk, po czym przysiadłem się do Mikołaja z Radkiem. \\
Kiedy wypili nieco, podjąłem rozmowę.
%
\sx Jak tam, Radek? Wszystko poszło zgodnie z\3k
\xx Ani słowa! -- przerwał mi brutalnie. -- Jak ci opowiem, co mnie spotkało w tej zasranej dziurze, to ci szczena opadnie.
\qd
Zachichotałem.
%
\sx No, opowiadaj. -- zachęciłem go i po raz trzeci już upiłem trochę piwa.
\xx No więc podjeżdżam pod tą szkołę. Robie to, co zwykle -- wyjmuje gnata, łomocze w drzwi i drę się ,,Służby Specjalne!'' -- sparodiował typowe zachowanie policjanta krzyczącego coś do aresztanta -- słyszę kroki, ktoś zza drzwi pyta się ,,Kto tam?''. To ja wołam znowu, a tu nagle BANG! -- rozłożył ręce, naśladując wybuch -- drzwi łamią się w pół, ja wpadam do środka i wale na oślep. Zmieniam magazynek, idę wzdłuż korytarza tej obskurnej szkółki, kiedy ktoś wypada dwa razy w ścianę, przy której stałem. -- szybko odsłonił łokieć, ukazując zakrwawiony bandaż. -- Jeden mnie trafił. Siedzę bezbronny jak dziecko pod ścianą, bez pistoletu, kiedy podchodzi do mnie jakiś facet w GARNITURZE -- zaakcentował -- gada mi coś o ,,tajemnicy tego miejsca'', po czym, i to jest najlepsze, unosi siłą woli pistolet w powietrze, wypala z niego siedem razy, a pociski utrzymuje nad ziemią, jak jakiś zasrany Neo. Wszystkie kieruje w moją stronę -- ślady po nich uczyniły pewnie zabawny kontur mojej sylwetki. Sam gość w garniturze
nie ma normalnych oczu, tylko same białka i jakieś czerwone, migające okręgi na nich. To jeszcze nic! Facet mnie zna, mówi mi po imieniu, a kiedy tracę przytomność, ocykam się w samochodzie i zamiast szkoły widzę co?
\xx Budkę z Hot-Dogami? -- spytał ironicznie Mikołaj.
\xx Mc’Donalda? -- spytałem podobnym tonem.
\qd
Radek z trudem powstrzymał się od śmiechu, ale szybko spoważniał i odpowiedział władczym głosem.
\sx Widzę, do cholery, drzewo. \qd
Mnie (i zapewne Mikołaja, bo wpatrywał się tępym wzrokiem niewiadomo gdzie) zamurowało. Obydwaj znaliśmy Radka od dawna i tylko dzięki temu go nie wyśmialiśmy. Kiedy on mówi takie rzeczy, nie zmyśla. Nie mogłem jednak ukryć lekkiego poirytowania.
%
\sx W tej szkole, siedem lat temu, znajdował się najlepiej strzeżony skład najdroższych i najrzadszych artefaktów, a nie sojusz Budowlańców I Ogrodników, którzy dążyli do sadzenia drzew w miejsce budynków!
\xx Dobra, wyżyj się, ale nie przeginaj. Aha. Kiedy wracałem, omal nie potrąciłem jakiegoś stalkera. Biegł w samo centrum Prypeci.
\qd
Minął Johna F.\\
I czemu ja nie kazałem mu pozaczepiać na drzwiach metrowych kolczatek\3k \\
To, że John urządził się gdzieś w Prypeci, specjalnie mnie nie zdziwiło. Nie zdziwiłby mnie nawet fakt, gdyby John F. mieszkał pod powierzchnią jakiegoś radioaktywnego jeziora. On był zdolny do niemal wszystkiego. \\
Zadzwonił telefon. \\
Mikołaj z Leonem spojrzeli na mnie niepewnym wzrokiem. Wiedzieli pewnie, tak samo jak ja, kto dzwoni. Wyjąłem komórkę, odebrałem ją, ale zanim zdążyłem coś powiedzieć ze słuchawki rozległ się nieludzki wrzask.
\sx Do mnie! Migiem!!! -- krzyknął Barry i rozłączył się. \qd
Włożyłem telefon powrotem do kieszeni i mruknąłem.
\sx Znowu się będę musiał ze wszystkiego tłumaczyć\3k ku*wa.
\xx Nie martw się -- pocieszył mnie Leon. -- Znasz Barry’ego. Trochę pokrzyczy, pomarudzi i mu przejdzie.
\xx Tak\3k tylko że ja nie jestem, ku*wa, odstresywaczem! Przyjechałem tu aby rozbijać organizacje łowców głów i tym podobnych, a nie zbierać nagromadzoną żółć!
\xx Lepiej tam już idź, żeby mieć to z głowy.
\qd
Pokiwałem głową, skończyłem piwo -- dwójka moich przyjaciół zrobiła to samo -- po czym wszyscy podnieśliśmy się z krzeseł.
\sx Idziemy z tobą -- obwieścił Mikołaj. \qd
Zamurowało mnie. Patrzyłem się na Pana Samobójcę tępym wzrokiem przez jakieś pół minuty, po których odpowiedziałem.
%
\sx Jesteś nienormalny.aaaa
\xx Wcale nie -- odrzekł radosnym głosem. Leon uśmiechnął się lekko.
\xx Posrało was. Albo jesteście nienormalni i życie wam niemiłe, albo coś kombinujecie. No? Co wam chodzi po głowie?
\xx Zobaczysz. Teraz idziemy do Barry’ego. Wszyscy troje.
\qd
Minęliśmy kilka stolików, wspięliśmy się po małych schodkach i wyszliśmy z baru na zewnątrz.  Mocno wiało, a szum kołyszących się drzew niósł się po całej okolicy. \\
Ruszyliśmy ku drewnianego, jednopiętrowego domku -- tam właśnie urzędował szef wszystkich tutejszych agentów. Cały domek był jego -- mieszkał w nim i sypiał na co dzień. \\
Stanęliśmy przed drzwiami -- eleganckie, z drewna ze żłobionymi nań wzorami. Po lewej stronie wisiał biały przycisk dzwonka.
Nacisnął go Leon. Dźwiękiem było ćwierkanie ptaka.
\sx Wejść! -- zza drzwi dobiegł donośny głos. \qd
%
\ro{6}
%
Ledwie Mark przekroczył próg baru, z jego wnętrza podbiegł do niego jakiś mężczyzna o szerokiej, gładkiej i ogolonej twarzy, piwnych oczach oraz krótkich, ciemnobrunatnych włosach. Miał na sobie, podobnie jak reszta barowego towarzystwa, zieloną kurtkę i grube, czarne, wytarte dresowe spodnie. Był przysadzisty, lecz w dobrej formie, i sięgał Markowi do podbródka.
%
\sx Ty jesteś ten nowy? Niejaki Mark? -- zapytał cichym, łagodnym tonem.
\xx Tak\3k
\xx Za mną. Pokaże ci co i jak ,i kto jest kim.
\qd
Obrócił się na pięcie i wolnym krokiem skierował się do środka. Mark ruszył za nim, zachowując metrowy odstęp. Przeszli przez zatłoczone stoliki, a gdy zamierzali się, by przy jednym usiąść, Przewodnik Marka wpadł na jakiegoś wysokiego stalkera. Przywitał się z nim uściśnięciem ręki, mrucząc pod nosem ,,Radek''. Zaraz za dryblasem, ku wyjściu szło dwóch innych gości -- Izaak, bo tak nazywał się ,,niski'' przywitał się z obojgiem, mrucząc ich imiona -- ,,Leon'' , ,,Mikołaj''. Po usadowieniu się, zamówieniu alkoholu -- piwa dla Izaaka oraz flaszki dla Marka, zaczęła się pomiędzy nimi rozmowa.
\sx Poka no swoje akta. -- rzucił Izaak. \qd
Mark wyciągnął zza pazuchy małą teczkę i położył ją na stole.
\sx Co my tu mamy\3k -- z uśmiechem zatarł ręce, otworzył akta i zaczął je czytać. \qd
Przewracał oczyma na lewo i prawo, co chwilę wydobywając z siebie głębokie ,,Hmm\3k''. Chwilę potem, obydwa rozszerzyły się w zdziwieniu. Izaak wytężył wzrok, pochylił się lekko, aby coś doczytać i prychnął śmiechem.
%
\sx Wolne żarty, gnojku. A teraz podaj swoje PRAWDZIWE akta, albo wylatujesz stąd na zbity pysk, i nawet do walk ze snorkami na arenie nie będą cię chcieli przyjąć! -- widać i słychać było w nim zdenerwowanie.
\xx To są prawdziwe akta, stary pierniku -- odpowiedział Mark z uśmiechem. Mina drugiego była nie do opisania, brakowało tylko pary bijącej z uszu i dźwięku gwiżdżącego czajnika.
\xx Aha -- odpowiedział po chwili -- Czyli w dzień przydziału do Zony, dostajesz zadanie\3k -- spojrzał akta i wyczytał z nich zdanie -- ,,Zabezpieczenia całego terenu szpitala w wschodniej Prypeci, aby dokonać tam specjalistycznych badań\3k'' Po przekroczeniu Strefy, jesteś jedną nogą w grobie. Tak długo śpieszy ci się do następnego kroku?! -- niemal krzyknął, rzucając mocno teczką o stół. -- Można się kłócić, co gorsze, Jantar, Dzicz, Prypeć, Czerwony Las, czy może ten zawszawiony Kordon. Ale ten szpital, to wraz z laboratoriami nad Jantarem i okolicami mózgozwęglacza, istne piekło. Nie pchaj się tam, po prostu ze zdrowego rozsądku.
\xx Poradzę sobie.
\xx Może z chmarami zombiech, stadami snorków i Bóg wie czym jeszcze, ale ten szpital\3k
\xx Jest nawiedzony, tak, wiem. -- spokój Marka był godny podziwy.
\qd
,,Niby nowy, zielony, ale zachowuje się, jakby miał naprawdę spore doświadczenie'' -- pomyślał Izaak.
%
\sx Może nie do końca nawiedzony -- uniósł ręce niczym duch z kreskówki -- w pełnym tego słowa znaczeniu. Tu da się powiedzieć tylko jedno -- ze szpitalem jest jak z resztą Zony. Niektórych\3k a właściwie to większości rzeczy nie da się racjonalnie wytłumaczyć. Wyglądasz na bystrego, ale powiem ci to z zasady -- cokolwiek zobaczysz, nie daj się zwariować. Niektórzy widują tutaj swoich zmarłych krewnych, jeszcze inni dawnych przyjaciół -- masz tu tego więcej, niż filmów w Bollywood. Sam widziałem tu ducha swojego dobrego kolegi\3k
\xx Nic mi nie będzie -- zapewnił Nowy. -- Ten szpital jest dla nas bardzo ważny.
\xx Taa\3k To, co się tam wydarzyło, daje do myślenia. Dziesięć lat przed wybuchem, a ten cały\3k jak mu tam było?
\xx Gomez. Doktor Gomez.
\xx Właśnie. 10 lat przed powstaniem Zony, ten cały Gomez miał już przedmioty o właściwościach dzisiejszych artefaktów. Niewiarygodne\3k Mimo, że masz o niejednym pojęcie, muszę cię zapoznać z paroma ludźmi, żeby jakieś gnojki nie odstrzeliły ci dupy dla nowych kaloszy. Najpierw ta trójka, która minęła nas niedawno. Ten wysoki to Radek -- jeden z lepszych naszych ludzi. Słyszałeś o ósemce?
\xx Ta grupa płatnych morderców, która została jakiś czas temu rozbita?
\qd
Izaak skinął głową.
%
\sx To Radek ich inwigilował. Męczył się szmat czasu, ale jak widać było warto. Ta dwójka, która mu towarzyszyła, to Leon i Mikołaj -- na razie mogę ci powiedzieć tylko tyle, że mają tu duże wpływy, znajomości całej w Zonie, a znają się od dzieciństwa. Nieoficjalnie do czas należą -- niejedno razem przeszliśmy\3k Reszta\3k jak cię polubią, to sami ci powiedzą. Ludzi, których warto znać, jest wielu, ale jak będziesz w dobrych kontaktach z Mastertonem, to sam Pan Bóg cię nie ruszy.
\xx A Masterton to\3k
\xx Jonathan Masterton -- można go skwitować jednym, bardzo miłym słówkiem -- psychol. Ma przeszłość brudną niczym gówno Mięsacza -- podejrzany o zabicie znajomego i wiele napadów na bank. Niegdyś prawdopodobnie płatny zabójca, wycofał się po rzekomym morderstwie szefostwa jego organizacji. Później, i to jest pewne, pracował w grupie zawodowych złodziei. Kolejne kilka okradzionych banków. Mniej więcej w wieku 40 lat wstąpił do nas. Cierpił na Mauhausena, ale do tej pory jakoś się trzyma. Ponadto, śmierć jego przyjaciela dogłębnie nim wstrząsnęła, i od tamtej pory ciągle łyka jakieś psychotropy, żeby kompletnie mu nie odbiło. Jest wiecznie podenerwowany, niepewny i podejrzanie łypie oczyma na wszystko co się rusza. Mimo wszystko jest to człowiek z zasadami -- nigdy cię nie oszuka, a w rozmowie i kontaktach z ludźmi jest miły i uprzejmy. Ubiera się jak ,,gangster z klasą'' i ma rozcięte lewe oko.
\xx A jego ,,przydatność bojowa''?
\xx Zna większość sztuk walk i jest mistrzem w strzelectwie, wyćwiczył większość broni znanych światu. Oho\3k idzie tu. Na razie siedź cicho, zobaczymy, czy mu się spodobasz.
\qd
Mark spojrzał przez ramię i ujrzał Jonathana. \\
Wysoki, około 180 Cm, z krótkimi, czarnymi włosami sięgającymi ledwie czubka czoła. ,,Zdrowe'', prawe oko miał błękitne, drugie zaś było trupio blade, a nad i pod nim widniał ślad po głębokim cięciu. Ubrany był, jak mówił Izaak, ,,po gangstersku'' -- zdecydowanie wyróżniał się z tłumu. Nosił czarne, materiałowe spodnie, ciemnobrązowe półbuty oraz czarny garnitur -- rozpięty luźno pod szyją, tak samo jak biała koszula. Mark zauważył jednak jeszcze jeden szczegół -- w prawej dłoni, palcem środkowym i serdecznym trzymał kastet z wbudowanym, krótkim, wystającym ostrzem. Dało się go spostrzec z daleka, i wraz z ubiorem robił ogromne wrażenie. Na twarzy malowało się zmęczenie, ale trudno się dziwić, po tym, co przeszedł. \\
Przysiadł się do Marka i Izaaka.
%
\sx Witam panów. -- miał wyraźny, czysty, lecz smutny głos. Wyciągnął rękę do obu agentów, uścisnął je i dosiadł się głębiej do drewnianego stolika.
\xx Co słychać? -- zapytał serdecznie.
\qd
Mark nie wierzył własnym oczom -- człowiek o wyglądzie ( nie licząc garnituru ) najgorszej męty, zachowywał się jak najlepszy uczeń szkoły savoir vivre’u. Psychotropy robiły swoje, czy też opowieści Izaaka były przesadzone? Nowo przybyły trzymał lewą rękę pod, a prawą nad stołem, co chwilę obracając w palcach kastet ze sztyletem.
\sx Ten tutaj -- Izaak wskazał palcem na Nowego -- dostał zadanie oczyszczenia szpitala. \qd
Jonathan gwałtownie szarpnął ostrzem w dół, jednocześnie odchylając głowę lekko w lewo, wydając przy tym nerwowy odgłos. Nóż z piskiem przejechał po stole, po czym głucho uderzył wraz z ręką posiadacza o krzesło. ‘Psychol’ był widocznie zdenerwowany, niemal wściekły -- zaczął się trząść, co chwilę, coraz mocniej wbijając kasteto-sztylet w spód blatu. Usta otworzyły mu się z widocznym zamiarem powiedzenia czegoś, lecz przerwał mu dzwoniący telefon. Migiem się uspokoił, sięgnął do kieszeni garnituru, otworzył klapkę i przyłożył telefon do ucha. \\
Trzy potaknięcia, pomruk oznaczający zgodę, pytanie ,,Gdzie?'', koniec rozmowy.
\sx Znaleźli jeden z ,,magazynów'' tutejszych handlarzy organów. Sala wyłożona kafelkami i przerobiona na kostnicę. sku*wiele\3k Lenny! -- zawołał, machając do jednego z obecnych w barze stalkerów ręką.
\qd
Przysiadł się do nich wychudzony, niewysoki, młody brunet.
\sx Dla całej czwórki? -- spytał, sięgając do kieszeni.
\xx Ta. -- odpowiedział Wariat.
\qd
Lenny podał wszystkim obecnym po parze białych, gumowych rękawiczek.
%
\ro{7}
%
Po dotarciu na miejsce, przeżyłem naprawdę szokujący dzień. Izaak nie odczuł tego zbytnio, a może udawał, oczekując tylko
okazji do puszczenia pawia? Lenny zachował kamienną twarz -- pewnie miał już do czynienia z gorszymy zachowaniami Jonathana, ale ja nigdy nie zapomnę, co zrobił w ,,prowizorycznej'' kostnicy. \\
Siedzieliśmy w niej od dziesięciu minut -- ja, czyli Mark, Masterton, Izaak i Leonard/Lenny. Miała ona szerokość siedmiu, a długość dwudziestu metrów. Po lewej i prawej stronie umieszczono chłodzone szuflady na nieboszczyków, na końcu pomieszczenia postawiono szafkę z krzesłem, wraz z blaszanym pojemnikiem na ,,sprzęt''. Sufit pomalowano na biało, a ściany i podłogę wyłożono białymi, kwadratowymi kafelkami z zielonymi przerwami. Do sufitu przymocowano dwie jarzeniówki, które skutecznie oświetlały cały ,,magazyn''. Poza tym panował tutaj silny zapach środków dezynfekujących. \\
Gdy wchodziliśmy do środka, Jonathan szedł przodem, ubrany tak, jak w barze, sprawie otworzył drzwi wytrychem, następnie wywarzając je kopniakiem. Trafił nimi w właśnie wychodzącego chłopaka (bo mężczyzną nazwać się go nie dało), zwalając go z nóg. \\
Masterton szybko wpadł do środka i spojrzał na jęczącego pod nim młodzieniaszka z zażenowaniem.
%
\sx Biedny\3k -- rzekł z politowaniem, strasznie cienkim głosem -- sku*wiel! -- zmienił ton na głośny, niemal krzykliwy i kopnął młodego w brzuch z wyraźnym obrzydzeniem na twarzy.

\qd
Wkroczyłem do ,,trupiarni'' wolnym krokiem, obserwując Psychola. . Zaraz za mną podążyli Izaak oraz Lenny. Ten ostatni powiedział na głos.
\sx Do roboty! -- zaczął zakładać rękawiczki. \qd
Reszta uczyniła to samo, nakładając lateks na lewą dłoń, obawiałem się najgorszego.
Jonathan chwycił młodzieńca za fraki i pociągnął go po ziemi, wbijając go głową w ścianę. Zawył z bólu i skulił się jeszcze bardziej.
%
\sx Niedobrze mi się robi -- mruknął niedoszły oprawca. -- Najgorsi z najgorszych, najpodlejsi z najpodlejszych\3k\qd
Chciałem mu przerwać, ale Lenny powstrzymał mnie szturchnięciem w łokieć i wymownym spojrzeniem.
\sx \3k wrócić do ciepłego domku i pogadać z przyjaciółmi, o tym jak zarobiłeś sobie na cudzej nerce?! -- krzyk i oczy, a raczej oko zaślepione szałem. -- Dziś nie wyjdziesz stąd w jednym kawałku, o nie\3k
\qd
Lewą ręką złapał młodego za szyję, zaś prawą przysunął ku twarzy, zbliżając ostrze na kastecie niebezpiecznie blisko oka ofiary.
%
\sx Możesz tylko zadecydować, w ilu. Jeśli szybko nam powiesz, kto jeszcze siedzi w tym\3k -- splunął z odrazą. -- czymś, to umrzesz szybko i bezboleśnie. -- wstał z kucków, odrzucając głowę mocno o ścianę. Usłyszałem kolejne wycie.
\qd
Siedzieliśmy teraz na 4 krzesłach obok siebie -- wszyscy, poza młodym, który wyznał, że nazywa się Johnson. Zajął krzesło za stolikiem i rozmawiał z Mastertonem. Jeśli można to było nazwać rozmową.
\sx Mów, skurwysynu! -- wrzeszczał tak głośno, że od czasu do czasu zabolało mnie ucho.
\xx Zajmowałem się\3k -- zaczął cicho Johnson.
\xx Wiem do ku*wy nędzy, że tylko ,,formalnościami'' -- dodał z pogardą -- Masz mi powiedzieć ważne osoby biorące w tym udział, a nie, kto zamiatał czy zmywał tę -- tupnął -- pierdoloną podłogę! Także, wkrótce, z twojego mózgu!
\qd
Nasze źródło informacji milczało.
\sx Dosyć tego, do ku*wy nędzy\3k -- odwrócił się gwałtownie na pięcie, głośno uderzając podeszwami butów o kafelki rzucił się na szafkę z narzędziami.
\qd
Wywrócił ją do góry nogami, wyrzucając z niej; skalpele, piły, brzeszczoty, worki, rękawiczki, igły, strzykawki i to, czego pewnie Jonathan szukał -- elektryczną piłę do otwierania czaszek.
\sx Te smarki nawet nie pozbyły się tego, co im niepotrzebne. Chyba że\3k -- zwrócił się do Johnsona -- lubiliście pobawić się takim cackiem na truposzach! Gadaj, do ku*wy nędzy, bo potne cię na kawałki! Nie zdajesz sobie sprawy, jak wiele mogę zrobić, zanim umrzesz.
\qd
Milczenie.
\sx Twardziel! -- Masterton wsadził kabel od piły do kontaktu i włączył ją.
\qd
Zapiszczała złowieszczo, siejąc zamęt w mojej głowie. Chciałem stąd wybiec, uciec jak najdalej, żeby dalej nie oglądać tego psychola. Serce podbiegło mi do gardła, byłem mokry od potu i oczekiwałem na makabryczny finał. \\
Piła piszczała i kręciła się z niesamowitą prędkością nieprzerwanie, z czego Jonathan był widocznie zadowolony. Przysunął ją blisko twarzy młodzieniaszka, który wyrywał się do tyłu.
\sx Dalej! -- wrzasnął młodemu do ucha.
\xx Barry Jefferson!
\qd
Wszyscy zaniemówiliśmy ze zdziwienia.
\sx Nasz Barry? -- zapytał z niedowierzaniem Izaak.
\xx Tak, wasz Barry, teraz mnie puśćcie, bo\3k
\qd
Lenny odwrócił głowę, Izaak zrobił to samo, jednocześnie odwracając mój łeb do tyłu.
Zdążyłem zauważyć, jak Jonathan błyskawicznie wyjmuje zza pazuchy obrzyna i wypala z niego prosto w twarz Johnsona. Zaszumiało mi w uszach od echa wystrzału, pełen wdzięczności do Izaaka, że uchronił mnie od zmywania krwi z czoła, spojrzałem na Jonathana. \\
Nie opuścił jeszcze lufy znad zmiecionej głowy -- wpatrywał się w nią tępym wzrokiem i nagle upuścił obrzyna z rąk. Lewa ręka zaczęła mu się panicznie trząść, wbił się nią dosłownie w kieszeń i wyjął pudełko tabletek. Odbijały się one o plastikową powierzchnię opakowania, dopóki Masterton nie wyrwał wieczka i nie wsypał sobie zawartości do ust. Na oko tabletek było z dziesięć, podłużnych, o długości około sześciu milimetrów. Pochylił się nisko, zaczął kaszleć i połykać co chwilę po jednej tabletce, z czym miał wyraźne trudności. Kiedy skończył, wstał i odwrócił się, ukazując twarz całą zakrwawioną. Rozejrzał się po swoim ,,dziele'' -- drobinach mózgu i czaszki.
\sx Mówiłem, że opuścisz to miejsce w kawałkach -- zwrócił się do miejsca, gdzie niegdyś dumnie widniała głowa. -- Na oko będzie ich z 5 tysięcy.
\qd
Lenny wyjął z kieszeni utleniacz, ręczniki i folię.
\sx Nie, nie dzisiaj\3k -- powiedział Psychol.
\xx To po ch*j mu rozwaliłeś łeb? Nie mogłeś po prostu mu skręcić karku, czy coś w tym stylu? Ku czemu zmasakrowałeś mu głowę?!
\xx Ku przestrodze. -- odpowiedział z uśmiechem. Spojrzał na mnie, nieszczęśnie wyglądającego z otwartymi w zdumieniu ustami. -- Ruszaj się, nie mamy całego dnia!
\qd
W trakcie drogi powrotnej mieli wyjątkowe szczęście -- nie spotkali ani wojskowego patrolu, ani grupy bandytów, nawet pojedynczego mutanta. Szybkim krokiem, zręcznie omijając anomalie, wkroczyli do obozu bezpieki. Lenny oraz Izaak pożegnali się z Markiem i Jonathanem, po czym udali się do swych biur. Pozostała dwójka miała umówione spotkanie z Mikołajem i Leonem w pubie.
%
\sx Nie mam czasu na szczegóły, po prostu o siódmej wieczorem macie być w barze. -- powiedział przez telefon Masterton, będąc przy bramie.
\qd
Kiedy przechodziliśmy koło budki strażniczej, strażnik wstał z krzesła i spojrzał zdziwiony na zakrwawionego Jonathana, miał już coś powiedzieć, ale nie zdążył.
%
\sx Ani słowa\3k -- rzekł pod nosem Psychol.
\qd
Strażnik uniósł ręce w przepraszającym geście i usadził się z powrotem na krześle. \\
Po wejściu do obozu, Jonathan szybko pobiegł do swojego pomieszczenia, ściągając na siebie spojrzenia okolicznych ludzi. Będąc w środku błyskawicznie wziął prysznic i przebrał się w czyste ubrania -- znowu garnitur, tylko tym razem szary, wraz z szarymi spodniami. \\
Teraz wraz z Markiem siedział przy stole w kącie. Dochodziła siódma, o tej porze było w pubie dość tłoczno -- większość stalkerów wracała wczesnym wieczorem, tylko nieliczni włóczyli się po Zonie w późnych godzinach. Masterton zamówił 3 kieliszki wódki, jego towarzysz piwo. Jonathan chwycił kieliszek, błyskawicznie wlał jego zawartość do ust i ze stuknięciem odstawił, to samo zrobił z dwoma następnymi, po czym przełknął. Cicho syknął i zwrócił się do Marka.
%
\sx Co dokładnie masz zrobić z tym szpitalem, co?
\qd
Rozmówca łyknął piwa.
\sx Przyjść. Zabezpieczyć. Zadzwonić po naukowców. -- odpowiedział. \qd
Masterton pokiwał głową.
\sx Widocznie nie zrozumiałeś pytania. Pytałem, co masz DOKŁADNIE -- zaakcentował silnie -- tam zrobić. \qd
Mark zrozumiał, że facet nie odpuści.
\sx Zbadać piwnicę. \qd
%
\ro{8}
%
Więc to on. \\
Jemu powierzono jedną z największych tajemnic Zony. \\
Spojrzałem na niego wymownie. \\
Ktoś, kto nie wytrzymuje widoku rozwalonego łba, ma być trzecią osobą, która pozna, co jest w piwnicy miejscowego szpitala?!
Największe w Zonie skupisko elektro to parter kliniki. W miejscu wejścia do piwnicy, dziewięć różnych anomalii w jednym punkcie, w jednym cholernym metrze kwadratowym znajduje się 9 anomalii! Nałożone na siebie, tworzą niesamowity widok.
W piwnicy na sto procent znajdowały się tylko 2 rzeczy -- zwłoki pewnej osoby ( teraz już pewnie szkielet ) oraz artefakt znany Pochłaniaczem. Trzecią, niepewną rzeczą, a raczej osobą, był wytwór owego artefaktu. \\
Jak mogli\3k \\
,,Przydzielimy naszego najlepszego człowieka, masz moje słowo!'' -- mówili mi tydzień po wybuchu, kilka godzin po dotarciu do Petersburgu.\\
Więc to jest ich najlepszy człowiek?\\
Pierwszy raz od 21 lat przypomniałem sobie tamten dzień\3k

Radek, Mikołaj i Leon wyszli z biura Barry’ego. Nie wiedzieli jeszcze, do czasu rozmowy w barze w Psycholem i Markiem, że ich szef był jednym z przywódców miejscowej grupy handlarzy organów. Zamykając drzwi, Leon odebrał telefon od Mastertona.
\sx Ta? Pod jego biurem. A o co chodzi? -- przerwa, Leon słuchał. -- Niech ci będzie\3k \qd
Schował telefon do kieszeni i oznajmił pozostałej dwójce:
\sx Idziemy do baru, Jonathan chce z nami mówić.
\xx Nie mógł wybrać sobie innego dnia na pogawędkę? -- zasmęcił Radek. -- Jestem wyczerpany, a jutro czeka mnie całonocne przesłuchanie kolejnego gnoja z 5.
\xx Widocznie nie. -- Leon wyprostował się i poszedł w stronę baru.
\qd
Radek ziewnął i ruszył za nim, idąc obok Mikołaja.
\sx Skurwysyn! -- Leon walnął pięścią w stół, przewracając dwa puste kieliszki. Gdyby nie szybka reakcja Marka, piwo wylałoby mu się na spodnie. -- Wiecie, co się stanie, jak inni się dowiedzą?!
\xx Anarchia -- Masterton.
\xx Chaos -- Mikołaj.
\xx To dopiero początek. -- Leon wsadził ręce do kieszeni. -- Paranoja. Skoro szef był w to zamieszany, powtarzam szef, to pomniejsi, mniej ważni członkowie obozu będą się podejrzewać na każdym kroku. Złe spojrzenie, w popłochu odebrany telefon będzie pretekstem, żeby kogoś rozwalić. Autorytet Barry’ego runie, nikt nie będzie czuł się bezpieczny. Pewnie wszyscy powyrzynamy się nawzajem\3k Jakieś pomysły?
\xx Zabić gnoja i powiesić jego łeb w miejscu publicznym. -- Jonathan wyrecytował swoją kwestię tak gładko, jakby obmyślał ją przez kilka minut.
\qd
Do rozmowy włączył się Mark.
\sx Nikt nie może się tego, prócz nas, dowiedzieć. Musimy się skontaktować z dowództwem w stolicy. Podstawią swojego człowieka, najbardziej zaufanego\3k
\xx Barry też miał być ,,zaufany'' -- zakpił Mikołaj.
\xx Drugi raz nie popełnią tego samego błędu. -- Mark kontynuował. -- Ich człowiek zajmie się tu dowodzeniem, a los Barry’ego spocznie w naszych rękach\3k
\qd
Lenny zapalił papierosa zapalniczką, zaciągnął się i po długiej chwili wypuścił dym z ust.
\sx Wyprowadźmy go po cichu z obozu i najlepiej, ku*wa, od razu wrzućmy do anomalii. -- zaproponował.
\xx Czy koniecznie musimy to robić? -- zapytał smutno Mark. \qd
Jonathan nachylił się do niego gwałtownie i rzekł głośno.
\sx Ziemia do Marka! To jest Zona, tu problemy rozwiązuje się tylko w jeden sposób, który przedstawiłem ci kilkadziesiąt minut temu, a nie\3k -- zawahał się, jakby miał trudności z powiedzeniem następnego słowa. -- czynami rodem z grzecznego komiksu, gdzie Ci Źli przeproszą i już na zawsze będą dobrzy. Jeśli masz opory, aby zabić byle gnoja, długo tu nie zagościsz. -- wyprostował się i oparł na krześle.
\xx Niekoniecznie\3k -- Izaak zamyślił się głęboko.
\xx Jak to?! -- zaprotestował Psychol. -- Chcesz się męczyć z namawianiem tego ścierwa, żeby nie wygadał innym, że dowiedzieliśmy się, w czym macza paluchy?
\qd
Mikołaj przejął inicjatywę.
\sx O Barrym wie wielu stalkerów, ale tylko nieliczni widzieli go na własne oczy. Z kim nigdy nie prowadziliśmy interesów?
\xx W\3k -- zaczął Lenny.
\xx \3kolność. -- dokończył Leon. -- Chcesz go zmusić do przyłączenia się do Wolności?!
\xx Jeśli go porządnie nastraszymy, nie puści farby. -- rzekł pomysłodawca.
\qd
Papieros Lenny’ego, spalony do filtra, wylądował w brązowej, porcelanowej popielniczce.
\sx Znam kilku gości z Wolności. Na pewno znajdzie się wśród nich co najmniej dwóch strażników. Jeśli będą go budzić codziennie rano i przypominać, że jeśli coś piśnie, skończy marnie, pewnie będzie siedział cicho. Mógłbym też załatwić kombinezon z ich oznaczeniami.
\xx Dobra, wiemy co zrobić z Barrym po wyprowadzeniu z obozu, ale jak go w ogóle z niego wywieziemy? -- Leon popadł w zadumę. -- Nie ma szans, żeby wyjść z nim przez główne drzwi, od razu zacznie wrzeszczeć na lewo i prawo, kim jest i tak dalej. Zamieszanie w jedynej siedzibie Służb w Strefie będzie mu, a raczej reszcie układu, tylko na rękę\3k Jakieś sugestie?
Mark wpadł na pomysł.
\xx Okno jego biura. Trochę ponad 5 metrów, pół metra do ogrodzenia, nie będzie problemów, żeby\3k Masterton i ja wejdziemy do środka i obezwładnimy Berry’ego. Leon, Mikołaj, Lenny, Izaak i Radek -- będziecie w ciężarówce, którą zaparkujecie tuż przy ogrodzeniu, od strony okna. Wezmę coś w rodzaju składanej kładki, kiedy Barry będzie nieprzytomny, otworzę okno, wyłożę kładkę i zrzucę z niej go wprost do ciężarówki.
Pojedziemy w jakieś ustronne miejsce, porządnie go nastraszymy\3k
\xx Ja się zajmę tym ,,zastraszeniem'' -- Masterton uśmiechnął się krzywo. -- Zrobię mu ,,głuchego''. \qd
Spojrzenia wszystkich, prócz Marka, dawały do zrozumienia, że wiedzą o co chodzi, więc zapytał się on, czym jest ,,głuchy''.
\sx Standardowa procedura -- piącha w ryło, kop w jaja, potem stawiam go przed anomalią, zawiązuję oczy, przystawiam lufę do tyłu głowy, ale wypalam cały magazynek koło ucha. Będzie miał farta, jak będzie w ogóle słyszał. W jego wieku, jest to mało prawdopodobne. Generalnie, mimo mocnego z pozoru charakteru i wysokiego stanowiska, żaden z niego twardziel. Wydzierał się na nas, bo gdybyśmy się postawili, ,,załatwił'' by nam posadę w jakimś podrzędnym więzieniu na zadupiu.
\xx Jedynym problemem są strażnicy na wieżach. -- głos Lenny’ego był wyraźnie niespokojny. -- Co z nimi?
\xx Proste. -- mruknął Masterton. -- Zróbmy w obozie jakieś zamieszanie, żeby się wami nie przejęli. Zawalę wieże, najlepiej posadzę jakieś małe bomby u ich podstaw. Strażnicy powinni być cali\3k Jak myślisz, moralisto? -- spojrzał wymownie na Marka. -- Dwie, góra trzy połamane kończyny to uczciwa cena za życie jednego śmiecia?
\qd
W odpowiedzi Mark wystawił mu środkowy palec, gapiąc się w jakiś kąt. \\
Jonathan uśmiechnął się pod nosem i kontynuował.
\sx Mogę też podłożyć okolicy małą bombę, do narobienia jeszcze większego hałasu, żeby nikt nie przejął się odjeżdżającą z ,,piskiem opon'' ciężarówką.
 Kiedy wysadzę ładunki na wieżach, wy w tym czasie obezwładnicie Barry’ego. Poczekajcie na drugi wybuch, wtedy zapakujecie do z okna do środka i pojedziemy siną w dal. Ale musicie uważać, bo jest tam tylko jedna droga, którą można przejechać przez drzewa, kiedy przetestuję karabin, powiem wam którędy jechać.
\xx Przetestujesz karabin? -- zagadnął Leon.
\xx Ta. -- odpowiedział rozmówca, dokańczając butelki alkoholu. -- Zaczaję się gdzieś, żeby mieć oko na bramę i zacznę walić w nią i okolice obozu, najlepiej pociskami zapalającymi. Nikomu nic się nie stanie\3k -- te słowa wyraźnie skierował do Marka. -- Po prostu niech pobiegają wokół, narobią szumu i chaosu ,i tyle.
\xx Co wy na to? Nie trafię w nikogo, tylko brama, może kilka drzew.
\xx Na pewno? \qd
Jonathan wstał, chwycił piwo Marka i wykonał gest picia, celowo przechylając butelkę za kołnierz, wylewając niego zawartości na podłogę. Po odstawieniu piwa na innym stole, ku zdziwieniu siedzących przy nim stalkerów, udał, że szuka krzesła, wymachując rękoma bez ładu i składu niczym niewidomy. Kiedy usiadł, uśmiechnął się.
\sx Masz mnie za ślepca? -- spytał. Gdy ucichły chichoty, Jonathan przeszedł do rzeczy.
\xx Jutro o tej samej porze go dorwiemy. Chodźcie, musimy znaleźć mi kryjówkę, a wam miejsce na bombę. \qd
%
\ro{9}
%
Trzy kwadranse później byli kilkaset metrów od obozu, na pasie lekkich, czasem wysokich, zasłaniających widok na obóz wniesień pokrytych gęstą, ciemnozieloną trawą -- dało się też spostrzec kilkanaście większych krzewów. Słońce zachodziło, rzucając wszędzie dookoła ciemnopomarańczową poświatę, niebo było błękitne i bezchmurne. \\
Licznik Geigera pykał powoli, dając znać, że promieniowanie jest w normie. Masterton, Leon i Mikołaj szli szybkim krokiem, rozglądając się na wszystkie strony. Pozostała trójka ,,wtajemniczonych'' załatwiała pozostały sprzęt -- paralizatory, chloroform, także ciężarówkę, dwa karabiny i mały ładunek wybuchowy, który mieli zakopać w ziemi tuż koło wschodniej części muru obozu. Psychol ubrał się w wzmocniony kombinezon w zielonych barwach maskujących i dźwigał na plecach skórzaną torbę. Miał w niej przerobionego M82 i siatkę maskującą, które miał zamiar przetestować, przed jutrzejszą akcją. \\
Jego towarzysze byli ubrani w kombinezony SEVA w kolorze czarno-zielonym i mieli przy sobie nic poza SIG’ami 550 z doczepionymi lunetami ACOG.
\sx Tutaj. -- Leon wskazał gęsty krzak, a raczej kilka krzaków rosnących blisko siebie. \qd
Łącznie miały one 3,5 metra długości oraz 2 metry szerokości, wysokie na ponad 140 CM.
\sx Idealnie\3k -- mruknął z zadowoleniem. \qd
Jonathan Położył torbę na ziemi, gniotąc źdźbła trawy, kucnął i wyjął zawartość swego bagażu. Najpierw wypakował ogromny, chroniący przed urazami mechanicznymi, pokrowiec na broń. Był bardzo długi, a sądząc po wyrazie twarzy Mastertona, dość ciężki. Pozostałe miejsce bagażu zajmowała mała, poręczna torba oraz zapakowana siatka maskująca w leśnym kolorze.
Najpierw rozpięto pokrowiec. \\
Oczom wszystkich ukazał się Barret M82 -- jeden z najpotężniejszych karabinów snajperskich znanych światu, o przerażającym kalibrze .50 Cala. Może miał swoje lata i do tej pory wypuszczono nowsze modele, ale nikt nie śmiał o tym wspomnieć. W Zonie nawet staromodne, niemal archaiczne AK ( można było spotkać tu jego najróżniejsze warianty ) sprawowało się wyśmienicie. Magazynek, luneta i dwójnóg ( złożony pod lufą ) były na swoim miejscu. \\
Z mniejszej torby Jonathan wyjął amunicję -- pociski przeciwpancerne umieszczone w specjalnym, usztywniającym każdy nabój opakowaniu. W magazynku znajdowała się amunicja zapalająca, Masterton wyjął cały magazynek i położył go bokiem na ziemi, po czym zaczął ładować amunicję przeciwpancerną. \\
Po napełnieniu i ponownym włożeniu magazynka, Psychol wyjął z pokrowca wielki, długi, czarny tłumik. Wiedział, że to niezbyt utrudnia namierzenie, skąd kto strzela, ale nie chciał po prostu zwrócić na siebie uwagi jakiegoś przechodnia. Zdjął dodatkowy kompensator i nakręcił w jego miejscu ,,tubkę'', bo tak można było tłumik ów nazwać.\\
Gdy broń była gotowa do użycia (nie licząc ustawienia parametrów w lunecie, ale tym zajmie się później). Jonathan przyłożył licznik do kępy krzaków -- promieniowanie było niewiele wyższe, a aby się upewnić, że w przyszłej kryjówce nie ma anomalii, rzucił w nią kilkakrotnie kilkunastoma małymi kamyczkami. Gdy nic się nie stało, wpełzł do środka, na razie bez kamuflażu. Jego zdrowe, ,,żywe'' oko i wyraz twarzy wyrażały zadowolenie.
\sx Podać ci siatkę? \qd
Masterton skrzywił się.
\sx Nie chce mi się z nią męczyć\3k w sumie to chciałem tylko postrzelać do ptaków czy coś\3k -- skinął głową na karabin. -- Podaj. -- rzucił do Leona.
\qd
Gdy Leon spełnił prośbę kolegi, rozejrzał się wokół. Z zachodu na wschód sunęły wielkie, ciemnogranatowe, niemal czarne deszczowe chmury. Słońce już prawie znikło za horyzontem i zaczęło się ściemniać. Zaczął wiać dość porywisty, zimny wiatr, wprawiający popłoch w roślinności. Trawy, drzewa i krzewy z szumem kołysały się w rytm wichru. Znajdowali się ponad kilometr od obozu, od strony jego głównego wejścia. Mieli stąd bardzo dobry widok na bramę, jak i kawałek obozu ponad nią, dzięki czemu mogli bezpiecznie pokryć środek siedziby niewielkimi ogniskami przy pomocy pocisków zapalających. \\
Masterton przyłożył karabin do siebie, wcześniej jednak załadował magazynek ze zwykłą, pełno płaszczową amunicją, po czym skierował go w okolice bramy obozu i zaczął ustawiać odległość w lunecie.
%
\sx Ile?
\xx Jeśli oddasz i zużyjesz do dwóch głowic, to za darmo. O chloroformie nie wspomnę, mam go całe tony. \qd
Mark zapakował paralizator i butelkę chloroformu do plecaka i zapytał Norberta ( jeden z wielu handlarzy w Zonie, w tym przypadku ważnego dostawcę Powinności, o coś jeszcze.
\sx A materiały wybuchowe?
\xx Mam trochę Semtexu. \qd
Mark pomyślał przez dłuższą chwilę. Semtex był potężny, ale niewielkie ładunki powinny spisać się nieźle.
\sx Daj mi z 5 kilo i to byłoby na tyle. \qd
%
\ro{10}
%
Minął dzień po nieudanej próbie zabicia dwóch pewnych stalkerów. \\
John przekręcił klucz w białych, odłażących z farby drzwiach, przekręcił okrągłą, zaśniedziałą klamkę i wkroczył do swego pokoju w Prypeci. \\
Zwykły, czteropiętrowy blok mieszkalny, z pięcioma klatkami schodowymi -- z każdej z nich można było się dostać do dziesięciu mieszkań. \\
John zamknął drzwi na klucz, ściągnął przewieszonego przez ramię czarnego L96 i odłożył go na stojaku. Zdjął buty oraz kombinezon, zostawiając na sobie jedynie dresowe spodnie i czarną podkoszulkę. Plecak z artefaktami został rzucony na zielony, obfity fotel. \\
Mieszkanie składało się z przedpokoju i 2 pomniejszych pomieszczeń. Przedpokój miał kształt litery L -- w jej lewym dolnym rogu znajdował się skręt do małego pokoiku, w którym Finn urządził magazyn broni, artefaktów, pancerzy i amunicji. Przy samym wejściu do środka, na prawo, urządził niewielki salon. \\
John miał do swojej dyspozycji 3 na 3 metry kwadratowe. Pod wschodnią ścianą postawiono czerwoną, obitą skórą, śliską kanapę, pod przeciwną stał 24-ro calowy telewizor z wystającą anteną, opartą o ścianę.
Same ściany, wraz z sufitem, miały biały, wypłowiały kolor. Wyłożenie dywanu ( w tym przypadku zielonego, a był to raczej rozwijany kawał płótna, niż dywan ) oraz sprzętu zapewniającego codzienne wygody a malowanie ścian w mieszkaniu, które od kilkunastu lat nie nikt nie odwiedzał, to co innego. Na ścianach nie wisiało nic, poza SPAS’em 15 ( umieszczonym na wwierconych uchwytach, rodem z garażu dla majsterkowicza ) z pełnym magazynkiem. Broń ta nie rozpowszechniła się zbytnio ze względu na potężny, niemal nieludzki odrzut, ale John F. był nieludzki, nie tylko pod względem nadnaturalnej siły. \\
Wszelkie potrzeby fizjologiczne załatwiał zwykle w toalecie w sąsiednim mieszkaniu obok, a czasem najzwyczajniej w świecie odlewał się na klatce schodowej. \\
Teraz siedział na kanapie i pił herbatę. Zrobił nią w kubku przy pomocy grzałki, którą zasilał artefaktem. Skakał po programach w telewizji, na większości z nich trąbiono o Mryńsku i wybuchu w nowo otwartej elektrowni atomowej. Na razie nie było wiadomo nic o promieniowaniu, albo też nikt nie chciał podać tego do publicznej wiadomości. \\
,,Bardzo dobrze.'' -- pomyślał.\\
Zadzwonił telefon. \\
John odebrał i słusznie spodziewał się najgorszego. W słuchawce rozległ się aż za dobrze znany mu głos.
\sx Leon i Mikołaj?
-Jeszcze ży\3k -- John nie dokończył, przez sekundę przemknęła mu myśl ,,Tylko nie to\3k''
\qd
Przez lewą rękę ( trzymał w niej kubek z niemal wrzącą herbatą ) Johna przeszedł paraliżujący ból, który zmusił go do wylania napoju prosto w krocze. Spodnie niewiele pomogły, Johna zapiekło tak mocno, że ledwie powstrzymał się od panicznego wrzasku. Zamiast tego zacisnął zęby, oblewając twarz krwiście czerwonym rumieńcem. Prawa ręka została zmuszona do zaciśnięcia jej palców na telefonie, również wywołując ból. Komórka ustawiła się na tryb głośnika i dobiegł z niej przeszywający, jeżący włos na głowie ton.
\sx Jutro o dwudziestej. Przy drzewie w Sali gimnastycznej. Blok 9A, klatka C, mieszkanie dziesiąte, lewe okno salonu. I przestań się przechwalać, czegoś to byś nie wytrzymał. Niszczysz garnizony, a wystarczy wrzątek i\3k po tobie. Nie jesteś jedyny -- gdyby Susarro się ujawnił, nie był byś godzien pucować mu butów. Za miesiąc wyruszasz do Mryńska.
\qd
Chwilę później w pokoju rozległ się dźwięk przerywanego połączenia. Ból ustał, a John ruszył do łazienki ( wybiegł z mieszkania i barkiem wparował do sąsiedniego apartamentu -- jego łazienka została z nieznanych przyczyn zamurowana ) by się przebrać i opatrzyć poparzenie.
\sx Jutro o dwudziestej\3k -- powtórzył, ściągając spodnie. \qd

\sx Jutro o ósmej wieczorem. -- oznajmił Masterton.
\qd
Siedział z Lennym, Leonem, Mikołajem, Markiem i Izaakiem w swojej kwaterze w obozie. Dochodziła jedenasta w nocy, obóz ucichł, bar opustoszał, a niebo poczerniało, tam, gdzie widoku nie zasłaniały chmury, świeciły gwiazdy. Blask księżyca był nikły z powodu chmur -- było to samo pasmo, które ponad dwie i pół godziny dostrzegł Leon, kiedy Masterton testował karabin. \\
Pomieszczenie miało sześć na dziesięć metrów. Z lewej strony widniało duże okno, w tej chwili zasłonięte grubą, nieprzepuszczającą światło, ciemnozieloną firaną. W lewym górnym roku stało piętrowe łóżko, naprzeciwko niego widniał mały telewizor, ustawiony na półce. \\
Wzdłuż ścian ustawiono pięć foteli z drewnianymi, prostymi oparciami. Wszystkie były zajęte -- tylko Mark siedział oparty o łóżko.
\sx Plan jest jasny, mamy wszystko co potrzeba? -- zapytał Jonathan. -- Paralizatory?
\xx Są. -- potwierdził Mark.
\xx Chloroform?
\xx Mamy.
\xx Broń?
\xx Też.
\xx A czym wysadzimy wieże?
\xx Mam już odpowiednie porcje Semtexu, wystarczy wsadzić detonatory.
\xx A jak zamierzasz je podłożyć? -- rzekł Leon.
\xx Otoczę kamieniami i tyle. Po dwa na wieże -- po jednym na nogę.
\xx Dobra\3k Masterton pokazał nam, którędy jechać. Mamy też ustalone miejsce przekazania Barry’ego do Wolności. Gdzie podłożycie główny ładunek?
\xx Już jest podłożony. -- odezwał się Mikołaj. -- Na środku obozu, głęboko w ziemi.
\qd
Jonathan wrzasnął. \\
Wszyscy utkwili w nim spojrzenia. Psychol złapał obydwiema rękoma za głowę ( sztylet w kastecie ocierał się o włosy ) i kiwał nią w górę i w dół, na zmianę, niczym ofiara migreny. Krzyknął jeszcze raz, ale tym razem ciszej, po czym głęboko się skulił -- utkwił głowę pomiędzy ramiona i podkulił ku nim kolana. Oparty na samych biodrach, monotonnie się kiwał, wydobywając, długi, nieprzerwany, zduszony jęk. \\
Izaak podniósł się z krzesła, podszedł do Jonathana i położył mu rękę na ramieniu.
\sx Nic ci nie jest? -- zapytał z troską i obawą w głosie.
\qd
Nagle Masterton gwałtownie odchylił głowę do tyłu, uderzając nią o ścianę. \\
Dyszał przez chwilę i oczy ,,podjechały'' mu ku górze, ukazując jedynie białka.\\
Wszyscy aż odskoczyli ze strachu, jednak ten stan nie trwał długo -- po chwili zarówno na i martwym, jak i zdrowym oku, można było dostrzec już tęczówki.
%
\sx Bierzemy ze sobą kombinezony.
\xx Przecież nie wyruszamy tam w strojach kąpielowych! -- powiedział głośno Leon.
\qd
Masterton wciąż dysząc, mruknął z zirytowaniem.
\sx Wiem, ku*wa. Chodzi mi o kombinezony antyradiacyjne. \qd
Leon zdenerwował się jeszcze bardziej.
\sx A po cholerę nam one?! -- teraz już niemal krzyczał.
\qd
Do dyskusji włączył się Mikołaj.
\sx Spokojnie! -- uspokoił kłócącą się dwójkę. \qd
Zwrócił się do Mastertona.
\sx Po co nam antyradiacyjne, Jonathan?
\xx Po prostu je weźmy. Nie mamy w nich chodzić non-stop. Ale czuję, że wydarzy się coś, co zmusi nas do ich użycia. Promieniowanie będzie rosnąć\3k wybuch\3k
\qd
Zaczął kompletnie bełkotać, mówiąc bez ładu i składu.
\sx Czy to objaw jego\3k -- Mark się zamyślił. -- ,,Choroby''?
\xx Nie. -- Izaak udzielił błyskawicznej odpowiedzi. -- To nie to. E! Ocknij się, człowieku! -- wielka ręka wystrzeliła ku twarzy Mastertona, zatrzymując się na prawym policzku z głośnym plaśnięciem.
\qd
Trafiony sapnął krótko i zaczął rozglądać się na lewo i prawo, jakby nie wiedział gdzie jest.
\sx Po prostu je weźmy. Coś się wydarzy, i będzie to początek końca tego miejsca. \qd
%
\ro{11}
%
Tej nocy Jonathan Masterton alias Psychol spał niespokojnie. Od czasu wieczornej rozmowy z znajomymi z obozu, powrócił jego najgorszy koszmar. Od dnia śmierci jego najlepszego przyjaciela, nigdy nie ujrzał snu z Adrianem. Został przez niego zastąpiony. \\
Aż do dziś, 21 Marca 2001 roku, Jonathan miał szczęście nie doświadczając swej najgorszej traumy. Każde, bez wyjątku, pojawienie się niej, zostawiało w psychice Jonathana głębokie, nieusuwalne piętno. To jak z poezją starożytnych -- stworzone, poczęte kiedyś i trwające do dzisiaj. Może kształty planet zmieniały się, ale trwało to setki lat -- Jonathanowi na odwrócenie zmian zaszłych w jego duszy zostało ich ponad 30.
Niektóre części snu uległy zatarciu przez podświadomość. Być może chciała ona nie dawać świadomości okazji do dalszego samozniszczenia, w pewnym sensie osiągnęło to swój skutek. Masterton pamiętał dobrze każdy szczegół, ale nic by tak nim nie wstrząsnęło, jak pełne przypomnienie sobie 1978 rocznika. \\
Tło było oślepiająco białe i bezkresne -- istniał tylko Masterton (widział obraz własnymi oczyma, nie wiedział, jak wygląda, był pewien tylko swego 15-to letniego wieku) z Adrianem. Był on wysoki na około 170 CM, był barczysty, umięśniony i miał nieprzyjemne, wredne spojrzenie. Tego dnia ubrał się w ciemnoszary dres, ciemne spodnie, adidasy, do tego zgolił się na łyso. Uśmiechnął się do Jonathana i zniknął. Sam Masterton zaś poczuł, że się cofa. \\
Do nieograniczonej bieli dołączył blok mieszkalny -- typowy na tutejszym osiedlu w Prypeci. Jonathan nie zdziwił się, gdy nie z własnej woli właśnie wychodził z klatki schodowej. W śnie ludzie byli obserwatorami, mało kto mógł na nie wpływać. Jonathan dowiedział się, jak jest ubrany oraz ile ma lat -- nosił sięgające kolan spodenki, śnieżnobiałą koszulkę, był 15-to latkiem. \\
Na koszuli zaczęły kwitnąć ciemne krople. \\
Masterton jęknął z bólu. Sprawiło to, że obserwującym go stalkerom zrobiło się go żal.
\sx Budzimy go? -- spytał Leon.
\xx Nie. Sam sobie poradzi. -- zapewnił Izaak. \qd
%
\ro{12}
%
Deszcz.\\
Tego felernego dnia, kiedy zaprzepaścił sobie życie (chociaż do 2003 nie będzie tego świadom) padał ulewny deszcz. Gospodarz snu stał pod blokiem, rozglądając się na prawo i lewo. Czuł krople deszczu, które chlustały go po twarzy, był nawet mokry, ale nie widział niczego poza bielą, blokiem, i sobą. \\
Z daleka, z potoku bieli, niczym z mgły, wyłonił się Adrian. Ubrany tak samo jak wcześniej, wykonał ten sam uśmiech.
\sx No witam. -- rzekł, zmieniając uśmiech na szyderczy. \qd
Jonathan poczuł jakiś metal w prawej dłoni. Pomiędzy dwoma palcami. \\
Uniósł rękę ku twarzy i spostrzegł kastet z wystającym sztyletem. Był nowy,
lśnił, a otaczająca biel jeszcze bardziej podkreślała jego upiorny, surowy połysk.
Nagle Mastertonem wstrząsnęły gwałtowne emocje. Przez moment poczuł się, jakby dowiedział się tego, co robił przez ostatnie 2 godziny, o których zapomniał z powodu amnezji. Wszystko wokół znowu się zmieniło -- posiadacz kasteto-sztyletu w sekundę zmienił pozę. Był nisko zgarbiony, ręce miał spuszczone swobodnie ku ziemi, patrzył na Adriana spode łba, tak jak nigdy. Mimo swego wieku wyglądał dorośle, a obecnym spojrzeniem złamałby niejednego dojrzałego mężczyznę. Nie patrzył w ten sposób na Adriana bez powodu.
Sekundę później poczuł w sobie wielką, ogromną wypełniającą całe jego ciało, niezdolną do zatrzymania, żądzę mordu.

Wszyscy patrzeli na Jonathana przerażonym spojrzeniem. \\
Łóżko, na którym leżał, było mokre od potu, którego zapach wypełniał całe pomieszczenie. \\
Widząc jego cierpienia, Leon musiał zareagować.
\sx Zbudźmy go, Izaak. Nie mogę na to patrzeć.
\xx Wiesz, co mu się śni, prawda? -- głos Mikołaja miał zatroskany i zaniepokojony ton.
\xx Nie mamy pewności. -- Izaak zamyślił się głęboko.
\qd
Mark nie wytrzymał.
\sx Koniec tego! -- wrzasnął, aż wszyscy podskoczyli.
\qd
Zerwał się z fotela (byli w kwaterze Mastertona, od dwudziestu minut wszystkie fotele ustawiono pod łóżkiem) i podniósł prawą rękę, bu uderzyć nią śpiącego w twarz. Równocześnie z jego, w powietrze wystrzeliła prawa dłoń Mastertona. Z nałożonym nań kastetem, wraz z złowieszczo błyszczącym ostrzem.
Kończyny trafiły się nawzajem. \\
Nikt nie potrafił określić tego dźwięku jednym słowem. \\
Charakterystyczny odgłos przebijanej skóry i krwi, która ochlapała ubranie Marka. \\
Ze środka dłoni wystawało zakrwawione ostrze, wokół którego powoli płynęła czerwona posoka. Wyraz twarzy Jonathana mówił, że jeszcze się nie zbudził -- Marka, że jest w szoku. Zaczął się trząść, a wraz z nim przebita dłoń, której kość wydawała upiorny, cichy zgrzyt, ocierając się o metal. Nie czuł jeszcze bólu -- szok był zbyt silny i przesłaniał wszystkie inne uczucia. \\
Błyskawicznie wyciągnął poranioną dłoń, ponownie wydając nieprzyjemny dla ucha dźwięk. Trysnęła kolejna porcja krwi i Mark wrzasnął przeraźliwie. Ręka Mastertona wciął sterczała w powietrzu, a jej właściciel trwał już nie we śnie, lecz w transie. Mark pojękując i trzymając rękę wysoko, doskoczył do wiszącej szafki, potężnym uderzeniem pięścią otworzył ją, po czym wydobył z niej silny lek przeciwbólowy oraz apteczkę. \\
Reszta patrzyła to na niego, to na Mastertona z sztywną kończyną, na zmianę, wciąż nie mogąc nadziwić się zaistniałej sytuacji. Leżący sprawca uszczerbku na ręku Marka uciszył się, opuścił ręce wzdłuż boku i otworzył oczy. Spocił się jeszcze bardziej -- twarz dosłownie błyszczała w jasnożółtym świetle żarówki. Miał też krótki, gwałtowny oddech. \\
Podniósł się, jak gdyby nigdy nic, spojrzał na zakrwawiony kastet przerażonymi oczyma.
\sx Co się stało? -- zapytał ledwie słyszalnym szeptem. \qd
%
,,Nie jestem ,,zaczepny'', raczej spokojny. Stary, przecież ja nikogo w życiu nawet nie uderzyłem! -- mówił, lecz nie wiedział, jak szybko to się zmieni. \\
,,Nigdy nic nie ukradłem\3k żal mi jest takich, co okradają rodziców dla marnego zarobku -- byle się upić, byle się upalić\3k'' -- mówił, lecz nie wiedział, że pierwsze morderstwo popełni będąc jeszcze nastolatkiem. \\
,,Będziesz wielki, świat jest twój i zdobędziesz go tylko dla siebie.'' -- mawiała mu rodzina, i on sam sobie. Nie widział, że zaledwie piętnaście lat po narodzinach, jego życie straci całkowity sens -- będzie karą, nie błogosławieństwem. Zwykłą udręką, którą będzie chciał jak najszybciej zakończyć.

Następny ranek był wietrzny i zimny -- mało kto wychodził ze swej kwatery, tudzież baru, chyba, że miał ku temu naprawdę ważny powód. Do tej wąskiej grupy zaliczał się Lenny -- musiał wyruszyć do Powinności po broń zamówioną dla Marka -- Walthera WA2000. \\
Przekazał Leonardowi torbę z pieniędzmi i przypomniał mu, że ma wrócić przed szóstą.
\sx Tak, mamo. -- odparł udawanym głosem małego dziecka wychodzącego na podwórko. -- Będę grzeczny i wrócę, zanim się ściemni.
\qd
Mark roześmiał się i życzył znajomemu powodzenia.
\sx A tak w ogóle. -- spytał ten drugi. -- Skąd do cholery masz tyle kasy? \qd
%
Nabywca drogiego karabinu zamyślił się głęboko. Po chwili ujrzał oczyma wyobraźni setki wielkich worków z drogocennym, białym proszkiem. Proszkiem, którego handlem i przemytem zajmował się przez większość swego życia.
\sx Odłożyłem sobie z poprzednich zajęć. -- Rozbawiła go ta myśl.
\qd
Procent pieniędzy, które posiadał, tych, które dzisiaj wydał na broń, mogłyby być równie dobrze spożytkowane w ramach papieru toaletowego. Handel narkotykami był niesamowicie, niewyobrażalnie dochodowy. Trzeba być człowiekiem z nie lada łbem, żeby chociaż wyobrazić sobie skalę zysków wynikających z tego samozwańczego rynku.
\sx No dobra\3k to ruszam\3k -- Lenny westchnął głęboko, jakby z obawą i niepokojem, po czym zszedł ze schodków prowadzących do pomieszczenia Marka, obrócił się na pięcie i szybkim chodem ruszył w kierunku bramy.
\qd
Ubrał typowy kombinezon, a pod spód kamizelkę. Z najbliższej leżącej w okolicy anomalii -- małej ,,Karuzeli'' położonej dziesięć minut drogi stąd, Lenny’ego miał do obozu Powinności zawieźć ich człowiek, a po wymianie odwieźć go z powrotem. Był więc bezpieczny. \\
Mark zaczął już się trząść od chłodnego powiewu. Ubrany w kurtkę i lekkie spodnie, w popłochu wrócił do ,,mieszkania'', zatrzaskując drewniane drzwi.
%
\ro{13}
%
Leon obudził się z momentem pierwszego blasku porannego słońca. \\
Jak zwykle (,,Niech to\3k'' -- pomyślał sobie) nie pamiętał, o czym śnił. \\
Wyciągnął się szeroko, ziewając i lewą dłonią oraz nogą zahaczając o ścianę. Spał w samej bieliźnie i było mu trochę zimno, przykrył więc szybko nagi tors kołdrą. Spojrzał zaspanym wzrokiem na kalendarz, wiszący tuż obok jego głowy, nad łożem.
\podro{22 Marzec. 2001 rok}
Trzydziestego będzie jego czwarta rocznica pobytu w Zonie.
Też mi, ku*wa, powód do świętowania\3k Powinienem tego dnia palnąć sobie w łeb na oczach wszystkich, ku przestrodze.
Leon jednak zbyt mocno szanował życie, by poświęcić je dla innych. Poza kilkoma, nielicznymi wyjątkami, generalnie nie wierzył w ludzi. Zdanie to wyrobił sobie to zbyt wielu niemiłych doświadczeniach. ,,Twa ofiara była próżna, Jezusie.'' -- mawiał niegdyś. Widząc, co niektórzy na tym świecie wyprawiają, męki na krzyżu wydały mu się bezcelowe. Czy był bezbożny? Na pewno łamał mnóstwo zasad Typowego Katolika. Pytać, czy agent służb specjalnych jest wierzący i przestrzega nauk Kościoła, jest równie mądre, jak pytanie Jimiego Hendrixa, czy miał kiedykolwiek w ręku gitarę.
Długopis, trzymany przez rękę Leona, powędrował w stronę kalendarza. Pod liczbą 22 Leon napisał niebieskim kolorem kilka słów.
,,Kolejny dzień w Zonie z kategorii posranych. Miłej zabawy, chłopie! I nie daj sobie odstrzelić dupy!''
Uśmiechnął się na myśl o własnym poczuciu humoru. Napisał coś pod dniem dwudziestym trzecim.
,,Jeśli przeżyję, będzie trzeba to opić. Zadzwonić do Pawła po kilka sześciopaków, i coś mocniejszego. Jeśli zaś nie, nagranie znajdą państwo pod deskami w podłodze mego pięciogwiazdkowego apartamentu. Miłego dnia.''
Spojrzał na ścienny, staromodny wahadłowy zegar. Dochodziła ósma rano. Masz dwanaście godzin do przekonania się, z jakiej jesteś gliny, Leon. Dwanaście godzin na przygotowania i zrobienie chyba najbardziej szalonej, zaraz po skoku w uran na dnie elektrowni, rzeczy w historii Zony. Co następne? Balansowanie w anomaliach? Próba morderstwa Johna F. przy pomocy zabawkowego pistoletu, z którego lufy wystawała chorągiewka z napisem ,,Boom!''?
Wstał z łóżka i doczłapał do szafy z ubraniami.

Mikołaj już o siódmej rano stał na nogach. Obudził się godzinę wcześniej, dwadzieścia minut poświęcił na ćwiczenia, ubranie się, zjedzenie śniadania i typowych nowo dziennych czynności. Po wyjściu z domu, został do niego niemal wepchnięty przez lodowaty wiatr. Z pewnością nie była to pogoda, podczas której można było spacerować w samej koszuli z długimi rękawami.
Po nałożeniu białego podkoszulka, grubego, czarno-pomarańczowego swetra, cieplejszych, czarnych, ocieplanych polarem spodni oraz ciemnej kurtki z kapturem, Mikołaj podbiegł do domu Mastertona. Wiatr był wyjątkowo nieprzyjemny i mówił wręcz ,,Wracaj do domu, póki się nie zaziębisz!''
Stanąwszy przed drzwiami kwatery Jonathana, Mikołaj nacisnął biały przycisk dzwonka. Stylizowany na ćwierkanie ptaka dźwięk nie zmusił właściciela do otwarcia drzwi. Mikołaj spodziewał się tego, wyjął więc zapasowe klucze. Mastertonowi nie podobało się, że może do niego wchodzić kiedykolwiek zechce, ale odkąd uratowano mu rękę tylko dzięki ,,wproszeniu się'', zmienił swe nastawienie.
Włożył kluczyk, przekręcił szybko dwa razy w lewo, lekko uchylił drzwi, wsunął się do środka, po czym powoli je zamknął.
Przedpokój w kształcie przybliżonym do sześcianu oraz dwie sztuki drzwi. Jedne na lewo, prowadzące do mini-kuchni oraz ubikacji, także prysznica, drugie do sypialni i czegoś, co na siłę można by nazwać salonem. Zanim Mikołaj zdążył dotknąć klamek tych drugich, usłyszał dźwięk tłuczonego szkła.
-Wiedziałem, ku*wa, że tak będzie! -- krzyknął, wywarzywszy drzwi barkiem.
Masterton, ubrany w bieliznę, także skarpety (białe) z narzuconą na siebie czarną marynarką, siedział na fotelu dosuniętym do stolika. Telewizor był włączony, grał jakiś stary film sensacyjny. Leżały pod nim drobiny kieliszka rzuconego przez Jonathana. Miał zapity wzrok, którym gapił się w pustą butelkę po wódce, trzymaną przez niego w lewej dłoni. Obracał ją na różne strony i podziwiał sposób, w jaki zniekształcała widok.
-Kompletnie pijany\3k -- mruknął Mikołaj, wyrywając przyjacielowi butelkę z ręki, po czym rzucił ją na łóżko. -- Jesteś napierdolony w trzy dupy, a o ósmej mamy robotę!
Pijany uniósł rozkołysaną głowę i spróbował skupić spojrzenie na stojącym nad nim człowiekiem. Na próżno -- gałki oczne błądziły na oślep, nie mogąc się chociażby na moment zatrzymać. ,,Kompletnie pijany\3k'' -- pomyślał w duchu Mikołaj. Z politowaniem, ale i troską.

O 10 rano John próbował przypomnieć sobie dane, które przekazano mu ostatniego dnia. Ładunek pojawi się o 19-tej przy drzewie, które wyrosło pośrodku sali gimnastycznej szkoły nieopodal. Anomalia wymagana do rzutu będzie\3k
Finn pomyślał przez chwilę\3k
Na razie pamiętał tylko, że będzie to w bloku położonego dość daleko od domu Johna, a dokładniej w salonie, pod lewym oknem. Cóż, został na to cały dzień. Prócz roboty zleconej mu przez Marvina, nie miał nic ważniejszego do roboty. Najlepiej było się po prostu gdzieś położyć i cały czas myśleć. Gdyby nie zdążył, była by to jego ostatnia okazja do jakichkolwiek czynności. Fakt, John Finn był silny, wręcz niezniszczalny, ale wczoraj, zaraz po poparzeniu go wrzątkiem, Marvin uświadomił mu pewną bardzo ważną rzecz -- gdyby Susarro się ujawnił, John nie miał by w Zonie czego szukać.

-Masterton? Pijany?! -- Mark schował twarz w dłoniach, wzdychając z załamanym tonem. Nie mógł uwierzyć, że Masterton się zwyczajnie upił. Musiał w akurat ten dzień?! Zresztą\3k może miał ku temu powody? Gówno prawda! Nawet byle frajer nie upija się, bo coś mu się przyśniło!
-A kiedy mu przejdzie? -- Leon był zniecierpliwiony i również podenerwowany z powodu zachowań Jonathana.
-Dobrze wiesz, jak ,,mocny'' -- ton przez chwilę zmienił się na drwiący. -- on ma łeb.
Niestety, wiedział. Warto było próbować sprowadzić go do kaca, by na kacu strzelał? Markowi niewiele brakowało, by się roześmiać. Oto ich najbardziej niebezpieczna misja, a udana tylko dzięki skacowanemu snajperowi!
Lenny zaraz po Jonathanie był pośród wszystkich ,,spiskowców'' najlepszym strzelcem, poza tym, jak wyniknęło ubiegło godzinnej rozmowy, miał on spore doświadczenie w używaniu Barreta. Drobna zmiana planów. -- stwierdził Mikołaj. -- I po krzyku.
-Izaak i Lenny -- zaczniecie podkładać bomby. -- oznajmił. -- Potem ty -- zwrócił się do Izaaka -- wsiadasz do ciężarówki, którą Radek będzie prowadził. Lenny do kryjówki i na mój znak zaczniesz strzelać. Ja, Mark i Leon zajmiemy się Jeffersonem. Wszystko jasne?
Grupowe skinięcie głową.
-Dobra\3k Radek. Zacznij obmyślać władczą gadkę, żeby Barry jeszcze bardziej narobił w portki.

Porcja Semtexu otoczona, wręcz oblepiona kamieniami, wypadła z kieszeni Izaaka tuż pod lewą nogę południowo-wschodniej wieży. Izaak dopalił papierosa, zgasił go o leżący pod nim kamień i przeszedł na środek obozu. Pod flagą stał Lenny.
\sx To już ostatnia. -- zakomunikował mu.
\xx Dobra\3k -- Lenny zarzucił plecak na ramię. -- Idę na stanowisko\3k
\xx Powodzenia.
\xx Taa\3k przyda mi się.
\qd
Południe oraz wczesny wieczór minęły spokojnie. Wszyscy, prócz Leonarda o 17 wybrali się do baru na piwo, obgadując ostatnie szczegóły. Mieli cały sprzęt przy sobie -- paralizator, chloroform, detonator oraz broń. Zsynchronizowali zegarki umówili się, że odpalą Semtex równo o 20. Za godzinę Radek miał pójść do ciężarówki -- zaparkowali ją w nocy.\\
Słońce chyliło się ku zachodowi, było ciepło i wilgotno, wiatr ustał więc rośliny, trawa i korony drzew trwały w bezruchu. Spora część obozu wybyła w głąb Zony. Ci którzy zostali -- Mark, Leon i cała reszta, była gotowa i z niecierpliwością oczekiwała ósmej. Chcieli to mieć za sobą -- oczekiwanie było dla nich udręką. Podobnie jak ostatnie losy Mastertona -- Mark czuł się niedoinformowany -- reszta zachowywała się, jakby znała przyczyny ostatniego zachowania Psychola. Czy znali się od dawna? Przyjaciele z dzieciństwa? Możliwe. Musiał się w końcu dowiedzieć, o co chodzi. Bycie ,,jedynym niewtajemniczonym'' od zawsze go denerwowało.
%
\podro{19:30}
%
John stał na klatce schodowej C bloku 9A na osiedlu położonym na południu Prypeci. Była odrapana z jakiejkolwiek farby, okna po lewej i prawej stronie zostały wybite, a drzwi wejściowych po prostu nie było. Połowę skrzynek pocztowych zerwano, ocalało ich jedynie pięć. Na ziemi leżało mnóstwo drobnego gruzu i szarego pyłu, oraz fragmentów porozbijanego szkła. Na zewnątrz było pusto i cicho. W promieniu kilometra znajdował się jeszcze jeden blok -- zbudowano go naprzeciwko tego, pod którym stał John. Lata świetlności miał dawno za sobą -- farba zeszła, okna były zdemolowane, nawet duże graffiti pokrywające wschodnią ścianę wypłowiało. Był przeraźliwie pusty i cichy, a myśl, że mieszkały tam dziesiątki ludzi, zawróciła Johnemu w głowie. ,,Takie rzeczy tylko w Zonie!'' -- pomyślał z rozbawieniem.
Balkon na wysokości czwartego piętra był zerwany, przez co można było dostrzec pomieszczenie, które niegdyś było kuchnią. Przestrzeń pomiędzy budynkami wypełniała droga -- na jej środku leżała zdemolowana, pordzewiała budka, pełniąca kilkanaście lat temu funkcję kiosku, a obok niej duży głaz. Na szosie walały się papiery, butelki i kamienie. Wzdłuż rosły 2 zaniedbane, obwisłe drzewa.\\
Powiał silny wiatr. Szkła na ulicy potoczyły się z donośnym dźwiękiem, zaszumiały leżące wokół papiery, podniosły się też tumany bladego kurzu. Niegdyś tętniąca życiem okolica, pełna rozkrzyczanych dzieci, dorosłych udających się do pracy oraz ich pociech wynoszących śmieci, teraz była opustoszała. Ani żywego ducha -- pustka, dobrze zachowane bloki pełne wspomnień, John F. i nic poza tym. Przynajmniej tak się wydawało, dopóki na ulicy nie rozległ się ryk. \\
Finn aż podskoczył i szybko wyrwał się z głębokiego zamyślenia. Ustalił źródło krzyku -- dobiegał on z pokoju bloku naprzeciw. Wiedział tylko, że na parterze, lecz nieznane mu było dokładne mieszkanie. I tak na niewiele by mu to się zdało, ponieważ twórca hałasu ujawnił się. Z położonego na prawo pokoju coś wyskoczyło. Otarło się to o wystające z ramy kawałki szkła, kalecząc się przy tym. Strużka krwi naznaczyła biały parapet.
Z okna wyłonił się mutant -- ewidentnie był to snork. Wylądował na plecach, lecz w sekundę zmienił pozę na typową dla tego gatunku stworów -- nisko pochylony, wręcz zgarbiony, poruszający się na nogach i dłoniach niczym pies. Ubrany w typową przeciwgazową maskę z urwanym przewodem, która zakłócała nierówne i nieprzerwane dyszenie poczwary. Zniekształcone i poranione, wręcz zniszczone ciało, okrywały pozostałości czarnego kombinezonu stalkera, niegdyś dumnego, teraz chroniącego jedyne resztki godności, które pozostały temu nieszczęśnikowi. Szkiełka maski były zaparowane i brudne, ukrywając ślepia. \\
Snork zwrócił się w stronę Johna, który stał tuż pod klatką schodową. Dzieliło ich ponad dwadzieścia metrów. Potwór podskoczył w miejscu i ruszył w stronę Finna. Krótkimi, szybkimi podskokami -- podczas każdego dyszenie stawało się głośniejsze, a rura maski stukała o asfalt. \\
Odległość odskoków rosła szybko -- przy czwartym snork pokonał odległość trzech metrów. Stalker czekał cierpliwie, lekko pochylony i z odchyloną prawą dłonią. \\
Odległość drastycznie się zmniejszyła, kiedy poczwara wybiła się przy pomocy większego głazu -- skok miał siedem metrów długości. Był zarazem ostatni, gdyż pomiędzy Johnem a snorkiem zostały już tylko cztery metry, ten drugi zaczął więc biec. Biegł, unosząc kończyny niczym pies, niczym lampart goniący ofiarę, poruszając się z wielką szybkością i, co zaskakujące, gracją. \\
Zatrzymał się nagle, uklęknął nisko i wzniósł się w powietrze z wystawionymi nogami, którymi miał zamiar powalić Johna. Całej scenie towarzyszył nieludzki ryk. \\
Finn ugiął kolana i wystającą dłonią sformowaną w pieść z wielką siłą (także Sprężyny, którą zawiesił sobie na pasie) grzmotnął snorka w sam środek jego maski. Rozległ się dźwięk łamanych kości i pękającego szkiełka, którego fragmenty wylądowały na drodze. Chlusnęła czerwona posoka i stwór uderzył w ścianę klatki schodowej, nie tylko z siłą uderzenia stalkera, lecz siłą jakiej użył do skoku. \\
Leżąc na brzuchu, wierzgał się na lewo i prawo -- nie dyszał już, lecz sapał, tonem pełnym wściekłości i upokorzenia. Z przekrzywionej osłony na twarz obfitym strumieniem płynęła krew. \\
John chwycił w prawą dłoń Berettę 81 i celnym strzałem w głowę skrócił agonię mutanta. Po włożeniu pistoletu z powrotem do kabury, poczuł w uszach ogłuszający pisk.\\
,,Co jest, do cholery?!'' -- przemknęło mu przez głowę.
Odgłos był tak mocny i intensywny, że John zapomniał o wszystkim innym, próbując go zniwelować. Okrycie uszu dłońmi nie dawało efektów. Po trzech sekundach przeraźliwego dźwięku Finna rozbolała już głowa. Momentalnie został on zastąpiony przez coś innego -- odgłos przypominający wybuch, gwałtowne tupnięcie dochodził z trzeciego piętra bloku 9A.\\
John wiedział co to. \\
Miotacz znalazł położenie.\\
Finn błyskawicznie chwycił dwa Elektro w ręce i nadludzko szybkim biegiem popędził do przepołowionej Sali gimnastycznej. Miał pięć minut, zanim Kula zniknie.
%
\ro{14}
%
\sx Za pół godziny zaczynamy\3k -- rzekł Leon po spojrzeniu na ręczny, srebrny zegarek, który błysnął w nikłym słońcu, kiedy zsunął się na niego rękaw. -- Za dziesięć idziemy do Barry’ego. Radek z Izaakiem są już w ciężarówce\3k -- spojrzał w ciemnoniebieskie, zachmurzone deszczowymi chmurami niebo. -- To czekanie mnie dobija\3k
\xx Trzy kwadranse i będzie po wszystkim. -- uspokoił go Mikołaj. -- Co z Mastertonem?
\xx Wciąż śpi, wciąż pijany, a ja wciąż jestem na niego wku*wiony. Chociaż\3kjeśli to naprawdę to, co myślę\3k
\xx Nie myśl. Wiedz. To na pewno\3k -- nie zdążył dokończyć, gdyż brutalnie przerwał mu Mark.
\xx Skończcie do cholery jasnej z tym ,,to''! O co wam chodzi? Co takiego Masterton zrobił, że tak mu odbiło? -- założył ręce na piersi i niecierpliwie czekał na odpowiedź.
\qd
W całym gronie zapadła głucha cisza. Wiatr przybrał na sile tak dużo, że Lenny się zachwiał.
\sx Powiedzieć mu? -- spytał Lenny. \qd
Reszta stalkerów pokiwała potwierdzająco głową.
Lenny nachylił się do ucha Marka i zaczął coś mówić. Cichym, słabo słyszalnym głosem, a Mark w pełnym skupieniu słuchał. Po minucie oczy rozszerzyły mu się w akcje przerażenia -- jęknął coś pod nosem i nasłuchiwał dalej, jakby wstrząśnięty i obawiający się dalszych informacji. Leonard swoją opowieść zakończył głośniejszym tonem, lecz wciąż nikt w okolicy nie miał szans dowiedzieć się, co dokładnie powiedział.\\
Mark stał przez chwilę zdziwiony, zszokowany, a może dotknięty tymi dwoma emocjami na raz. Myślał, próbował sobie wyobrazić, jak można sobie w tak głupi sposób zmarnować życie. Taki młody\3k
\sx Współczuję\3k -- mruknął. \qd

Wiatr smagał Johna silnie po twarzy. Pędził po opustoszałych ulicach Prypeci od ponad dwóch minut -- do Sali gimnastycznej zostało mu ponad dwieście metrów. Omijał kratery po wybuchach, kupy gruzu odłamane od okolicznych budynków, zardzewiałe budki, pręty, przystanki autobusowe. Ciemne niebo było ledwie widzialne zza deszczowych, szarych chmur, które po chwili zasłoniły także słońce. Prypeckie osiedle zostało ogarnięte przez lekki cień i półmrok -- Finn poczuł się nieswojo. Po chwili zauważył wielką wyrwę w ścianie, za którą znajdowała się stara hala do ćwiczeń uczniów pobliskiego gimnazjum. Wbiegł do niej tak szybko, że nie zdążył wyhamować, potknął się o wystający kamień i z krzykiem wpadł do środka pomieszczenia, łomocząc przy zderzeniu z podłożem.
\\
Sala miała dziesięć na dziesięć metrów, po lewej stronie dalej przyczepione były bale do wspinania się, pod nimi zaś stały dwie długie, drewniane ławki. Ściany były pokryte niegdyś beżową farbą, której mizerne, odpadłe resztki leżały na ziemi, ukazując surowe betonowe płyty, z których zbudowany został budynek. Oznaczenia boisk z desek dawno się starły, ocalała jedynie jedna, długa, czarna kreska, wyznaczająca środek. W tej sali także było duszno, głównie z powodu wiszącego w powietrzu brudu oraz wdzierającego się do nosa i gardła białego tynku oraz nieprzyjemnego, przyprawiającego o mdłości zapachu.\\ Wiejący na zewnątrz wiatr tylko częściowo docierał do wnętrza, a zderzając się z zniszczonym murem, świszczał przeraźliwie.
Z samego jej środka wyrastało niewielkie drzewko -- z niewielką ilością zielonych liści, dość zdrowe, można by powiedzieć, zadbane. Gałązki były szeroko rozpostarte i rzucały wokół chude, długie cienie. Gdy John wstał na równe nogi, zauważył na ścianie na prawo wyryte cyrklem serce i inicjały R.I + H.N.\\ Uśmiechnął się na myśl o szalejących tu lata temu dzieciach. W miejscu, gdzie zwykle nauczyciel WF’u przywykł kłaść piłkę, leżała Kula. John aż zaniemówił z wrażenia, mimo, że widział ją już niepierwszy raz. Mała, o średnicy maksymalnie siedmiu centymetrów, błyszczała nienaturalną, wręcz oślepiającą bielą. Co najciekawsze, deski, na której leżała, nie miały na sobie żadnych świetlnych refleksji. Kula pełna była blasku, którym jednak nie chciała się dzielić z otoczeniem. Nie wydawała żadnego dźwięku, nie poruszała się też -- trwała w deskach, niczym wryta, nie ruszając się o ani milimetr, nie wydając najmniejszego dźwięku. \\
Finn schylił się powoli i dotknął Kuli palcem wskazującym. W miejscu, w którym spoczął, biel ustąpiła całkowitej, smolistej czerni. Tak samo stało się, gdy została chwycona całą dłonią -- miejsca, które cokolwiek dotykało, były całkowicie czarne. Uniósł ją przed siebie, wpatrując się weń z podziwem. Nie wierzył, że taka potęga mieściła się w czymś tak małym.

Za dziesięć dwudziesta. Podczas, gdy John F. dotarł już, przedzierając się przez puste osiedla, do Miotacza, Mark zapukał do drzwi Barry’ego Jeffersona. Serce waliło mu jak młot, tak głośno, że zagłuszało (z punktu widzenia Marka) otoczenie. Niewiele brakowało do zapadnięcia na zewnątrz całkowitej ciemności -- mimo, że nie było wcale tak późno, wchłaniające wszelkie światło słońca chmury stwarzały wrażenie, iż jest około jedenastej w nocy. \\
Wiatr ucichł, zaś z niewiadomych przyczyn, poza trójką stojącą pod domkiem Jeffersona (Markowi towarzyszył Leon i Mikołaj, tak, jak się umawiali) na zewnątrz nie było nikogo. Bar za to aż pękał w szwach od naboru gości. Wrócili zmęczeni po ,,całym dniu pracy'' i zwyczajnie odpoczywają. -- pomyślał z tęsknotą Mikołaj, ściskając walizkę, w której trzymał kładkę. O niczym tak nie pragnę, jak usadowieniu się w barze i\3k \\
Drzwi uchyliły się -- otworzył je młody mężczyzna koło trzydziestki, z krótkimi czarnymi włosami. Ubrany był w bluzę i dżinsy, mający około 170 CM wzrostu.
\sx Słucham? -- spytał grzecznie. \qd
Mikołaj przestał myśleć o zimnym piwie w towarzystwie znajomych (a być może próbował sobie tylko wmówić, że tak robi, gdyż myśl ta była niezwykle pociągająca, szczególnie w obecnych okolicznościach) i wkroczył na pierwszy schodek prowadzący do kwatery szefa.
\sx My do Jeffersona. -- odpowiedział. -- A ty\3k
\xx Chciał dzisiaj wzmocnionej ochrony.
\xx On? Tutaj? A co miało niby miałoby mu się stać? -- odwrócił się, upewniwszy, że nikt go nie widzi.
\qd

Strażnicy mogli co najwyżej powiedzieć o ,,trzech mężczyznach w kapturach'' -- cały ubiór Leon z Mikołajem i Markiem mieli jednakowy, a żaden z nich nie miał niczego w stylu ,,cechy, po której poznam, że należy do tego i tego''. Nie widział jednak, ilu strażników jest w środku. Spokojnie\3k -- uspokoił się w duchu. -- Najpierw wejdźcie do środka, macie Rękę Kontrolera, więc w razie czego możecie jej użyć. \\
Czemu więc, do ch*ja, nie użyjecie jej na Jeffersonie, geniusze? -- durne alterego Mikołaja dało o sobie znać. Nic poważnego, pewnie jego podświadomość kazała wszystkie jego życiowe wątpliwości wyrażać w postaci drugiej tożsamości.
Bo współpracowników Jeffersona nie należy lekceważyć. Gdyby dorwali go w stanie ,,Chryste, gdzie ja jestem, kim ty i ja jestem?!'' zbadali by go na wszystkie możliwe sposoby i odkryli powód amnezji, a do tego odwrócili by jej efekt. -- odpowiedział alteregu\3k \\
Sobie? \\
Po czym myślał dalej, co robić. Myśl\3k\\
Chwila\3k znał tego gościa, który im otworzył. Jego skłonności do alkoholu były wszystkim znane. Na imię miał Irek, ale wszyscy nazywali go Pijaczyną. Tak, upozorowanie upojenia alkoholem to dobry pomysł. W przypadku Barry’ego na przeszkodzie stało by im co innego -- jego ciągłe zażywanie leków przeróżnej maści -- wliczając w to antybiotyki. Ręka Kontrolera nie interweniowała bezpośrednio w organizm, jednym słowem, była bezpieczna nawet dla staruszka, którego powiew dałby radę zamienić go w pył. Barry trzymał u siebie sporo alkoholu -- dla gości, sam nigdy nie pił. Upić strażników, użyć artefaktu, a do tego narozrabiać w biurze Barry’ego.
Pomyślą, że strażnicy zrobili mu awanturę po pijaku.
\sx W takim razie\3k -- kontynuował Mikołaj. -- Możemy wejść?
\xx Jasne\3k -- Irek uchylił drzwi szerzej, wpuszczając trzech przyszłych porywaczy do środka.
\qd
Gdy weszli do środka, zauważyli dwóch innych strażników -- jeden siedział na małym krzesełku pod zachodnią ścianą, przy drzwiach do toalety, drugi zanurzył się w wytartym, zielonym fotelu, stojącym na prawo od drzwi wejściowych. Sam znajdujący się na dole przedpokój miał niewiele ponad cztery i pół metra kwadratowego. \\
Mikołaj postawił walizkę przy drzwiach. \\
Ściany, podłoga oraz sufit były z czystego, jasnego drewna -- nie było go widać jedynie pod dwumetrowym, rozłożonym przed schodami na piętro, czerwonym dywanem. Irek usiadł zamknął drzwi i stanął pod nimi, trzymając dłoń na kaburze, z której wystawała rękojeść potężnego rewolweru. Broń pozostałych dwóch strażników oparta była o ich siedzenia -- po dwa AKM.
Leon i Mark nie wiedzieli, co począć. Czekali na jakiś znak od Mikołaja. Ten drugi podszedł do dwójki, wyjął paczkę papierosów\3k
\sx Można? -- odwrócił się i spytał pozostałych w pokoju. \qd
Strażnicy skinęli głowami na zgodę.
Wyciągnął z niej trzy papierosy -- podał każdemu po jednym i sięgnął do kieszeni po zapalniczkę. Mark z Leonem zbliżyli się z wyciągniętymi w dłoniach papierosami. Zapalniczka zapłonęła ogniem, i w momencie, w którym wszystkie papierosy były zapalone, Mikołaj powiedział cicho.
\sx Po trzecim ich załatwiamy. \qd
Oddalili się od siebie, paląc. Zaciągali się długo, a jeszcze dłużej trzymali dym w ustach. Mark zerknął na zegarek -- do dwudziestej zostało siedem minut. Nie wiedział, co później zamierza Mikołaj, ale ufał mu. Cokolwiek chciał zrobić, ufał, że wymyślił coś sensownego. Po drugim zaciągnięciu się Mikołaja, dwaj agenci zbliżyli się nieco do strażników. Sam Mikołaj stanął dziesięć centymetrów od Irka.
\sx Barry zdradzał wam jakieś szczegóły? Czuje się zaniepokojony, czy co?
\xx Nic na ten temat mi nie wiadomo\3k
\qd
Nastała ta chwila, w której Mikołaj zaciągnął się po raz trzeci. Długo, ,,przyspawając'' papierosa i jednocześnie patrząc prosto w oczy Irka. Odchylił się lekko do tyłu, i uśmiechając się krzywo, wypuścił dym przez nos. Zauważywszy to, Mark z Leonem zadziałali, lecz Mikołaj był pierwszy. \\
Z wielką szybkością uderzył Irka w żołądek -- tak mocno, że ten chwycił się za niego oburącz i zaczął kurczowo chwytać powietrze. Następnie wymierzył ,,haka'' w tył kolana, co zmusiło Irka do zgięcia się w prawą stronę, niemal zmuszając się do upadku. Został chwycony przez ramiona i szybko odwrócony plecami. Lewa ręka spoczęła na plecach, przyciskając Irka do drzwi, prawą zaś Mikołaj wyjął z kabury swojego Colta. Uderzył nim całą siłą w tył głowy strażnika, pozbawiając go przytomności. Lewa dłoń przytrzymała bezwładne ciało, by nie huknęło przy zetknięciu z deskami. \\
Leon rozwiązał sprawę podobnie, lecz nieco prościej. W lewej dłoni pojawił się nagle pistolet z długim, szarym, tłumikiem. Była to była broń ukraińskich agend bezpieczeństwa, prawdziwa spuścizna po ubiegłej epoce. Chwycony przez lufę, pistolet kilkakrotnie uderzył strażnika w głowę, który zdążył poderwać się na równe nogi i chwycić broń. W momencie, w którym sięgnął dłonią do bezpiecznika, Leon zaczął tłuc go kolbą pistoletu po twarzy. Każde uderzenie kolbą sprawiało, że ochroniarz słabł coraz bardziej i ,,wracał'' do pozycji siedzącej. Na szczęście dla porywaczy, nie zdążył odbezpieczyć kałasznikowa. Czwarte uderzenie wystarczyło -- siedzący stróż bezwładnie opadł głową w dół. Mark rozprawił się z ostatnim strażnikiem. w wyjątkowo prymitywny sposób. Wystarczył potężny cios w podbródek. \\
Palce ofiary Leona zesztywniały i wysunął się z nich karabin. Upadł z łomotem na ziemię, przez co w pokoju zapadła głucha cisza. Wszyscy nasłuchiwali, czy z góry dobiega jakiś odgłos. Nie było na co czekać -- może właśnie szykował się do ogłoszenia alarmu.
\sx Kończy nam się czas. -- powiedział zaniepokojony Leon.
\xx Na górę. Szybko -- odpowiedział Mikołaj krótko, chwytając neseser. -- Ja zajmuję się Barrym, wy bierzcie cały jego alkohol i\3k macie wlać go w tą trójkę. Mark, szykuj Dłoń Kontrolera.
\qd
Szybkim krokiem zaczął się wspinać po schodach. Przygotował paralizator.
Po wejściu na piętro, zapukał dwa razy do biura.
\sx Wejść! -- krzyknął Barry. \qd
Mikołaj powoli nacisnął klamkę, powoli otworzył drzwi (które zostawił uchylone) i wszedł do środka. Z wielką ulgą dostrzegł, że Jefferson siedzi na swoim krześle niewzruszony, niczego nie podejrzewając. Gabinet był mały, jego ściany były pokryte trofeami, w tym zakonserwowaną głową mięsacza oraz dzika.\\ Resztę wyposażenia stanowiło już tylko drewniane, ciemne biurko, za którym siedział Barry -- miał grubo ciosaną twarz, duży, wystający brzuch i siwiejące włosy. Był w wieku około pięćdziesięciu lat, reszta nie interesowała się dokładną datą jego narodzin. Kładka ukryta w walizce wylądowała na podłodze.
\sx Taa? -- spytał Barry niedbale, przekładając w rękach jakieś papiery.
\xx Zapłacisz za to. -- Mikołaj uniósł paralizator w górę i odpalił dwie głowice, które trafiły Jeffersona w brzuch.
\qd
Nacisnął, pełen gniewu, ale i wyczucia (serce dowódcy mogło nie wytrzymać zbyt dużego natężenia), przycisk zwiększający przepływ prądu. Barry krzyknął, dostając lekkich drgawek. Miał zamiar wrzasnąć mocniej, ale nie zdążył, gdyż chwilę później szmata nasączona chloroformem okryła jego nos i twarz.

Trzecie piętro. Mieszkanie dziewiąte. \\
Odrapane drzwi ledwo trzymały się w nawiasach -- pierwszy kop Johna od razu je wyważył. Jeszcze pięć minut. \\
Miotacz wirował zaraz przy lewym oknie, które w całości wyjęto. Pokój był niewielki, typowy w ,,tej okolicy'' -- zdemolowany i pełen duchów przeszłości. Drzwi na lewo i prawo -- prowadzące kolejno, do łazienki i kuchni, były zamknięte. Salon nie przekraczał swymi wymiarami pokoju Johna, był za to o wiele mniej zadbany i pełen drobnego gruzu, szeleszczącego przy każdym postawionym kroku. \\
W skrócie można go było opisać miniaturową wersją Karuzeli, z tą różnicą, że nie wciągał od do siebie niczego, poza specjalnymi przedmiotami. Takimi jak Kula.\\
Finn podszedł bliżej, odwinął rękaw i spojrzał na zegarek. Ponad trzy minuty. To będzie ostrzeżenie, a jeśli go nie posłuchają, Susarro i reszta sięgną do bardziej radykalnych środków.

Okno zaskrzypiało przeraźliwie, gdy Leon je otwierał. \\
Kładka stabilnie opierała się o framugę okna i wystawała dobre ponad pół metra za pokrytym drutem kolczastym wierzchu ogrodzenia. Barry na oko mieścił się w oknie -- na szczęście pozory się potwierdziły -- bez problemu wyszedł poza granice swego biura. \\
Mark trzymał kładkę z drugiej strony, by się nie przechyliła zbytnio w drugą stronę.
\sx Ile? -- spytał.
\xx Minuta. -- Mikołaj z Leonem opróżnili już wszelkie ,,zapasy'' Jeffersona i przenieśli trzech nieprzytomnych strażników na schody.
\qd
Zostali poddani już działaniu Dłoni Kontrolera -- nie pamiętali niczego z sprzed dwudziestu minut. Rozważali użycie jej na Barrym, by jeszcze bardziej ,,wtopił'' się w Wolność, postanowili jednak trochę jeszcze z niego wycisnąć. \\
Mieli całą frakcję po swojej stronie, nie mieli więc po co się spieszyć.
\sx Pół minuty\3k -- Leon chodził nerwowo w kółko. W końcu też będzie musiał wskoczyć razem z nimi do ciężarówki, a miał paniczny lęk wysokości. -- Co myślicie o tej radzie Mastertona, żeby zabrać skafandry?
\xx Też bym plótł takie bzdury, gdyby przypomniało by mi się coś podobnego. Nie jego wina\3k -- Mark mocno trzymał koniec rozkładanej deski, patrząc z tęsknotą w ręczny zegarek. Niech to już się skończy jak najszybciej, niech już będzie po wszystkim\3k -- powtarzał sobie w duchu niczym mantrę.
\qd
Kula wydała lekki stukot, gdy John położył ją na pokrytej brudem i gruzem podłodze. Do Miotacza z tego miejsca dzieliło ją ponad dwadzieścia centymetrów. Miała zostać rzucona trzydzieści sekund przed nastaniem dwudziestej, aby dać jej czas na pokonanie odległości. Finn przełknął nerwowo ślinę i skupił się, by nie stracić orientacji w czasie. Mimo, iż miał ustawiony alarm w zegarku, który aktualnie nosił w kieszeni, John wolał mieć pewność, że gdyby ten nie zadzwonił, pośle Kulę w odpowiednim czasie. Z nerwów zaczął się gwałtownie pocić, zaschło mu w gardle, a nogi lekko zwiotczały, do tego poczuł w żołądku nerwowy skurcz. Ten kto uważa, że najbardziej uczuciowym organem jest serce, powinien spróbować w życiu czegoś naprawdę śmiałego -- reakcja żołądka zmusiła by go do zmiany zdania. \\
Alarm w zegarku dał o sobie znać.\\
John jeszcze raz przełknął ślinę i przestał przytrzymywać Kulę dwoma palcami, a unosząc dłoń, popchnął ją palcem serdecznym ku Miotaczowi. \\
Jako, że nie była aktualnie dotykana przez żadnego człowieka, Kula lśniła swym niezachwianym, nie wydostającym się z niej, białym blaskiem. Cała scena przypominała obraz, w której Kulę stanowiła pomyłkę malarza, który niechcący chlapnął płótno białą farbą. Kontrast między toczącym się obiektem, a otoczeniem, był przeogromny. Kulając się, wydawała ona przytłumiony, suchy dźwięk. \\
Pięć centymetrów od ,,pola'' Miotacza, dwadzieścia sekund do godziny dwudziestej. Dwie tak małe, z pozoru nieistotne rzeczy, miały odmienić los co najmniej pięciu osób. Działanie Kuli to zaprzeczenie wszelkich praw fizyki, zasad związanych z chemią, atomistyką, matematyką, ogólnie rzecz biorąc, nauką. Nie miała ona tu nic do powiedzenia -- to cały urok Zony. \\
Miotacz pochwycił Kulę i zadziałał w typowy dla siebie sposób. Wchłonięty przedmiot uniósł się powoli w niewidocznym wirze powietrza, jednocześnie zataczając spirale. Gdy osiągnął wysokość dwudziestu, może trzydziestu centymetrów, wystrzelił bezgłośnie z prędkością i torem lotu odbitej piłki baseballowej. Po trzech sekundach prędkość zmieniła się gwałtownie -- Kula przyspieszyła tak bardzo, że dosłownie zniknęła Johnowi z oczu.
\swk[7em] -- Dziesięć\3k -- \qwk
Leon niemal się trząsł. \\
Oczekiwanie na wybuch czterech wież i upozorowany atak oboz nie dawały o sobie zapomnieć.
\swk -- Siedem\3k Sześć\3k Pięć\3k -- \qwk
głos z sekundy na sekundę napełniał się coraz większą paniką i niepewnością. Leon miał chęć krzyknąć coś w stylu ,,Już się nie bawię, odkręćcie to, ja idę do siebie, mam tego dosyć!''

Kula minęła Prypeć i Czerwony Las. Leciała jeszcze szybciej, a do obozu bezpieki dzieliło ją już tylko\3k
\swk -- Trzy\3k Dwa\3k Jeden\3k -- \qwk
Całą siłą skupioną w obydwóch dłoniach nacisnąłem przycisk detonatora. Wybuch semtexu, dźwięk łamanych desek, krzyczących ze strachu strażników oraz Jeffersona wpadającego do ciężarówki zagłuszyło coś innego. Wybuch, nie mający konkretnego źródła, niewyobrażalnie, wręcz ogłuszająco głośny, dochodził wprost z mojej głowy. Krótki, trwający dwie sekundy, jednak zmusił mnie do kurczowego dociśnięcia dłoni do głowy i uszu -- był tak potwornie donośny, że omal nie zemdlałem. Bębenki pulsowały rytmicznym, mocnym bólem, podobnie jak cała głowa. W tej samej chwili ujrzałem przez ułamek sekundy oślepiający, biały błysk. \\
Zalał on dosłownie wszystko, co widziałem -- wnętrze ciężarówki, nieprzeniknione, blisko rosnące korony drzew za oknem. Nic nie powstrzymało białego światła, które było tak silne, że sprawiało wrażenie, iż całe otoczenie spłonęło. Przenikało ono przez wszystko, niczym światło ogromnego reflektora przez skrawek najcieńszego na świecie pergaminu. Kolejny ułamek sekundy później, wszechogarniającą biel zastąpiła ciemność. Moje oczy chwilowo kompletnie straciły zdolność widzenia, uszy słyszenia, zdziwiłem się więc, że mogłem przynajmniej pomyśleć. Myśl nie była zbyt ważna, ale dowodziła, że przynajmniej pod względem umysłowym tajemnicza eksplozja mnie nie dosięgnęła.
\sx Co to, ku*wa, jest?! \qd
Leon, Mikołaj i Mark otrząsnęli się z skutków szoku po dwudziestu sekundach. Wciąż piszczało im w uszach, wciąż bolały ich niemiłosiernie głowy, ale mogli w końcu przejąć inicjatywę.
\sx Chryste panie! -- Mikołaj oparł się o biurko, chwytając powietrze niczym dusząca się ryba. -- Uran w elektrowni piznął, czy co?
\xx Nieważne, co to było. -- Leon trzymał się lewą dłonią na lewe ucho. -- Idziemy do ciężarówek! Szybko, zanim\3k zanim\3k -- stanął gwałtownie na równe nogi, szybko się pochylił i oparł ręce o kolana, po czym zwymiotował długo i obficie, wydając przy tym typowy dla tej czynności odgłos.
\qd
Wyciągnąwszy się jeszcze bardziej, pochylając twarz w dół, Leon zacharczał kilkakrotnie i zaczął kaszleć. Długie strugi śliny spływały mu z ust. W obozie zawył alarm. Dobrze znana wszystkim częstotliwość, której nie chcieli by usłyszeć za żadne skarby. \\
Dźwięk, którym straszono nowych nie wiedzących, że wybuch leżącego na dnie elektrowni uranu jest bardzo mało prawdopodobny. \\
Alarm mówiący o silnym wzroście promieniowania i skażenia. \\
Leon nie dał rady dalej stać -- ze stukotem opadł na pośladki, zderzając się z twardą podłogą. Zwrócił jeszcze raz -- głośniej, ale wypluwając mniej wymiocin. Po raz kolejny zacharczał, po czym głęboko wciągnął w powietrze w płuca. Wyglądał beznadziejnie -- z sekundy na sekundę zaczął przypominać ciężko chorego. Promieniowanie robiło swoje.
\sx Trzeba było posłuchać Mastertona\3k -- kolejne, głośne kaszlnięcie.
\qd
Alex siedział przy piwie z dwójką znajomych w barze. Było około ósmej wieczorem i w barze było aż ciasno od siedzących w nim ludzi. Panował harmider i hałas, wszyscy o czymś rozmawiali, obowiązkowo przy jakimkolwiek alkoholu. Alexa dobiegł głos puszczanego pawia.
\sx Ho ho ho! -- powiedział z dezaprobatą, odwróciwszy się przez ramię. -- Ktoś nieźle sobie gulnął, co chło\3k \qd
Poczuł ogromną falę nadpływających mdłości. Całe jego dobre samopoczucie nagle się ulotniło i zostało zastąpione przez osłabienie i ból głowy. Coś białego błysnęło mu przed oczyma, niczym światło skierowanego wprost na jego oczy reflektora. W głowie usłyszał przytłumiony, lecz i tak bardzo głośny dźwięk wybuchu. \\
Gdy rozległa się syrena alarmowa, w barze wybuchła panika.
%
\ro{15}
%
\podro{Rok 1977}
%
Za oknem pokoju czternastoletniego Jonathana Mastertona świeciło słońce. Rozświetlało ono wnętrze pokoju jasnym blaskiem, na co wpływ miało zdjęcie firanek -- były one w praniu. Okna niedawno umyto i lśniły one niczym szyby z reklamy środków czystości. Młody Jonathan mieszkał w czteropiętrowym bloku o literze D, niedaleko centrum, w trzypokojowym mieszkaniu nr 6. Z jego własnego pokoju miał widok na plac zabaw zbudowany zaraz pod blokiem -- składał się z kilku drabinek, trzymetrowej piaskownicy oraz dość wysokiej, blaszanej zjeżdżalni. \\
Cała przestrzeń ,,należąca'' do osiedla miała ponad dwieście metrów -- poza wcześniej omawianym placem, ustawiono tu cztery ławki (dwie obok siebie, cztery metry od piaskownicy, kolejne dwie pod balkonami bloku D, w odstępie dwudziestu metrów. Wszystko wokół porastała gęsta, nisko przycięta, zielona trawa. Przecinała ją betonowa, wąska ścieżka spacerowa o kształcie niskiej elipsy, przypominającej wręcz kwadrat. Wzdłuż niej stały jeszcze dwa budynki -- bloki A i C, blok B znajdował się na południe od tego, w którym mieszkał młody Masterton. Co kilka metrów przy ścieżce, jak i po całym osiedlu, rosły niskie, lecz obfite w liście drzewka. Jedyne w okolicy większa drzewo posadzono lata temu tuż przy klatce schodowej Jonathana -- miało ono dziesięć metrów wysokości, a gałęzie rosły tak szeroko, że można się było pod nimi schować w razie deszczu.

Był 21 czerwiec -- przedsionek wakacji, wszystkich uczniów okolicznych szkół zdążyła już ogarnąć euforia i wyjątkowo dobre poczucie humoru -- został tydzień do rozdania świadectw, oceny wystawiono, pogoda dopisywała, a każdy miał już ułożone plany na wakacje, których wcielenia w życie nie mogli się doczekać. Na niebie nie było ani jednej chmurki, słońce grzało przyjemnie, a nawet, kiedy komuś zrobiło się za gorąco, został natychmiast schłodzony lekkim powiewem. Ptaki ćwierkały radośnie, tworząc niemal sielankowy obraz całości. \\
Czterdziesto metrowe mieszkanie składało się z dwóch małych pokojów (jeden należał do Jonathana), sypialni, przedpokoju, sporego salonu, kuchni oraz łazienki. Pomieszczenie, w którym rezydował Masterton, miało dwa i pół metra kwadratowego, do jego wyposażenia należało: składane łóżko, wysokie biurko, telewizor i cztery, ustawione pod ścianą, wysokie szafy na ubrania, książki i różne inne rzeczy.\\ Wspominany telewizor był oparty o kredens, a biurko ustawiono w rogu pokoju. W pomieszczeniu rozlegały się dźwięki dobiegające z odbiornika wiadomości -- Jonathan już za młodu chciał znać wszystkie aktualności i ważniejsze wydarzenia, czasem denerwowało go nawet, gdy ktoś nie znał takich rzeczy jak nazwa którejś partii czy nazwiska obecnego Prezydenta. \\
Oglądanie reportażu o pewnym morderstwie na północy Ukrainy przerwało pukanie do drzwi. \\
Podniósł się powoli z łóżka (obecnie złożonego) i wolnym krokiem podszedł do malowanych na biało drzwi wejściowych. Spojrzał przez wizjer i ujrzał w nim szóstkę swoich przyjaciół. \\
Lenny\dotfill
Mikołaj\dotfill
Izaak\dotfill
Leon\dotfill
Radek\dotfill
Kamil\\
Wszyscy, prócz ostatniego, siedzieli na schodach. Byli ubrani niemal tak samo -- nosili krótkie spodnie sięgające kolan. Jedynym wyjątkiem był Mikołaj, który jak zwykle miał na sobie długie, czarne bojówki. Obecni przed mieszkaniem Mastertona przywdziali koszulki o różnych kolorach, których rękawy pokrywały jedynie kawałek ramion.\\
Jonathan został, jak za każdym razem, serdecznie przywitany przez swych znajomych. Pierwszy odezwał się Leon.
\sx W końcu wolne, co? -- spytał z uśmiechem. -- Idziemy po zakupy, idziesz z nami?
\qd
Masterton bez zastanowienia się zgodził -- wykonał jedynie gest mówiący ,,chwilkę\3k'', wszedł powrotem do mieszania, ubrał buty oraz zmienił spodnie. Chwycił klucze, znów wyszedł na klatkę schodową, zamknął drzwi wejściowe na cztery zamki i z całą ,,paczką'' udał się do Prypeci -- pięknej i tętniącej życiem, przynajmniej ówczesnych latach.

Jonathan z przyjaciółmi spotkał na klatce schodowej sąsiada z mieszkania nr 4 -- miłego starszego pana, zwykle broniącego całą miejscową ,,paczkę'' przed czepiającymi się wszystkiego (według Leona) sąsiadami. Harry, bo tak nazywał się 54-ro letni mężczyzna, mówił ,,Dajcie im się wyszaleć, zostało im niewiele lat beztroski!'' Leon mawiał wtedy (o reszcie, nie Harrym) coś dosadniejszego.
,,Banda idiotów'' -- takie miał zdanie o ponad połowie osiedla.
Kamil przytrzymał drzwi klatki schodowej, poczekał aż wszyscy wyjdą na zewnątrz, po czym je puścił -- dobrowolnie się zamknęły. Wiatr pomiędzy dwoma blokami był wyraźnie słabszy i jednocześnie chłodniejszy. Na Mikołaja i resztę padał długi, szeroki cień bloku B.
\sx Idziemy tamtędy.- Izaak wskazał drogę na prawo, do centrum miasta.
\qd
Ruszyli chodnikiem i wkrótce wyszli z cienia, na co Kamil zareagował westchnięciem pełnym ulgi. Nienawidził wszelkich przejawów chłodu, poza tym miał ostatnio problemy ze zdrowiem i ,,przewianie'' mogłoby mu zwyczajnie zaszkodzić.\\
Szli ramię w ramię po zatłoczonym chodniku, tuż pod innymi blokami. Po ulicy jeździło wiele samochodów, a z okien i balkonów okolicznych budynków wyglądali ludzie -- jedni ze sobą rozmawiali, inni wytrzepywali dywany lub pościel, jeszcze inni po prostu cieszyli się słoneczną pogodą i bezchmurnym niebem. O tej porze roiło się ono od ptaków wszelkiej maści -- między innymi wróbli, srok i gołębi, które chętnie, co jakiś czas, siadały na szczycie pobliskiej lampy. W końcu doszli na koniec ulicy, znajdując się w centrum. \\
Otoczony blokami mieszkalnymi i nie tylko, szeroki plac zajmował sporą część ulicy Kurczatowa. Pośrodku niego ciągnęły się dwie szosy, pomiędzy którymi zbudowano pewnego rodzaju alejkę -- chodnik z posadzonymi na lewo i prawo wysokimi drzewami, które otoczono równo przystrzyżonym trawnikiem. Chodnik po lewej i prawej stronie także pełen był trawy i drzew, stojących na tej samej wysokości, można by rzecz, w jednej linii. Najwięcej ludzi było na wyższych chodnikach -- tych, do których szło się po niskich schodach, tych, przy których znajdowało się najwięcej sklepów i salonów wszelkiej maści, głównie fryzjerskich. Dawało się wyczuć tutaj wakacyjną aurę -- wielu dorosłych ludzi wraz ze swymi pociechami już dziś udawało się na urlop, powierzając odebranie świadectw znajomym. \\
Nastolatkowie w grupach podobnych do tej, w której szedł Leon, także kłębili się po okolicy, rozmawiając na każdy temat, głównie jednak o czasie wolnym i sposobu uczczenia zakończenia roku nauki. Całe otoczenie wraz z święcącym radośnie słońcem skutecznie stwarzało poczucie sielanki.
%
\sx Chryste\3k aż nie mogę w to uwierzyć. -- westchnął Leon. -- Czy ktoś, poza Kamilem, jeszcze stąd na wakacje wyjeżdża? -- spytał.
\xx Prawdopodobnie jadę na wieś do połowy lipca\3k -- powiedział Mikołaj. -- Co ja tam będę, cholera, robił? -- mruknął z pretensją. \x Karmił świnie? Rąbał drewno do kominka? Będę odliczał tylko sekundy, aż wrócę tu, do was.
\qd
Wszystkim spodobała się ta myśl. Zawsze trzymali się razem, i nie dopuszczali do siebie myśli, że ich więź się kiedykolwiek zmieni. Na gorsze.
\sx I co chcecie tu robić przez całe dwa miesiące, co?
\xx Mamy basen, dom kultury, teatr\3k
\qd
Izaak skrzywił się na to słowo.
\sx Intelektualna nędza\3k -- mruknął Mikołaj, po czym zaczął wymieniać dalej. -- Mamy też kawiarenkę i stadion -- zajęć nam nie zabraknie. No i wesołe miasteczko\3k
\xx Poskaczmy w kulkach! -- krzyknął Lenny udawanym dziecięcym głosem. -- Mimo wszystko, nie widzą mi się dwa miechy siedzenia tutaj. Wyjedźmy gdzieś kilka razy, chociaż na jezioro. Nie wytrzymam tu dwóch tygodni bez konkretnego zajęcia.
\xx A czy\3k -- zaczął Masterton. -- Nie marzyło ci się nigdy usiąść gdzieś poza domem, w wieczór, najlepiej w namiocie, i całą noc, przy świetle świecy, grać w karty albo po prostu porozmawiać? Tematów nam na pewno nie zabraknie.
\xx Masz dziwne wyobrażenia.
\xx Cieszę się drobnymi rzeczami, po prostu nie wymagam wycieczki do Watykanu, żeby być szczęśliwym.\
\qd
Jonathan do marca przyszłego roku jest wierzący i dość uduchowiony, w przeciwieństwie do jego kolegów. Mimo wczesnej śmierci ojca, Jonathan miał wciąż pozytywne nastawienie do życia. Miał je wkrótce zmienić, ale póki co, pozostawał wielkim optymistą.
\sx A tak w ogóle. -- rzekł. -- Co macie kupić?
\xx Jako, że nie idziemy do kolejnej kilometrowej kolejki pod spożywczym, skończy się na chlebie i zapasie octu. -- brzmiało to śmiesznie, ale wcale takie nie było, chociaż chyba nikt nie miał wątpliwości, że za kilkanaście lat cały świat będzie zrywał boki z uroków komunizmu.
\xx Aha\3k -- mruknął rozczarowanym tonem. -- No, w końcu jesteśmy na miejscu.
\qd
Dom handlowy miał kształt kwadratu i składał się z czterech pięter. Miał w sobie mnóstwo okien -- te, które umieszczono w ,,słupku'' po lewej i prawej stronie domu, ukazywały schody na następne jego piętra, zaś te ułożone poziomo, należały bezpośrednio do zakładów wewnątrz -- między innymi placówki poczty, salonu fryzjerskiego i małego sklepu spożywczego. Kilka z nich było otwartych na oścież, inne zostały jedynie nieśmiało uchylone.\\
Pod budynkiem wyłożono długi i szeroki jasnoszary chodnik, do którego prowadził przecinająca trawnik równie jasny, lecz o wiele węższy chodnik. Przy jednym z trzech wejść stała zielona ławka. Tu także rosło jedno drzewko -- wysokie na trzy metry, gładko przystrzyżone. Z wnętrza kompleksu co chwilę wychodzili jacyś ludzie o niezbyt bogatych zasobach w torbach -- chleb, jakieś butelki i słoiczki, prawdopodobnie ocet i musztarda.
\sx Witajcie w naszym kochanym Związku, gdzie ludziom żyje się lepiej, a towarów mamy więcej od kapitalistów i to bez pogoni za pieniądzem. -- mruknął Jonathan.
\xx Ooo tak! -- zawtórował mu Lenny. -- Kraina miodem i mlekiem płynąca\3k \qd
W centrum było dość wielu ludzi -- najwięcej z nich kłębiło się w biurze poczty, odbierając i nadając przesyłki oraz listy, przy okazji rozmawiając ze sobą o wszystkim i niczym -- ludzie byli tu na ogół szczęśliwi i szybko zawierali nowe znajomości po krótkiej pogawędce. Kilku mieszkańców Prypeci spędzało obecny czas w salonie fryzjerskim , także ucinając sobie krótkie pogawędki ze znajomymi. Mikołaj otworzył i przytrzymał wielkie drzwi, pozwalając reszcie wejść do środka. Nie mieli czasu na stanie w kilometrowej kolejce po jajka pod okolicznym sklepem spożywczym, a asortyment tutejszego potwierdził obawy o skromnych zakupach tego dnia.\\
Ocet\dotfill Ocet\dotfill Chleb\dotfill
Jeszcze więcej octu, no i trochę musztardy.\\
Każdy wziął po bochenku, zaś jedynie Jonathan kupił butelkę octu -- zapas na następny rok, pomyślał, po czym opuścił centrum budynek. Lenny podczas zeskakiwania z ostatniego stopnia schodów na parterze, potrącił jakiegoś chłopaka. \\
W białej koszuli, czarnych wytartych spodniach, średniego wzrostu, szeroki w barakach i w miarę umięśniony. ,,Zasrany paker'' -- tak mawiał o nim Masterton. Nie zmieni zdania, bo nie będzie do tego okazji -- żywot tej osoby skończy się krótko i nagle.
\sx Uważaj, jak, ku*wa chodzisz! -- krzyknął Adrian.
\qd
Gdy zobaczył, kto na niego wpadł, wyszczerzył się w swoim wrednym uśmieszku. Zaś kiedy spostrzegł, kto towarzyszył Lenny’emu, uśmiechnął się jeszcze szerzej i szyderczo, głównie z powodu obecności Mastertona.

Trwało to od dłuższego czasu, ale nasiliło się ostatniego ,,poważnego'' dnia w szkole. Połowa czerwca, wystawienie ocen i świadomość wśród uczniów, że nauka w tym roku się zwyczajnie zakończyła, że mogli udać się do domów i wieść dwa miesiące nieskrępowanego obowiązkami żywota. Pierwsi z budynku szkoły wypadł Jonathan, z plecakiem przewieszonym przez lewe ramię, prawą wymachując na lewo i prawo. Tuż za nim z budynku wybiegł Leon, potem po kolei -- Mikołaj, Radek, Lenny, Kamil oraz Izaak. Stanęli oni w wypukłym rzędzie, rozglądając się na wszystkie strony, co jakiś czas zasłaniając twarz przed bijącym w oczy słońcem. Wszyscy byli uśmiechnięci i szczęśliwi, roznosiła ich energia i poczucie euforii -- myśli o wakacjach naprawdę nie dawały o sobie zapomnieć.
%
\ro{16}
%
\podro{Rok 2001}
%
Godzinę przed wybuchem, przed alarmem w obozie służb bezpieczeństwa, godzinę przed śmiercią pewnej ważnej osoby i powstania kilometra kwadratowego śmiertelnego skażenia, Graham udał się do miejsca zamordowania Johnsona -- podrzędnego pachołka biorącego udział w procederze sprzedawania organów osób umarłych (bynajmniej nie przypadkowo) w Zonie. W okolicach wejścia do swego rodzaju kostnicy wiatr był dość silny i wyginał okoliczne trawy oraz drzewa do granic wytrzymałości. Graham ubrał się w czarną kurtkę i niebieskie spodnie, nie chciał wyróżniać się z tłumu.\\
% 
Podszedł do blaszanych drzwi, otworzył je i od razu przykrył nos rękawem, by poczuć jak najmniej smrodu ciała pozostawionego przez Jonathana. Był okropny, i mimo, iż Graham czuł go niejednokrotnie, nie zdążył się do niego przyzwyczaić oraz nie miał takiego zamiaru. Znał ludzkie emocje i doznania od ledwie dwóch lat -- rzeczy takie jak zabijanie, mimo, że był do nich niejednokrotnie zmuszany, uważał za chore i niepotrzebne.\\
% 
Do czasu, gdy wysłuchiwał materialnych korzyści, jakie te czynności dawały.\\
% 
Korytarz kostnicy cuchnął i wyglądał przeraźliwe -- skrzepnięta krew i wyschłe drobiny mózgu Johnsona były wszędzie i wyjątkowo kontrastowały z bielą kafelków. Graham wkroczył do środka, dalej trzymając rękaw prawej ręki przy twarzy, po czym wolnym krokiem ruszył do końca pomieszczenia. Każdy krok odbijał się szerokim echem -- po wykonaniu czterdziestego Graham stanął jak wryty, wyczuwając pod sobą pewnego rodzaju energię.\\
% 
Tuż przy trupie. -- przemknęło Grahamowi przez głowę. -- Wybrałeś idealne miejsce, Sussaro.\\
% 
Kopnął z całej siły w kafelek, na którym kończyła się zmieciona przez śruciny głowa Johnsona -- pękł on w pół z charakterystycznym odgłosem. Grahamowi zrobiło się niedobrze -- typowy efekt kontaktu z Pochłaniaczem. Chwycił go szybko w lewą rękę (wyglądało to komicznie, ale prawa ręka aż do wyjścia z kostnicy nie odrywała się od ust i nosa) i schował go do specjalnego pojemnika, który nie przepuszczał na zewnątrz negatywnej energii artefaktu.\\
% 
Kolejna porcja energii dla Susarro -- wystarczy jeszcze odwiedzić Mryńsk, by wcielić cały plan w życie.\\
Chryste panie\3k
 
Powtórzyłem to już czwarty raz -- dwa razy na głos i dwa razy w myślach. W obozie wszyscy albo wrzeszczeli, albo padali nieprzytomni, jednak większość wymiotowała i zawodziła z bólu. Należałem do tych ostatnich -- puściłem już dwa obfite pawie i z trudem powstrzymywałem następne, że nie wspomnę i wszechogarniającym bólu. To nie było zwyczajne promieniowanie -- działało zbyt szybko i gwałtownie, ale to, co w normalnych świecie zmusiłoby całą planetę do powołania ekipy śledczej, w Zonie było często spotykane.\\
Z trudem stanąłem na równe nogi i rozejrzałem się po biurze Barry’ego -- Mark usadził Leona na krzesło, chwyciwszy go za ramiona. Miał problemy z oddychaniem, kaszlał i co chwilę nerwowo mrużył oczy.\\
Cokolwiek to było, mamy Barry’ego w ciężarówce i musimy go stąd jak najszybciej wywieźć. Trzeba działać, potem będzie czas na rozmowę i cogodzinne bieganie w krzaki w celu zwrócenia zawartości żołądka. Działać, do cholery\3kDziałać\3k
% 
\sx Do cholery jasnej, wynośmy się stąd! -- krzyknąłem do dwójki stalkerów.
\qd
Boże, jak bardzo brakowało mi uroków dzieciństwa. Żadnego zabijania, żadnego podkradania się w celu dywersji, żadnej przemocy, żadnych kłopotów. Tylko ja, Masterton, Radek, Kamil, Lenny, Leon i Izaak -- tylko my i żyjąca Prypeć. Niczego więcej nie wymagaliśmy. Tęskniłem za normalnym życiem, ale było niestety za późno, by je odzyskać.
% 
\sx Dobra\3k -- Mark widocznie był za moim pomysłem. -- Leon\3k -- rzekł cicho. -- Leon!
\qd
Leon otworzył szeroko oczy i jeszcze raz zakaszlał.
% 
\sx Dam sobie radę, jedźcie z Radkiem i Izaakiem, ja pójdę do Mastertona.
\qd
Poderwał się na równe nogi i wybiegł z pokoju, zostawiając drzwi otwarte na oścież.
\sx Co teraz? -- spytałem.
\qd
Mark odpowiedział mi wzruszeniem ramion. Podobnie jak ja, oczekiwał on powrotu Leona. W tej nerwowej sytuacji sekundy dłużyły się niesamowicie -- po połowie minuty, kiedy Leon do nas wrócił, miałem wrażenie, że minął cały dzień.
\sx I co? -- zdziwiony i podenerwowany ton zadziwił samego mnie.
\qd
Wyjdę przez okno, a potem bramą, ale mniejsza z tym. W obozie nie ma ani śladu dymu. Nawet najmniejszego płomienia.
Lenny. -- pomyślałem przerażony.
Chwyciłem do lewej dłoni radiostację ustawioną na częstotliwość tej należącej do Lenny’ego i wywołałem go. Brak odpowiedzi. Nacisnąłem przycisk i wyszeptałem :
\sx Lenny?\qd
Po dziesięciu sekundach z słuchawki rozległ się krzyk. Był tak głośny, że sam wrzasnąłem, wypuszczając krótkofalówkę na podłogę. Wrzask nie trwał on zbyt długo, ale wystarczył, by wprowadzić mnie w szok -- spociłem się i strach niemal mnie sparaliżował -- przez moment schowałem głowę między kolana w akcie kompletnej paniki. Gdy nieco oprzytomniałem, lęk i ból straty (niemal pewnej) Lenny’ego zostały zastąpione przez determinację.
Po drugim kroku w kierunku leżącej kładki w mym prawym biodrze rozkwitł potworny ból -- zatrzymałem się nagle i jęknąłem cicho, lecz po chwili przestałem zwracać uwagę na cierpiący staw -- zajmę się nim później, teraz musiałem przede wszystkim wydostać się z obozu. Dobiegłem\3k Nie, dokuśtykałem do okna i wyjrzałem przez nie.
\sx Radek! Izaak! -- krzyknąłem w stronę ciężarówki.
\qd
Barry leżał bezwładnie wśród skrzynek z szeroko otwartymi oczyma.\\
Lewe drzwi pojazdu otworzyły się szeroko i wyszedł z niego Radek. Widocznie kręciło mu się w głowie, jednak wciąż pozostawał świadomy tego, co się wokół niego dzieje.
\sx Izaak jest nieprzytomny, dalej, skaczcie i jedziemy stąd w piz*u!
\xx Dasz radę prowadzić?
\xx Tak, tak, tylko się pospiesz! -- Radek wsiadł do ciężarówki na miejscu kierowcy i odpalił silnik, po czym zatrzasnął mocno drzwi.
\qd
Odwróciłem się przez plecy i chwyciłem oburącz leżącą na deskach kładkę -- wystawiłem ją przez okno, tuż nad naszym jedynym obecnym środkiem transportu. Pierwszy wszedł na nią Leon -- nie skakał on jednak do pojazdu, lecz ześlizgnął się w wąską szparę pomiędzy ścianą domku Jeffersona, a obozowiska. Upadł twardo na ziemię, leżał na niej chwilę, co mnie dość mocno zaniepokoiło.\\ Miałem już zawołać, czy wszystko z nim w porządku, lecz po kilku kolejnych sekundach Leon powstał, przecisnął się przez wąską przestrzeń w lewo, znajdując się na środku obozu. Kiedy tylko to zrobił, popędził on biegiem do kwatery Jonathana Mastertona.\\
Drugi był Mark -- widocznie najmniej dotknięty przez\3k cokolwiek to było. Zsunął się najpierw nogami w dół, chwilę później lądując na nich twardo w ciężarówce. Rozstawiłem tylne podpory kładki, co zapewniało jej stuprocentową stabilność, do momentu, gdy nie przekroczyłem framugi okna. Po wspięciu się na długą deskę i dojściu do progu okna, kucnąłem nisko.\\
Musiałem wyczuć moment przechylenia się pewnego rodzaju mostu, na którym stałem, by bez szwanku znaleźć się w pojeździe.\\ Posuwając się do przodu co milimetr, coraz bardziej odczuwałem chęć odjechania gdzieś daleko -- dźwięk włączonego silnika działał na mnie niczym syreni śpiew.
Długi kawał drewna zaczął sunąć w dół -- podpórki oderwały się od ziemi, pozostawiając mój los w rękach praw grawitacji i szczęścia.
% 
\ro{17}
% 
Leon, wciąż zmagający się z mdłościami, otworzył drzwi do pomieszczenia Jonathana. Miał jedyny klucz należący do Mastertona, ten sam, którym zamknął go dziś rano -- z Psycholem nie dawało się nawet porozmawiać -- leżał na swym łóżku, trzeźwiejąc po wypitym ostatnio alkoholu.\\
Łóżko było puste, drzwi łazienki otwarte na oścież, ukazując jej puste wnętrze.
\sx Geniuszu? -- powiedział Leon do siebie w myślach. -- A za drzwiami, obok których właśnie stoisz?
\qd
Były one uchylone do połowy, prostopadle do framugi.
Powoli, będąc pewien, co za nimi zobaczy, Leon zaczął je zamykać, jednocześnie wycofując się na lewo. Gdy ujrzał skrawek spodni Jonathana, popchnął drzwi na tyle mocno, że się zatrzasnęły.\\
Spodnie od wieczorowego stroju, które miał na sobie Masterton, były czarne, podobnie jak reszta ubioru -- buty oraz garnitur, jedynie koszula pod spodem miała biały kolor. Cóż, przynajmniej jest już sobą. -- stwierdził cicho Leon. Zmienił zdanie, gdy stojący przed nim człowiek rzucił się na niego z nieludzkim rykiem.

Na szczęście idealnie wyczułem moment, w którym kładka zaczęła całkowicie lecieć ku ziemi. Wybiłem się obiema nogami od drewnianej powierzchni z całych sił, przez sekundę wisiałem w powietrzu, po czym plecami upadłem na tyle ciężarówki. Gdy Radek to usłyszał, ruszył do przodu -- najpierw powoli, by ominąć drzewa, chwilę później przyspieszył, gdyż wjechał na coś w rodzaju ścieżki -- szerokie odstępy między grubymi pniami dawały możliwość przejechania pomiędzy nimi z dość dużą prędkością, bez ryzyka zderzenia.\\
Usiadłem na tyłku, opierając się bolącymi plecami o bok pojazdu. Barry leżał przede mną, dalej pod działaniem chloroformu, zaś Mark siedział za nim, naprzeciwko mnie, w podobnej pozie, tyle, że z rękoma opartymi o kolana.
\sx Ledwie tu przybyłeś, a spotyka cię taka masa atrakcji! -- krzyknąłem do niego, by przebić się przez odgłos kół i warkotu silnika. -- Strach pomyśleć, ile będziesz miał za sobą za rok!
\xx Strach się bać\3k -- mruknął w odpowiedzi.
\qd
Spojrzałem na jego zabandażowaną dłoń. Wątpię, czy ot tak puści Mastertonowi w niepamięć ten incydent, lecz nawet jeśli mu wybaczy, niesmak pozostanie do końca, choć krótkiego, życia obojga stalkerów.
\sx Mówiłem\3k -- Jonathan uderzył pięścią w twarz Leona.
\qd
Obaj leżeli na podłodze
\sx Żeby\3k -- zamachnął się jeszcze raz. -- Wziąć\3k -- trzecie uderzenie złamało Leonowi nos. 
\qd
Krzyknął on krótko zalanymi krwią ustami, próbując jednocześnie uspokoić Jonathana. 
\sx Kombinezony! -- w prawej dłoni szaleńca pojawił się zakończony sztyletem kastet. 
\qd
Jego posiadacz przyłożył ostrze do prawego ramienia Leona, powoli obracając je w lewą i prawą stronę, demonstrując swe narzędzie na wszelkie możliwe sposoby. Ofiara Psychola nawet przez gruby strój czuła zimne, wręcz lodowate ostrze -- było to wyjątkowo nieprzyjemne doznanie.
\sx Nie\3k -- jęknął cicho. -- Nie rób tego! Nie\3k -- Leon doprowadził swoje gardło do granic wytrzymałości w momencie, gdy Jonathan wbił się kastetem w jego ramię.
\qd
Jego skóra w mgnieniu oka zmieniła swój kolor -- w krótkiej chwili stała się trupio blada. Okolice rany natychmiast wypełniła czerwona, rzadka posoka, której kilka kropel wylądowało też na deskach. Ostrze zagłębiało się coraz bardziej i bardziej, zwiększając głębokość i rozmiar rany, wydając przy tym przeszywające odgłosy stali ocierającej się o mięso.\\
Na chwilę Masterton utkwił swe spojrzenie w Leonie.
Jego oczy były po brzegi wypełnione czystym obłędem -- rządzą mordu, czynienia krzywdy i cierpienia, które w danej chwili sprawiało mu nie lada radochę. Dysząc ciężko, przybliżał swą twarz coraz bliżej twarzy Leona, niemal stykając się nosami.
\sx Zupełnie jak z Adrianem, co? -- mruknął Leon.\qd
Jonathan natychmiast umilkł.\\
Jako, że przestał się wierzgać na wszystkie strony, dało się w mu się w końcu dokładnie przyjrzeć -- każda powierzchnia skóry lśniła potem, którego niespodziewanie zaczęło się tworzyć jeszcze więcej. Wyraz twarzy także uległ drastycznej zmianie -- usta, jeszcze chwilę temu wygięte w straszliwym grymasie, teraz opadły nisko w geście smutku. Szalone spojrzenie znikło w mgnieniu oka -- na ich miejscu pojawił się żal i ogromne, wręcz niewyobrażalne poczucie winy, połączone z czystym strachem. Myślami znów wrócił do pewnego deszczowego dnia.
\sx Przestań\3k -- jęknął głosem pełen rozpaczy, dalej rozmyślając o tamtym dniu.
\qd
Leon odzyskał inicjatywę -- zrzucił z siebie Jonathana lewą ręką, gdyż prawa wypełniła się tak potwornym bólem, że nie miał odwagi nią poruszyć. Masterton bezwładnie wylądował na plecach -- po chwili Leon powstał, (z zranionym ramieniem sprawiło mu to nie lada trudność -- podnosząc się z zakrwawionej podłogi poczuł tak silny ból, że omal nie zemdlał) patrząc na Jonathana z politowaniem. Mimo tego, że kręciło mu się w głowie, musiał zachować w obecnej sytuacji zimną krew.
\sx ,,Tylko tobie mogę zaufać, tylko ty nikomu nie wygadasz.'' albo ,,Nie dzwoń po pogotowie, i tak z tego nie wyjdzie.'' 
\qd
Płakałeś jak małe dziecko nad jego trupem. -- ton Leona i słowa, które właśnie wypowiadał, przerażały go samego, ale nie widział innego wyjścia, by ujarzmić Jonathana, którego mimo pozorów, było mu dogłębnie żal. Sięgnął lewą dłonią do długiej kieszeni spodni, przybliżywszy się do Mastertona o trzy kroki.\\ Usłyszawszy szydercze słowa Leona, wyrzucił on kastet na podłogę i chwycił się oburącz za głowę, chowając ją jednocześnie między kolana. Próbował sobie ,,wszystko poukładać'', ale niestety, jak zwykle mu się nie udawało.

Chwyciłem stalową szpitalną strzykawkę w lewą dłoń, po czym błyskawicznie wyciągnąłem ją z kieszeni. Zanim Jonatan zdążył spojrzeć na mnie swym smutnym spojrzeniem, wstrzyknąłem mu środek usypiający. Mój przyjaciel skrzywił się boleśnie i już po sekundzie opadł bezwładnie w dół.

Gdy ciężarówka zbliżała się obozu, Barry na chwilę odzyskał przytomność. Leżąc na plecach, poobijany i posiniaczony na całym ciele, spojrzał w lewo. Mikołaj zerknął na niego i uniósł prawą dłoń na wysokość lewej piersi. Pomachał nią energicznie, jednocześnie przywdziewając swą twarz w uśmieszek pełen udawanej serdeczności. Po dwóch sekundach tego szyderczego zachowania, Mikołaj pozbawił Barry’ego przytomności, kopiąc go czubkiem buta w podbródek.
\sx Ciesz się, że w ogóle żyjesz\3k -- powiedział cicho pod nosem.
\qd
Gdy Radek zatrzymał swą ciężarówkę, kilkaset metrów od niego, w tej samej chwili, Lenny odzyskał przytomność. Leżał w krzaku, który wcześniej został ,,wybrany'' przez Jonathana, który nie mógł jednak uczestniczyć w dzisiejszej akcji. Prawa ręka Lenny’ego podniosła się i obmacała pulsujące przeszywającym bólem czoło. Lewa zaś położyła się na pokrytej trawą ziemi. Lenny przewrócił się z brzuchu na plecy i odetchnął ciężko. Zrzucił z siebie siatkę maskującą i spojrzał na leżący obok niego karabin. M82 leżało na lewym boku, ze złożonym dwójnogiem i wyjętym, opróżnionym z pocisków zapalających magazynkiem. Szkiełko lunety było pęknięte w dwóch miejscach, podobnie jak wielki tłumik, który leżał luzem w trawie. Leon wyczołgał się (nogi miał kompletnie bezwładne) z krzewu na otwartą przestrzeń. W oddali dojrzał jakiś pędzący pojazd -- ciężarówkę Radka.\\ Szybko chwycił w prawą dłoń swą krótkofalówkę i wywołał Leona.
\sx Harlan? Ta\3k Tak\3k Dobra, później pogadamy, jesteś mi potrzebny. Ta, znowu mu odbiło. A skąd mam wiedzieć? Może coś wspólnego z tym alarmem\3k Właśnie -- nikomu nic nie jest? Naprawdę? Nikomu?! Dobra, wiesz, co masz ze sobą wziąć, czekam, byle szybko.
\qd
Rozłączyłem się i schowałem telefon do kieszeni. Zranione ramię bolało nieprzerwanie i mocno, jednak na szczęście sporą ulgę przynosiły mi leki przeciwbólowe, których Jonathan miał tutaj całe mnóstwo. Wliczając w to masę broni palnej, nie zdziwiłbym się, gdyby Masterton sięgnął po tego rodzaju ,,metody'' na znieczulenie. Opatrzyłem ranę i nie zamierzałem nawet o niej myśleć do momentu, gdy nie wypożyczę sobie na parę godzin porządnego artefaktu -- dokładnie miałem na myśli Duszę. Wizja całkowitego zagojenia się rany (co prawda zostanie po niej blizna) była niezwykle pociągająca.\\
Każdy w obozie wiedział o większości zachowań Jonathana, jednak tylko Harlan był na tyle godzien zaufania, by poznać go głębiej, a przy okazji móc od czasu do czasu się nim zajmować. Razem z Nami (czyli mną, Izaakiem, Radkiem, Mikołajem i Lennym) Harlan pomagał Mastertonowi od dość długiego czasu i robił to bardzo dobrze. Jednak nie wiedział o nim jeszcze bardzo wielu rzeczy -- kto wie, może kiedyś dowie się, co spotkało Kamila, Adriana, i tym samym, czemu Jonathan jest, jaki jest.\\
Ma radiostacja dała o sobie znać, krótkim, piszczącym odgłosem. Po odebraniu, zanim zdążyłem cokolwiek powiedzieć, usłyszałem głos Lenny’ego.
\sx Leon?
\qd
Bogu dzięki, że żyje.
\sx Gdzie jesteś? -- spytałem podniecony. 
\qd
Naprawdę uszczęśliwił mnie fakt, że Lenny jest cały -- jego krzyk w biurze Barry’ego nie pozostawał mi żadnych wątpliwości. Aż do teraz.\\
Po pięciosekundowej przerwie Lenny znów dał o sobie znać.
\sx W kryjówce. Ten krzak, w którym Jonathan\3k
\xx Tak, wiem który. Dasz radę\3k Chodzić? -- usłyszałem sapnięcie i dźwięk gniecionej trawy -- mój rozmówca najwyraźniej próbował się podnieść. Odczekałem krótką chwilę.
\qd

Lenny oparł się rękoma i ziemię, po czym począł podnosić się w górę. Udało mu się, gdyż odzyskał czucie w nogach. Schylił się po radiostację, chwiejąc się niczym pijak, i powiedział na głos:
Raczej\3k -- w momencie, kiedy miał rzec ,,Tak'' nogi znowu odmówiły mu posłuszeństwa, powalając go a hukiem na glebę. -- \sx Nie\3k
\xx Co się dzieje?
\xx Co chwilę tracę czucie w nogach. Lepiej po mnie czymś przyjedź, nie zamierzam tu tak długo leżeć.
\xx Dobra, spokojnie, niech tylko Harlan przyjdzie, to\3k
\xx Harlan?
\xx No. Harlan, nie pamiętasz\3k
\xx Wiem, który Harlan. Chwila\3k Czy ty jesteś u Mastertona?
\qd
Mruknięcie oznaczające zgodę.
\sx No tak\3k Co tym razem?
\xx Najpierw się na mnie zaczaił, a potem podziurawił ramię tym swoim sztyletem.
\xx Od momentu odkrycia powiązań Barry’ego całkiem mu odbija, że nie wspomnę o tym jego wczorajszym śnie. Wszystkich dziurawi -- najpierw Mark, teraz ty. Chyba znowu będzie trzeba ściągać tu jakiegoś psychologa.
\xx A nie możemy\3k Wybacz, zapomniałem.
\qd
Lenny odetchnął ciężko.
\sx On nigdy. -- rzekł smutno. -- Przenigdy nie opuści Zony. Wolałby zgnić w starej rurze, niż resztę życia spędzić w ekskluzywnym hotelu z widokiem na ogromne miasto. -- Lenny’ego ogarnęło poczucie beznadziejności. -- Od momentu narodzin nie opuścił okolic Czarnobyla na ani jeden dzień. To chore.
\xx Nie ma się co użalać. Trzymaj się tam, niedługo po ciebie przyjadę.
\xx Taa\3k Przeżyję\3k
\qd

Mikołaj kontynuował swe chamskie zagrywki. Gdy ciężarówka stanęła gwałtownie, tuż pod lekkim wzniesieniem, wydał on z siebie odgłos naśladujący dźwięk poprzedzający komunikatom na dworcach pociągowych.
\sx Przystanek -- Wolność. Dziękujemy za podróż naszymi liniami, mamy nadzieję, że znów z nich skorzystacie w przyszłości! Nasz personel dokona teraz prezentacji poziomu, z jakim obchodzimy się z naszym drogim klientem! -- kiedy skończył recytować, chwycił Barry’ego za barki i wyrzucił go z całej siły z tyłu pojazdu. 
\qd
Opierał się, ale w obecnym stanie nie miał z Mikołajem szans w siłowaniu się, więc po chwili leżał już na twardej, pokrytej kamieniami ziemi, z obolałą twarzą i brzuchem. Nie zdążył nawet podnieść głowy, bo została ona brutalnie wdeptana w podłoże czyimś ciężkim butem. Barry, charcząc, przez dwie sekundy ocierał się głową o kamienie, kalecząc ją w dwóch miejscach, po czym odetchnął z ulgą, kiedy Mark skierował swoją prawą nogę z powrotem ku ziemi. Chwila wytchnienia nie trwała jednak długo, bo sekundę później Mark z całej siły kopnął Barry’ego w żołądek, wywracając go jednocześnie na plecy. Po chwili usłyszał otwierane i zamykane drzwi ciężarówki -- spojrzał w lewo i ujrzał idącemu ku niemu Radka.\\
Zatrzymał się on metr przed Jeffersonem i założył ręce na piersi.
\sx Takie rzeczy\3k -- zaczął. -- Tylko u nas!
\qd
Po chwili krótkiej ciszy (nie licząc lekkiego powiewu i tym samym, szumu liści) odezwał się Mikołaj.
\sx Co z Izaakiem? -- skierował to pytanie w wyraźny sposób do Radka.
\xx Źle się czuje, ale to raczej nic poważnego -- mdłości, ból brzucha i takie tam\3k
\xx Tobie ten ,,wybuch'' też nie daje spokoju?
\xx Ooo tak. Ale bardziej\3k -- Radek spojrzał na Barry’ego. -- Spokoju nie daje mi on.
\qd
Z prawego okna ciężarówki wychylił się Izaak.
\sx Szybciej! -- ponaglił. -- Chce mi się srać, a wy się bawicie w niewiadomo co!
\qd
Mikołaj zamyślił się na chwilę.
\sx Racja\3k -- mruknął. -- Miejmy to za sobą.
\qd
Sięgnął lewą dłoń za pazuchę i wyjął z niej rakietnicę, po czym błyskawicznie skierował ją ku niebu.
\sx Od tej pory pół Zony będzie cię miało na oku. -- powiedział, po czym odpalił zieloną racę.
\qd
Poszybowała ona z typowym dla siebie świstem wysoko w górę, pozostawiając za sobą strugę ciemnozielonego dymu. Po chwili umilkła i zaczęła powoli, leniwie opadać w dół, jarząc się radośnie. Drzwi ciężarówki otworzyły się na oścież i wypadł z niej Izaak. Oparł się o maskę i zapalił papierosa. Po minucie ciszy, odezwał się, wyrzucając w tej samej chwili niedopałek.
\sx Chmurzy się.
\qd
Wciąż leżący na ziemi Barry próbował spojrzeć za plecy, jednak wszechobecny ból mu na to nie pozwalał. Zrobił to natomiast Radek i Mikołaj z Markiem, dostrzegając na południu ogromną, szeroką, granatową chmurę. Sunęła ona ku obozowi Wolności powoli, z racji niezbyt silnego wiatru, była też ona dość daleko. Mimo to wszyscy obecni poczuli chęć powrotu do obozu, a tym samym do zadaszonego domku, gdzie będą mogli spokojnie spędzić resztę dnia, nie zważając na deszcz i pioruny.\\
Pierwszy rozbłysnął kolejne pół minuty później, w samym centrum odległej chmury. Po kilku sekundach rozległ się wszechobecny, przytłumiony, charakterystyczny odgłos grzmotu. Na moment wiatr przybrał na sile, obsypując Jeffersona kolejną kupą liści i piasku.\\
Mark skierował swe spojrzenie na wzgórze.
\sx W końcu\3k \qd
Zza wzniesienia wyłoniła się piątka stalkerów -- każdy miał zielono-czerwony kombinezon, na których rękawach widniały zielone emblematy o wyglądzie wilczej głowy. Cała grupa była uzbrojona w OC-14 -- dwa karabiny miało dodatkowo doczepione lunety. Wszyscy, prócz członka Wolności idącego przodem, mieli zasłonięte twarze. Powoli, ostrożnie, cała piątka zeszła z trawiastego wzgórza, by po chwili stanąć tuż przed samym Mikołajem. Niejaki lider nowoprzybyłej grupy miał koło czterdziestu lat, krótko przycięte ciemne włosy i piwne oczy oraz niewielki, lekko przekrzywiony nos, zaś dolną część jego twarzy pokrywał szorstki zarost.\\
Po kolejnym grzmocie i nikłym, fioletowo-granatowym błysku, który zalał na ułamek sekundę całe otoczenie, Mikołaj podszedł do Anarchisty i uścisnął mu dłoń.
\sx Więc to on? -- odezwał się ten drugi.
\xx Taa\3k -- Mikołaj spojrzał niepewnym wzrokiem na zbliżającą się deszczową chmurę. 
\qd
Wiatr nasilił się, i to na stałe, powodując wszechobecny, głośny szum.
\sx Chcesz zdążyć przed burzą, co?
\xx Tak, a poza tym, Jonathan znowu nie miewa się zbyt dobrze.
\xx Pozdrów go ode mnie. -- zarośnięty mężczyzna klepnął Mikołaja po plecach.
\xx Dzięki\3k Przekażę\3k No\3k -- stalker zatarł ręce, jednocześnie drepcząc przez chwilę w miejscu. -- Zabierajcie go, wiecie co mu powiedzieć.
\xx Damy sobie z nim radę\3k Nie takich kozaków już tu mieliśmy. -- dodał członek Wolności, uśmiechając się ironicznie. -- Wstawaj, Barry! -- krzyknął.
\qd
Jefferson spojrzał na Marka i Radka, obawiając się kolejnych kopniaków, wgniatania w kamienistą drogę, czy innych tego typu zagrywek -- jego obawy rozwiał Izaak.
\sx Podnoś się, na dziś już dostałeś wystarczająco! -- zachęcił go. \qd
Barry usłuchał i chwilę później stał już wyprostowany, obejmując dłońmi obolały brzuch.

Usłyszałem pukanie do drzwi. Podniosłem się z łóżka, spoglądając na Harlana.
\sx Tylko spokojnie. -- mruknął, celując we mnie paralizatorem. \qd
Zignorowałem go, jednak wciąż było mi wstyd, tego, co zrobiłem w przeciągu ostatnich godzin. Doczłapałem do drzwi, wciąż będąc lekko pochylony i spojrzałem przez wizjer.\\
Lenny\dotfill Mikołaj\dotfill Izaak\dotfill Leon\dotfill Radek\\
Zupełnie jak tego dnia, pod koniec czerwca, kiedy poszliśmy razem na zakupy. Puste miejsce po prawej stronie wypełniła moja wyobraźnia -- młody Kamil uśmiechnął się do mnie, po czym zniknął.
% 
\ro{18}
% 
Wszyscy usiedli przy okrągłym, drewnianym stoliku w biurze Mikołaja. Pomieszczenie miało piętnaście metrów kwadratowych i łączyło w sobie kuchnię, salon i sypialnię. Łazienka znajdowała się zaraz za drzwiami wejściowymi -- cała kwatera rozplanowaniem przypominała tą należącą do Mastertona, jedyną różnicą było wyposażenie łazienki i większy pokój główny.\\
W jego rogu stało spore, rozłożone, zaścielone łóżko, zaś zaraz przy nim, drewniana szafka, na której stały -- elektroniczny zegarek oraz niewysoka lampka z wąskim, jasnobeżowym abażurem. Kuchenka gazowa wraz z aneksem kuchennym (do którego przymocowano zlew) zajmowały zachodnią część pokoju. Na samym jego środku stał zaścielony pieniędzmi i kartami stół, przy którym w pokera grało siedem osób -- Leon, Lenny (miał podbite lewe oko i prawą nogę w gipsie -- okazało się, że jest złamana), Mikołaj, Radek, Mark, Izaak (który wciąż borykał się z kłopotami trawiennymi, toteż co jakiś czas wybiegał w popłochu do toalety) oraz Masterton. Ten ostatni prezentował swym wyglądem definicję słowa ,,rozbity'' -- środki uspokajające widocznie go uciszyły i pozbawiły wigoru, stwarzając prawdziwy obraz zmęczenia i wyczerpania -- mówił on i zachowywał się niemrawo, bez jakichkolwiek chęci. Melancholia i przybicie w jednym.\\
Reszta jego przyjaciół (Izaak, Leon, Lenny wraz z Mikołajem i Radkiem) i jeden znajomy (Mark) co chwilę spoglądali na Jonathana zaniepokojeni, w główniej mierze z obawy, czy znowu mu nie odbije. ,,Ostatnich dwóch dni moich wyskoków tak łatwo nie zapomną\3k'' -- pomyślał Jonathan.\\
Za jedynym w kwaterze oknie panowała ciemność i cisza, przerywane jedynie przez jasne, białe błyski i potężne grzmoty burzy -- deszczu jak na razie nie było. Pół godziny temu, kiedy wybiła dziesiąta, Mark postanowił zgasić lampy i spędzić noc w otoczeniu świec, lecz okazało się, że Mikołaj ma tylko jedną -- stała ona teraz dumnie pośrodku blatu, roztaczając wokół żółtawe światło. Telewizor także stał wyłączony -- grało jedynie radio, ustawione na stację z której nadawano na żywo relację z pewnego meczu piłki nożnej. Głębiej interesował się nią tylko Leon, toteż co chwilę uciszał resztę towarzystwa, by z podnieceniem nasłuchiwać opisów potencjalnych bramek -- kiedy jakaś padła, Leon za każdym razem zasłaniał usta rękawem, by nie krzyknąć z podniecenia.
\sx I z czego tu\3k -- pojękiwał Masterton co chwilę, niezwykle leniwym głosem.
\qd
,,Morda w kubeł, wariacie!'' -- tak Leon jak i reszta mieli zamiar odpowiedzieć, jednak powstrzymali się -- musieli zrozumieć Mastertona, którego stan zakrawał o chorobę. W swych zachowaniach łączył objawy przeróżnych schorzeń -- od koszmarów sennych poprzez omamy (także słuchowe) na atakach szału i objawach schizofrenii skończywszy. Mimo wszystkich krzywd, jakich sprawił im Jonathan (gdzie największą wcale nie było zranienie na przykład ręki Marka czy ramienia Leona), nadal go wspierali i podtrzymywali na duchu. Każdy z nich, może poza Markiem, miał tą smutną świadomość, że zostanie w Zonie do końca swych dni, wiedzieli też, że spędzą je u boku Jonathana -- zostawienie go samego było by gorsze od dokonania na nim morderstwa.\\
Atmosfera była niewątpliwie, mimo wszystko, przyjemna. Granie w karty i jednoczesne słuchanie meczu całkowicie odprężało stalkerów, nie licząc Mastertona. Był spięty, zdenerwowany i przygnębiony, ale według Izaaka powinno mu niedługo przejść. Psychol umilkł o północy -- wcześniej udał się do siebie.
\sx Jestem wykończony, idę spać\3k -- rzekł, wstając z krzesła.
\qd
Pozostali wciąż grali w karty -- prowadził Mark, który niemal całkowicie ogołocił resztę z ich ,,majątku'' -- pieniądze o które grali, miały specjalną, sentymentalną wartość, o której Mark nie miał jeszcze pojęcia.
\sx Idę po piwo\3k -- oznajmił po chwili.
\xx Idę z tobą\3k mam ochotę na coś mocniejszego\3k -- Leon przerwał grę i po chwili wraz z Markiem opuścił pokój, udając się do baru.
\qd
Leonard, zwany przez żonę Lennym (nie lubił zbytnio, gdy nazywał go tak ktoś poza małżonką) założył nogę na nogę i zaczął tasować karty.
\sx Znowu to samo\3k -- powiedział po chwili milczenia.
\xx To znaczy? -- spytał Izaak, jednocześnie wyglądnąwszy przez okno. 
\qd
Ciemność, wielka, bezkresna ciemność, rozświetlana kilkoma nikłymi światłami z okien innych agentów. Poczucie bezsilności, ogromu Zony i jej niebezpieczeństw, a także wrażenie zaniepokojenia o swe życie było potęgowane przez tego typu widoki, jak Strefa nocą.
\sx Te myśli\3k
\xx Pesymistyczne?
\xx Pełną gębą\3k -- Lenny westchnął. -- Mimo, że jesteśmy to bardzo długo, to ja wciąż nie czuję się pewnie. Wciągnięcie nas do Zony było tak gwałtowne, wręcz brutalne, że chyba nigdy się do Niej nie przyzwyczaję. Raz w życiu byłem poza Jej terenem, do ku*wy nędzy! -- walnął pięścią w stół. -- Ona tam na mnie czeka\3k A ja chyba nigdy się z nią nie spotkam.
\xx Spokojnie\3k Wszystko będzie\3k
\xx W porządku?! -- Lenny zmienił ton na lekko kpiący. -- Tkwimy tu od tylu lat, a jak próbujemy wyjść, to zawsze jest jakiś pretekst, przez który zostajemy! ku*wa, czuje się jak w ,,Truman Show''!
\qd

Spokojny sen dla Jonathana był czymś niemożliwym w ostatnim i zapewne przyszłym tygodniu. Cała sytuacja z Barrym dała mu się porządnie we znaki, jednak dokładnych powodów Jonathan wciąż nie był świadomy. Szukał go usilnie -- jakiegoś wytłumaczenia, po prostu odpowiedzi, lecz niczego się nie dowiadywał. Wszelkie tego typu rozmyślania nie dawały żadnego efektu -- Masterton musiał czekać, aż rozwiązanie pojawi się samo, podane na tacy, czarno na białym.\\
Tego dnia śniło mu się coś z goła innego, niż ostatnim razem. Dzień, w którym poznał Marka, dzień, w którym w brutalny sposób zabił Johnsona na oczach nowoprzybyłego. Sen zaczynał się od momentu pociągnięcia za spust.\\
Masterton aż za dobrze, za dokładnie przypomniał sobie eksplozję czaszki nieszczęśnika, podobnie, jak jej fragmenty, które gęsto go pokryły. Czuł wszystko z tamtego wydarzenia z przerażającą dokładnością -- dźwięk, obraz, wszystkie możliwe szczegóły. Za każdym razem, kiedy we śnie garnitur Jonathana trafiła jakaś zakrwawiona, obrzydliwa drobina, czas zwalniał, by w pełni przedstawić Mastetonowi to przerażające ,,widowisko'', przy okazji obdarowując go krótkotrwałym, za to obezwładniająco mocnym bólem głowy. Jako, że użył obrzyna, jego katusze trwały przez ponad godzinę. Przez ten czas spocił się jak harujący cały dzień wół, spadł też z łóżka, lądując boleśnie na brzuchu. Oddech miał krótki i bardzo nierówny -- przez chwilę niemal zapragnął zawału, byle zaprzestać cyklicznego rozkwitu bólu w czaszce.\\
Kiedy podświadomie odetchnął z ulgą, schował obrzyna znów za pazuchą. Po krótkiej chwili nastąpił punkt kulminacyjny koszmaru.\\
Bezgłowy trup chwycił Jonathana za szyję i zaczął go dusić.
\sx Niemożliwe! -- jęknął Mikołaj. -- Jak to ,,odsyłają''?!
Stał wraz z Leonem pod barem, rankiem następnego dnia, rozmawiając pod fioletowym niebem.
\xx Załamał się. Nie ma z nim żadnego kontaktu, po prostu ,,odleciał''.
Leon ukrył twarz w dłoniach. Niewiele brakowało, by nie dał rady powstrzymać się od płaczu.
\xx Muszę go zobaczyć. -- rzekł zdecydowanym tonem.
\qd
Z swego pokoju wybiegł Izaak. Wciąż w pidżamie, nie zważając na wiatr i chłód, stanął zaraz przy Leonie, błądząc oczyma na wszystkie strony.
\sx Co z Jonathanem? Słyszałem, że\3k
\xx Odsyłają go. -- Mikołaj spuścił głowę. -- Masterton opuszcza Zonę\3k
\qd
Radek zerknął na przypiętego do noszy Jonathana.
\sx Katatonia? -- spytał przestraszony.
\qd
Masterton miał puste, martwe spojrzenie, ręce spuszczone u dołu i nie dawał żadnych oznak życia. Po prostu leżał bez ruchu, wpatrując się nieprzerwanie w sufit. W tym tragicznym stanie znalazł go Harlan -- nad ranem zajrzał do niego, by upewnić się, czy znów nie odstawił żadnego numeru. Po wejściu do jego kwatery usłyszał głośne sapanie -- ustało, kiedy Harlan wparował do sypialni Jonathana -- Psychol leżał bezwładnie na brzuchu, na podłodze. Kiedy Harlan do niego podbiegł i obrócił go na plecy, wydawane do tej pory sapanie ustało.\\
O stanie Mastertona wiedział już praktycznie cały obóz, jednak nieliczni się głębiej tym faktem przejęli -- trudno było się dziwić, dla większości tutejszej społeczności był on jedynie typowym wariatem, z którym znajomość ograniczała się do kilku wspólnych akcji i wypraw. By zawiązać z nim bliższe więzy, potrzeba było o wiele więcej, nie mówiąc już o byciu jego przyjacielem -- do tej pory miał ich sześciu i liczba ta raczej nie miała ulec zmianie -- od wybuchu w 86 Masterton w pewnym sensie odizolował się od reszty świata i ludzi, spędzając jak największą ilość wolnego czasu głównie z Leonem, Mikołajem i pozostałą czwórką dawnych znajomych. Nikt nie wiedział co -- podświadomość, on sam, czy nawet Bóg podczas rzekomego objawienia, ale pewna siła upewniła go, że jego starym przyjaciołom nic nie grozi. Być może dlatego każdej nocy spał niespokojnie, a podczas ostatnich dni nie sypiał prawie wcale -- porwanie Barry’ego było naprawdę niebezpiecznym zagraniem, które narażało jego uczestników na śmierć. Czy to 
dlatego Jonathan tak się zachowywał? Obawiał się, że Radkowi, Izaakowi albo innemu bliskiemu jemu stalkerowi stanie się coś złego? Prawdopodobnie, gdyż wtedy Masterton został by sam -- nic nie łączyło by go z ,,dobrymi (choć niekoniecznie) czasami'' -- z Kamilem i Adrianem. W takiej sytuacji obawiał się dodatkowo, że zwariuje szybciej, niż zdąży się zabić.
\sx Możesz nas na chwilę zostawić samych? -- spytał cicho Leon. Poza nim, Radkiem, Mikołajem, Izaakiem i Lennym w pokoju znajdował się Mark, który pod względem znajomości Jonathana znacznie odstawał od reszty.
\xx Jasne\3k -- rzekł tonem pełnym zrozumienia, po czym wyszedł, zamykając za sobą drzwi.
\qd
Krótką chwilę później Radek sięgnął prawą ręką do lewej kieszeni marynarki Jonathana -- pogrzebał w niej przez chwilę, po czym wyłowił zeń plastikowe pudełko z lekami, które Masterton musiał zażywać regularnie, trzy razy dziennie. Wpatrywał się w żółte pudełko przez moment, zaś po chwili odkręcił czerwoną nakrętkę, którą obrócił do góry nogami i zaczął w niej grzebać palcem wskazującym.\\
Krwiście czerwona, wyjątkowo mała tabletka w kształcie koła o średnicy co najwyżej milimetra. Nikt nie znał dokładnego składu tej trucizny -- nawet sam Jonathan nigdy nie zapamiętał ilości śmiertelnych środków chemicznych, których do niej napakowano. Miała zabić natychmiastowo w razie potrzeby, tylko to się dla niego liczyło. Od ostatniego tygodnia wizerunek Jonathana wielce podupadł -- początkowe ,,odpały'', które dla większości stalkerów nie były niczym niezwykłym i wrażenie wielkiej pewności siebie zaczęły ustępować wizji doszczętnie zniszczonego człowieka -- kogoś, kto boryka się z problemami od dłuższego czasu i nie może sobie z nimi poradzić. Kogoś, kto już w młodym wieku przeszedł więcej, niż nie jeden, który z tego świata zszedł w wieku stu lat. Wariat, Psychol -- słowa te były wypowiadane z przymrużeniem oka (do pewnego czasu), gdyż to, co Jonathan czasem robił, nadawało się głównie do opowieści przy ognisku -- spokojnych, gdzie nie było miejsca na poważne i dobijające historie. Mimo, że jego
zachowanie nie jeden raz jeżyło włos na głowie i autentycznie przerażały, uważano go za ,,porządnego'' człowieka.\\
Sytuacje, w których wykazywał się niebywałym talentem, te, podczas których olśniewał poczuciem humoru, pomocą, grzecznością -- wszystkie one mieszały się z momentami, kiedy Mastertonowi prawdziwie odbijało -- kiedy zabijał w okrutny sposób czy dostawał ataków szału i wściekłości. Powoli i stopniowo na jaw zaczęła wychodzić prawda o Jonathanie -- non stop nosił przy sobie nafaszerowaną truciznami tabletkę, obawiając się całkowitego zatracenia w szaleństwie -- szaleństwie, które być może ogarnęło go ostatniej nocy, podczas snu o Johnsonie.\\
Cały czas myślał o możliwości samobójstwa i nic nie przemawiało mu do rozsądku -- pozostawał głuchy na wszystkie argumenty jego przyjaciół, a jednego omal nie zatłukł na śmierć, kiedy próbował podmienić tabletki -- z trującej na pozbawiającą przytomności, jednocześnie umieszczając w niej mały nadajnik, dzięki któremu Mastertona szybko by znaleziono.\\
Wszystkie próby pozostawały jednak bezskuteczne -- Jonathan w pewnym stopniu żył we własnym świecie, do którego wielu rzeczy po prostu nie dopuszczał.
\sx Co mu się mogło stać? -- zapytał Izaak.
\xx Pewnie znów coś mu się przyśniło. -- mruknął Leon. -- Trzeba było przy nim zostać na noc. Czemu, do ku*wy, puściliśmy go sami?
\xx A jeśli komuś powiesz\3k -- ton Mikołaja tworzył nową definicję słowa ,,niesamowity''. -- Pożegnasz się nie tylko z drugim okiem\3k
\qd
\podro{Rok 2006}
Druga Zona.\\
Hazardzista, duchy, tworzenie anomalii, wszelkie możliwe artefakty.
Wyprzedanie codzienności, sześć poziomów, potencjalnych siedem. Co z tym wszystkich wspólnego ma Pochłaniacz, Oko Stwórcy i\3k te\3k te\3k Wytwory? A ta cała grupa? Co połączyło wszystkich uczestników incydentu w Mryńsku i pierwszych, którzy spenetrowali szpital z tymi, którzy po raz pierwszy weszli na Pola i tym samym do domu Gomeza? A ten chudzielec, którego z nimi spotkałem? Moją ostatnią nadzieją są akta wszystkich służb mundurowych na terenie Ukrainy -- może znajdę powiązanie.
% 
\podro{Rok 1978}
% 
\sx Jezu Chryste\3k -- Kamil padł bezwładnie na ziemię, uderzając twardo o posadzkę.
\qd
Nawet Leon i jego wielki refleks nie uchroniły go przez bolesnym upadkiem. Reszta poza Kamilem na szczęście utrzymała się na nogach, widząc Jonathana.\\
Znajdowali się na trzecim piętrze szpitala w sali 12 -- typowym szpitalnym pomieszczeniu z białymi kafelkami i równie śnieżnobiałymi ścianami, jak i sufitem. Znajdowały się w niej trzy szpitalne, zaścielone łóżka -- dwa były puste, ostatnie, to przy oknie, zajmował Masterton. Powodem tak gwałtownej reakcji, w tym Radka, który niemal zwymiotował z nadmiaru emocji, był ogromny opatrunek na lewym oku pacjenta. Widoczne wielkie plastry i pokryte różnego rodzaju środkami bandaże, także przez nie dodatkowo zawinięte, wraz ze specjalną taśmą. Wszyscy wiedzieli już, co się mniej więcej stało, dowiedzieli się od samego poszkodowanego, a raczej znajomego ojca Mastertona, który pracował w szpitalu i w krótkim czasie zwołał całą jego grupę znajomych.\\
Zastanie przyjaciela w takim stanie zawsze było szokiem -- zwłaszcza, kiedy coś takiego przeżywa się po raz pierwszy w życiu. Cała szóstka (Kamil ocknął się po chwili) dopadła łóżka Jonathana, zajmując resztę jego wolnego miejsca, nie przejmując się nawet stojącymi pod ścianą krzesłami. Wszyscy byli zatroskani i zaniepokojeni, zasypywali Jonathana mnóstwem pytań wszelkiej maści. Trwało to ponad pięć minut -- wszyscy próbowali myśleć więcej o tym, że mimo wszystko Mastertonowi już nic nie będzie, że są gorsze tragedie, niż utrata oka, że łatwo się po tym pozbiera i wróci do normalnego życia. Całą atmosferę wymieszanego cierpienia i szczęścia zakłóciło z goła inne pytanie -- zapadło ono głęboko w pamięci każdemu z siódemki nastolatków, jednocześnie rozpoczynając koszmar Jonathana Mastertona.
-To Adrian ci to zrobił, tak? -- Kamil założył ręce na piersi, wpatrując się w leżącego bezlitosnym spojrzeniem.\\
Zapadła martwa cisza.
% 
\podro{Rok 2001}
% 
Graham raz jeszcze sprawdził zawartość plecaka, po czym oparł się o betonowy murek, patrząc w niebo. Było czarne i pełne gwiazd, których blask podświetlał niewielkie, białe chmurki. Była pierwsza w nocy -- zimno i wiatr dawały się we znaki każdemu, kto miał o tej porze odwagę zapuścić się w głąb dzielnicy przemysłowej, potocznie zwanej Dziczą. Nazwa mówiła sama za siebie -- tereny te wprost przypominały dżunglę -- ciasne, otoczone budynkami, dźwigami i wagonami przestrzenie, gdzie głównym tłem były place budowy i tory kolejowe. Całe masy napromieniowanego żelastwa i złomu wszelakiej maści -- od znaków drogowych na wrakach samochodów kończąc. Woń rozkładu, rdzy, pyłu i brudu niemal wisiała w powietrzu, mieszając się ze zwietrzałym smrodem oleju i smaru.\\
Dookoła panowała przeraźliwa ciemność, bezruch i cisza, przerywana jedynie szelestem papierów, blaszek oraz niewielu liści, wprawianych w ruch przez każdorazowy podmuch wichru. W okolicy nie było widać żadnego źródła światła -- jedyną rzeczą, która nieco oświetlała powierzchnię, był księżyc, który, gdy wyłaniał się na moment zza chmur, pokrywał wszystko bladoniebieskim kolorem. Zza blaszanych dachów starych budynków Graham mógł dostrzec w oddali także słaby, wzbijający się wysoko w górę snop, którego źródłem był reflektor. Prawdopodobnie z obozu wojska położonego za wzgórzem, między terenami wokół starej wieży zegarowej, a wschodnią częścią Jantaru.\\
Graham stąpał po zardzewiałych torach, obok niewielkich budynków, z których kontrolowano zwrotnicę i opuszczano szlabany -- przynajmniej, kiedy stacja jeszcze działała. Przez stare szyny rosły kępy ciemnozielonej trawy, które miejscami były urozmaicane także przez nienaturalnej wielkości kwiaty i chwasty. Krótkofalówka Grahama wydała dźwięk, kiedy mijał on zdezelowany wagon kolejowy z kilkunastoma otworami po kulach niewielkiego kalibru.
\sx Gdzie jesteś? -- spytał Graham po odebraniu sygnału.
\xx Zaraz przy szlabanie.
\qd
Po ponownym przypięciu słuchawki do pasa, Graham raz jeszcze sprawdził noszoną za pazuchą czarnej kurtki kopertę z pieniędzmi -- dalej była na swoim miejscu. To samo zrobił z bronią -- upewnił się, że jego Spectre jest odbezpieczony i odpowiednio załadowany.
\sx Spokojnie, Leon! Poradzą sobie z nim! -- Jerome powstrzymywał Leona, który usilnie próbował przebić się przez dwóch strażników, ustawionych niczym mur przed wejściem do podziemnego wejścia -- klapy, leżącej tuż przy domku Barry’ego Jeffersona.
\xx Nie dam ci go zamknąć w jakiejś cuchnącej celi przesłuchań! -- Leon wpadł na dwójkę goryli, wymachując wściekle ramionami we wszystkie strony, przeklinając szpetnie. -- Ty skurwielu, jak możesz! Mikołaj! Lenny! -- krzyknął wściekle. Po chwili nabrał powietrza w płuca i wydobył z siebie tak nieludzki i przeraźliwy ryk, że stojący przed otwartą klapą strażnicy, wraz z Jeromem aż podskoczyli.
\xx Izaak! -- ryk był tak głośny, że Leon upadł na ziemię, niczym zemdlony. Podniósł się po chwili, po czym znów najechał na dwójkę agentów, z zaciekłym ,,Ach!''. Dalej jednak nie dawał rady, krótko powiedziawszy, stał w miejscu tracąc po prostu siły -- na próżno.
\xx Wy\3k -- Leon pochylił się nisko, łapiąc oddech, niczym ryba wyjęta z wody. -- sku*wiele\3k -- wysapał, po czym twardo opadł pośladkami na trawę, wpatrując się tępo w pordzewiałe drzwiczki.
Rozległ się dźwięk zamykanych gwałtownie drzwi. Po chwili do miejsca całej szamotaniny podbiegł Izaak z Radkiem i Mikołajem u boku.
\xx Co się tu, ku*wa dzieje? -- wyszeptał ten ostatni zimnym, zniecierpliwionym głosem.
\qd
Leon odciągnął głowę do tyłu, oczami błądząc po nocnym niebie.
\sx Te\3k -- przez odchylenie głowy zachłysnął się własną śliną.
\qd
Gwałtownie spuścił ją między kolana, kaszląc sucho. Kaszel palacza. Kiedy skończył, zaciągnął się powietrzem, tak, jakby miał to być ostatni oddech w jego życiu. 
\sx Te gnoje chcą wywalić Jonathana z jego kwatery i zamknąć go w podziemiach do czasu, gdy po niego przyjadą! Rozumiesz? Zamknąć go w\3k -- zawiesił głos i zmienił go na ledwie słyszalny szept. -- Celi\3k
\qd
Wszystkim trzem nowo przybyłym przysłowiowo odpadła szczęka. Wpatrywali się w Jerome’a z oburzeniem i wściekłością. Milczenie przerwał Izaak.
\sx Dlaczego? -- zapytał.
\qd
Radek, Leon i Mikołaj, a w pewnym stopniu on sam wyjątkowo zdziwili się jego grzecznym tonem. Administracja obozu miała zamiar zapakować Jonathana Mastertona na ,,czas bliżej nieokreślony'' do swego rodzaju lochu, a Izaak dokonał właśnie zapytania, które pasowało by do skrajnego pacyfisty, który pyta się listonosza, dlaczego nie dostarczono mu przesyłki na czas.
\sx Żeby go zbadać.
\xx Ooo nie! -- Leon poderwał się na równe nogi i niemal zetknął się nosem z twarzą Jerome’a. -- Już ja wiem, jakie badania masz na myśli! Zaraz\3k
\qd

Skąd on, do cholery wiedział?!
Każdy postawiony tutaj krok trzeba było uzgadniać z Jeffersonem. Ile piwa maksymalnie mogłeś wypić, od kogo zlecenia przyjmować i jakie artefakty przechowywać mogłeś na terenie obozu. Jerome już wcześniej miał w planie jakieś podejrzane badania -- z naciskiem na testowanie działań artefaktów, w sposób, gdzie humanitarność nie miała nic do gadania. Barry na szczęście zakazał mu tego typu praktyk pod groźbą ,,pójścia do piachu'' -- poskutkowała. Lecz teraz, gdy w obozie nikt nie sprawował rzeczywistej władzy, uświadomieni tym faktem mogli robić, co żywnie im się podobało. Jerome widocznie skorzystał z tego przywileju -- nie zdziwiłbym się, gdyby znalazł na dole cały ,,sprzęt'' rodem z siedziby Frankensteina. Do czasu przybycia nowego dowódcy trzeba było potrzymać resztę w nieświadomości -- dom Jeffersona był stale obserwowany -- nikt do niego nie wszedł od czasu wywiezienia Barry’ego, zaś strażnicy widocznie jeszcze się nie ocknęli. Podsłuch w jego pomieszczeniu też zdjęliśmy. Ku*wa, przecież Jerome nie wszedł 
tam do środka przez drabinę!

Zaskakująco wiele myśli przeszło mi przez głowę, w momencie, w którym dobywałem pistolet. Był to jednak bezcelowy manewr -- właśnie dowiedziałem się, dlaczego Jerome wciąż trzymał prawą dłoń za płaszczem. Upadając na ziemię, z pociskiem w brzuchu, obejrzałem się na Radka, Mikołaja i Izaaka.\\
,,Myśl o dawnych czasach.'' -- powtarzałem sobie panicznie w duchu.\\
,,Same dobre chwile, Prypeć\3k Prypeć\3k sierpień 78!''\\
78?!\\
Przestałem ufać własnemu umysłowi -- właśnie przywołałem najgorsze wspomnienie z czasu bycia młodzieniaszkiem. Jedyną pociechą był Kamil -- jedyny, który z naszej grupy pozostał nieskażony aż do chwili śmierci. Może tak było dla niego lepiej?
% 
\ro{22}
\podro{Rok 1978}
% 
\sx Odjebał nie jedno, to fakt. Ale żeby zrobić coś takiego\3k -- Kamil urwał nagle, po czym znów zmierzył Jonathana twardym spojrzeniem.
\qd
Wszyscy obecni w Sali popadli w głęboką zadumę -- co zrobić z Adrianem? \\
Oczywistym wnioskiem na samym początku była zemsta -- każdy z przyszłych stalkerów (z wyjątkiem Kamila) wyobrażali sobie przeróżne okropieństwa, które można było wyrządzić Adrianowi. Radek, Mikołaj, Izaak i reszta nie byli bandą kradnących oprychów, tak zwanym ,,złym towarzystwem'', ale do grzecznych, bojących się bójek chłopców też się nie zaliczali.\\
Wiedzieli, że ich przyszły czyn odmieni ich życie na zawsze, lecz satysfakcja z zemsty była dla nich ważniejsza. Pod tym względem odstawał jedynie Kamil -- z początku był za zgłoszeniem tego na policję, lecz szybko wycofał się z tego pomysłu -- uległ najzwyczajniejszej żądzy odwetu. Wciąż jednak w Mastertonie i jego przyjaciołach kłębiło się coś w rodzaju lęku -- podświadomość podrzucała im okropne obrazy, przedstawiające konsekwencje -- więzienie, problemy w rodzinie, duże prawdopodobieństwo zostania w przyszłości zwykłym bandziorem. Zero życia towarzyskiego -- nie licząc prostytutek, zszargana reputacja, prawie stuprocentowa szansa na popadnięcie w alkoholizm czy narkotyki. Pośród grupy znajomych panowała prawdziwa burza mózgów. Na moment przerwał ją gwałtowny grzmot.\\
Wszyscy, w tej samej chwili obejrzeli się w stronę okna. Za wieżowcami, położonymi przed dziedzińcem szpitala, wysoko w górze kłębiły się granatowo-szare burzowe chmury. Poruszały się one w miarę szybko, płynęły wraz z wiejącym z północy wichrem, z sekundy na sekundę coraz bardziej zasłaniając słońce i zbliżając się do szpitala. Wokół robiło się coraz ciemniej -- na podwórzu ilość światła zmieniała się jak w regulowanej świecy, zaś słońce zaczęło rzucać ogromny, półprzezroczysty cień, sunący powoli w stronę kliniki niczym morska fala. Kiedy pokrył on cały budynek szpitalny, w momencie, w którym słońce zostało całkowicie przyćmione a w ,,dwunastce'' zrobiło się tak ciemno, że trzeba było zapalić wszystkie światła, Masterton głośno przełknął ślinę. Patrząc nieobecnym spojrzeniem za okno, widocznie się czymś martwił. Częściowo uspokoił go Kamil.
\sx Nie martw się. -- rzekł, położywszy dłoń na lewym ramieniu Jonathana. -- Zostajemy tu z tobą do samego rana.
\qd
Jonathan zdobył się na lekki uśmiech.\\
Na zewnątrz rozległ się tak silny podmuch wiatru, że Masterton aż podskoczył, słysząc zgrzytający przy oknie parapet. Ludzie na dziedzińcu z widocznymi trudnościami utrzymywali równy krok, zmagając się z coraz mocniejszymi podmuchami. Drzewa były ugięte pod nienaturalnym kątem, obficie gubiąc stare liście. Po kolejnym, wyraźnie głośniejszym grzmocie rozpadał się ulewny deszcz.
% 
\podro{Rok 2001}
% 
Graham oparł się o stary, przekrzywiony kolejowy szlaban. Po prawej miał niewielką budkę, po lewej zaś przewrócony, zniszczony wagon towarowy. Jego odłażące z farby drzwi były przesunięte w lewo, ukazując wnętrze -- stos złomu i spleśniałych, starych drewnianych skrzyń, z których kilka było roztrzaskanych. Gdzieś, wokół całego tego bałaganu powiewała delikatnie niewielka pajęczyna z oczekującym nowych ofiar pająkiem na samym jej środku. W kilku miejscach wisiało niewiele martwych, wysuszonych much. Jeszcze jedna, srebrna w świetle księżyca pajęczyna wisiała na całej szerokości otworu w starej budce po jej lewej stronie, dokładnie w miejscu, gdzie niegdyś znajdowało się niewielkie okienko. Całe otoczenie tonęło w rzucanych przez chmury cieniach, przez co podłoże wyglądało jak niezbyt głęboka tafla jeziora podczas lekkiego powiewu. Miało się wrażenie, że z gruntu zaraz wypłynie jakieś morskie zwierzątko.\\
Po odczekaniu dwóch minut, Graham miał zamiar po raz trzeci wywołać Gavina, lecz w chwili, w której sięgał po radiostację, nieznany sprzedawca dał o sobie znać.
\sx Jestem, jestem\3k -- mruknął przepraszającym tonem, wytaczając się spod wagonu (gniotąc lekko swój nowoczesny kombinezon), któremu Graham jeszcze przed chwilą się przyglądał. 
\qd
Niewiele brakowało, by podskoczył ze strachu, dał radę jednak w porę się opanować, wydobywając z siebie jedynie jęk zdziwienia i uznania. Przewrócił oczyma z dezaprobatą, patrząc na wykonującego jakiś gest Gavina. Po chwili przestrzeń w pobliżu dziesięciu metrów wypełniła się stalkerami. Było ich czternastu, wszyscy, co jeszcze bardziej zdumiało Grahama, mieli na sobie jedne z najrzadszych kombinezonów w Zonie -- SEVA. Uzbrojeniem także robili wrażenie -- pięć sztuk G3, kolejne pięć L1A1, dwójka stalkerów trzymała po jednym DAO-12. Po krótkiej chwili Gavin wydobył zza pazuchy L22A2, po czym skierował go lufą do ziemi.\\
,,Ku*wa, a spodziewałem się zwykłych oprychów\3k'' -- pomyślał przez moment Graham. Sam sposób pojawienia się współtowarzyszy Gavina także dawał do myślenia, że są nie byle kim -- połowa czyhała schowana pod kilkoma innymi wagonami, dwóch z nich ukryła się w wagonie, przylegając do jego wnętrza, zaś ostatnich czterech wybiegło z krzaku rosnącego około dwudziestu metrów od szlabanu.
\sx To jak? -- ton Gavina wskazywał jasno, że jest wyjątkowo pewny siebie, cóż, trudno mu było się dziwić, mając taką obstawę\3k -- Robimy interes? -- po tym zapytaniu z wyraźną uciechą zatarł o siebie ręce.
\xx Taa\3k -- mruknął Graham, po czym zaczął kierować broń do wewnętrznej kieszeni ubrania. 
\qd
Kilkanaście luf, z wyjątkiem tej Gavina, wystrzeliło w jednej sekundzie w powietrze, biorąc Grahama na celownik.
\sx Hej hej! Tylko kasę wyjmuję\3k
\qd
Bronie zostały opuszczone -- po krótkiej chwili Graham wyjął dużą, grubą kopertę. Wystawił ją powoli ku Gawinowi, który po paru sekundach przyglądania się, wziął ją do ręki. Ocenił jej grubość, gwiżdżąc z zadowoleniem, po czym opróżnił ją i zaczął liczyć pieniądze, które w niej schowano. Robił to szybko, sprawie i zręcznie. Kiedy skończył, kiwnął kilka razy porozumiewawczo, mrucząc coś do siebie. Ok, ok., ok., ok\3k
\sx Ok., zgadza się\3k Henry, przynieś tego\3k Jak mu tam było?
\xx Vychlopa, szefie, Vychlopa. -- rzekł zrezygnowanie jeden z stalkerów trzymających 
DAO-12. 
\qd
Kiedy podrapał się po głowie, odwrócił się przez plecy i wolnym krokiem ruszył w stronę wagonu. Wszedł do środka przez boczne drzwi, krusząc niewielkie kawałki drewna. Włączył przyczepioną do ramienia w prowizoryczny sposób latarkę, oświetlając stary, pordzewiały blaszany dach. Złapał w nim jakąś wypukłość prawą dłonią i mocno pociągnął do siebie. Z wagonowego sufitu odpadła wielka, szeroka blacha, ukazując coś, co wyglądało na zamknięty w pokrowiec teleskop -- to, w jaki sposób trzymał się sufitu, było dla Grahama wciąż niewiadome, prawdopodobne posłużyły do tego wyjątkowo silne magnesy. Henry z trudem nie wypuścił zawartości skrytki z rąk podczas jej wyjmowania -- nic dziwnego, Vychlop ważył ponad osiem kilogramów. Chwyciwszy go oburącz, niosąc niczym ceremonialny miecz, Henry podszedł do Grahama.
\sx Proszę bardzo\3k -- rzekł, położywszy pokrowiec na ziemi.
\qd
Nagle stojący na samym końcu ,,formacji'' stalker odwrócił się momentalnie przez prawe ramię i wystrzelił trzy razy ze swojego~G3.\\
Reszta gwałtownie odwróciła się w stronę oddanych strzałów, podnosząc swoje lufy przed siebie -- powstrzymał się od tego jedynie Graham -- po takiej reakcji na gwałtowny ruch wolał nie wyciągać przy Henrym i reszcie broni.
\sx Dzięki, Ben\3k -- podziękował Gavin tonem pełnym ulgi, opuszczając broń.\\
\qd
Pijawka leżała tuż przed stalkerem (Benem) z trzema wielkimi otworami po kulach w swym brązowym, niemrawie owłosionym cielsku. W bardzo szybkim tempie lała się z ich krew, czerwieniąc beton wokół miejsca, w którym potwór padł na ziemię, nadając szynom upiornego, czerwono-szarego koloru. Macki mutanta rzucały się jak szalone we wszystkie strony, łącznie z przekrzywiającego się z boku na bok łbem -- również białe, świecące w ciemności ślepia poruszały się szybko i chaotycznie. Kończyny zachowywały się podobnie -- próbowały pomóc ich właścicielowi przewrócić się na brzuch, lecz w konsekwencji łomotały jedynie wściekle o tory.
\sx Wróćmy do naszej wymiany. -- powiedział Gavin po wystrzeleniu jednego pocisku w głowę pijawki -- uniesione ręce i nogi wraz z głową i mackami zastygły na moment w powietrzu, po czym opadły bezwładnie.
\qd
Ben usiadł po turecku, sapiąc. Czując na sobie spojrzenia innych, zapewnił ich, że nic mu nie jest.
\sx Skończcie, to zaczęliście, mi nic nie będzie, muszę tylko odpocząć\3k -- wysapał.
\xx W porządku\3k -- rzekł Gavin. -- Słyszałeś, skończmy to jak najszybciej. Sprawdź sprzęt, spieszy nam się.
\qd
Graham pochylił się nisko, rozpiął pokrowiec i wyjął z niego karabin snajperski VSSK Vychlop.\\
Broń ta autentycznie go przerażała -- miała ponad 800mm długości, z czego połowę stanowił niesamowicie gruby tłumik, co w połączeniu z potężną amunicją 12.7MM dawały temu karabinowi wielkie możliwości. Jego korpus miał ciemno oliwkowy kolor. Sam karabin zbudowano w układzie bull-pup -- magazynek mieszczący pięć sztuk amunicji mieścił się za spustem, przed którym zaś zamontowano składany dwójnóg, obecnie złożony. VSSK używał kultowej już lunety PSO.
\sx Idealny\3k -- mruknął Graham. -- Wprost idealny.\qd
Zachwyty Grahama przerwał kolejny wystrzał, nie dochodził on jednak z żadnej z broni ludzi Gavina. Jeden z nich padł na ziemię, łapiąc się za przestrzelone kolano, krzycząc:
\sx Wojsko! je*ane trepy czekały, aż\3k
\qd 
Słowa rannego stalkera zagłuszył dźwięk eksplodujących granatów dymnych. Po chwili ich cichy syk mieszał się z jękami rannego, którego Gavin przyciągnął za fraki do przeciwległego wagonu.
Całą przestrzeń wypełniły opary szarego, smolistego dymu. Ktoś z wojskowych wystrzelił jeszcze dwa razy na ślepo, nikogo nie trafiając. Cała trzynastka, w sumie czternastka, jeśli doliczyć Grahama, leżała płasko na torach kolejowych, nie mając żadnej osłony, poza dymem.
\sx Pod wagony! -- Gavin powiedział to dość głośno, by przebić się przez dźwięk opróżniających się granatów.
\qd
Graham czołgał się pierwszy -- z trudem pokonywał kolejne metry nierówności, kierując się na wschód. Sapnięcia stalkerów, którzy za nim podążali zdawały się nie mieć końca.
Dym zaczął powoli opadać -- zauważywszy to, Gavin odbezpieczył i upuścił jeszcze jeden pojemnik.\\
Cały spód wagonu wypełnił się oparami -- wyglądało to tak, jakby na torach, na których stał, palił się ogień. Stalkerzy z trudem pod niego dopełzli, zaś kiedy w końcu im to się udało, Gavin kazał wszystkim założyć tłumiki. Jego podopieczni usłuchali -- każdy miał czarną, wąską tubę przyczepioną do tyłu kombinezonu, w okolicach potylicy. Przez moment zrobiło się głośno od dźwięku zakręcanych tłumików, zaś posiadacze strzelb wydobyli z kabur wytłumione już pistolety -- Graham spojrzał na nie z ciekawości i poznał między innymi P99. Graham także miał przygotowaną tego typu broń -- w prawym ręku trzymał teraz Berettę 93R z zamontowanym wcześnie tłumikiem.
\sx A teraz\3k -- zaczął Gavin. -- Nie robicie nic, dopóki wam nie każę.
\qd
Na górze rozległo się głuche stuknięcie -- to postrzelony stalker doczołgał się do kryjówki. Graham miał wraz z resztą jego chwilowych towarzyszy wyjątkowe szczęście -- wagon miał szeroki spód, dzięki czemu zasłaniał on całkowicie ciała ukrywających się pod nim stalkerów.\\
Dym zniknął, rozwiany przez lekki powiew. W oddali, według Grahama w granicy złomowiska z terenami przemysłowymi, rozległo się wycie wilka. Po chwili, tym razem znacznie bliżej, rozległy się kroki kilku par butów.

Trzy godziny wcześniej, w chwili, w której noc zaczęła dominować nad dniem, w okolicach Jantaru przebywało pięciu członków~S. Była to ta sama grupa, której znaczna część w tym samym czasie wyruszyła w stronę Dziczy, by sprzedać Vychlopa. Jedna z najmniej znanych i najbardziej elitarnych frakcji w Zonie -- na równi z Monolitem, ustępowała jedynie grupie Susarro, lecz z nim nikt nie mógł konkurować.\\
Jej nazwa -- S -- pochodziła od nazwy kombinezonu SEVA -- sztandarowego typu kombinezonu używanego przez jej członków, a raczej z historią, w jaki założyciele S zdobyli owe pancerze.\\
Głównym organizatorem był Gavin -- człowiek, którego w 1993 porwał oddział dezerterów z Specnazu, którzy zrobili mu pranie mózgu i wcielili do siebie w roli żołnierza. Rok później, listopada 1994 Gavin dowiedział się, że oddział od dłuższego czasu zajmuje się wyłącznie ogromną akcją na terenie Zony -- Strefy, terenu elektrowni atomowej w Czarnobylu. Planowali oni dokonać wielkiego przewrotu -- wyeliminowania lub zrekrutowania wszystkich wojskowych na Jej terenie i wykluczyć jakąkolwiek rolę oficjalnych sił jakiegokolwiek rządu -- Zona miała należeć wyłącznie do nich. Potem ta całkiem pokaźna grupa (należało do niej ponad dwustu byłych komandosów, w tym Gavin, który miał za sobą wiele specjalistycznych szkoleń) chciała utworzyć z elektrowni atomowej swoją twierdzę -- symbol ich potęgi i pozycji, jaką zajmowali by w Strefie. W 1992 zatrudnili oni sztab naukowców, którzy badali Zonę od momentu nastąpienia wybuchu -- mieli oni opracować kontrolowany sposób tworzenia anomalii, by nikt poza nimi nie mógł 
przekroczyć granicy do Czerwonego Lasu -- początku ich oficjalnego terytorium. Do tego oddziału badaczy należał między innymi Doktor Gomez, który zainteresował Dezerterów pod wieloma względami.\\
Każdy inny naukowiec miał zostać zabity, a wyniki ich badań -- przejęte.
Dwudziestego lipca 1996 roku Gavin został uprowadzony spod jego domu -- dwójka nieznanych ludzi ogłuszyła go, kiedy wychodził z mieszkania, po czym zapakowali do samochodu. Po przejechaniu kilkunastu kilometrów Gavin odzyskał przytomność -- był w południowej części Petersburga, a dokładniej w pewnej obskurnej kawalerce. Dwaj nieznajomi okazali się, jak sami powiedzieli, ,,oficjalnymi przedstawicielami ukraińskiego rządu'', który ,,dowiedział się o zamiarach grupie dezerterów i zamierza nie dopuścić do ich zrealizowania'' ponieważ ,,rząd ma własne plany co do sposobu wykorzystania Zony, a ,,jakikolwiek rozruch na tle przejęcia władzy będzie dla niego wyjątkowo niekorzystny''. Dwójka agentów zaproponowała a raczej rozkazała Gavinowi co tygodniowe raporty na temat planów Dezerterów i działania ,,mające doprowadzić do całkowitego rozpadnięcia się grupy''. W zamian oferowali mu szansę stworzenia własnej, potężnej frakcji, której jednak działalność miała być tajna -- Gavin i członkowie P.S (bo tak brzmiała wtedy
ich nazwa) mieli dążyć jedynie do ,,zaspokojenia własnych korzyści'' i ,,utrzymywania w Zonie równowagi'' do ,,bliżej nieokreślonego czasu''. Gavin do 1999 roku był w Strefie jedynie raz -- wizja bycia Jej ,,nie koronowanym królem'' wraz z chęcią zemsty na sprawcach wyprania mu rozumu przekonały Gavina -- dwa miesiące po nastaniu roku 2000 szefowie Dezerterów zostali zabici (po wczesnym, ponad czteroletnim inwigilowaniu) zaś członkowie rozeszli się w popłochu, obawiając się podzielenia losu ich przełożonych.
\\
Już dwa lata przed tym wydarzeniem Gavin zebrał jeszcze trzech ludzi, którzy mieli stworzyć P.S -- miał też plany co do ich wyposażenia i roli, jaką mieli odegrać w Zonie.\\
Kiedy byli żołnierze Specnazu przestali zagrażać planom Ukrainy, zaczął on tworzyć w Strefie własną władzę -- stworzyli oni obóz, gdzie pracowali ich najlepsi agenci -- ich zadanie po części pokrywało się z celem P.S -- zbieraniem wszelkich informacji o Zonie.\\
Po zbudowaniu obozu i umieszczeniu tam większej części swoich ludzi, ukraińska władza posłała pierwszy z dwóch transportów z wyposażeniem dla ośrodka służb specjalnych. Zawierał on osiemdziesiąt sztuk kombinezonów SEVA i ponad setkę sztuk broni wszelkiego rodzaju -- od strzelb po karabiny szturmowe na snajperskich skończywszy.\\
Tym razem to Gavin dokonał dezercji.\\
Zaatakował i przejął cały transport, w chwili przekroczenia przez niego granicy. Gdy otwarcie uznano go zdrajcą, ukrył się w głębi Zony wraz ze swoimi ludźmi, których miał już dwudziestu dwóch. Z racji ich wyposażenia szybko opanowali oni północną część Zony, zaś wszędzie, prócz grupy Susarro i Monolitu, posiadali informatora w każdej frakcji. Ze starej nazwy usunięto ,,P'' -- P.S przestało oficjalnie istnieć, lecz i tak tylko nieliczni w ogóle wiedziało o istnieniu jego ,,następcy''. Być może nazwa znowu uległa by zmianie, gdyby ludzie Gavina zdecydowali by się na atak drugiego transportu, który miał miejsce 2001 roku -- nie chcieli jednak całkowicie wykluczać ingerencji rządu w Zonę -- posiadanie w ich obozie informatora wydało się kuszącą wizją, którą zresztą wcielono w życie.\\
Teraz niepisany zastępca, a także przyjaciel Gavina, który wielce pomógł mu podczas ataku na pierwszy transport, imieniem Teodor, zwany zwykle ,,Tedem'' wraz z czwórką innych żołnierzy S oczekiwało wojskowego śmigłowca. Dwaj z nich na chwilę zostawili swoje RPG -- oparli je o wrak autobusu, przy którym też usiedli, oczekując raportu od zwiadowcy, który miał im dać znać, kiedy wojskowy Mi-28 przekroczy granicę Jantaru. Teodor i pozostała dwójka uzbrojeni byli w G3 z celownikami optycznymi.
\sx Ciemno jak w dupie\3k -- mruknął Pete, jeden z posiadaczy RPG.
\qd
On jak i pozostała czwórka miał zasuniętą osłonkę twarzy, wbudowaną w kombinezon.
Jego głos był w charakterystyczny sposób stłumiony i zniekształcony przez szczelne zamknięcie.
\sx Chyba bym zdechł, gdybym miał nosić jeden z tych podrzędnych kombinezonów, jakich używają ci tutejsi.
\xx Co masz na myśli? -- spytał rozglądający się we wszystkie strony Teodor.
\xx Chodzi mi o to, że znaczna większość ich wbudowanych noktowizorów bardziej pomaga oślepnąć, niż ułatwić widzenie w nocy. Ci, którzy je robią, dają je chyba dla szpanu, nie mają zielonego pojęcia o tego typu technice!
\xx Fakt\3k -- Ted na chwilę odsunął przesłonę na twarz, by pociągnąć łyk wody z manierki.
\qd
Kiedy to zrobił, zasunął ją z powrotem, znów upodabniając się do reszty. W takiej sytuacji odróżnić go można było poprzez dwa wielkie, czerwone pasy, które wszyto w prawe i lewe ramię kombinezonu. Podobne miała SEVA należąca do Gavina, która miała dodatkowo jeden czerwony pas biegnący od lewego barku do prawego uda. Tego wszycia nie dało się pomylić z żadnym innym, dlatego Gavin zawsze był łatwo rozpoznawany przez swoich ludzi wśród oddziału identycznie ubranych stalkerów.\\
O tej porze Jantar był rzeczywiście mroczny i ciemny -- słońce kierowało się ku horyzontowi, lecz jego promienie i tak prawie nigdy nie gościły na wyschniętym jeziorze -- mało kto wiedział dokładnie co -- gęsta, szarawa mgła, wiecznie stojące pasmo grubych chmur (Takie rzeczy tylko w Zonie! -- rzekłby John Finn), ale coś sprawiało, że Jantar niemal o każdej porze roku i dnia był nieoświetlony przez słońce.\\
W tym momencie widoczność ograniczała się do ponad dwudziestu metrów -- dalej była tylko mgła, której srebrzysty kolor ledwie wybijał się ponad panujące ciemności. Teodor czuł się jak na planie filmowym, gdzie autobus był jednym z rekwizytów, a reszta świateł na całym planie została wyłączona.\\
Pete potrząsnął kilkakrotnie puszką z białą farbą.
% \dd
\sx Minęła godzina. Ściągnij go spowrotem.
\qd
Jonesy nacisnął na radiostacji odpowiedni przycisk i przyłożył słuchawkę na lewego ucha. Po pięciu dłużących się sekundach wydobył się z niej głos.
\sx Szefie? -- spytał przestraszony. -- Tu jest\3k
\xx Wracaj natychmiast. -- rozkazał Jonesy. -- Wynoś się stamtąd, za parę godzin będzie Zwarcie!
\xx Dobra\3k Jeśli stąd nie wyjdę\3k powiem wam jedno -- nigdy, przenigdy tu nie wchodźcie! Mutanty wyłażące ze ścian, wizje\3k -- ze słowa na słowo głos wysłannika stawał się coraz głośniejszy i mniej zrozumiały. -- Anomalie, o których wam się nie śniło, do tego ruchome, omamy, halucynacje\3k Niech Zona wstydzi się, że zrodziła takie miejsce! -- żołnierz Monolitu wykrzyknął ostatnie słowa tak głośno, że Jonesy o mało nie ogłuchł. 
\qd
Teraz ze słuchawki dobywał się jedynie szaleńczy bełkot -- miało się wrażenie, że obecny wewnątrz budynku stalker krąży w kółko, gadając do siebie.\\
Minęła minuta cierpliwego oczekiwania na kompana z oddziału, jednak niczego one nie dały -- radiostacja całkowicie umilkła, przestała wydawać z siebie jakiekolwiek odgłosy.\\
W oknie na trzecim piętrze kompleksu rozkwitło spore pęknięcie -- coś zostało rzucone z wnętrza pomieszczenia. Coś czerwonego i widocznie mokrego -- za tajemniczym przedmiotem ciągnęły się niczym smugi czerwonawe krople. Nieznany obiekt leciał z wyjątkową prędkością i wyraźnie kierował się w stronę Jonesy’ego -- w ciągu dwóch sekund, przed zderzeniem oślizgłej kuli z jego twarzą, rozległ się świst przecinającego powietrza, który stopniowo narastał podczas krótkiego lotu serca żołnierza, którego wybrano do wejścia na przedostatnie piętro.\\
Trafiło ono Jonesy’ego prosto w nos, łamiąc go -- po tym trafieniu także świeżo wyrwane serce uległo zmiażdżeniu, oblewając wszystko w promieniu kilkunastu centymetrów ciepłą krwią. Jonesy z trudem ustał na nogach, gdyż siła uderzenia była naprawdę wielka. Ku zdumieniu wszystkich, dowódca oddziału był niewzruszony -- reszta podwładnej mu drużyny była w kompletnym szoku. Jonesy stał twardo niczym słup soli z zakrwawioną głową, wbijając się tępym spojrzeniem w trzecie piętro prypeckiego szpitala. Na chwilę dostrzegł w budynku coś więcej niż mury i fundamenty -- zobaczył klinikę jako żywy, myślący własnym rozumem organizm, który robi Monolitowi na złość, nie zapraszając nikogo otwarcie do swych progów. Na chwilę Jonesy był nieobecny -- przeniósł się do własnego, niedostępnego dla nikogo, poza nim świata.
\sx Rób sobie co chcesz\3k -- wyszeptał nieludzko chrypliwym głosem. -- Oko Stwórcy i tak wkrótce będzie nasze.
\qd
W oddali huknął potężny grzmot. Jonesy potraktował to jako pogardliwą odpowiedź szpitala. Splunął szpetnie, po czym odwrócił się przez plecy i zaczął przecierać twarz.
\sx Idziemy\3k -- rozkazał, trąc lewe oko. -- Nie chcę tu być, kiedy Zwarcie się zacznie.
\qd

Teodor wstał i przeciągnął się, chrupiąc nadgarstkami, szyją i każdym palcem z osobna. Siedzący obok niego Pete nawet nie zwrócił na to uwagi -- był przyzwyczajony do rozmaitych rzeczy, które Ted wyczyniał ze swoim ciałem.
\sx No\3k -- mruknął, rozglądając się na boki. -- Długo mamy jeszcze na niego czekać?
\qd
Zapadła już niemal całkowita ciemność -- pojęcie widoczności przestało zobowiązywać na odległości większej niż pięć metrów. O tej porze Jantar był szczególnie przerażający -- odgłosy zamieszkujących to miejsce mutantów nie przestawały o sobie znać, a wiatr był jeszcze silniejszy, niż zwykle. Po zapytaniu Teodora, gałęzie i korony drzew świszczały niemal ogłuszająco. Wszyscy członkowie S. poderwali się na równe nogi, po czym biegiem schowali się do starego wraku autobusu. Do czasu ustania Powiewu stalkerzy wcisnęli się na koniec autobusu, obserwując bacznie każde wybite okno pojazdu -- Gavin nie zamierzał dopuścić do zranienia drugiego człowieka w ten naiwny sposób -- mianowicie chodziło o Pabla, który dał się wciągnąć Snorkowi przez kabinę kierowcy innego autobusu (w tym przypadku Prypeckiego). Gdyby nie znaleziony przypadkiem odłamek Duszy, Pablo pożegnał by się z lewą nogą.\\
Teraz cała frakcja miała opanowany każdy ,,wzór'' zachowania do każdej sytuacji -- znali całą Zonę niemal na wylot, miejsce każdej anomalii, położenie każdego wraku, każdej rury lub większego kawałka gruzu. S. ryzykowała wiele, przemierzając Strefę każdego dnia po ponad piętnaście godzin -- było to zwyczajowe Jej poznanie, które trwało trzy miesiące nieustannych tułaczek, nieprzespanych nocy i walk z mutantami oraz innymi stalkerami. Do stuprocentowej wiedzy na temat Zony podwładnym Teodora i Gavina zostało już tylko dokładne przeszukanie elektrowni jądrowej. Niestety szanse na to drugie były dość nikłe dopóki Monolit całkowicie się nie rozpadnie, nie wspominając o ilości wojska broniącego Sarkofagu.\\
Autobus drżał silnie i miało się wrażenie, że Powiew zaraz wywróci go na dach. Zgniłe siedzenia i podłoga autokaru pokryły się starymi liśćmi, pyłem i suchymi źdźbłami trawy.\\
Ted spośród całej gamy chaotycznych dźwięków zdołał ,,wyłowić'' odgłos radiostacji. Pochylił się nisko, niemal chowając głowę między kolana, po czym odebrał sygnał.
\sx Teodor?! -- okolicę wokół kryjówki zwiadowcy także wypełniała kakofonia wywołana przez anomalię. -- Śmigłowiec się zbliża! -- informator S. ledwie przebijał się głosem ponad szum.
\qd
Ten, ku zdziwieniu wszystkich, nagle ustał. Wszystko wróciło do normy -- najszybciej o tym fakcie zorientował się Pete, który z RPG w rękach w mgnieniu oka wybiegł z autokaru.\\
,,Pete szybko biega, Pete’a nikt nie dogoni.''

Dobiegł mnie dźwięk łopat wirnika tnących bezlitośnie powietrze. Z granatnikiem gotowym do strzału, opartym o prawe ramię, włączyłem zintegrowany z kombinezonem noktowizor. Jantar nabrał jasno fioletowej barwy, podobnie jak wiszące nad nim niebo. Teraz, wzmacniając blask gwiazd, mogłem bez trudu dostrzec nadlatujący śmigłowiec. Mi-24WP, u którego zamiast czterolufowego karabinu zamontowano działko 23mm. Oparłem swego RPG7 wyżej, po czym przyłożyłem oko do lunety. Mi leciał nisko, dziesięć metrów przede mną, przez co mogłem trafić go w dowolne miejsce. W momencie, w którym wzbijany przez wehikuł kurz zaczął stukać w mój kombinezon, odpaliłem pocisk.
% 
\podro{Rok 1978}
% 
\sx No, pokazuj, co tam masz! -- zachęcił Kamil podnieconym głosem.
\qd
Izaak rzucił niedbale karty na plandekę, ukazując cztery asy i króla.
\sx Ja pier*olę\3k -- Mikołaj gwizdnął z podziwem.
\qd
Wszyscy, prócz przebywającego w szpitalu Jonathana, grali w karty pod namiotem w lesie, który za kilka lat zostanie nazwany lasem czerwonym. Był to bardzo duży namiot, z czymś w rodzaju trzech pomieszczeń oddzielonych przegrodami. W jednej znajdowało się radio i prowizoryczna kuchnia -- stojak na patelnie, których były dwie wraz z przyczepionym u dołu palnikiem zasilanym na gaz. Największa, środkowa część namiotu zasiedlało pięć materacy -- to tutaj niekompletna ,,paczka'' przyjaciół przesiadywała wieczory. Spędzali je na długich, nie mających końca rozmowach na wszystkie tematy oraz na graniu, głównie w karty, kości, lub inną zabawę, którą ,,dasz radę wziąć ze sobą pod namiot''. Zwykle pierwszy ,,padał'' Leon, przyzwyczajony do wczesnego chodzenia spać -- od dwunastego roku życia zaczął biegać i zachowywać się jak sportsmen -- wstawał wcześnie rano, biegał, ćwiczył, generalnie, wielce dbał o kondycję i zdrowie. Największym Nocnym Markiem był Kamil -- nie widział on żadnych przeszkód, by nie przespać nocy, a
zmrużyć oko dopiero w południe następnego dnia, by obudzić się rześkim po dwóch godzinach snu. Była to jedna z wielu osobliwości, które posiadał, które jednak upodabniały go do reszty uznawanego za ,,nieodpowiednie'' towarzystwa. Teraz, o pierwszej w nocy, prócz niego na nogach trzymał się jeszcze Mikołaj, Izaak i Radek -- Leon zasnął już po jedenastej i przewrócony na prawy bok, zapewne śnił o Wielkim Kanionie (miał lęk wysokości, zaś ten konkretny sen nawiedzał go już od szóstego roku życia).\\
Stawką były, jak zwykle, zapałki -- Kamil od dawna przestrzegał przyjaciół przed poważnym hazardem, posługując się ojcem-bankrutem jako przykład. Przynajmniej do tej pory jego ostrzeżenia odnosiły skutek -- nikt z obecnych w namiocie, a także Jonathan, nie wygrali ani nie przegrali w tego typu zagrywkach żadnego pieniądza.\\
,,I oby tak dalej\3k'' -- powtarzał Leon co jakiś czas.\\
Po skończonej grze i zebraniu całej ,,puli'' przez Izaaka, Radek wrócił do bolesnego tematu.
\sx Masterton ostatnio dziwnie się zachowuje, no nie? -- spytał.
\qd
Izaak przewrócił oczyma.
\sx A jak ty byś się zachowywał, gdyby ktoś rozciął ci twarz? -- odparł. 
\qd
Miał rację -- w takim sytuacjach osobowość człowieka może się diametralnie zmienić, a nikt, nawet sama ofiara nie jest w stanie przewidzieć, jak.
\sx Jestem w stanie zrobić wszystko. Dosłownie wszystko, dla Jonathana. -- obwieścił Kamil. 
\qd
Spodziewano się takiej wypowiedzi -- Kamil zawsze dawał o sobie znać, że to on jest najbliższą Jonathanowi osobą zaraz po rodzinie. Nie wiadomo było jeszcze, co dokładnie zbliżyło tą dwójkę w tak dużym stopniu, lecz Radek, Leon, Izaak i Mikołaj mieli się już niedługo o tym dowiedzieć. Do tej pory pozostawały im jedynie domysły -- czy jeden uratował drugiego podczas napaści? Pożyczył pieniądze w naprawdę kryzysowej sytuacji? Ilość podejrzeń była nieograniczona, pozostawało więc jedynie oczekiwać prawdy.
\sx Zauważyłem coś innego, Kamil. -- zagadnął Izaak.
\xx Co masz na myśli?
\xx Ty zachowujesz się dziwnie. Jesteś zduszony, wyciszony, gapisz się w glebę jakby Bóg ci się na niej objawił. Żal? Poczucie winy? Czujesz się odpowiedzialny za to, co stało się Jonathanowi?
\xx Adrian! Tak, do ciebie przecież, ku*wa mówię! Choć no tu! -- Kamil nie czuł się pewnie w obecnej sytuacji. 
\qd
Nie wróżył sobie świetlanej przyszłości, jeśli jego plan nie wypali. Cóż, twierdził, że Adrian zalicza się mimo wszystko do grupy szpanerów -- takich, którzy są mocni tylko w gębie i pięści, innymi słowy -- jeśli takiemu się dobrze przygada, to wymięknie, tak samo, jak jego wielkie mięśnie.\\
Nazywający samego siebie tego dnia Pan Ryzykant niósł dżinsy, adidasy i białą koszulkę z krótkim rękawem. Adrian zaś miał na sobie koszulkę na ramiączkach oraz spodnie z dresu. Obaj byli podobnego wzrostu, jednak Kamil był o dwa lata młodszy.\\
,,Bez ryzyka nie ma zabawy!'' -- powiedział sobie w duchu.\\
Nie znali się -- gdyby Adrian wiedział, że Kamil kumpluje się z Jonathanem i resztą jego osiedla, na dzień dobry wybił by mu ząb. Nie pierwszy raz zresztą.\\
Zamiast tego serdecznego powitania, stanął on przed Kamilem jak słup soli, po czym zaczął mierzyć go morderczym spojrzeniem. Na szczęście dla Ryzykanta nie odniosło ono skutku -- Kamil, zanim przeprowadził się do Prypeci, mieszkał w innym podobnym miasteczku, które tworzyło zupełnie nowe znaczenie słowa ,,chuligaństwo''. Na ulicy ,,rządziła'' grupa zwykłych nastoletnich bandytów, okradających nie tylko sklepy i napadających na ludzi dla pieniędzy. Kamil mieszkał tam od urodzenia i widział nie jednego ,,kozaka'' -- w jego rodzinnej miejscowości, Adrian byłby pomniejszym frajerem -- tu, w Prypeci, próbował uczynić z nich między innymi Jonathana\\.
,,W moich snach, gnoju!'' -- rzekł Kamil do siebie, gdy dowiedział się od Jonathana, że Adrian go, jak to określił, gnębi.
\sx Życie ci nie miłe? -- spytał lekceważącym tonem Adrian.
\qd

Pęka.\\
Taki z niego twardziel, a wystarczyło, że nie zlałem się patrząc mu prosto w oczy.\\
Teraz to ja mierzyłem go spojrzeniem, które można było nazwać zabójczym. Teraz to on zaczynał być trzęsącym spodniami frajerem. Tam, gdzie się urodziłem, miałem wyjątkowe szczęście, będąc dobrym znajomym ,,od przedszkola'' Mariusza Kirowskiego -- chłopaka starszego ode mnie o trzy lata, który w pewnym stopniu przekonał swoich znajomych (czytaj -- resztę osiedlowych chuliganów) do mojego towarzystwa, przez co byłem świadkiem niejednego wybryku. Trzy razy doprowadziło to do poważnej bójki, miałem za sobą dwa tygodnie w poprawczaku, o czym jeszcze nikt poza Jonathanem nie wiedział.\\
Dziwi mnie, jak nietypowe mogą być powody stworzenia się takiej zażyłości.
\sx No?! -- głos Adriana był już widocznie zaniepokojony. 
\qd
W takich chwilach, jak spodziewał się Kamil (co widział kilka razy na własne oczy) tego typu osiłki przestają rozmawiać, za to zaczynają się bić. Do tego momentu brakowało niewiele -- Kamil chciał jeszcze trochę poucierać gnębicielowi nosa.
\sx Takiś mocny, co? -- zapytał sarkastycznym tonem.
\qd
Śmiałość.\\
Śmiałość zawsze łamie takich gnojków, którzy oczekują od innych, że zaczną płakać po podniesieniu przez nich tonu głosu.\\
Dobra, czas przejść do konkretów.\\
Założyłem ręce na piersi, jeszcze intensywniej świdrując stojącego przede mną nastolatka.
\sx Podobno zaczepiasz Jonathana Mastertona.
\qd
Adrian wydawał się oburzony.
\sx Właśnie! -- teraz jego głos był niemal paniczny. Podstawowa zasada -- nie daj po sobie poznać, że się boisz. -- Jonathan! To nawet nie jest ukraińskie imię, że o nazwisku nie wspomnę! -- wykrzyczał.
\qd
Zaczął zachowywać się jak hipokryta. Jak gość na przyjęciu, który próbuje przekonać otoczenie do kiepskiego kawału, obawiając się, że inni się nie zaśmieją. Nie myślał nad tym, co mówi -- plótł jęzorem na ślepo, próbując załagodzić sytuację albo zboczyć z tematu. Teraz jego jedynym (i tak mocnym) atutem była siła.\\
Fakt, Masterton nie był z Ukrainy tylko z zachodu, lecz to nie miało teraz nic do rzeczy.
\sx Pytałem, czy go gnębisz, a nie skąd pochodzi.
\xx A co cię to?! Zresztą, kim ty jesteś, żeby tak do mnie gadać, co?
\qd
Ostatnie próby uratowania swej godności.

Adrian wystrzelił prawą, sformowaną w pięść dłonią w kierunku nosa Kamila. Ten sprawnie uniknął ciosu, uskakując w prawo. Po chwili całą siłą, na jaką potrafił się zdobyć, kopnął Adriana prosto w krocze.\\
Oczy wyszły mu na wierch, a twarz nabrała komicznego choć i przerażającego wyrazu. Odruchowo złapał się za trafione miejsce obiema rękoma, jęcząc przeraźliwie i osuwając się na chodnik. Kamil złapał go za ucho i podniósł ogoloną głowę na wysokość jego pasa, po czym schylił się nisko i wyszeptał:
\sx Jeszcze raz go tkniesz, to będę cię kopał w jaja tak długo, że ci odlecą i będziesz musiał je zbierać z chodnika. Od dziś Jonathan Masterton jest dla ciebie nietykalny i święty -- jeśli spadnie mu choć włos z głowy\3k -- po wypowiedzeniu tych słów doprawił swą ofiarę jeszcze jednym kopniakiem, tym razem w czoło.
\qd

Dyszałem ciężko, patrząc na zwijającego się z bólu Adriana. Od drugiego tego typu występku mego autorstwa przyzwyczaiłem się do emocji, które wcześniej wywoływały u mnie drżenie nóg. Teraz, choć, co oczywiste, podekscytowany, byłem opanowany. Moje zachowanie ściągnęło na siebie spojrzenia przechodniów -- kilku przechodzących obok stadionu ludzi omijało mnie szerokim łukiem. Gdyby nie oni, zmusiłbym się do wymiotów wtykając do ust palca, zwracając rzecz jasna na leżącego pseudo twardziela. Wziąłem głęboki oddech i szybkim krokiem ruszyłem do bloku Jonathana, z wieścią, że jego problemy z Adrianem się skończyły.\\
Lecz miały się one dopiero zacząć.
% 
\podro{Rok 2009}
% 
\sx Cóż\3k -- rzekł Jonathan pocieszającym tonem, patrząc w bezchmurne niebo. -- Przynajmniej nie mogłeś się dłużej obwiniać\3k
\qd
% 
\podro{Rok 2001}
% 
\sx Po pierwszej piątce trupów reszta spękała. -- pochwalił się Gavin, paląc papierosa. -- Gdybyś tylko widział ich miny -- takie! -- Gavin wytrzeszczył oczy i w komiczny sposób rozszerzył usta, niczym wyjęta z wody ryba.
\qd
Siedzący na dachu wieżowca wybuchli śmiechem. Zamierzali spędzić w nim noc, a konkretnie na czwartym piętrze najwyższego w Prypeci budynku. Nieoficjalnie stanowił on główną kwaterę S. -- piętra powyżej trzeciego były zupełnie niepodobne do pozostałych -- Gavin starannie je odnowił, zamontował kilkanaście nowoczesnych, stalowych drzwi, a ściany dodatkowo opancerzył, podobnie jak okna. Tylko kilka pokoi nie uległo gruntownemu remontowi -- większość z pełnych gruzu pomieszczeń zamieniły się w rusznikarnie, laboratoria i centrum łączności, wręcz pęczniejące od sprzętu elektronicznego -- laptopów, telewizorów i wielkich komputerów, wszystkie z nich były zasilane artefaktami. Na najwyższym piętrze wieżowca zburzono ściany, upodabniając je do wielkiej sali, które było połączeniem jadalni i sypialni, choć, jeśli ktoś chciał przebywać sam, nie miał ku temu żadnych przeszkód -- każdy z członków S. miał do dyspozycji własny pokój.
\sx A jak wam poszło? -- zapytał Gavin po chwili.
\qd
Ted strzelił nadgarstkiem, po czym odpowiedział.
\sx Zestrzelony, na nasze szczęście, walnął w tą górę na południu, więc załoga nie miała żadnych szans.
\qd
% 
\ro{23}
% 
\sx Gdzie ty jesteś, do cholery? -- spytał podenerwowany John.
\qd
Czekał ponad minutę na odzew od Grahama. W tym czasie, jak zwykle zresztą, nerwowo rozglądał się na boki, wypatrując naukowców i wojska. Odpowiedź drugiego podwładnego Sussaro była następująca:
\sx Zwiedzam sobie centrum. -- nie tyle treść, co beztroski ton tej wypowiedzi doprowadził Finna do szału.
\xx Kurwa!- zaklął. -- Nie jesteś tu na wycieczce krajoznawczej, więc przestań się bawić w turystę i jak najszybciej tutaj przychodź! -- wykrzyczał.
\qd
W odpowiedzi usłyszał kilka uspokajających pomruków typu ,,Tak, tak\3k'' Westchnął głęboko, po czym oparł się o pniak najbliższego drzewa i założył ręce na piersi, wyczekując partnera.
\sx Wycieczkę sobie zrobił\3k -- mruknął pretensjonalnie.
\qd

Smutas.\\
Rodzi się druga Zona, a on mi nie pozwala mi obejrzeć centrum kilkanaście godzin po wybuchu. Mimo wszystko usłuchałem -- Finn może miał i gorsze relacje z Sussaro niż ja, ale nawet taka drobna sprzeczka mogła je pogorszyć. Sussaro należał do ludzi strasznie przewrażliwionych, także w kwestii wykonywania jego zleceń. Zaś po wybuchu tutejszej elektrowni, dostał wręcz obsesji na punkcie Mryńska. Promieniowanie.\\
Wszystko przez powstałe promieniowanie, którego, według wcześniejszych ustaleń, mogło być nawet dwa razy więcej, niż zdoła pomieścić Pochłaniacz. Naładowanie go w połowie pozwoliłoby Gomezowi na poszerzenie Strefy o kilka kilometrów, zaś całkowite zapełnienie artefaktu energią stanowiło prawdziwy szczyt marzeń. Wolałem nie myśleć, co zrobi Sussaro, kiedy dostanie do dyspozycji ,,pełnego'' Pochłaniacza. Szybko odpędziłem od siebie wizje, jaki kształt przybierze wtedy Zona, więc przyspieszyłem kroku i zacząłem skupiać swą uwagę na wymarłym mieście.\\
Mryńsk dość znacznie różnił się on Prypeci -- nowocześniejsza (czytaj -- nie ta rodem z ,,komuny'') zabudowa -- wyraźnie lepsze drogi i chodniki, teraz tonące w kolorowych papierkach i wstążkach, które przy mocniejszym wietrze lądowały także na pobliskich kamienicach.\\
Kamienice.\\
Tutaj to one stanowiły większość budynków mieszkalnych, w przeciwieństwie do tonących w blokach i wieżowcach Prypeci. Na ulicy, którą właśnie się przechadzałem, kamienice były szare, trochę zaniedbane i bardzo wysokie, przez co rzucały długie cienie, pogrążając centralną część Mryńska niemal w całkowitej ciemności. Była siedemnasta, zaś czułem się jak o dziesiątej w nocy.\\
Idąc, bacznie obserwowałem otaczające mnie budynki. Żadnej odrapanej farby, gruzu, pyłu czy śladów zniszczeń. Zdawało się, że wszyscy mieszkańcy gdzieś wyjechali, nie zaś, że uciekli w popłochu. Tutaj wrażenie zatrzymania się w czasie było jeszcze wyraźniejsze, niż w Prypeci.\\
Wiele okien było szeroko otwartych, ukazując białe sufity wnętrz. Przez nieliczne balkony i okiennice wywieszono pranie -- ku memu zdumieniu wiele z nich jeszcze nie wyschło.\\
Po wyjściu z wąskiej i zacienionej uliczki, znalazłem się na sporym placu. Miał kształt przybliżony do prostokąta -- w każdym jego rogu znajdowało się wejście do uliczki podobnej do tej, z której właśnie wyszedłem. Wyłożony został kostką brukową, która co kilkanaście metrów została urozmaicona niewielkim paskiem wypolerowanego granitu. Na całej powierzchni placu rosły drzewa, sadzone dwoma rzędami z pięciometrowym odstępem. Zaczynał się on w połowie placu, biegł z zachodu na wschód -- omawianą przestrzeń pomiędzy nimi wyłożono podobną kostką, lecz w znacznie innym kolorze i wzorze. Ziemia, z której wyrastało każde drzewo, była ogrodzona granitem, tworząc coś, co wyglądało jak kwadratowa doniczka. Na końcu pewnego rodzaju alejki ustawiono czarną furtkę, prowadzącą do niewielkiego kościoła.\\
Jego wielka wieża zegarowa wyrastała ponad korony drzew i dachy kamienic, była zbudowana z czerwonej cegły, sam zegar składał się z białej, oszlifowanej kamiennej tarczy i dwóch wielkich, czarnych wskazówek. Rzymskie cyfry wskazujące godziny wykonano z czarnego kamienia pomalowanego dodatkowo złotą farbą. Ten zegar późnił się trochę w stosunku do mojego -- wskazówki na kościele wskazywały za pięć siedemnastą.\\
Przez chwilę miałem ochotę uciec do Finna, obawiając się w pewien sposób dzwonów, które niewątpliwie zadzwonią, kiedy nastanie pełna godzina. Bałem się wielu dziwnych rzeczy, nabyłem wiele nietypowych fobii, ale nie mogłem dokładnie opisać, czego dokładnie obawiałem się, gdy w pustym, opustoszałym niedawno mieście rozlegną się dzwony kościelne. Przygnębienia? Załamania? Wizja, że kolejne duże, tętniące życiem miasto podzieliło losy Prypeci, niewątpliwie mnie smuciła.\\
Postanowiłem jednak, czym prędzej udać się do Johna i elektrowni. Skręciłem w prawo, szerokim łukiem mijając kościół. Po tej stronie placu, otaczały go kawiarnie, sklep z alkoholami, zaraz obok z tytoniami, zaś na jego końcu, u wylotu uliczki, jeszcze niedawno działała cukiernia. Skręcając w ulicę Ukrainki, omiotłem spojrzeniem wystawione przed sklep plastykowe stoliki i krzesła.\\
Pomyśleć, że jeszcze niecałe dwa dni temu były pełne zajadających się ciastami, pączkami i innymi słodyczami ludzi. Pod jednym ze stolików leżała przewrócona porcelanowa filiżanka, pełna zaschniętej kawy na dnie.
% 
\sx Poszły! -- warknął John zduszonym przez hełm głosem, strzelając ze swego P99 do stada psów. 
\qd
Zastanawiało go, czemu psy zachowują się w ten sposób w takich sytuacjach -- odczuły swobodę, nieobecność swych panów? Ich całkowite zdziczenie, (jeśli wcześniej wszystkie nie padną trupem przez skutki promieniowania) pozostało już tylko kwestią czasu.\\
Stado rozbiegło się w popłochu we wszystkie strony, kiedy jeden z jego uczestników został trafiony w głowę, która niemal pękła w pół w miejscu, gdzie wilczur miał oko. Przez moment zdawało się, że pysk zwisa w powietrzu na niewielkim płacie pokrytej sierścią skóry.\\
Z zarośli tuż za martwym kundlem wyszedł Graham, rozglądając się na boki. Wraz z Johnem nosił on szarawy kombinezon naukowca, z hermetyczną przesłonką na twarz, podobnej do tej, jaką miały pancerze SEVA. W kwestii ochrony przed promieniowaniem ten kombinezon był niekwestionowanym liderem, z którym cała grupa Sussara obchodziła się niemal z namaszczeniem, na co wpływ miał też sposób ich zdobycia. Mianowicie chodziło o sprzedaż trzech Dusz (zdobycie dwóch z nich wymagało napaści na bazę wojskową) naukowej organizacji, która miała swój wkład w odkrywaniu Zony. Odkryli oni i nazwali wiele artefaktów, anomalii i mutantów.\\
Całe te zamieszanie na szczęście się opłaciło.\\
John ujrzawszy idącego ku niemu Grahama, wyjął z plecaka Pochłaniacza w specjalnym pojemniku -- tym samym, w którym Graham umieścił artefakt ukryty w miejscu, gdzie Masterton zabił Johnsona. Gdy sięgnął ręką do otwierającego przycisku, powstrzymał go przed tym Graham.
\sx Jeszcze nie! -- niemal krzyknął. -- Dzwony.
\xx Jakie znowu dzwony? -- spytał John podenerwowanym głosem. 
\qd
Miał już zamiar dodatkowo ochrzanić swego towarzysza, jednak zmienił swoje zamiary, gdy pokazał on mu zegarek. Za kilka sekund miała nastąpić piąta.
\sx Dalej się ciebie to trzyma? -- dociekał Finn. -- Chociaż\3k -- zamyślił się głęboko. -- Też nie jestem pewien, jakbym zareagował. Wiesz, jak działa Pochłaniacz. Nieprzewidywalnie, jak alkohol i leki.
\qd
Kilka sekund po wyjęciu kilkumetrowej liny przez Johna, rozległo się pierwsze uderzenie dzwonu. Grahama przeszył zimny dreszcz. Wszystko, całe otoczenie i powstała sytuacja -- opuszczenie miasta przez mieszkańców, chłodne powietrze, szare, wręcz smutne niebo wraz z bijącym dzwonem kościelnym strasznie go dobijały. Gdyby ktoś nieświadom wybuchu przechadzał się nieopodal, myślałby o udających się do kościoła ludziach, jak i tych, których sprawy duchowe w ogóle nie interesowały.\\
Jedyna rzecz stwarzająca pozory funkcjonowania miasta dawała o sobie znać. Krótkie uderzenia i rozchodzące się we wszystkie strony głuche echo wypełniało nie tyle uszy, co myśli Grahama, trwało przez ponad pół minuty. Kiedy John upewnił się, że jemu asystentowi nic nie jest, wyjął Pochłaniacza z pojemnika. W momencie odchylenia przezroczystego wieka, Finn natychmiast poczuł się o wiele gorzej, jakby właśnie dowiedział się o śmierci bliskiej osoby. Nie mógł pomylić tego uczucia z żadnym innym -- żal, smutek dosłownie wypełniły go po brzegi. Uczucie było tak silne i rzeczywiste, że Finn niemal nie spytał Grahama: ,,Jak umarł?''\\
,,Weź się w garść!'' -- pomyślał. Zaczął sobie wmawiać, że to uczucie jest jedynie efektem działania artefaktu, że kiedy tylko zamknie go z powrotem w pojemniku, negatywne emocje znikną jak za dotknięciem czarodziejskiej różdżki. Kiedy w końcu uświadomił sobie, że nie umarł mu nikt bliski (zresztą, żadnego nigdy nie miał) połączył drut z pochłaniaczem za pomocą przyssawki, po czym odczepił od paska przeciwpancerny granat.
\sx Niedługo mają przysłać pierwsze oddziały, więc lepiej się sprężmy. -- rzekł, odbezpieczając ładunek.
\qd

Powoli, z ostrożnością sapera na polu minowym, ominąłem Elektro. Światło dobywające się z Blasku Księżyca dobywało się z niewielkiego dołku w glebie, tuż pod ścianą magazynów z wagonami, na wschód od centrum terenów przemysłowych. Od elektro po lewej dzieliło mnie niewiele ponad kilka centymetrów, byłem więc, powiedziawszy szczerze, o krok od śmiertelnego porażenia prądem. Jedynie odłamki Baterii, którymi byłem obwieszony pod koszulą, mogły dać mi szansę na oderwania się od ładunku elektryczności. Kucnąłem, wziąłem głęboki oddech, po czym odniosłem wrażenie, że całe życie przelatuje mi przed oczyma -- od młodości w Kijowie po zawitanie do Zony. Wciąż pamiętałem moment wjazdu na Jej tereny. Ogromne, ciągnące się po horyzont pola uprawne, gdzie jedynym urozmaiceniem od wszechobecnych traw była niewielka szosa. Po kilkugodzinnej jeździe, przejechaniu przez rządowy punkt kontrolny (dobre kilka kilometrów przed właściwą Strefą, gdzie nikt bez sporego wsparcia i arsenału broni palnej się nie zapuszczał -- w pewnym
momencie sam byłem zdziwiony, dlaczego) zacząłem sobie uświadamiać okropieństwo tego miejsca.
\swk[27em]
\texttt{Witamy w Zonie!\\ Miejscu, gdzie po paru dniach zaczniesz zabijać dla kawałka chleba, miejscu, które wypierze cię z emocji i wymaże z twej świadomości pojęcia takie jak ,,racjonalne'', ,,prawdopodobne'', ,,możliwe''. To nowy, inny świat, z którego za cholerę nie wyjdziesz, chyba, że założysz tu burdel, albo urwą ci nogę, więc, jeśli życie ci miłe, zawracaj i módl się, żebyś nigdy nawet nie pomyślał o tym szyldzie.}
\qwk
-- głosiły litery nabazgrane biała farbą na sporym kawałku drewna. Pod szyldem wykopano spory i głęboki dół. Zbliżyłem się do niego i omal nie upadłem z wrażenia po tym, co zobaczyłem w środku.\\
Setki, wręcz tysiące ludzkich zębów. Po zwróceniu niewielkiego śniadania, dostrzegłem jeszcze jedną tabliczkę. Pisało na niej:
\swk[27em]
\texttt{Każdy ząb to jeden głupiec, który stracił tu życie. Oczywiście nie każdy trzyma się tego zwyczaju -- te zęby to niewiele więcej niż połowa zmarłych tu stalkerów. Wciąż chcesz tutaj zostać? W takim razie miło było cię poznać.}
\qwk
Przełknąłem głośno ślinę i pozbierałem myśli. Musiałem, musiałem tu wejść. Nie było innego wyjścia. Wsiadłem do samochodu i kazałem memu kierowcy jechać.
\sx To jeszcze nic. -- mruknął.
\xx Co? -- spytałem zaniepokojony.
\xx Nic\3k zobaczysz.
\qd
Po dziesięciu minutach zobaczyłem.
\\
Ze stojącego przede mną pnia, na szubienicy wisiał człowiek. Musiał się zabić (albo ulec tym, którzy go zmusili) dawno, bo ciało było wyjątkowo przegniłe, a w paru miejscach było widać już gołe kości. Nie mogłem dokładnie opisać tego okropnego widoku -- powierzchni mięsa, ilości mieszających się z sobą obrzydliwych wydzielin, a także niewątpliwego przesłania, które miał rozpowszechniać wisielec. Jego ręce ułożono wzdłuż boków -- trzymał w nich kolejną tabliczkę. Trudno było nazwać to trzymaniem -- kawałek drewna przybito do palców nieszczęśnika gwoździami.
\swk[12em]\texttt{Skończysz tak samo\3k}\qwk
\sx Hej, ty! -- obcy głos zza pleców wyrwał mnie z zamyślenia.
\qd
Odruchowo wyrwałem mego USP Experta z kabury i wycelowałem w przybysza. Ten z kolei mierzył do mnie Browningiem HP. Miał na sobie SEVA’ę, z odsłoniętą twarzą. Patrząc na nią i sposób, w jaki kombinezon układał się na jego sylwetce, można było dojść do oczywistego wniosku. Był chudy.\\
Niewiarygodnie chudy.\\
Wszystkie kości sterczały w niesamowity, wręcz nienaturalny sposób.
\sx Nic z tego! -- krzyknąłem. -- To jest mój artefakt, byłem tu pierwszy!
\qd
Blady chudzielec prychnął, po czym powiedział:
\sx Nie interesuje mnie twój artefakt, widzisz\3k -- po tych słowach odchylił głowę do tyłu w bok, ukazując spore rozcięcie na kościstej szyi.
\qd
Świeże rozcięcie, z którego jeszcze niedawno płynęła krew. Teraz skrzepła i szpeciła szyję stalkera. 
\sx Nie powiodła mi się walka z mutantem, a mam takiego pecha, że\3k -- znów skierował głowę, a tym samym swe spojrzenie na mnie. -- Zgubiłem środek dezynfekujący. Nieszczęśliwy wypadek, który mógł się nawet zdarzyć Sus\3k każdemu.
\qd
Dobrze, że przynajmniej nie osępi mnie na Blask Księżyca. Wtłukłem zbyt wielu ludziom, żeby tak po prostu zrezygnować. Wciąż celując w chudzielca, lewą ręką sięgnąłem do wielkiej kieszeni na lewym udzie. Otworzyłem ją, po czym wymacałem w jej wnętrzu zminiaturyzowaną apteczkę, z której wyjąłem niewielką buteleczkę. Po zamknięciu kieszeni, rzuciłem środkiem oczyszczającym ponad Elektro, w stronę nieznajomego, który zręcznie ją chwycił.\\
Przytrzymując flakon kciukiem, uniósł w moją stronę rękę w geście wdzięczności, po czym odwrócił się i z pistoletem u boku ruszył w stronę przejścia do Rostoka. Odetchnąłem z ulgą, kiedy zniknął mi z oczu. Położyłem pistolet na ziemi, kucnąłem i wyciągnąłem rękę w niewielkie zagłębienie w ziemi, w którym radośnie świecił Blask Księżyca. Pewność ogromnej zapłaty i wizja posiadania nowego karabinu jeszcze bardziej mnie zmotywowały.

Zapukałem do przyczepy Adama.
\sx Wejść! -- rozległo się zza drzwiczek.
\qd
Pociągnąłem klamkę, otworzyłem szeroko drzwi, po czym wszedłem do środka, zamykając je za sobą. Adam siedział na kanapie i oglądał telewizję. Wnętrze bez wątpienia można było nazwać schludnym -- dywan, obicia ścianek i sufitu, meble i sprzęty kuchenne były czyste i zadbane.
\sx Jakież to wspaniałości przynosisz mi dzisiaj, o Michale? -- zawołał do mnie Adam, nie odrywając oczu od kineskopu. 
\qd
Sprzedawałem mu różności od dawna, tak często, że prawie się zaprzyjaźniliśmy. Jednak przejście na ,,ty'' w zupełności mi wystarczyło -- uprzejmie, miło, bez niepotrzebnego włażenia w dupę i klepania w plecy przy drinku. Zrzuciłem z siebie torbę wraz z kombinezonem, pod którym miałem oliwkowy sweter i cienkie spodnie. Schyliłem się nisko, po czym wyjąłem swą zdobycz z plecaka. Lazurowy blask natychmiast wypełnił pomieszczenie, z racji późnej pory i faktu, że dopóki nie wyjąłem Blasku Księżyca, jedynym źródłem światła w przyczepie był telewizor. Handlarz bronią natychmiast zauważył zmianę.
\sx Noo, w końcu jakieś konkrety! -- rzekł z uznaniem, zacierając ręce. -- Co tak stoisz? -- spytał uprzejmie. -- Siadaj. -- ruchem głowy wskazał boczne siedzenie kanapy.
\qd
Z artefaktem w dłoniach, zająłem miejsce obok Adama.
\sx Słyszałem\3k -- zaczął. -- Że wkurza cię ten twój, nie oszukujmy się, złom.
\qd
Miał rację -- przez ostatni tydzień zabiłem trzy mutanty, zaś strzeliłem ze swego zdezelowanego M16 dwa razy. Strzały służyły przestraszeniu pewnego informatora, jak więc zabiłem mutanty? Powiedzmy, że częściowo, bo przez szczęście, obaliłem tezę wielu ekspertów twierdzących, że kolba M16 może pęknąć przy mocnym uderzeniu. Karabin, którym posługiwałem się od przybycia do Zony był zwyczajnie stary, miał też kilka usterek, ale bez wątpienia nie wyrzucę go na śmietnik. Miałem sentyment do wielu rzeczy, w tym broni, planowałem, więc, po zakwaterowaniu się u Powinności, ładnie go wyeksponować -- zapewne wstawię go w gablotę.\\
Potrzebowałem czegoś nowoczesnego, najlepiej egzemplarz typu ,,prosto z taśmy produkcyjnej''. W idealnym stanie.
\sx Proponuję ci snajperskiego Gaila. Oraz coś naprawdę\3k sam zobaczysz.
,,Może być.'' -- pomyślałem. Gail był dobrą bronią, w pełni mnie zadowalał. 
\qd
Bardziej interesowało mnie to, co Adam zapewne chciał określić mianem niespodzianki.
\sx Umieram z ciekawości. -- powiedziałem całkiem szczerze, starając się, by w mym głosie nie dało się wyczuć wazeliniarskiej nuty. 
\qd
Na te słowa Adam jeszcze raz zatarł ręce, po czym zanurkował głową i rękoma pod stolik. Usłyszałem jakiś łoskot, po chwili dźwięk odskakującej deski. Adam wynurzył się z spod stołu i położył na stole teczkę z aktami. Była dość nowa, miała biały kolor i zamykano ją na mocny rzep. Pękała w szwach.
\sx Słyszałeś o tej grupie naukowców, jak im tam\3k Autorach?
\xx Ci, którzy jako pierwsi wjechali do Zony? -- spytałem.
\xx Tak. I zarazem ci, którzy\3k -- Adam zawiesił głos i dał mi znak, żebym dokończył za niego zdanie.
\qd
Zamyśliłem się na moment, po czym odparłem, a raczej odpowiedziałem pytaniem na pytanie:
\sx \3kzginęli przez wejściem do Czerwonego Lasu?
\qd
Adam patrzył na mnie przez chwilę, po czym wybuchł śmiechem tak głośnym i gwałtownym, że musiałem zatkać prawe ucho. Śmiał się przez dwie sekundy, wystawiając moją cierpliwość na próbę. Kiedy umilkł i otarł łzę (albo zrobił to, by podkreślić, jak bardzo go bawię) jego mina pozostała dalej rozbawiona, przyozdobiona szerokim uśmiechem.
\sx Co ty? W Boga wierzysz? -- zapytał z lekką dozą cynizmu. -- Każdy tak sądzi, dopóki nie przeczyta tych akt. Powiem ci to w skrócie.
\qd
Założyłem nogę na nogę i z wyraźnym podnieceniem oczekiwałem dalszych słów Adama.
\sx To ich zapiski, dokładnie z 87-ego roku. Autentyczne, wierz mi na słowo. Według ich\3k cała grupa Autorów ginie w Prypeci, a dokładnie w samym jej centrum, na placu.
\qd
Więc jednak przeszli przez Las. Co tam spotkali? Jakie nieznane nam anomalie i mutanty? Cholera, czy tam w ogóle coś żyje? Sam fakt, że przeprawili się przez Czerwony Las dawał wiele do myślenia. Po wydarzeniu w 97-ym roku chyba nikt się nie zapuszczał w tamte strony.\\
Gwałtownie otworzyłem teczkę i ujrzałem ostatnią rzecz, jakiej bym się spodziewał.\\
Mnóstwo zapakowanych płyt DVD.
\sx Masz u siebie odtwarzacz? -- spytał mnie Handlarz, widocznie widząc w mych oczach coś w rodzaju żądzy wiedzy.
\xx Musiałbyś mi pożyczyć\3k -- mruknąłem, wpatrując się tępym wzrokiem w zawartość teczki.
\xx Jasne, jasne\3k powiem ci trzy rzeczy i dam ci go na czas ,,nieokreślony'' byś wszystko dokładnie sobie przestudiował. Pierwsza -- pewnie zastanawiasz się, dlaczego daję to akurat tobie. Stalkerów w Zonie są niezliczone rzesze, a ja dołączam ciebie do wąskiej grupy, że tak powiem, wtajemniczonych. Powiem ci dlaczego -- spośród tych wszystkich samotników narażających życie dla kasy, którą im daję, ty jesteś najostrożniejszy. Większość handlujących ze mną nie żyje ze zwykłej głupoty, wiadomo, jak to wypadki chodzą po ludziach w Strefie. Nikt nie czuje się bezpieczny, a przeciwko tobie jest wszystko -- od anomalii po mutanty, na chciwych bandytach skończywszy. Mam wrażenie\3k Nie, jestem pewien, że ty jako jeden z nielicznych tu gości dokonasz czegoś naprawdę wielkiego -- zwiedzisz elektrownię, pokonasz Monolit, coś z tego kalibru. Nie skończysz jako anonimowy krzyż wbity w glebę na jakimś wzgórzu -- pewnego dnia zrobisz coś, co naprawdę odmieni twoje życie, a kiedy już inni się o tym dowiedzą, okrzykną cię
prawdziwą legendą. Poznanie prawdziwej historii Autorów to twój pierwszy krok -- oby następne były podobne. Dwie pozostałe rzeczy, to coś, co nazywam zwiększaniem apetytu.
\qd
Siedziałem oniemiały, wciąż wbijając się tępym spojrzeniem w Adama. Naprawdę wierzyłem w to, co mówi.
\sx Nie chcę mi się liczyć wszystkich anomalii, które są powszechnie znane, ale jest ich około dwudziestu. Autorzy spisali ich dokładnie czterdzieści sześć. Słyszałeś zapewne o miejskim szpitalu. Autorzy byli w środku, przemierzyli każdy metr poza piwnicą. Ich zapiski do chwili wejścia do tego budynku urywają się, w lipcu. Następny, a zarazem ostatni wpis ma datę dwudziestego trzeciego września, a brzmi mniej więcej tak -- Autorzy wychodzą ze szpitala, biegną na plac w centrum Prypeci, po czym giną. Jak i ,,dlaczego'' -- sam zobaczysz.
\qd

Lenny ostatniej nocy znów śnił o Sammym.\\
Obudził się gwałtownie, lekko spocony, dokładnie o dziewiątej rano.\\ Przyzwyczaił się on już w pewnym stopniu do bólu dobywającego się ze złamanej nogi, który był teraz znacznie mniej uciążliwy. Leonard podniósł się z łóżka, oparł głowę o lewą dłoń i zaczął rozmyślać, co właściwie stało się dnia, w którym porwano Barry’ego Jeffersona. Pamiętał ten ,,wybuch'' -- ogłuszająco głośny trzask i oślepiające, wszechobecne światło, tak jak i niewielkie promieniowanie, które w wyniku tajemniczej eksplozji powstało. W momencie ujrzenia blasku, Lenny stracił przytomność -- w tej samej chwili, w której miał ostrzelać obóz pociskami zapalającymi, dokładnie w tej samej sekundzie, w której odpalono podłożone u podstaw wież strażniczych ładunki wybuchowe.\\
Kiedy się ocknął\3k Przypomniał sobie. Po radioaktywnym wybuchu Leonard wciąż leżał w krzakach (kryjówce, którą dzień wcześniej znalazł Jonathan. Zdrowy Jonathan.) z karabinem u boku. Z niewiadomych przyczyn jego radio było odczepione od paska i leżało przez nim o dwa metry, przysypane kilkoma grudkami ziemi. Po chwili wydobył się z niej głos Marka. Wzywał on snajpera, zapewne zaniepokojony brakiem jakichkolwiek śladów pocisków, które miał wystrzelać.\\
Lenny zaczął się powoli czołgać w stronę słuchawki, zatrzymując się co kilkanaście centymetrów, by podjąć próbę jej sięgnięcia. Po dziesięciu sekundach w końcu dopiął celu.

Obróciłem się na plecy i pochyliłem nisko, z rękoma (w prawej trzymałem krótkofalówkę) wciąż partymi o ziemię, wziąłem głęboki oddech. Targały mną koszmarne mdłości, kręciło mi się w głowie, a w uszach wciąż echem odbijały się dźwięki spowodowane wybuchem. To z pewnością nie było Zwarcie. W chwili naciśnięcia przeze mnie przycisku odpowiedzialnego za wysłanie wiadomości, w krzaki wskoczył rozszalały Snork.\\
Odruchowo wrzasnąłem i wypuściłem krótkofalówkę z ręki.\\
Snork wyczuł albo dostrzegł, w jakiej pozycji siedzę, bo zakończył swój nieludzko długi skok potężnym kopnięciem końcem swej stopy w moją prawą łydkę, trafiając w sam środek kości piszczelowej. Zamarłem, słysząc dźwięk pęknięcia. Nie wiem co -- zapewne adrenalina -- o tym przesądziło, ale udało mi się przemóc ból.\\
Po kopnięciu, Snork chwycił mnie obiema rękoma za ramiona i przeskoczył mi nad głową. Wciąż trzymając swe pokryte skrzepniętą krwią łapskami, pociągnął mnie nimi mocno w dół. Z głuchym łomotem uderzyłem i tak już obolałymi plecami w twardą glebę, krzywiąc się boleśnie. W tym czasie mutant, człapiąc niechlujnie, obrócił się o sto osiemdziesiąt stopni, znowu będąc naprzeciwko mnie. Sapnął, a raczej prychnął niczym byk przed szarżą, po czym znów ruszył w moją stronę. Po drodze wypadł nieco z rytmu, uderzając lewą dłonią o M82, tłukąc szkiełko w zamontowanej na nim lunecie.\\
Podniosłem lekko głowę i próbowałem przewidzieć, w jakiej pozie stwór dokona następnego skoku, próbując w końcu mnie wykończyć.\\
Odzyskując wcześniejsze tempo, Snork znów wybił się w powietrze.\\
Leciał z przekrzywioną głową i wyciągniętymi przed siebie rękoma, rycząc przeraźliwie.\\
Poderwałem się do góry i chwyciłem potwora za jego przerażającą, wykrzywioną w grymasie niepohamowanej żądzy zabijania twarz.\\
Udało mi się zmienić kierunek, w którym mutant miał zamiar polecieć, a jako, że włożył w to bardzo dużo siły, dość łatwo udało mi się obalić go na ziemię. Gwałtownie przekręciłem dłonie.\\
Puściłem bezwładne ciało, po czym odetchnąłem z wyraźną, przeogromną ulgą. Ciężko dyszałem, ale i tak byłem o wiele spokojniejszy. Wszystkie zapasowe rezerwy energii, w tym adrenalina, powoli mnie opuszczały, o czym świadczył też ból w złamanym piszczelu. Narastał stopniowo, podobno jak każde miejsce, w które zostałem uderzony w przeciągu ostatnich kilku godzin. Zamarzyłem o lekach przeciwbólowych.\\
Wezwałem Marka przez krótkofalówkę.

Lenny obrócił się, podniósł leżące u nóg łóżka kule i już po chwili kuśtykał nimi w stronę kuchni. Wciąż nosił na sobie pidżamę -- wróżył sobie taki stan aż do dnia zagojenia się złamania.
Drzwi były otwarte -- przeszedł przez próg, oparł lewą kulę o aneks, by wolną ręką przysunąć do niego taboret. Po ułożeniu obu kuli na podłogę, Lenny usiadł na nim bokiem, po czym zaczął szykować sobie śniadanie. Wyciągnął się i otworzył lodówkę. Wyjął z niej trzy plasterki żółtego sera, kilka kawałków pokrojonego małosolnego ogórka. Chleb, schowany w pojemniku opartego o bok lodówki, wylądował na aneksie, zaraz przy kostce masła leżącej na małym talerzyku. Leonard posmarował chleb, ułożył na nim ser z ogórkiem, po czym zabrał się do jedzenia, oczekując przy okazji herbaty -- woda już się gotowała.\\
Po skończeniu pierwszej kromki, Lenny przysunął się wraz z taboretem w stronę okna. Jego kwatera była na pierwszym piętrze budynku, miał więc bardzo dobry widok na plac obozu.
\\
Widział  stąd też korony drzew wzbijające się ponad mur. O tej porze panował niewielki ruch, nie licząc ganku tutejszego baru. Siedziało przy nim pięć osób -- Lenny rozpoznał wśród nich Aleksa, Damiana i Radka -- pozostała dwójka stała plecami do Leonarda. Pomiędzy ubranymi w skórzane płaszcze agentami toczyła się ożywiona dyskusja, w której najaktywniejszy udział brał Radek.\\
W kuchni rozległ się gwizd czajnika. Lenny wyłączył gaz i zalał szklankę z torebką miętowej herbaty wrzątkiem. Wkrótce silny, miętowy aromat wypełnił pokój. Oczekując, aż napój wystygnie, Lenny wrócił do lektury książki, którą zaczął czytać od wczorajszego późnego wieczora, jakiś czas po założeniu gipsu. ,,Observatory Mansion'' Edwarda Carey’a zapowiadała się bardzo obiecująco.\\
Po skończeniu rozdziału, Lenny napił się herbaty, po czym skończył śniadanie. Spojrzał na gips, czując zadowolenie i ulgę z racji tego, że złamanie było niezbyt groźne i kość szybko się zrośnie. Następne pięć minut Leonard spędził na rozmyślaniach -- o dealerze narkotyków, swojej żonie i Sammym. Następne dwadzieścia minut myślał o Jonathanie.\\
Ostatnie dni drastycznie różniły się od pozostałych -- głównie za sprawą zachowań Mastertona, którego stan ze stabilnego zmienił się gwałtownie w krytyczny. Wszystko zaczęło się od Johnsona, pogoni Johna F. za Leonem i Mikołajem, od tego, co spotkało Radka. Ale tak naprawdę wszystko zaczęło się od Prypeci -- od roku 78-ego wszystkie sprawy przybrały zupełnie inny obrót, zaś zajście w Prypeckim szpitalu udowodniły, że mimo wszystko może być jeszcze gorzej. Szpital, jego piwnica, Gomez i Kamil.\\
Teraz, po wybuchu w elektrowni, powrót do szpitala był śmiertelnie niebezpieczny, lecz w głębi duszy Lenny wciąż czuł, że prędzej czy później wyląduje w klinice wraz ze swymi przyjaciółmi. Był zaś pewien, że tam wszystko się skończy.

Graham trzymał oburącz stalową linkę, która biegła w górę, ku dachowi elektrowni atomowej, podczas gdy John przy pomocy laptopa operował niewielką kamerą. Jej obiektyw został umieszczony przez Finna w otworze wywierconym w ścianie budynku -- mógł być on ręcznie obracany za pomocą programu komputerowego, którego John włączył na swym przenośnym komputerze kilka minut temu. Jego monitor ukazywał wnętrze elektrowni oraz przyczepiony do liny Pochłaniacz, który sunął powoli w dół, obniżany przez Grahama.
\sx I jak? -- spytał.
\xx Jeszcze trochę. -- rzekł John Finn.
\qd
Kellerman westchnął i poluzował uchwyt, tym samym opuszczając artefakt w dół. Mimo ,,świeżości'' Mryńska wolał on jednak starszą, ,,poczciwszą'' Zonę na terenie Czarnobyla. Wiedział jednak, że to Mryńsk i jego okolice będą wkrótce nowym terenem działań jego, Johna i Sussaro. Było niemal pewne, że w przeciągu kilku tygodni powstaną tu nowe anomalie, artefakty, a nawet mutanty. Gomez miał zamiar ,,poeksperymentować'' na drugiej Zonie, głównie używając do tego Pochłaniacza -- był ciekaw efektów, jakie wywołają Wybuchy i Cykle na obszarze, w którym tak niedawno doszło do radioaktywnego skażenia.
Na razie wiedział tylko, że nowo powstałe artefakty znacznie przybliżą go do odkrycia położenia Złotej Kuli. Oko Stwórcy, Pochłaniacz, Złota Kula -- te trzy artefakty były mu potrzebne, by bezpiecznie przemieszczać się po szpitalu. Drugim szpitalu.
\sx Już! -- krzyk Johna wyrwał Grahama z zamyślenia. Chwycił on drut z całej siły, co uchroniło Pochłaniacz od zderzenia z kafelkami w elektrowni. -- Teraz go połóż.
\qd
Graham usłuchał, po czym rzucił linę na ziemię.
\sx Ile to potrwa? -- zapytał.
\xx Jakąś godzinę, potem wracamy.
\qd

Gomez przysiadł się do starego, obsypanego tynkiem stolika. Znajdował się w Sali nr 6 Prypeckiej szkoły podstawowej nr 2. Pomieszczenie to służyło niegdyś uczniom placówki na kształcenie się z dziedziny historii. Oczywiście odpowiednio przeforsowanej przez radzieckich ,,specjalistów'' w tym zakresie.
Sporych rozmiarów klasa tonęła w postrzępionych kawałkach drewna (niegdyś stanowiących ławki -- wciąż dało się dostrzec wyryte cyrklem inicjały, pseudonimy uczniów) i gruzie, któremu towarzyszył biały pył i chmury tynku. Ściany i sufit były w opłakanym stanie -- farba łuszczyła się ogromnymi płatami, niczym skóra gada, ukazując goły beton. Wszystkie osiem wysokich okien zostało niemal całkowicie wybitych -- przez ich pozostałości z gwizdem wiał wiatr, który co jakiś czas powiewał stronnicami porozrzucanych po sali książek i podręczników szkolnych.\\
Zza framugi wyważonych drzwi klasy, z szkolnego korytarza rozlegały się odgłosy stawianych kroków. Narastał z sekundy na sekundę -- w końcu u progu drzwi stanął Sussaro.\\
Nosił na sobie czarny, sięgający ziemi płaszcz z długim kapturem, który miał obecnie zsunięty na plecy. Pod tym nietypowym strojem nałożony był Całun.
Sussaro oparł się o zakurzoną ścianę.
\sx Aleksander Gomez. -- mruknął pod nosem.
\qd
Gomez wstał z ławki i uśmiechnął się ironicznie.
\sx Obecny! -- powiedział głośno udawanym tonem ucznia wywołanego przez nauczyciela. -- Jak im idzie w Mryńsku?
\qd
Sussaro sięgnął prawą dłonią pod płaszcz i nacisnął przycisk. Dyktafonu. Ciągle nosił przy sobie ten cholerny dyktafon, na który nagrywał większość słów, które wypowiedział przez swoje krótkie życie. Co, co znajdowało się na taśmach, było o wiele bardziej wartościowsze niż najpotężniejszy artefakt. No, może z wyjątkiem Złotej Kuli. Po włączeniu nagrywania Sussaro powiedział:
\sx W pełni naładowany Pochłaniacz, zapowiadana burza, sytuacja Jonathana. Czego chcieć więcej? -- po krótkiej chwili milczenia dodał. -- Pamiętasz, jak szukałeś następnego? Te podszywanie się pod psychologa i takie tam? Ale opłaciło się, co?
\qd
Gomez uśmiechnął się pod nosem, przypomniawszy sobie 86-ty rok. Bynajmniej nie z powodu wybuchu w elektrowni.
\sx O tak. -- powiedział pełnym zadowolenia głosem. -- Opłaciło się, i to bardzo.
\qd

\sx Michał! -- krzyknął ktoś do mnie w chwili, gdy zamykałem pokój w kwaterze dowodzenia Wolności.\qd
Chwilę później uderzył w stojące przede mną drzwi, wytrącając mi z rąk teczkę z aktami na temat Autorów klucze i odtwarzacz. Zawartość teczki przyozdobiła podłogę, wypluwając z siebie kartki, zdjęcia i stosy zapakowanych płyt DVD. Sprzęt RTV z hukiem grzmotnął o podłogę, cudem nie łamiąc się w pół.\\
Kopnąłem w drzwi ze złości, trafiając nimi, jak się okazało, Dawida -- jednego z podobnych do mnie stalkerów zaprzyjaźnionych z Wolnością, tych, którzy sypiali na terenie magazynów wojskowych za niewielką opłatą. Dawid specjalizował się w zbieraniu trofeów -- spodziewałem się też, jak się okazało słusznie, że właśnie w tej sprawie do mnie przylazł.
\sx Przy śmigłowcu widziałem piękną Pijawę! -- krzyknął do mnie podekscytowanym tonem, masując trafione drzwiami, spocone czoło. 
\qd
Dawid miał dwadzieścia osiem lat, ponad metr siedemdziesiąt wzrostu i niemal całkowicie łysą głowę. Był sporej postury i dobrej budowy ciała z wyjątkiem nabytego w wyniku nadmiernego spożywania piwa brzucha. Odstawał od niego niczym włożona pod koszulkę piłka do koszykówki.
\\
Schyliłem się i zacząłem zbierać porozrzucane akta, próbując odgonić od siebie wizję zapolowania na mutanta. Powstałe w wyniku rozmowy z Adamem podniecenie nie dawało o sobie zapomnieć. Miałem ochotę jak najprędzej wyprosić (a jeśli to nie poskutkuje, to ,,wywalić na zbity pysk'') Dawida, zamknąć się na klucz, zdemontować dzwonek i do rana chłonąć treść dokumentów. Mimo to, znając Dawida, byłem pewien sporych zarobków, które od niego otrzymam, jeśli mu pomogę. Próbując stłamsić w sobie ciekawość i głód wiedzy, szybko odłożyłem pozbierane papiery z nośnikami na kanapę. Dawid miał zapewne zamiar mnie o coś spytać, lecz uciszyłem go gestem ręki i pobiegłem do kuchni.
\\
Szerokim zamachem, niemal nie wyważywszy drzwiczek, otworzyłem lodówkę. Zostawiwszy ją otwartą, wyjąłem z niej butelkę wódki -- pociągnąłem łyk i naprędce zakręciłem ją z powrotem, po czym niedbale rzuciłem na drugą półkę lodówki, którą równie niedbale zamknąłem.
\\
Wciąż mając w ustach smak alkoholu, wróciłem do salonu. Dawid czekał z założonymi na piersi rękoma.
\\
Spojrzałem na ułożoną z desek podłogę i z całej siły kopnąłem tą z wyrytymi nań nożem dwoma kreskami. Odskoczyła ona na wysokość pół metra, po czym z głośnym trzaskiem upadła na pozostałe, ukazując drewnianą, błyszczącą kolbę karabinu. Kucnąłem i powoli, uważając, by nie zahaczyć o wciąż ,,sztywne'' deski, wydobyłem swojego SWD. Wstałem na równe nogi, odciągnąłem zamek, sprawdzając, czy nabój jest w komorze. Był. Sprawdziłem także magazynek, upewniwszy się, iż jest on pełen.
\\
Zabezpieczyłem karabin i zarzuciłem go sobie na plecy przy pomocy oliwkowego, wytartego materiałowego paska.
\sx Byle szybko\3k -- mruknąłem do Dawida, pokazując mu gestem dłoni, żeby wyszedł z pokoju. 
\qd
Kiedy to zrobił, ruszyłem za nim i zamknąłem drzwi na klucz, które schowałem w kieszeni kombinezonu.
\sx Na tobie zawsze można polegać. -- rzekł do mnie Dawid z uznaniem, przeciągając się szeroko.
\qd 

{\noindent\em Szpital.
\\
Autorzy.
\\
Nowe anomalie i artefakty.
\\
Miesiąc by błądzili po klinice?}
\sx Kurwa, gdzie ona jest?! -- spytałem zdenerwowany, opierając się o drzewo.
Dawid oderwał oczy od lornetki i spojrzał na mnie zdumiony.
\xx Spokojnie. -- rzekł lekko przestraszony. -- Prędzej czy później nas wywęszy. -- wycedził, po czym wrócił do obserwacji horyzontu. Niemal wyczytałem w jego myślach słowa ,,Psychol\3k''.
\xx Przepraszam. -- powiedziałem po chwili. -- Po prostu mam coś bardzo, ale to bardzo ważnego do załatwienia i nie mogę się tego doczekać.
\xx W porządku\3k Hmm\3k -- Dawid przejechał palcami po pokrętłach, wytężając wzrok. -- Generalnie -- co słychać?
\qd
Z wielkim trudem powstrzymałem się od wzmianki na temat nietypowej formy zapłaty, którą wręczył mi dziś Adam.
\sx Bez zmian, ciągle to samo. No, może z wyjątkiem Mryńska.
\xx Ta\3k Wciąż trudno mi\3k -- nastąpiła chwila przerwy, podczas której Dawid rozważał dobranie odpowiedniego słowa. -- Przyjąć to do wiadomości. Myślisz, że wyjdzie z tego druga Zona?
\xx Zapewne. Ale znając doświadczenia naszego rządu z Czarnobylem, nie dadzą pewnie się tam rozwinąć stalkerostwu. Poślą tam pół swojej armii i nikogo nie wpuszczą. Zagarną wszystko dla siebie, a nam każą się ślinić za szklaną szybą.
\xx A artefakty, anomalie i mutanty? Tam one też powstaną?
\xx Być może\3k -- wziąłem głęboki oddech.
\xx To niewielkie miasto, podobno promieniowanie nie rozeszło się za daleko.
\xx Wojskowi pewnie zajmą tam każdy budynek, a całość ogrodzą murem.
\xx Widzę ją. Na drugiej.
\qd
Ponownie przeniosłem się z rozmyślań o centrum Prypeci i placu, na którym zginęli najbardziej zasłużeni stalkerzy w historii do ,,prawdziwego świata''. Przywołałem sobie z pamięci obraz tarczy zegara z perspektywy Dawida i skierowałem lunetę w tamtym kierunku. Tak, niewątpliwie. Wyuczenie się na ,,blaszkę'' gdzie jest, która godzina i tym samym natychmiastowa reakcja, to jedna z rzeczy, której musiałem się nauczyć w najbliższym czasie. Na tle innych bez tej podstawowej umiejętności wyglądam jak nieudacznik i zwykła, nie warta zachodu oferma.\\
A przeszedłem nie jedno.\\
Niemal nie do odróżnienia na tle bujnej roślinności mutant zastygł w chwili, w której skierowałem na niego swe spojrzenie. Wciąż pozostawał pod działaniem ,,kamuflażu'', którego nie dało się określić inaczej niż ,,Taki, jaki miał Predator.'' Na podstawie zniekształconej przez mutanta powierzchni roślin i pnia drzewa oceniłem jego wzrost na około dwa metry. Dawid ma przed sobą nieprzespaną noc przebytą na patroszeniu.\\
Pijawka znowu ruszyła do swego charakterystycznego marszu, tym razem prosto w stronę moją i Dawida. Dzieliło nas około trzystu metrów.\\
Wciąż nie odrywając oka od lunety, którą bez przerwy podążałem za Pijawką, spytałem:
\sx Gdzie strzelać?
\xx Na pewno nie w łeb. Jeśli możesz, w środek klatki piersiowej. Jeśli nie, odstrzel jej nogi.
\qd
Znowu nabrałem powietrza we płuca. Tym razem wydawało mi się, że było go o wiele mniej niż poprzednim razem. Zawsze miałem takie wrażenie, kiedy się denerwowałem -- podobnie zresztą jak wydzielanie żółci. Dlatego przezywano mnie czasem ,,Gilbert''. W momentach silnego stresu, a czasem tylko lekkiego podenerwowania obficie się pociłem, oddechy stawały się krótsze z sekundy na sekundę -- zdawało mi się, że siedzę w ciasnym pomieszczeniu, chociażbym stał na polu uprawnym, to zdenerwowany będę się czuł jak w szybie wentylacyjnym. Dlatego miałem spore trudności z zabieraniem się na ,,wycieczki'' -- większość stalkerów miało mnie za niezrównoważonego. Jednak stawałem się coraz bardziej opanowany -- była to głównie zaleta leków, ale lepsze to niż nic.
\\
Przełączyłem bezpiecznik.
\\
Gdy mutant zbliżył się do odległości stu metrów i ,,zdjął'' kamuflaż, strzeliłem.
\\
Dzięki pozycji i własnym usprawnieniom niemal w ogóle nie poczułem odrzutu i podrzutu. Dźwięk wyrzucanej z zamka, dymiącej łuski został zagłuszony przez odbijające się o korony drzew echo wystrzału. Dobiegł mnie jedynie krótki, niemal niemożliwy do usłyszenia syk, powstały w momencie zetknięcia się gorącej łuski z chłodnym, źdźbłem trawy.
\\
W okolicy mostka Pijawki rozkwitła niewielka chmurka ciemnej krwi -- kiedy jej drobne kropelki opadły na ziemię, mutant padł gwałtownie na plecy.
\sx Byle szybko.
\qd
Już niedługo się go pozbędę i w spokoju przejrzę akta. Życie jest niewątpliwie piękne.
\\
Podniosłem się, założyłem zabezpieczony już karabin na plecy i wyjąłem z kabury Rudera -- mały, bogato zdobiony przez znajomego mi rusznikarza, małokalibrowy pistolet, którym zawsze dobijałem mutanty. Od czasu, kiedy rzekomo martwa Chimera omal nie odgryzła mi ręki, zawsze nosiłem przy sobie tą, strzelającą pociskami kalibru .22 cala, broń. Siły wyższe -- w tym przypadku ta część mózgu, której nie kontrolowałem -- kazały mi od tamtej pory dobijać każdego mutanta strzałem w serce.
\\
Ten, którego przed chwilą ustrzeliłem, leżał bez ruchu z prawą ręką komicznie wyrzuconą nad głowę, patrząc się tępo w niebo gasnącymi ślepiami. Wyciągnąłem przed siebie pistolet, przymrużyłem lewe oko i strzeliłem potworowi w serce -- pod takim kątem, by wlot drugiego pocisku był jak najbliżej otworu po kuli SWD -- ważny był każdy, nienaruszony centymetr wartościowego, brązowego futra.
\\
Po strzale Pijawka wykonała coś w rodzaju westchnięcia.
Dawid założył na głowie Pijawki pętle z grubego sznura, którego dwumetrowy pęd przewiesił sobie przez ramię. Ja w tym czasie ułożyłem pod trupem twardą, odporną plandekę, którą spiąłem nad obrośniętą mackami głową -- każdego upolowanego mutanta Dawid wraz ze mną zabezpieczał w ten sposób, by podczas ich ciągnięcia po ziemi nie poharatać pleców.
\\
Schodząc ze wzgórza, Dawid wolną ręką bawił się swym nożem. Na przemian chował i obnażał ostrze przy pomocy ponacinanego pionowo przycisku.
\sx Co z Jonathanem? -- tak brzmiały moje pierwsze słowa po tym, kiedy ocknąłem się w podziemiach obozu.
\qd
Miałem obolały brzuch i żebra, poza tym kręciło mi się w głowie. Leżałem na rozłożonym, skórzanym fotelu -- przebrano mnie w koszulę bez ramion i luźne spodnie. Podniosłem głowę i ujrzałem siedzącego przede mną Mikołaja. Obok niego, z nogą w gipsie opartą o drewniane krzesło, usadowił się Lenny, nad którym, oparty o jego lewe ramię, stał Izaak. Pod ścianą stała stara kanapa -- siedział na niej Mark i Radek.
\\
Od bardzo wielu lat w tego typu sytuacjach brakowało mi Kamila, ale tym razem, chyba po raz pierwszy w życiu, bardzo zatęskniłem za Mastertonem. Miałem z związku z nim spore obawy, które w pewnym stopniu rozwiał Radek, odpowiadając na moje niedawne pytanie.
\sx Harlan się nim opiekuje z jakimś lekarzem. Jak mu tam\3k -- Radek zaczął kręcić nadgarstkiem, co, według niego, pomagało mu się skoncentrować. -- Pawłem. -- rzekł ożywionym głosem, kiedy w końcu przypomniał sobie imię naukowca.
\qd
Zdołałem przemóc ból w żebrach i podniosłem się nieco, przysunąłem do tyłu i oparłem plecami o miękkie obicie fotela. 
\sx Jerome? -- wyjąkałem. -- Czy Jerome go\3k -- zdążyłem wydukać te słowa, zanim przerwał mi Radek.
\xx Już wrzuciliśmy go do anomalii. -- zapewnił mnie. 
\qd
Odruchowo spojrzałem na Marka i ujrzałem, jak krzywi się z wyrzutem sumienia. Przez moment poczułem chęć, by spytać go, skąd u niego takie przewrażliwienie na punkcie zabijania, a tym samym, po co przybył do Zony z takim ,,urazem''. Zamiast tego spytałem:
\sx Możesz na chwilę wyjść?
\qd
Mark zrozumiał, że to pytanie jest wyraźnie skierowane do niego -- kiwnął głową ze zrozumieniem, westchnął cicho, podniósł się na równe nogi i wolnym krokiem wyszedł z pomieszczenia, zamykając za sobą dźwiękoszczelne drzwi.
\sx Coś jest nie tak. Coś naprawdę nie gra. -- wyszeptałem. To, co powiedział Lenny, całkowicie zbiło mnie z tropu.
\xx Musimy iść do szpitala. Do piwnicy. -- po krótkiej chwili ciszy Leonard powtórzył na głos swoją poranną, stworzoną rano, myśl. -- To tam się wszystko zaczęło. Tam i w Prypeci. Co wam się ostatnio śniło? Mi Sammy.
\qd
Każdy z obecnych w pokoju stalkerów, prócz Lenny’ego, popadł w zadumę.
\sx Prypeć przed wybuchem. -- oznajmił Mikołaj.
\xx Mi też. -- powiedział lekko zdezorientowany Radek.
\xx Mi\3k to samo\3k -- wydusiłem z trudem.
\xx Ja ostatnio prawie w ogóle nie sypiam. -- rzekł Izaak. -- Ostatni sen, jaki pamiętam, to Jonathan wybiegający ze szpitala. Nigdy tego nie zapomnę\3k
\qd
Mikołaj wstał z krzesła, okrążył pokój dookoła, po czym stanął jak wryty i powiedział:
\sx Radek. Załatw mapę Prypeci i Ukrainy. Jutro idziemy do centrum. Mam pewien pomysł.
\qd
% 
\podro{Rok 1986}
% 
\sx Po co oni tam w ogóle poszli? -- spytał Leon, paląc papierosa.
\qd
Radek w odpowiedzi wzruszył ramionami.
\sx Za cholerę nie wiem. -- mruknął, po czym zerknął na zegarek. Musiał użyć palącego się papierosa jako źródła światła. -- Siedzą tam już piętnaście minut.
\qd
Radek z Leonem i Mikołajem stali na placu w centrum Prypeci, pod schodami prowadzącymi na wyższy chodnik, z którego wchodziło się do miejskiego szpitala. Była noc -- cicha, przerywana przez nielicznych przechodniów, z których większość właśnie wracała już do domu. Leon skończył papierosa i wrzucił niedopałek do śmietnika po wcześniejszym jego zgnieceniu.
\sx Coś mi tu nie gra. -- powiedział zaniepokojony. -- Zaczekaj tu.
\qd
Leon obrócił się, powiewając swoją skórzaną kurtką i szybkim krokiem ruszył w stronę kliniki. Wchodząc na schody, odchrząknął i splunął w stronę trawnika, po czym otarł usta lewą dłonią i wspiął się po ostatnich trzech stopniach. Będąc kilka centymetrów od drzwi, wyciągnął prawą dłoń ku drzwiom.\\
Nagle wyskoczył z nich Jonathan, powalając Leona barkiem na ziemię.\\
Jego ubranie -- czarna marynarka i spodnie -- było postrzępione i osmalone w kilku miejscach. Miał zakrwawioną twarz i ręce, którymi miotał wściekle we wszystkie strony. Biegł, a raczej człapał nisko pochylony, niczym ranny żołnierz, który zauważył idącą mu na pomoc kompanię.\\
Jednak najbardziej w pamięci Izaaka, Leona i Mikołaja zapadł jęk -- nie, ryk i rozpaczliwy wrzask, które dobywały się z gardła Mastertona. Nieludzkie, przywodzące na myśl zdziczałe zwierzęta wycie zwróciło uwagę przechodniów, którzy ze strachem i zdumieniem w oczach obserwowali pokrytego krwią Jonathana.\\
Nie zwrócił uwagi na schody -- stoczył się z nich, ciągle rycząc, lądując boleśnie na plecach. Przez chwilę wierzgał się w szaleńczy sposób, lecz po chwili przewrócił się na brzuch i, kuśtykając, ruszył przed siebie. Jego krok stawał się bardziej chwiejny z każdym pokonanym metrem -- skończył się w chwili dopadnięcia go przez Leona. Złapał go on oburącz za brzuch i powalił na ziemię.\\
Jonathan był bardzo silny fizycznie, lecz widocznie w tej chwili siły go opuściły -- wciąż będąc w uścisku Leona, zdołał jedynie doczołgać się do przodu o ponad metr. Kiedy to zrobił i żałośnie zawył, wyciągnął przed siebie lewą rękę. Trudno to było nazwać zwykłym wyciągnięciem -- kończyna Mastertona dosłownie wystrzeliła w powietrze, jakby jej właściciel próbował coś złapać -- owada czy upadającą monetę.\\
Palce zaciskały się i rozwierały bez przerwy, jakby Masterton usiłował pochwycić klamkę, która co chwilę mu się wyślizgiwała. Gapił się w niebo opętańczym, nieobecnym spojrzeniem -- coś, co próbował w wyobraźni złapać, najwyraźniej było bardzo daleko.
\\
Ręka wraz z resztą ciała Jonathana nagle zesztywniały i upadły -- umilkł też nieludzki wrzask i szloch, którego był sprawcą.
\\
Cała trójka jego przyjaciół stała nad nim, próbując znaleźć racjonalne wyjaśnienie dla tego typu zachowania. Byli tak zszokowani, że odjęło im mowę i było stać ich jedynie na wpatrywanie się w nieprzytomnego Mastertona. Przez ten krótki moment byli oszalali z rozpaczy. Czuli się okropnie -- mimo wszystkiego, co przeszli z Jonathanem do tej pory, ta sytuacja dosłownie ich zamurowała.
\\
Jonathan otworzył czerwone od posoki powieki i zapłakał.\\
Chwilę później w czarnobylskiej elektrowni jądrowej doszło do eksplozji.
Jonathan ponownie uniósł swoją rękę, tym razem powoli i ostrożnie -- wskazał ten sam kierunek, co przed paroma chwilami, po czym znowu stracił przytomność.
% 
\podro{Rok 2001}
% 
\sx Dobra. -- zapewnił Radek. -- Ale najpierw wam coś powiem. Mark! -- krzyknął.
Ciężkie drzwi uchyliły się z cichym piskiem. Wychylił się zza nich Mark.
\xx Już? -- spytał. 
\qd
Kiedy Leon skinął mu głową, otworzył drzwi szerzej i wszedł do pokoju, zajmując miejsce Mikołaja, który stał oparty pod ścianą, rzucając we wszystkie strony podejrzliwe spojrzenia.
\\
,,Do Prypeci\3k'' -- pomyślał. Po chwili się odezwał, nie mogąc dłużej tłumić swoich wątpliwości.
\sx Do Prypeci? -- spytał tonem niedowiarka, któremu właśnie opowiedziano wątpliwą historię o duchach. -- Mamy dwudziesty trzeci -- Monolit znowu zaczyna patrole. A tak w ogóle -- jak nasze punkty kontrolne w Czerwonym?
\qd
Radek odchrząknął.
\sx Właśnie między innymi o tym chciałem z wami pogadać. Rano rozmawiałem z Damianem, Aleksem i jakimś samotnikiem, który określił miejsce tego wczorajszego wybuchu. Cokolwiek to było, to walnęło to na środku jednej z tych dróg prowadzących do Prypeci. O dziwo, o co zapewnili mnie ci z Wolności, całe promieniowanie powstałe w tej eksplozji zwyczajnie wyparowało, nikt nawet nie zginął. Po prostu pół Zony puściło wielkiego pawia. A punkty na szczęście zwinęły się na czas. Po co w ogóle chcesz iść do Prypeci, Mikołaj?
\xx Słyszałeś Lenny’ego. -- odpowiedział stalker. -- W tym szpitalu coś jest.
\xx No coś takiego! -- Radek machnął rękoma, niczym naukowiec krzyczący ,,Eureka!'' -- Co za odkrycie, kiedy Ameryka?
\xx Nigdy nie chciałeś wejść do środka? W sumie, to jestem zdziwiony, że po 86 nasza stopa nigdy tam nie postała.
\xx Gdyby nie ilość artefaktów, jaką wtedy mieliśmy. -- powiedział Leon. -- Nie miałbym czym stąpać. Ten sposób działania anomalii mnie przerasta.
\qd
Widząc minę Mikołaja, Leon kontynuował:
\sx Przecież próbowaliśmy tam wejść. -- powiedział pretensjonalnie. -- Zresztą nie tylko my. Niejeden sławny stalker próbował i co? Teraz ich zmielone szczątki są przenoszone razem z wiatrem. Tam się nie da wejść! -- krzyknął, zrywając się na równe nogi. Spojrzał na Marka rozwścieczonymi oczyma. -- Ty! Ty podobno miałeś go ,,zabezpieczyć'' -- wyrecytował udawanym tonem urzędnika. -- Zabezpieczyć. Co to, ma ku*wa znaczyć?! I kto w ogóle kazał ci to zrobić? Nasz rząd? Do 1999-ego posłał do szpitala sześćdziesięciu dziewięciu ludzi, i nikt z niego nie wrócił! Nie wiadomo nawet, czy ktoś dostał się do środka. Jesteś Straceńcem. Co takiego żeś przeskrobał, że kazali ci tam zdechnąć? Powiedzieli ,,Zginiesz, ale możesz spróbować wejść do szpitala, jeśli dowiesz się, co jest w piwnicy, darujemy ci życie.''. Byłeś już kiedyś chociaż dziesięć metrów od szpitala, Mark?
\qd
Spytany stalker, choć zdziwiony i wyraźnie obrażony, nie dał po sobie tego poznać. Stał niewzruszony, mierząc Leona wzrokiem. Powoli, nie odrywając od niego oczu, pokręcił głową.
\\
Leon, jakby spodziewając się tej odpowiedzi, powiedział:
\sx Będziesz żałował, że stworzyłeś sobie okazję, by to zrobić. -- po tych słowach wyraźnie ochłonął.
\qd
Westchnął z wysiłkiem i oparł się nisko o obłażący z farby mur. Powoli usiadł i oparł ręce o kolana, chowając głowę pomiędzy nimi.
\sx Dlaczego posłali kolejnego? -- spytał, nie zmieniając pozycji.
\qd
Mark otarł czoło lewą ręką, po czym założył ręce na piersi.
\sx Dokumentacja Autorów. -- rzekł. -- Ona istnieje, a ja wiem, gdzie jest.
\qd
Izaak nawet nie próbował ukrywać zdumienia. Chciał (podejrzewał, że jego towarzysze też) rozwiać wszelkie wątpliwości, zanim da się do końca ponieść emocjom.
\sx Żeby wszystko było jasne. -- wyszeptał. Głos wyraźnie mu drżał. -- Mówimy o tych samych aktach? O tym, co spisali Autorzy, pierwsi w Zonie?
\qd
Mark skinął głową.
\sx Z tym, co zawierają te akta, zdobycie szpitala staje się\3k -- zawiesił głos, czując, jakby miał zaraz obwieścić wyjątkową, niesamowitą rzecz, która zmieni dotychczasowe oblicze Zony. -- \3kmożliwe.
\qd
Minęło trochę czasu, podczas którego emocje opadły. Kiedy to nastąpiło, Mikołaj ponowił prośbę.
\sx Dajcie mi mapę Prypeci, kompas i mapę Ukrainy. -- powiedział. -- O\3k
\xx Czyli komputer Pawła. -- wtrącił Radek, uśmiechając się pod nosem. -- Zaraz przyniosę. -- rzekł, idąc w kierunku drzwi.
\qd

Uporządkowałem wszystkie, zapisane wyjątkowo drobnym drukiem, kartki -- od strony pierwszej do czterdziestej drugiej. Były z bardzo dobrego (i niewątpliwie bardzo drogiego) papieru, który od strony jedenastej, co kilka stron, zdobiony był krwią. Najwidoczniej Autorzy walczyli przez połowę tworzenia swej ,,twórczości'' podczas swego pobytu w Zonie. Pisali i nagrywali, nie zważając na ataki mutantów i prześladowania przez wojsko.
Po ułożeniu gotowych do czytania i uporządkowanych akt, przejrzałem kilka płyt.\\
Były one podpisane czarnym markerem -- wiele z nazw zaczynało się słowem ,,Pierwszy'' -- ,,Pierwsza Anomalia'', ,,Pierwszy Snork'', ,,Pierwsza Pijawka''. Co najmniej cztery płyty oznaczono słowem ,,Zwarcie''.\\
Płyta pierwsza -- ,,Zwarcie 1-7''.\\
Płyta druga -- ,,Zwarcie 7-16''.\\
Płyta trzecia i czwarta -- ,,Zwarcie 17-23'' i ,,24-31''.\\
Po prostu nie mogłem się doczekać.\\
Przejrzałem jeszcze sześć krążków -- ,,Elektro'', ,,Snork nr 3'', ,,Zdjęcia Porównawcze 1-7'', ,,Zdjęcia Porównawcze 17-23'', ,,Obserwacje Burzy -- 2'',\\ ,,Rzeka''.\\
Rzeka.
Czyżby jedna z nowych anomalii? Nieznany artefakt? Wkrótce się dowiem.
Usiadłem wygodnie na fotelu -- na stoliku obok mnie leżał stos ułożonych kartek, obok których stała bogato zdobiona filiżanka z kawą.
\sx W końcu\3k -- wyszeptałem niemal błogim, pełnym zadowolenia głosem, sięgając po pierwszą stronę.
\qd
Kompletny raport z eksploracji Zony.

Pod tym tytułem znajdowała się szczegółowa historia wybuchu w Czarnoblskiej elektrowni -- o nieudanym teście bezpieczeństwa, ewakuacji ludności, Likwidatorach, utrzymywaniu wieści o katastrofie przez rząd i pierwszych niepokojących zjawiskach na terenie, który wkrótce nazwano Zoną.W ostatnich zdaniach pierwszego ,,rozdziału'' najczęściej pojawiającymi się słowami były synonimy wyrazu ,,zaprzeczenie''.
\\
,,\3kzaprzeczenie prawom grawitacji\3k''
\\
,,\3ksprzeczność ze wszystkim, do czego przyzwyczailiśmy się przez wszystkie lata bycia na Ziemi\3k''
\\
Ostatnie zdanie, które było czymś w rodzaju podsumowania, brzmiało:
\\
,,Wszystkie znane nam prawa fizyki, grawitacji i związane z nimi teorie, nie mają tu nic do gadania.''
%
\ro{25}
%
Wchodząc do pokoju z laptopem pod pachą, spytałem:
\sx A gdzie Mark? \qd
Zdążyłem zamknąć za sobą drzwi na wszystkie trzy, stalowe zasuwy i antywłamaniowy zamek, zanim usłyszałem odpowiedź Leona.
\sx Odesłałem go, prędzej czy później i tak byśmy go wyprosili. Chciałem to mieć za sobą. -- powiedział.
\qd
Usiadłem wygodnie na krześle, położyłem komputer na kolanach, otworzyłem go i uruchomiłem. Kiedy startował system, podłączyłem do komputera kartę zapewniającą dostęp do Internetu. Sprawdziłem łączność, po czym przez przeglądarkę wszedłem na stronę z szczegółowymi mapami niemal całego globu. W niewielkie okienko wpisałem słowo ,,Prypeć''.
\\
Wirtualny wizerunek planety obrócił się o kilkadziesiąt stopni w lewo, po czym strona automatycznie zwiększyła powiększenie, ukazując me rodzime miasteczko. Zwiększyłem procentowy wskaźnik powiększenia na dziewięćdziesiąt -- na tej mapie zaznaczone było każde drzewo, każdy blok i niegdyś sprawna instytucja. Wliczając w to szpital.
\\
Serwis ten współpracował z wieloma rządami, którym potrzebne były szczegółowe, satelitarne zdjęcia, wliczając w to rząd Ukraiński. Podając odpowiednie hasło, mogłem uzyskać dostęp do zdjęć o poziomie szczegółów i jakości, które nadawałyby się do przewodników turystycznych. Na razie zrezygnowałem -- obecny plan robotniczej miejscowości i czyjś dobry pomysł, jakim było umieszczenie opcji pozwalającej na zaznaczenie linią dowolnego dystansu i przeliczenie go na jakąkolwiek jednostkę długości, pewnie wystarczyły.
\sx Jeśli dobrze zgaduję. -- zacząłem. -- Mam wskazać okolice szpitala?
\qd
Ucieszyłem się, widząc skinięcie Mikołaja. Oddaliłem nieco obszar i na moment zerknąłem poza monitor, widząc zainteresowanie w oczach pozostałych osób obecnych w pokoju. Wszystkich, prócz Mikołaja, który miał minę pacjenta czekającego na wyniki badania obecności na jakąś śmiertelną chorobę. Podejrzewałem, że jeśli jego przypuszczenia się potwierdzą, będzie w szoku. My pewnie też.
\sx Co dalej? -- spytałem po nakierowaniu obrazu na plac w centrum Prypeci.
\qd
Mikołaj sięgnął po pilota i włączył wiszący w rogu pokoju telewizor. Przeskoczył przez pięć kanałów informacyjnych -- na wszystkich wręcz trąbiono o Mryńsku.
\sx Mikołaj. -- powiedziałem zniecierpliwiony.
\qd
Ten jakby ocknął się z głębokiego zamyślenia, po czym odparł:
\sx Dzień wybuchu. Pamiętasz, co zrobił wtedy Jonathan?
\qd
Miałem wrażenie, że ktoś wymierza mi mocny cios w policzek. Powrócił żal i poczucie winy. Poczucie, którym ja, Leon, Izaak, Mikołaj i Lenny do dziś darzą Jonathana. Wiedział, co stało się w szpitalu, i nigdy z nikim nie podzielił się tą wiedzą.
\\
Wczoraj Izaak opowiedział mi, jaki kit wcisnął on Markowi na temat Jonathana, w dniu, w którym też nowoprzybyły poznał Psychola. Bajeczki o bankach, organizacjach najemników, nieludzkich umiejętnościach strzeleckich i poziomie opanowania walki wręcz. O tym, jak przebył świat długi i szeroki, zostawiając po sobie stosy trupów.
\\
,,Kiedyś może zasłuży.'' -- powiedział mi wtedy Izaak, kiedy przestałem się panicznie śmiać.
\\
Do tej pory Mark żył w przekonaniu, że Jonathan Masterton to zwykły zabijaka i rzeźnik. Najlepiej będzie, jeśli tak pozostanie. Chyba, że Mark ,,zasłuży''.
\\
Ani on, ani ja, jednym słowem nikt do tej pory nie przekonał Jonathana do wyjawienia prawdy o tym, co stało się w szpitalu dnia, kiedy nastąpiła atomowa eksplozja. Ostatnia tego próba nastąpiła W południe jedenastego kwietnia 1998-ego roku. Po tym, jak Jonathan omal mnie do siebie nie upodobnił pod względem przeciętego oka, przestaliśmy go wypytywać o tamten dzień.
\\
Kolejne dwie minuty upłynęły na śledzeniu kierunku wyznaczonego przed laty przez Jonathana.
\\
Las.
\\
Jakaś rzeka.
\\
W pewnej chwili mapa zaczęła wyświetlać granicę niewielkiego, położonego na dość zalesionym terenie, miasteczka. Oddaliłem nieco widok, by ujrzeć nazwę miejscowości.\\
Mryńsk.

Zastukałem do mieszkania 21 dwa razy, lewą ręką, używając środka palca wskazującego. Oczekując na to, aż Damian otworzy mi drzwi, spojrzałem na lewo. Zza okna klatki schodowej padał obfity i puszysty śnieg, który przysłonił szarawe niebo wiszące nad Luksemburgiem. Odwinąłem kołnierz uniformu listonosza i spojrzałem na srebrny zegarek -- była siódma wieczorem. Przystąpiłem z nogi na nogę, znowu wpatrując się w antywłamaniowe, drewniane drzwi. Z zewnętrznej strony judasza wynikało, że w przedpokoju mieszkania paliło się światło. Po chwili zostało przysłonięte, zapewne przez Damiana.
Odkaszlnąłem i przybrałem twarz w mój typowy, wzbudzający zaufanie uśmiech. Usłyszałem dźwięk przekręcanego kluczyka w zamku, któremu towarzyszył odgłos odsuwanego rygla. Po zdjęciu łańcucha drzwi otworzyły się, ukazując stojącego za nimi człowieka i wąski, pokryty ściennymi panelami i czerwonym dywanem na podłodze, przedpokój. Dalej, zapewne w salonie, grał włączony telewizor, ustawiony na kanał informacyjny.
\sx Tak? -- spytał Damian ochrypłym głosem.
\qd
Miał sześćdziesiąt dwa lata, metr siedemdziesiąt wzrostu i szczupłą sylwetkę. Nosił na sobie zielony, wełniany szlafrok i włochate, czerwone kapcie, na nosie zaś oparte były czarne okulary o grubych, prostokątnych, przezroczystych szkłach, zza których bacznie przyglądały mi się duże, brązowe oczy.
\\
Jego skóra była blada, pokryta ciemnobrązowymi plamami i dużą ilością zmarszczek, zwłaszcza w okolicach ust i na czole.
\\
Odpowiedziałem mu w chwili, w której zaczął lewą dłonią gładzić się po błyszczącej łysinie, oblizując usta.
\sx Przesyłka dla Pana, Panie\3k Damianie, tak?
\xx Tak, to ja. -- rzekł, uchyliwszy drzwi szerzej, co pozwoliło mu zauważyć wielkie, wysokie ponad na metr, szerokie pudło pokryte pocztowym papierem. 
\qd
Nie krył radości.
\sx Ktoś najwyraźniej mnie kocha. -- rzekł zadowolony, uśmiechając się.
\qd
Zapewne.
\\
Ktoś bardzo cię kocha.
\sx Od kogo to? -- spytał mnie, wciąż uśmiechnięty.
\xx Od brata. -- odpowiedziałem. -- Pachnie kwiatami.
\xx W końcu temu staremu pierdzielowi przeszło\3k -- zachichotał ochryple. -- Podpisać?
\qd
Wyjąłem spod prawego ramienia kwit pocztowy, oparty o mały kawałek drewna z uchwytem na długopis. Przybliżyłem się do Damiana, pozwalając mu dosięgnąć ręką mego Wattermana. Wyjął go z uchwytu i wciąż nie wychodząc zza progu swego mieszkania, podpisał się imieniem i nazwiskiem. Podał mi długopis, który schowałem w przyszytej na lewej piersi uniformu kieszeni.
\\
Starzec postawił lewą nogę poza próg domu, schylając się po przesyłkę.
\sx Nie, nie\3k -- powiedziałem przepraszająco. -- Pomogę Panu. -- uśmiechnąłem się szeroko, po czym wziąłem pudło w obie ręce, opierając je o brzuch.
\xx Dziękuję bardzo. -- powiedział zadowolony Damian, wchodząc do pokoju. 
\qd
Ruszyłem za nim, oglądając się zza wielkiej przesyłki na boki.
\\
Szeroki na dwa metry i długi na półtora, przedpokój, był obficie obwieszony wszelkiej maści obrazami, przedstawiającymi morskie i górskie krajobrazy. Największy z nich, umieszczony w pięknej, złoconej oprawie, ukazywał wielki, dziewiętnastowieczny statek towarowy zmagający się ze silnym morskim sztormem. Sposób oddania szczegółów i przedstawienie przesłaniającego widok deszczu wzbudzały uznanie malarza, który wykonał ów malunek.
\\
Markowski doczłapał, szurając kapciami o dywan, do salonu, wpuszczając mnie w jego głąb. Potem odwrócił się przez plecy, wrócił do drzwi wyjściowych i zamknął je, kiedy wrócił, usiadł na beżowym fotelu, po czym założył nogę na nogę.
\sx Połóż na stole, młodzieńcze. -- rzekł cicho.
\qd
Spełniłem jego prośbę -- powoli, uważając, by nie ściągnąć obrusu, ustawiłem pudło na dużym, prostokątnym stoliku z dębu. Westchnąłem głęboko i wyprostowałem plecy, ziewając głośno.
\sx Trudny dzień w pracy? -- spytał mnie Damian.
\xx Taa\3k -- westchnąłem, drapiąc się po szyi. -- Ale to już moja ostatnia robota na dzisiaj.
\xx Może usiądziesz? -- siedzący na fotelu przede mną wskazał na ustawioną pod ścianą sofę.
\xx Chętnie, jeśli można\3k -- rzekłem, obchodząc stolik dookoła.
\qd
Salon miał około czterech metrów kwadratowych -- naprzeciwko sofy, na której usiadłem, znajdowało się wejście do kuchni, obecnie zamknięte staromodnymi drzwiami z półprzezroczystym, zniekształcającym obraz szkłem. Wyposażenie pokoju było skromne -- prócz rozkładanej sofy, na której właśnie zająłem miejsce i fotelu ,,Marka'', stołu i kolejnego pięknego obrazu wiszącego mi nad głową, resztę wyposażenia salonu stanowił telewizor, ustawiony na lewo od wejścia do kuchni kredens i szafa. Wykonano je z tego samego dębu, co stół, co pasowało z drewnianymi panelami i pomalowanym na beżowo sufitem, nie wspominając o dobrze dobranym dywanie.
\sx Pana brat polecił mi, by po zdjęciu tego całego zabezpieczającego syfu, od razu rozpakował pan zawartość.
\xx Chętnie. Proszę się nie gniewać, ale chciałbym jak najszybciej zostać sam.
\xx Ależ nie ma sprawy. -- moje słowa brzmiały jak przeprosiny. 
\qd
Wyjąłem z prawej kieszeni niebieskich spodni nóż do kartonu, wysunąłem jego ostrze i przysunąwszy stół do siebie, przeciąłem karton z góry na dół. Położyłem nóż na stole, obróciłem paczkę o sto osiemdziesiąt stopni, po czym powtórzyłem ostatnią czynność. Wsunąłem ostrze z powrotem w plastikową rękojeść, schowałem go do kieszeni, tym razem lewej, po czym, znów lekko pochylony, rozwarłem przecięty karton na bok.
\\
Oczom Damiana ,,Marka'' Markowskiego ukazało się wysokie na metr i nieco chudsze od kartonu plastikowe, białe pudełko, po środku którego umieszczony był plastikowy, czarny przycisk. Przez jego położenie biegło złączenie pudełka -- naciśnięcie go pozwalało w pewnym sensie przepołowienie opakowania na dwie części, które opadały bezwładnie na boki, ukazując zawartość.\\
Pod przyciskiem napisano czerwonymi literami ,,Nacisnąć''.\\
Szczyt perfidii.\\
Damian nie mógł się powstrzymać. Zerwał się z fotela, z trudem obrócił pudło w swoją stronę i nacisnął przycisk, łamiąc ,,otoczkę'' przesyłki na dwie części, które opadły na dywan.\\
Z miejsca, na którym siedziałem, dostrzegłem jedynie tył ramki na zdjęcia opartej o wazon, z którego sterczał bukiet róż. Czarnych róż.\\
Skupiłem się jednak na twarzy starca, której wyraz diametralnie się zmienił -- uśmiech wyparował w mgnieniu oka, zastąpiony przez grymas przerażenia. Otworzył szeroko usta i zaniemówił. Zanim zdążył spojrzeć na mnie, błyskawicznie wyjąłem zza pazuchy pistolet i strzeliłem. Damian w chwili śmierci nie wydał żadnego dźwięku, choć prawdopodobne westchnięcie mogło zostać zagłuszone przez telewizor. Po trafieniu w głowę, i przyozdobieniu ścianą za nią swym mózgiem, upadł bezwładnie na plecy, z szeroko rozłożonymi ramionami.\\
Zabezpieczyłem swojego Colta, po czym zacząłem odkręcać jego tłumik. Po schowaniu jego i broni, wstałem na równe nogi i podszedłem do stygnącego trupa.
\sx Rzeczywiście, ktoś cię kocha. -- po tych słowach wolnym krokiem ruszyłem w przedpokój i chwyciłem klamkę drzwi.
\qd
Zanim je otworzyłem i opuściłem blok, obejrzałem się przez plecy, dostrzegając rzecz, która wołała w Markowskim takie zdumienie.\\
W kwadratowej ramce oprawiono cztery zdjęcia. Pod nią, na grzbiecie wazonu, ustawiono czarną tabliczkę.
\swk[16em]
Aleksander Jurkowski -- 1959-1979.
\qwk
Zdjęcie w lewym górnym rogu było czarno-białe i przedstawiało młodego, uśmiechniętego chłopaka o krótko przystrzyżonych włosach. Obok niego także znajdował się wizerunek Aleksandra -- z tą różnicą, że zdjęcie zrobiono w kostnicy, ciało było nagie i potwornie okaleczone, w oczy rzucały się głównie siniaki po uderzeniach pałką i pięściami.
\\
Pod tym okropnym widokiem znajdowało się inne, również czarno-białe zdjęcie, ukazujące żyjącego jeszcze Aleksandra przyjmującego cios od milicjanta służbową pałką w brzuch. Zdjęcie po lewej, przedstawiało jej właściciela. Młodego Damiana Markowskiego, milicjanta, który za głoszenie haseł przeciw komunistycznej władzy, zatłukł dwudziestoletniego mężczyznę. Takich zleceń nigdy nie odmawiałem.
% 
\begin{compactenum}
\item Wstęp
\item Skład Grupy Autorów
\item Historia Wybuchu
\item Powstanie Zony
\item Stalkerstwo I Stalkerzy
\item Anomalie
\item Artefakty
\item Miejsca, Okolica, Zona
\item Monolit
\item Ruchy, Frakcje, Grupy
\item Zjawiska Pogodowe, Zwarcia
\item Szpital
\end{compactenum}

Zajrzałem do części siódmej. Artefakty.\\
Wiele z znanych mi ,,cudeniek'' znajdowało się na liście.\\
Gwiazda Wieczorna.\\
Blask Księżyca.\\
Korale.\\
Kulebiak.\\
Czy, wręcz cudotwórcza, Dusza. Jednak wiele z podanych tu nazw było dla mnie nieznanych.\\
Ręka Kontrolera.\\
Pochłaniacz.\\
Czerwoną czcionką u dole strony napisano:\\
,,Zak. -- Zabójca''\\
Najbardziej jednak zaintrygował mnie artefakt o nazwie ,,Oko Stwórcy''.\\ Jedyny, do którego nie zamieszczono zdjęcia -- na papierze znajdował się sam, i tak dość krótki i niezbyt treściwy opis.\\
Rzec by można, oszczędny.

Po kilkugodzinnej dyskusji stalkerzy udali się do swych kwater na odpoczynek. By ,,przespać się z nowinami''. Najbardziej poruszony był rzecz jasna Radek, Leon i Mikołaj -- świadkowie całego zdarzenia pod budynkiem szpitala, na krótko przed wybuchem. Wciąż nie mogli otrząsnąć się z, prawdę powiedziawszy, szoku -- Jonathan Masterton przed ponad piętnastoma laty niemal w chwili katastrofy w czarnobylskiej elektrowni atomowej, dokładnie wskazał miejsce następnej.
\sx Daje do myślenia, co? -- powiedział jak zwykle Izaak.
\qd
Mówiąc to miał pewnie, podobnie jak reszta, bardziej na myśli fakt, że Jonathan nikomu nie zdradził, co stało się w podziemiach szpitala. Teraz już może nie być okazji.
Nie pozostaje  nic innego, jak zastosować się do słów Marka -- znaleźć rzekome akta Autorów, by z kolei zastosować się do słów Lenny’ego; wejść do szpitala, co z kolei wiąże się ze słowami ,,Tam wszystko się skończy''.

,,Oby, ku*wa, nie.'' -- powiedział Lenny na głos, wychodząc z piwnicy obozu na świeże powietrze. -- ,,Oby nie\3k''

Zatrzasnąłem drzwi tak głośno, że pod moją kwaterę przybiegł zaniepokojony Bogdan, myśląc, że ktoś na mnie napadł.
\sx Nie, nie\3k wyjąkałem poirytowany. -- Po prostu za mocno trzasnąłem drzwiami i\3k -- Bogdan uniósł ręce w przepraszający sposób i oddalił się w stronę baru.
\qd
W takich chwilach tęskniłbym za możliwością pobytu w barze. Gdyby nie pokaźny zbiór osobisty. Oczywiście brakowało mi ,,ludzkiego towarzystwa'', lecz mi w całości wystarczało lustro.\\
Po wzięciu prysznicu, o pierwszej w nocy, położyłem się do łóżka z litrem wódki pod pachą i mnóstwem rzeczy do przemyślenia. Czas umilał mi telewizor, z oczywistych chyba względów włączony na czymś innym niż kanał informacyjny. Nie mogłem się jednak powstrzymać, by choć nie zerknąć na ,,Czerwone paski sunące u spodu ekranu''. Siedem stacji opisało różnymi słowami jedno wydarzenie -- zjazd najważniejszych osób z Europy, mających związek z energią atomową -- politycy, specjaliści, eksperci, znawcy.\\
Francja powinna czuć się dość niepewnie. Nie dziwię im się.\\
Odkręciłem butelkę i postawiłem ją na stoliku obok, zaraz obok szklanki. Czekałem, aż woń alkoholu powoli wypełni wąskie pomieszczenie. Bardzo odpowiadał mi ten gorzki, surowy, wręcz drażniący nozdrza zapach.\\
Przełączyłem na kanał ze starymi, klasycznymi filmami z starej epoki. Nalewając wódki do szklanki wsłuchiwałem się w odgłosy ,,Wściekłego Byka''. Jeszcze raz po cichu podziękowałem ówczesnej komisji akademii za przyznanie Robertowi De Niro Oscara. Należał mu się, i to bardzo.\\
Zacznijmy od początku.\\
Z zdarzeń w 86 wynika prosty, jednoznaczny i niezaprzeczalny wniosek -- od tamtego dnia Jonathan stał się Psycholem. Niezbyt oryginalne, niestety prawdziwe. Cokolwiek tam się stało, musiało wpłynąć na fakt odejścia Kamila. Zostawił go, a raczej nas, na zawsze. Byłem ciekaw, co się z nim stało -- gdzie mieszka, czemu nie próbował się z nami skontaktować, jeszcze zanim narodziła się Zona -- zanim przystąpiliśmy do służby. Uprzedzał nas jednak przed taką sytuacją, przynajmniej nikt nie mógł mieć do niego pretensji.\\
,,Może już nigdy się nie zobaczymy.'' -- tak wtedy powiedział, choć nie dam głowy za zgodność co do wyrazu. Sens niestety był ten sam.\\
Po dwóch szklankach zrobiłem się wyjątkowo senny. Myśląc o 86-tym, tej nocy śniłem o 86-tym.
% % 
\podro{Rok 1986}
% 
\sx To już dziewiąty, Leon. Za darmo masz dwa, resztę mi odkupisz, jasne?
\qd
Leon bez słowa wyrwał mi papierosa z ręki, wsadził go sobie w usta, po czym trzęsącymi się dłońmi, zapalił przy pomocy zapalniczki. Kiedy się zaciągnął, wyraźnie się rozluźnił.
\sx Nie bój dupy. -- rzucił niedbale. -- Kasy mi starczy. A ty ileś spalił, że masz do mnie pretensje, co?
\qd
Uniosłem głowę w górę, w stronę nieba.\\
Westchnąłem ciężko i zrezygnowany obejrzałem się przez ramię, spojrzawszy na szary chodnik zaścielony niedopałkami. Naliczyłem dwadzieścia trzy.
\sx Dziewiętnaście. -- rzekłem zrezygnowany, obróciwszy głowę z powrotem w stronę Izaaka, Leona i Mikołaja. -- Co? Chcesz sprawdzać?
\xx Żebyś wiedział. -- Leon wykonał chwiejny krok w przód.
\xx Dobra, dobra, dwadzieścia trzy! -- niemal wykrzyczałem, powstrzymując Leona ręką. -- Cztery\3k -- dodałem, wyrzuciwszy kolejny niedopałek na śmierdzącą stertę. -- A tobie co odbiło, chcesz mi robić problemy z powodu niedopałków?
\xx Taa! -- rzekł niemal z zadowoleniem. Najwidoczniej zasada Kamila działa. -- Od dzisiaj będę ci robił problemy za każdym razem, kiedy będziesz się mnie czepiać o ilość wypalonych fajek!
\qd
Zasada Kamila brzmiała -- ,,Od czasu wydłubania Jonathanowi oka przez Adriana, dzwoniąc do was na ten numer, zawsze będę miał złe wieści. Bardzo złe, więc nie obwiniajcie się, jeśli ktoś będzie wam robił problemy z byle czego. Będzie to\3k uzasadnione.''
\sx Uspokój się Leon. -- wycedził Mikołaj przez kaszel. -- Jakoś to przetrwamy. Po Adrianie nie może być gorzej.
\qd
Leon zaśmiał się upiornie, załamując ręce. Przejechał prawą ręką po sztywnych, krótkich włosach, obrócił się w stronę mego obrońcy i ruszył w jego stronę. Dzieliły ich ledwie dwa metry, odległość ta zmieniła się dopiero po trzydziestu sekundach -- Leon szedł wyjątkowo wolno, chwiejnie, a po ,,drodze'' dwa razy zatoczył się niczym pijany.\\
Stanąwszy przez Mikołajem, omal na niego upadł -- wyglądało to jak omdlenie. Stojący obok Izaak uchronił go przez upadkiem; w odpowiedzi otrzymał kolejny dwuznaczny uśmiech.\\
Leon przestał się chwiać, stanął z szeroko rozstawionymi nogami i zbliżył swą twarz w stronę twarzy Mikołaja. Jeden czuł oddech drugiego, dwójka przyjaciół wręcz stykała się nosami.\\
Leon zmierzył Mikołaja groźnym spojrzeniem.
\sx Dziś\3k -- wyszeptał, po czym głęboko nabrał powietrza w płuca. -- Stanie się coś o wiele gorszego.
\xx Hej!
\qd
Wszyscy odwrócili się jednocześnie w stronę zawołania.
Całe dotychczasowe napięcie wyparowało jak kamfora. Wszyscy obecni pod szpitalem odetchnęli z głęboką, niesamowicie przyjemną ulgą. W jednej chwili uśmiechnęli się niemal jednakowo -- tak samo szczerze, tak samo serdecznie.\\
Kamil i Jonathan szli szybkim krokiem by ,,dołączyć'' do reszty. Jonathan miał na sobie marynarkę i eleganckie spodnie i buty, Kamil, czarną koszulkę z krótkimi ramionami, wytarte dżinsy oraz brązowe półbuty. Obaj mieli podobne, srebrne zegarki na lewych nadgarstkach. Obaj się uśmiechali.\\
Kiedy zbliżyli się do nas na jakieś pół metra, Leon pobiegł w stronę Kamila i chwycił go w ramiona, kołysząc się. Kamil wciąż trzymał ręce u boku, nie rozumiejąc gestu swego znajomego. Kiedy powoli, jakby ze smutkiem zwolnił uścisk i odsunął się do tyłu o pół metra, zapłakał.
\sx Eee\3k -- jęknął zrezygnowanie Jonathan. -- Co jest? -- podrapał się po bliźnie po cięciu Adriana lewą dłonią, po czym położył ją na ramieniu szlochającego Leona. -- Co jest, Leon? Co się dzieje? -- z jego głosem było coś nie tak.
\qd
Jak głos lekarza, który daje fałszywą nadzieję rodzinie chorego -- wiedział, że coś jest nie tak, mimo to udawał, że tak nie jest.\\
Dawno nie czułem się tak zmieszany. Poruszenie udzieliło się też Izaakowi, który w nerwowy sposób kroczył z nogi na nogę, równie nerwowo mrużąc oczy. Wszyscy byliśmy zdezorientowani zachowaniem Leona.\\
Ten umilkł, otarł oczy i wychylił się w lewo zza stojącego przed nim Mastertona.
\sx Zostań. -- powiedział.
\qd
Kamilowi dosłownie oczy wyszły z orbit. Spojrzał pretensjonalnie na Jonathana, który zaś wyglądał jak niesłusznie oskarżony o morderstwo. Teraz nie udawał -- naprawdę zgłupiał.
\sx Nie pisłem słowa, naprawdę! -- wydusił z trudem, co chwilę zerkając dookoła, jakby szukał wsparcia.
\qd
Ja, Izaak, Mikołaj, Leon i Radek nie mogliśmy wydusić ani słowa. Zamurowało nas. Pierwszy, po ponad minucie milczenia, odezwał się Kamil.
\sx Leon, Radek, Mikołaj -- zostańcie tu. Lenny, idź z Izaakiem do mojego domu. -- po tych słowach wyłowił z kieszeni pęk kluczy i rzucił mi go. -- Przynieście broń. Szybko. Jonathan -- idź do szpitala. Wiesz, co i gdzie dla ciebie zostawiłem. Ja\3k Jonathan wam wszystko wyjaśni, kiedy wróci. Żegnam\3k -- po tych słowach Kamil obrócił się na pięcie, po czym biegnąc truchtem, zniknął za rogiem bloku. 
\qd
Zdenerwowanie, niepewność i strach znowu powróciły -- tym razem ze zdwojoną siłą.
Nie doczekałem się obiecywanych przez Kamila wyjaśnień.
% 
\ro{26}
% 
% 
\podro{Rok 2001}
% 
% 
Radek wyszedł z pokoju i cicho zamknął za sobą drzwi.
\sx I jak? -- spytał zdenerwowany Mikołaj.
\xx Nic. Ciągle to samo. -- odpowiedział Radek, drapiąc się po głowie. Na kołnierz posypało się kilka drobin łupieżu.
\xx Kiedy ostatnio myłeś głowę?
\xx Z trzy tygodnie temu.
\qd
Mikołaj wystawił język, sycząc z obrzydzeniem. Kiedy spoważniał i znowu zrobił zmartwioną minę, zaprosił Radka do siebie na rozmowę.
\sx A co? -- dociekał, strzepując łupież. -- Chcesz znowu się poużalać nad Jonathanem, 86-tym, Mryńskiem i całą tą sytuacją.
\xx Nie\3k -- odpowiedział Mikołaj ku rozczarowaniu rozmówcy. -- Dwie sprawy. Dzwonili do mnie z dowództwa i\3k
\qd
Radek gwizdnął z udawanym podziwem.
\sx Ważniak! No, i co ci powiedzieli? -- zapytał.
\qd
Poczekał chwilę na odpowiedź -- zszedł za Mikołajem po schodach, czekał, aż wygrzebie klucze ze schowka w suficie, potem wyszedł z nim z kwatery i obserwował, jak zamykał drzwi wejściowe na klucz -- zarówno zamek pod klamką jak i trzystopniową zasuwę wyżej.
\sx
Wybrali nowego szefa obozu. -- powiedział Radek, chowając klucze do kieszeni, po czym ruszył w stronę baru. -- Niech go sobie przypomnę\3k Dy\3k Dymitr K\3k K\3k kurewsko trudne nazwisko. Dadzą mu monitoring dwa razy większy, niż ten, którym Wolność obarcza Barry’ego.
\xx A propos Barry’ego -- coś już niego wycisnęli?
\xx Nie wiem, szczerze mówiąc to\3k
\xx To co?
\xx Nie przerywaj mi, jak mówię. Jeszcze z nim ,,nie rozmawiali''.
\xx Co?!
\xx I z czego tu\3k
\qd
Mikołaj nie krył oburzenia:
\sx Leon.
\xx Co ,,Leon''?
\xx Sprawdź, czy brakuje mu nerki.
\qd
Radek prychnął złośliwie.
\sx Bardzo, ku*wa śmieszne. Myślisz, że jak nie zaczną go męczyć pierwszego dnia, to\3k
\xx Tak, geniuszu! To siatka będzie działać dalej, ba, pewnie już znalazła sobie nowego ,,kierownika''. Może właśnie w tej chwili ktoś chlasta jakichś samotników elektrycznym nożem chirurgicznym -- co ty na to?
\xx Zadzwonię do nich, żeby się za niego zabrali. -- wyszeptał Radek po długim wdechu. Wypuścił powietrze z sykiem i ulgą. Przejdźmy do rzeczy\3k -- Druga sprawa to nowa robota, od tego całego Dymitra.
\xx Słucham, słucham\3k -- zachęcił Mikołaj, otwierając drzwi do pubu.
\qd
Z jego twarzy wynikało, że poczuł woń wnętrza -- starego alkoholu, wymiocin i środków czyszczących oraz odświeżających, które próbowały je zamaskować. Wchodząc do środka, powęszył nosem i mruknął z zadowoleniem.
\sx Waniliowe. No, dalej, dalej\3k
\qd
Z racji gwaru obecnych w barze i muzyki dobiegającej z radia, Radek musiał mówić głośniej, niż zwykle. Próbował przekrzyczeć otoczenie, jednocześnie przeciskając się wraz ze swym kompanem przez tłum, szukając wolnego stolika.
\sx Będzie okazja do sprawdzenia tego genialnego pomysłu! -- wykrzyczał, powoli posuwając się naprzód. Mikołaj co chwilę przybliżał się do niego, by jak najwięcej usłyszeć. -- Dowództwo znowu kombinuje jakąś dostawę dla naukowców. Mamy skombinować dwóch stalkerów znających\3k
\xx Tylko nie to\3k
\xx \3kDolinę Mroku, by\3k
\xx Kurwa! Nienawidzę tego miejsca!
\xx To jeszcze nic. Mamy im przynieść po jednym trupie każdego mutanta, jakiego spotkamy.
\qd
Mikołaj stanął jak wryty. Radek też.
\sx Może oszczędźmy sobie tego zachodu. -- zaproponował Mikołaj.
\xx Co masz na myśli?
\xx Oblejmy się wodą i wskoczmy do Elektro. -- rozmówca Radka zachichotał ironicznie, uśmiechnął się przez dwie sekundy i wskazał prawą ręką wolny stolik. Stały przy nim dwa drewniane krzesła.
\xx Tylko ty robisz w gacie na samą myśl, by tam wejść. -- wycedził Radek, siadając na krzesło i przysunąwszy się do stolika.
\xx Możliwe, ale wiesz\3k -- Mikołaj zamilkł, po czym usiadł na krześle. -- Wciąż rozmyślam nad moją propozycją co do skoku w Elektro. -- dokończył, przybliżając się do przodu wraz z krzesłem. -- Przynajmniej oddamy nasz majątek, komu trzeba.
\xx Daj spokój. Wiem, że to, co tam się stało cię\3k
\qd
Mikołaj uniósł rękę na znak protestu.
\sx Co?
\xx Hmm\3k -- Mikołaj zamyślił się, po czym opuścił dłoń z powrotem na kolano. -- Chciałem ci walnąć jedną z tych gadek typu\3k -- zmienił ton na ten, który uważał za ,,parodiujący brazylijskie tasiemce'' -- ,,Nic nie wiesz, o tym, co przeżyłem!'' albo ,,Nic o mnie nie wiesz!'', ale darujmy to sobie. Ja tam nie pójdę. Zresztą, po co ten cały Dymitr chce tam nas?
\xx Po prostu jesteśmy dobrzy.
\xx Tak jak cały ten obóz.
\xx Dobra, to niewinne zrządzenie losu. Szkoda, że nie wróżyłeś ostatnio z tarota -- pewnie dowiedział byś się o tym zawczasu i postrzelił się w jądra, albo skoczył w Elektro, by się od tego wymigać.
\xx Nie. Zmieniłbym łachy i na jakiś czas poudawał samotnego, lub, wkręciłbym się do bazy wojskowej na parę dni. Słuchaj\3k czy ja jestem tam konieczny?
\qd
Radek oparł czoło o dłoń, kiwając głową z dezaprobatą.
\sx Wiesz, że mamy napięte terminy. -- rzekł.
\xx Ta, i co z tego?
\xx To, że prędzej czy później, dojdzie do takiej sytuacji, w której wszyscy dostaniemy jakąś robotę w tym samym czasie i nie będzie okazji, byś na przykład, zamienił się ze znajomym, by iść do takiego Agropomu, zamiast Doliny. Nie można być w Zonie i zwyczajnie bać się wejścia na któreś z jej terenów. Dowództwo będzie miało w dupie, co tam przeszedłeś -- masz tam iść i koniec. Po za tym, to ty masz najlepsze znajomości u samotnych. Znasz, kogoś, kto\3k
\xx Michał.
\qd
Radek kiwnął kilkakrotnie głową z zadowoleniem.
\sx Noo\3k -- zawołał ciepło, klepiąc Mikołaja po ramieniu. -- Dobrze. Ile będzie za to chciał?
\xx On i gość imieniem Dawid.
\xx Zaraz, zaraz. Kto jest kim?
\qd
Mikołaj wyprostował się na krześle i ruchem ręki dał znak barmanowi, że dziś niczego nie zamawia. Ziewnął, po czym odpowiedział:
\sx Obaj są dobrymi stalkerami. Michał zna tamte tereny, oprócz gospodarstwa, Dawid zbiera trofea. Pomoże nam z przewożeniem zwłok do sekcji -- zna się na tym jak nikt inny.
\xx Może być\3k Więc -- ile będą za to chcieli?
\xx Nie są zbyt chciwi. Pewnie będziemy im płacić w sprzęcie -- będą mieli z tego większe korzyści. Sama kasa niewiele daje, jeśli nie wiesz, gdzie jej wydawać\3k Chyba jednak coś sobie wezmę. Też chcesz?
\qd
Radek pokręcił głową\\
Mikołaj wstał z krzesła i wolnym krokiem skierował się w kierunku lady. Podniósł lewą stopę, by nie zahaczyć o wystającą rączkę służącą do otwierania klapy schronu. Po ominięciu stojących metr przed ladą stalkerów, zagadał do barmana i poprosił o małą szklankę. Chwycił ją w palce lewej ręki i wrócił do Radka. Postawił szklaneczkę na stoliku i zaczął się w nią wpatrywać.
\sx Od gapienia się w szklankę nie napełni się ona alkoholem. Chyba, że sprawdzasz, czy jest w niej woda i chcesz zabawić się w Chrystusa. A może\3k
\xx Mam swoje. -- przerwał Radkowi Mikołaj, wyjmując z kieszeni butelkę whisky.
\xx Hmm\3k Lenny ci dał?
\xx Owszem. -- potwierdził Mikołaj, przymierzając się do zakręcenia butelki po napełnieniu szklanki jej zawartością.
\xx Daj łyka. -- poprosił Radek zrezygnowanym tonem, wpatrując się w butelkę Lenny’ego niemal z uwielbieniem.
\xx Na zdrowie.
\xx Zdrowie. -- Radek sięgnął po butelkę, i w chwili, w której pociągnął łyk, w obozie rozległy się syreny.
\qd
Alarm inny, niż w przypadku wybuchu. Chwilę później wszyscy w obecni w barze, ba, wszyscy obecni w Zonie poczuli się jak podczas trzęsienia ziemi. Z daleka, a dokładne z północy zaczął napływać ogłuszający, szybko narastający ryk. Niektórzy nazywali to gwizdem.\\
Niebo za oknem powoli nabrało czerwonawej barwy, która po chwili zamieniła się w fiolet. Kolejną sekundą później kolor nieba zmieniał się w szalony sposób, niczym szkła w kalejdoskopie. Rozdzierający ryk wypełnił całą przestrzeń, zagłuszając typowy, barowy harmider. Teraz każdy miał wrażenie, że huk rozlega się bezpośrednio z jego głowy; nie miał żadnego konkretnego źródła.
\sx Co za ironia\3k -- mruknął Radek, odstawiwszy butelkę z alkoholem na stoliku.
\xx Zwarcie! -- wrzasnął barman, wyskakując zza lady w kierunku klapy schronu. 
\qd
Wyprzedził go Mikołaj, który właśnie schylał w kierunku dwóch stalowych uchwytów. Złapał je, i wstając otworzył schron.

Obudziły mnie szalejące błyskawice.\\
Otworzyłem oczy i spojrzałem nimi w stronę okna. Zwarcie.\\
A ja miałem zamiar się wyspać\3k\\
Przetarłem oczy i ze stojącej obok szafki zdjąłem leżącą na niej kartkę.

Pochłaniacz.\\
Wyjątkowo zagadkowy artefakt. Po raz pierwszy spotkaliśmy go pod Super Samem -- sklepem, który miał zostać niedawno otwarty, był to pewnego rodzaju dyskont spożywczy, taki market. Pochłaniacz leżał pod jednym ze sklepowych wózków. W wózku nad artefaktem leżały zwłoki żołnierza. Raport z sekcji -- płyta 27, tytuł ,,Sekcja Wojskowego-Pochłaniacz''.
Artefakt wygląda jak idealnie okrągła kula(o średnicy dwudziestu pięciu centymetrów) o nieprzeniknionej, czerwono złocistej barwie, która pokryta jest czymś w rodzaju narośli, bogatą ilością cienkich włókien, chropowatych w dotyku. Jest twardy jak głaz i równie odporny -- testy nacisku nie stworzyły jakiegokolwiek odkształcenia; mimo to, ma się wrażenie, że wszystkie włókna poruszają się niczym żywa istota.\\
Nazwaliśmy go tak, ponieważ absorbuje pobliskie promieniowanie. Wykonaliśmy pięć testów -- mierzyliśmy dwa metry kwadratowe powierzchni wydzielone specjalną siatką -- przed i po umieszczeniu artefaktu pośrodku wyznaczonej przestrzeni na pięć minut. Po tym, dokładnie odmierzonym czasie, promieniowanie w odmierzonym obszarze wyraźnie zmalało.

Przerwałem lekturę. Zwarcie powoli milkło -- cienie w moim pokoju powoli przestały tańczyć, tworzone przez błyski różnorakich barw, przez które poczułem się jak we wnętrzu kalejdoskopu.\\
Byłem zbyt zmęczony na czytanie. Położyłem kartkę niedbale obok łóżka, przewróciłem się na prawy bok i wkrótce zasnąłem.\\
Ostatnią myślą przed zapadnięciem w sen było -- ,,Miałeś rację, Adam.''

Siedząc w schronie, zacząłem wypytywać Radka o szczegóły zleconej nam misji. Błyskawice i ryk Zwarcia niewątpliwie nam to utrudniały, podobnie jak szepczący coś do siebie obecni w piwnicy stalkerzy, jednak umiejscowienie się w kącie skutecznie pomogło nam cieszyć się dyskrecją.
\sx Dostaniemy jakiś pojazd?
\xx Specjalną ciężarówkę.
\xx Kto jeszcze z nami idzie?
\xx Prócz mnie, ciebie, Michała i Dawida? Leon i Mark. Izaak jest chwilowo bez pracy, podobnie zresztą jak Lenny. Pijawkę, psa i wilczura będziemy mogli od razu przywieźć tutaj, bez zapuszczania się w Dolinę. Te długoszyje, połamane gnoje, kontrolerzy, zombie i\3k
\qd
Radek zanucił piosenkę śpiewaną przez siódemkę krasnoludków z ,,Królewny Śnieżki'' podczas wybierania się do kopalni.
\sx \3k Krasnale będziemy musieli taszczyć stamtąd aż tutaj na pokładzie ciężarówki. Nie ma innego wyjścia, ale cóż\3k
\xx Lepiej powiedz, co będziemy z tego mieli.
\xx Jakieś artefakty, kupę szmalu i dowolną, możliwą do zdobycia przez nich, broń.
\xx No\3k -- Radek gwizdnął. -- To chyba jest wart naszego zachodu.
\xx Powinieneś się cieszyć z tego, że przełamiesz swój lęk. Nie wyobrażam sobie ciebie bojącego się tam wejść.
\xx Taa\3k teraz tylko wejść do szpitala i mogę uważać swoje życie za spełnione. Pogadam z Michałem. Może słyszał coś o tych aktach Autorów.
\qd
% 
\ro{27}
% 
Mark zakręcił pojemnik ze środkami przeciwbólowymi, położył jedną z białych tabletek na stoliku, po czym ostrożnie usiadł na brązowym krześle. Schylił się i wyciągnął spod niego butelkę wody mineralnej, otworzył ją i upił z jej dwa długie łyki. Spojrzał na zabandażowaną dłoń -- wciąż był wkurzony na Jonathana, a jeszcze bardziej na jego znajomych, którzy ciągle wciskali mu kit. O jego przeszłości, tym co zrobił i co się z nim dzieje.\\
Albo go nie lubili, albo uważali wiedzę na temat Jonathana za jakiś niewyobrażalny skarb, którym nie można się dzielić z byle kim.\\
,,Władza dla elit'' -- tak nazywał to Mark. W tej hierarchii oni byli królami, Mark zaś Dworzańskim błaznem.\\
Po tej myśli sięgnął zdrową ręką po tabletkę, wsadził ją w sobie usta i popił wodą. Zadziałała po kilku sekundach, uśmierzając ból w dłoni i wypełniając Marka błogim spokojem. Było mu tak dobrze, że miał chęć położyć się do łóżka, co też zrobił.
\sx Walić ich. -- szepnął do siebie, kładąc głowę na poduszce. -- Nie chcą mi opowiadań o swoim zwariowanym chłopcze, trudno; nie po to tu przyjechałem.
Koka. To wszystko przez nią.
I tak, było mi szkoda Jonathana. Choć, może mu się należało?
\qd

Maks potknął się i z impetem wpadł na drzewo, uderzając dołem szczęki o twardy pień. Zawył z bólu, podniósł się z trudem, prawą ręką ocierając krwawiące usta, po czym pobiegł dalej. Zaczął dostrzegać już blaszane ogrodzenie obozu.\\
Mutant, który go zaatakował, przestał go ścigać ponad pół godziny temu; Maks jednak, nie mogąc otrząsnąć się z szoku, pędził nieprzerwanie, odnosząc wrażenie, że wciąż goni go stado pijawek. Mimo, że stwór, którego spotkał, był o wiele gorszy od stada pijawek. Może nawet od gromady Burerów, tu jednak miał by większe wątpliwości. Jednak sam fakt, że jedna istota może być gorsza od stada innych, napędzała mu wystarczającego stracha. Był przestraszony i dumny jednocześnie -- przerażony spotkaniem z Iluzjonistą, dumny, że uszedł z niego z życiem. Jeśli dojdzie do siebie, jeśli w obozie się nim zajmą, nie nachwali się znajomym swym osiągnięciem do końca życia.\\
Jedyna broń, jaka mu została, a mianowicie gołe pięści, nie powinny być wystarczającym pretekstem dla wartowników, by zabili go na sam jego widok. Poza tym, po grudniowej masakrze w Boże Narodzenie, bezpieka powinna zacząć odbudowywać swój wizerunek.\\
Prędzej, czy później, nie zrzeszeni stalkerzy zbiorą się w grupę i otwarcie zaatakują obóz bezpieki, by pomścić swych znajomych. Tutejsza armia, wraz z urzędem bezpieczeństwa w Zonie, umywała ręce od grudniowej masakry, nie mogła znaleźć dobrego kontrargumentu na argument stalkerów -- zwłok żołnierza, jedynego, który zginął podczas rzekomej akcji.\\
Taki wstyd\3k
\sx Stój! -- krzyknął ktoś znad wzniesienia, w kierunku którego zmierzał Maks.
\qd 
Stanął jak wryty i podniósł ręce wysoko w górę.

Zbudził mnie odgłos rozsuwanej, ciężkiej bramy.\\
Czując się jak w młodości -- jak sportowiec, biegacz, poderwałem się z łóżka bez żadnych oporów i doskoczyłem do okna. Brama powoli przesuwała się w lewą stronę, zaś przed nią stało trzech stalkerów z uniesioną bronią.\\
Wsłuchując się w dźwięk skrzypiącego metalu, otworzyłem okno na oścież. Przetrzymałem chłodne powietrze, które wdarło się do pomieszczenia, i wystawiłem głowę nad parapet, po czym rozejrzałem się dookoła. Niebo pełne było ciemnych, szarych deszczowych chmur, które prawie zasłoniły nisko położone słońce, którego blady blask dało się dostrzec jedynie pomiędzy pniami drzew.\\
Całkowicie przesunięta w lewo brama ukazała samotnego stalkera. Jego wygląd był okropny.\\
Niegdyś sprawny, domowej roboty kombinezon był postrzępiony i pełen śladów pazurów, jak i skrzepłej krwi. Materiał od pasa w dół był zerwany i postrzępiony, ukazując zdartą skórę na kolanach i ślad po postrzale w okolice piszczela. Stalker nie miał prawego buta, plecaka, nawet kabury. Jego włosy kleiły się od potu i krwi, podobnie jak posiniaczona i pokryta zadrapaniami twarz. Prawe oko znikło za obfitym, sinym obrzękiem, który zdobiła krew z rozciętego łuku brwiowego.\\
Stalker padł na pośladki, niemal kładąc się na ziemi, po czym próbował odsapnąć.
Mimo dzielącej mnie i jego odległości, słyszałem jak rozpaczliwie łapał powietrze potwornie świszczącym i wręcz nienaturalnie zniekształconym oddechem. Każdy wdech i wydech sprawiał mu wyraźnie ból, o czym świadczył każdorazowy grymas na twarzy, kiedy tylko wypinał i wciągał pierś.\\
Nie pozostanie nam nic innego jak się nim zająć. Jaki mamy wybór, odkąd wrobiono nas w rzeź?\\
Do rannego samotnika podbiegło dwóch sanitariuszy z noszami -- jeden z nich chwycił go za nogi, drugi za ręce, po czym jednocześnie położyli go na łożu, które pędem przenieśli do piwnicy.\\
Mijając moją kwaterę, stalker zaczął krzyczeć:
\sx Iluzjonista! Widziałem Iluzjonistę!
\qd
\dd
\sx Ilu\3k! -- Maks, jak się ,,przedstawił'' urwał w połowie zdania; była to reakcja po podaniu środków uspokajających.
\qd Całe powietrze włożone w zamierzony krzyk powoli ulotniło się z ust z cichym sykiem i błogim westchnięciem Maksa. Wyraźnie mu ulżyło.
\sx Odkaźcie rany, zszyjcie łuk, zróbcie coś z tym obrzękiem i utrzymujcie go w tym stanie. -- wyrecytował Doktor, drapiąc się po prawej dłoni. -- Zaraz wracam, zbadam jego płuca i gardło. -- dodał, po czym wstał i wyszedł z pomieszczenia.
\qd
Mick, Leon, Izaak i Radek siedzieli na skórzanej kozetce, naprzeciwko łóżka z Maksem. Wciąż wyglądał strasznie, jednak przynajmniej na jakiś czas przestał też dawać o swoim stanie przy pomocy głosu. Wcześniejsze wrzaski o Iluzjoniście, hordzie mutantów i tym, co przeszedł, doprowadzały do furii.
\sx Myślicie, że naprawdę go spotkał? -- spytał Radek, który od momentu wejścia na oparcie, dziecinnie machał nogami na przemian.
\qd
Leon wzruszył szeroko ramionami, wzdychając.
\sx Może to zwidy od Kontrolera. -- zaproponował.
\xx Raczej nie. -- zaprzeczył Radek. -- Szum, purpurowa barwa, kołysanie, mdłości i powolny paraliż, tak, ale to mi wygląda bardziej na emisje Zwęglacza. Byłeś tam? -- to pytanie zostało skierowane do Maksa.
\qd
Ten odpowiedział natychmiast:
\sx Nie od ponad trzech lat. -- mówił wolno i trochę niewyraźnie, na co wyraźny wpływ miały podane przed chwilą leki. -- Jego spotkałem gdzieś w centrum Wysypiska, szedł na północ.
\qd
Radek oparł brodę o prawą dłoń i zaczął myśleć.\\
W całej historii Zony oficjalnie spotkano dwóch Iluzjonistów -- nieoficjalnie, czterech, wliczając w to przypadek Maksa.\\
Pierwszy z nich, pojawił się w okolicach bazy bandytów w Dolinie Mroku w roku 1991 -- w tym samym czasie spora grupa stalkerów, w większości członkowie Powinności, przeprowadzała akcję. Ówczesny dowódca frakcji zaprzyjaźnił się z wysoko postawionym porucznikiem wojska -- po roku przyjaźni zaczął wykonywać zlecenia, które wojsku mogły ,,zepsuć reputację''.\\
Pierwszym z nich było całkowite wyeliminowanie bandytów z Zony.\\
Tak więc, Powinność zebrała rozsianych po całej Strefie swych żołnierzy do Baru, wyposażyła ich w sprzęt dostarczony przez wojskowych, po czym ruszyła na plac budowy -- odwieczną kwaterę bandytów.\\
Z trzydziestu sześciu stalkerów i czterdziestu trzech bandytów, ocalały trzy osoby -- dwóch ludzi z Powinności i jeden bandyta, który jednak w pewnym sensie umarł -- całkowicie oszalał i do dziś siedzi w jednym z Ukraińskich szpitali psychiatrycznych, cierpiąc na ciężką schizofrenię. Pozostała dwójka opuściła Zonę raz na zawsze, pozostawiając na dożywotnej opiece psychologa.\\
Według ich relacji, zdążyli zabić już kilku bandytów i walka wkrótce rozgorzała na dobre. Po dziesięciominutowej wymianie ognia, kolejnych pięciu ofiarach, zza głównego budynku stacji benzynowej, stojącej bardzo blisko bramy na ruiny, wyłonił się Iluzjonista. Nazwę tą nadano mu na podstawie relacji ocalałych -- było to pierwsze spotkanie z tym mutantem.\\
Nazwano go tak, gdyż po wkroczeniu przez potwora na teren budowy, wszyscy na jej terenie zaczęli strzelać do siebie nawzajem. Wszyscy, prócz pilnujących furtki dwóch stalkerów, którzy na widok mutanta zwyczajnie padli trupem.\\
Bandyta do bandyty, członek Powinności do swego niedawnego kompana -- wszyscy wystrzelali się nawzajem niemal do ostatniego, krzycząc przeraźliwie, jakby ze świadomością tego, co robili. Sam stwórca całej rzezi stanął obok jednej z ciężarówek i napawał się swym dziełem. Podsuwał stalkerom wizje -- ludziom wydawało się, że stojący obok niego człowiek zamienia się w mutanta, że próbuje zabić swego towarzysza, lub, jak powiedział bandyta: ,,Wmówił mi potworny ból, naprawdę go czułem, oczekując, że sam się zabiję. Trzech moich kolegów tak zrobiło -- jeden skoczył z wysoka, pozostali dwaj przyłożyli sobie lufy do skroń\3k''\\
Po dwóch minutach na placu budowy zapadła głucha cisza, przerywana jedynie odgłosem kroków stawianych przez Iluzjonistę -- kroczył on pośród stygnących ciał, napawając się dumą i satysfakcją z rzeczy, której dokonał.\\
Trzech bandytów i dwóch ludzi Powinności wytrzymało potworne wizje -- albo zawarli ze sobą okresowy rozejm, albo zwyczajnie zapomnieli o swych niedawnych wrogach, całą swą uwagę skupiając na mutancie. Zaatakowali go, jak podał jeden z ocalałych (obserwator, który nadzorował akcję z daleko położonego wzgórza), z okien ceglanej ruiny, kiedy szczupła, dwumetrowa postać w czarnym płaszczu przechodziła pod jednym z dwóch zardzewiałych dźwigów.\\
Jeden ruch, machnięcie bladej ręki, uśmierciło dwóch bandytów i jednego pozostałego żołnierza; pozostała dwójka została sparaliżowana od pasa w dół.\\
Po chwili Iluzjonista ściągnął z głowy kaptur, ukazując swoją głowę.\\
Jak do tej pory, snajper-obserwator pamiętał wszystkie szczegóły z wydarzenia, którego był świadkiem, jednak nie potrafił opisać wyglądu głowy mutanta żadnym konkretnym słowem. Kiedy ją zobaczył, całkowicie odjęło mu mowę; zaczął się panicznie trząść i mimo chęci oderwania oczu (a raczej lornetki) od potwornego widoku, nie mógł tego zrobić. Wpatrywał się tępo w, jak określił, ,,niewyobrażalnie przerażającą'' twarz, siejąc w swoim mózgu coraz większe spustoszenie.\\
Po upływie pięciu minut potwór założył kaptur z powrotem, po czym pobiegł na zachód, pozostawiając po sobie kilkadziesiąt okaleczonych zwłok.\\
Druga oficjalnie potwierdzona obecność Iluzjonisty -- Jantar.\\
Siedmiu naukowców z dziesięcioosobową eskortą żołnierzy ocalał z ponad dwóch setek żołnierzy, którzy zabezpieczyli opustoszały ośrodek badawczy. Tutaj jedynym dowodem był gość szczegółowy opis mutanta, który ,,wybił'' większość ludności na terenie zakładu. Jego autor, naukowiec, podał zgadzający się w większości z poprzednim przypadkiem, opis potencjalnego Iluzjonisty. Dwa dni po przejęciu przez naukowców w bunkrze nad bagnami postradał resztki rozumu.\\
Trzecia wizyta Iluzjonisty w południowej Zonie, a dokładniej, w Dziczy, nie znalazła potwierdzenia. W napromieniowanym domu przy jednym z torów kolejowych, w sejfie położonym w piwnicy dawnej siedziby kolejarza, znaleziono nagranie na dyktafonie. Było to coś w rodzaju relacji ,,na żywo'' ze spotkania z Iluzjonistą -- zaczynało się ono normalnie; relacją pewnego stalkera, który podawał szczegółowy opis otoczenia. Po minutowym raporcie w nagraniu rozlega się ogłuszający wrzask, a właściciel dyktafonu, twórca nagrania, zaczyna się panicznie, nieprzerwanie śmiać, nawet bez przerwy zaczerpnięcia oddechu. Obłąkańczy rechot zostaje przerwany słowami ,,Iluzjonista\3k'', po czym kończy się dźwiękiem głośnego strzału, mokrego chluśnięcia (prawdopodobnie krwi) i ciężkim łomotem; prawdopodobnie ciała.\\
Wielu ekspertów -- z grubsza ludzi pracujących dla rządu i Sacharowa, jak i doświadczeni stalkerzy, potwierdzili, że ryk w nagraniu to krzyk Kontrolera.\\
Przypadek czwarty -- Maks, stalker, który z północy dotarł aż tutaj.\\
Rozmyślania i dedukcje Radka przerwał Leon:
\sx I co o tym sądzisz? -- spytał, wpatrując się tępo w przykutego do łóżka, poranionego człowieka. -- To naprawdę mógł\3k
\xx To był Iluzjonista! -- krzyknął Maks, próbując się uwolnić. -- Podam wam jego dokładny opis, to wam wystarczy, tak?!
\qd
Izaak aż się wzdrygnął.
\sx Taa\3k To nam wystarczy. -- powiedział powoli, schodząc z pryczy.
\qd
Przeciągnął się szeroko, po czym podszedł do szuflady. Otworzył ją i wyłowił z niej zapakowaną w folię strzykawkę.\\
Zerwał folię i zdjął zabezpieczenie igły, po czym rzekł do Maksa:
\sx Jeszcze raz wydrzesz tak japę, to wbije ci to w pęcherz. Jak widzisz, jest dość długa.
\qd
Maks skrzywił się ze strachu, po czym powoli obrócił głowę przed siebie i zaczął wpatrywać się w sufit.
\sx Nie ma takiej potrzeby\3k -- mruknął spokojnie. Mam nadzieję, że po tym mnie odeślą. Mam dość Zony.
\qd
% 
\ro{28}
% 
% 
% 
Nocą, kilka minut po północy, Radek wyszedł z obozowego baru. Natknął się na kuśtykającego o kulach Lenny’ego, który zmierzał do środka pubu. Było w nim tłoczno, gorąco i parno od oddechów wszystkich jego gości. Jak zwykle o tej porze, zapach papierosów i alkoholu dało się wyczuć dobre trzy metry przed progiem.\\
Lenny miał na sobie ciepły płaszcz, spod którego świtała pidżama. Biały gips także mocno rzucał się w oczy.
\sx Chcesz tam nocować? -- spytał ironicznie Radek, podnosząc rękę w powitalnym geście.
\qd
Lenny skinął mu głową na ,,dzień dobry'', po czym odparł:
\sx Nie, chce odebrać nagrodę.
\xx Jaką? I za co?
\qd
Lenny uśmiechnął się pod nosem, kręcąc głową. Wyglądało, jakby zaraz miał zamiar się roześmiać. Odchrząknął, po czym zaczął odpowiadać Radkowi na pytania:
\sx Mick. Wiesz, który?
\xx Uratował Leona i Mikołaja przez Finnem. Jest w środku\3k
\xx A ci jeszcze mu nie podziękowali\3k
\xx Przypomnę im. -- zapewnił Radek uspokajająco. -- Więc, Mick\3k
\xx Założył się ze mną o pięć paczek ekskluzywnych cygar, plus papierkowa robota z ich załatwieniem.
\xx A o co? -- spytał Radek z wyraźnym zainteresowaniem w głosie. Założył ręce na piersi i czekał.
\xx Czy Igor znowu pokaże się w ,,stroju roboczym'' na oczach całego obozu, podczas odbierania ładunku.
\xx Nie mogę\3k -- Radek z trudem nie wybuchnął głośnym rechotem.
\qd
Przyłożył zaciśniętą pięść do ust i zaczął chichotać, kołysząc się w lewą stronę. Wylądował na ścianie baru, wciąć śmiejąc się pod nosem. Kiedy skończył, opuścił rękę z powrotem u boku z pełnym samozadowolenia westchnięciem, który przypominał dźwięk wydawany przez narkomana, zaraz po kolejnej dożylnej dawce heroiny.\\
Spoważniał, po czym jeszcze bardziej zaciekawiony zapytał:
\sx Naprawdę to zrobił? Opowiadaj, nie było mnie dziś cały wieczór.
\qd
Lenny oparł się o mur zaraz obok swego przyjaciela, oparł prawą kulę o barierkę otaczającą trzy niskie stopnie, po czym już wolną ręką sięgnął do kieszeni.
\sx Godzinę po tym, jak wyszedłeś, przyszli do obozu pohandlować. Z siedmiu, ośmiu, bardzo wpływowi i znani wśród ,,nie zrzeszonych'' stalkerów. Chcieli trochę broni, amunicji, jeden wykupił wiele sprzętu do oprawiania zwierzyny. Generalnie, oni chyba zajmowali się głównie polowaniami -- specjalne naboje, w tym zatrute, głównie małokalibrowe. Zamówili, jak mu tam\3k Iver Johnsona.
\xx Muszą nieźle na tym zarabiać.
\xx To oczywiste. Dwa dni łowienia w takiej Dziczy, kilkanaście minut jazdy okrężną drogą na Yantar i masz kilkanaście tysięcy. Ubicie i przetransportowanie całej, nienaruszonej pijawki to dla nich normalka. -- Leonard wygrzebał z kieszeni paczkę papierosów -- otworzył ją kciukiem i skierował w stronę Radka.
\qd
Ten pokręcił głową na ,,nie''. Autor propozycji wyłowił jednego papierosa przy pomocy zębów z prawie pełnej paczki, którą schował do kieszeni, tym razem tylnej w spodniach.
\sx Przydadzą się nam. To znaczy\3k -- Radek spojrzał na gips. -- Ty to masz farta\3k
\xx Co masz na\3k Aha\3k Dwu tygodniowy zastój, a kiedy dostajecie nową robotę, ja muszę być ,,niedysponowany''. -- po tych słowach Lenny zapalił trzymanego w ustach papierosa. Zaciągnął się głęboko, przetrzymał dym przez trzy sekundy, po czym wypuścił go nosem.
\xx Zmienisz zdanie, jak ci powiem, co mamy zrobić. Potem skończysz to o zakładzie.
\qd
Radek urwał, gdy z baru wyszedł Leon. Zatrzymał się on na drugim kamiennym stopniu, widząc stojącą pod ścianą dwójkę.
\sx Radek! -- rzekł rozpromieniony, wciąż pozostając na miejscu. -- Gdzieś ty był do nocy?
\xx Później ci powiem.
\xx Dobra\3k w takim razie idę do siebie. Jestem zbyt zmęczony na typowe, swojskie lanie wody. -- wycedził, ziewając krótko.
\qd
Machnął ręką na pożegnanie, schodząc ze schodków, po czym wolno skierował się do swojej kwatery.
\sx Chyba ma zły humor\3k -- mruknął Leonard, oprowadzając Leona spojrzeniem, do momentu, w którym całkiem zniknął w ciemnościach. Kiedy to się stało, wrócił do poprzedniego tematu:
\xx To co macie zrobić?
\xx Przytaszczyć tu trupa każdego mutanta, jaki stąpał po Zonie. Z wyjątkiem Olbrzyma i Iluzjonisty, rzecz jasna.
\xx Po kiego?
\xx Dla ,,elity intelektualnej'' kilku współpracujących z nami krajów.
\xx Ile wam to zajmie?
\xx Z tydzień. Siedem dni łażenia po Zonie wzdłuż i wszerz.
\xx Miałeś rację. Zmieniłem zdanie. Dzięki ci, Snorku, za mą nogę\3k
\xx Wariat. -- mruknął Radek żartobliwie.
\xx Może trochę. -- przyznał Lenny podobnym tonem. Skończył palić, niedopałek rzucił na ziemię. -- To chcesz wiedzieć, jak było z tym zakładem? -- dodał po chwili.
\xx Mów.
\xx Po ,,zakupach'' cali ci Łowcy zaczęli wsiadać do samochodów, wracając, mijali dom Barry’ego\3k
\xx Dymitra.
\xx Racja\3k -- Lenny odkaszlnął. -- Igor wypadł z podziemi, dokładnie, kiedy mijali ją ci stalkerzy.
\xx I co?
\qd
Lenny wrzasnął w sposób parodiujący nisko budżetowe horrory.
\sx Odskoczyli jak porażeni! Jeden zemdlał, a pozostali zaczęli uciekać. Dopiero, jak Igor zaczął się śmiać, to zorientowali się, że to nie mutant.
\xx Cały we krwi?
\xx Jak zwykle. Biały fartuch, co ja mówię -- czerwony i zachlapany posoką, podobnie jak jego twarz! Nawet ja się wzdrygnąłem, a widziałem to wszystko z okna.
\xx Prawdziwy wariat. -- skomentował z pogardą Radek.
\xx Może. -- przyznał Lenny. -- Ale czy ktoś z taką robotą może być normalny?
\xx Nie.
\qd
% 
% 
% 
\podro{Rok 1978}
% 
% 
Ktoś zapukał do moich drzwi. Trzy razy, w krótkich odstępach. Odłożyłem gazetę, podniosłem się z krzesła i szybkim krokiem podszedłem do drzwi. Spojrzałem przez judasza i zobaczyłem spoconego Kamila, stojącego na klatce schodowej. Miał rozczochrane włosy, wyglądał na wyczerpanego. Nawet stojąc za drzwiami słyszałem jego zmęczony oddech.\\
Oby to nie było to, o czym myślę.\\
Przekręciłem klucze w zamku i otworzyłem szeroko drzwi. Kamil wpadł do środka, jakby opierał się o ścianę, którą właśnie usunięto. Zataczając się, obleciał cały przedpokój, lądując na ścianie, tuż obok wiszącego lustra. Uderzając plecami o panele, zsunął się po nich; po chwili siedział już na tyłku, z rękoma opartymi o kolana. Ciężko dyszał.
\sx Co znowu? -- spytałem, trąc lewe oko.
\xx Cóż\3k Jonathan\3k -- Po każdym krótkim słowie Kamil brał głęboki wdech, zmagając się ze zmęczeniem. -- Pogadałem sobie z nim. Już więcej cię nie zaczepi.
\xx Na pewno? -- mruknąłem, wciąż niepewny i lekko przestraszony.
\qd
Nie mogłem przekonać samego siebie, że postąpiłem słusznie mówiąc Kamilowi o ostatnich wybrykach Adriana. Teraz albo się odczepi albo jeszcze bardziej wnerwi.
\sx Co robisz? -- zapytał mnie Kamil, który oddychał już normalnie. Wstał z wysiłkiem, podrapał się po czole i poszedł do mojego pokoju.
\xx Czytałem gazetę.
\xx Coś ciekawego?
\xx Słyszałeś o tym psycholu, który zabija dwójkami?
\xx Ta. Nawet w telewizji o tym gadają. Co musi siedzieć we łbie takiego człowieka, żeby\3k Eee, szkoda gadać. Mam nadzieje, że jak go złapią, to powieszą. I bynajmniej nie za głowę.
\qd

Kolejna wzmianka o mordercy nastąpiła wieczorem, także w wiadomościach.\\
Bez, motywów, i w pewnym sensie świadków -- odkryto jedynie powiązanie -- jedna ofiara śmiertelna, druga, świadek zabójstwa dostaje świra i ląduje w psychiatryku. I niczego nie da się z niego wyciągnąć -- dwóch z sześciu naocznych świadków już próbowało popełnić samobójstwo. Całkiem im odbiło. Tak jak policjantowi, który odkrył pierwsze miejsce zbrodni.
% 
% 
\podro{Rok 2001}
% 
% 
Igor odbezpieczył granat błyskowy i wrzucił go do środka dawnego salonu. Kiedy jego wnętrze wypełnił oślepiający blask i niemal zwalający z ziemi huk, obecny towarzysz Igora, Radek, wpadł do środka z dzikim wrzaskiem. Zaraz po nim do salonu wtarł Igor, rozglądając się na boki.\\
W czterometrowym pomieszczeniu, na pozostałościach z kanapy, wiła się zdezorientowana pijawka.
\sx Wal! -- krzyknął Radek, odpaliwszy w stronę pijawki dwie głowice paralizatora.
\qd
Ustawił napięcie na maksimum -- pijawka przestała przejmować się działaniem granatu -- teraz trzęsła się pod wpływem prądu; rzucała się dziko na lewo i prawo, rycząc wściekle. W ciągu tych dwóch sekund Igor dobył z kabury sześciostrzałowy miotacz strzałek -- dwie z nich nasączono trucizną paraliżującą, kolejne dwie środkami uspokajającymi. Ostatnie zatopiono w zabójczej truciźnie, mieszanki sporządzonej przez samego Igora.\\
Sześć strzałek przecięło powietrze ze świstem, po czym utkwiły w trzęsącym się ciele mutanta.
Radek wciąż nie zmieniał napięcia -- było tak mocne, że na piersi wierzgającego się stwora zaczęły tworzyć się oparzenia.\\
Nagle, bez żadnej zapowiedzi, z sekundy na sekundę, Pijawka zesztywniała, po czym zsunęła się z fotela na podłogę, lądując na plecach.
\sx Wyłącz to! -- wrzasnął Igor do Radka, który wciąż nie zmieniał pozycji. -- Wyłącz! -- Igor krzyknął jeszcze głośniej, co poskutkowało. 
\qd
Operator paralizatora wypuścił broń z ręki, po czym wyraźnie otrzeźwiał. Schylił się po pistolet, załadował go dwiema nowymi głowicami, po czym schował do udowej kabury. Nagle złapał się za nos, wolną ręką odpychając smród.
\sx Pieczona pijawka\3k Zaraz się porzygam. -- wycedził, zmagając się z odorem.
\xx Wynieśmy ją stąd, zanim zasmrodzi całą chatę. -- Igor nacisnął włożoną w ucho słuchawkę. -- Przynieście nosze. -- następne słowa skierował do Radka. -- Na pewno nie żyje?
\xx Mnie pytasz o skuteczność swoich trucizn? Nawet, gdybyś pomylił je z alkoholem, pewnie padła od prądu. Mutant mutantem, ale czegoś takiego nie przetrzymałby Zły Przewodnik.
\qd
Do pokoju przez pustą framugę wtargnęło dwóch stalkerów z noszami -- Mick i Leon. Pierwsze co zrobili, to skrzywili się z obrzydzeniem, czując smród spalenizny.
\sx Bierz nogi, ja za ręce. -- rozkazał Leon, ustawiając się za kanapą. Po chwili, na ,,raz, dwa, trzy'' wraz z Mickiem unieśli pijawkę do góry, po czym, po kilku krokach, niedbale, acz celnie, rzucili na nosze. Mick ułożył ręce na tułowiu i przybliżył nogi, po czym zapiął pasy.
\xx Kto z was jest najsilniejszy albo chce mieć wymówkę do tego, że dał się postrzelić.
\xx Ja. -- zgłosił się Radek. Podszedł do noszy i odczepił pasek, który umocował na kombinezonie.
\xx Chodźmy stąd, bo zaraz padnę od tego smrodu\3k -- szepnął, odganiając od siebie nieprzyjemny zapach, po czym skierował się w stronę wyjście. 
\qd
Reszta kompanów podążyła za nim. Po wyjściu na zewnątrz, pod ciemniejące niebo, rozległy się oklaski.
\sx Brawo! -- powiedział głośno Michał, patrząc z zadowoleniem na ciągnącego Pijawkę Radka. 
\qd
Stłumił kolejny odruch przepony i podszedł do pijawki, otwierając kolejną butelkę wody mineralnej.
-No\3k -- mruknął, po czym wypił połowę pół litrowej butelki jednym łykiem. -- Ten smród\3k nie przejmuj się, mi też się to zdarzało. -- po tych słowach pocieszenia Michał zrobił zdziwioną minę. Przejechał palcem prawej reki po nosie, po czym potarł go o kciuk. Jednocześnie wystawił lewą przed siebie. Po dwóch sekundach oczekiwania, założył kaptur kombinezonu, po czym schował ręce w kieszeniach.
\sx Zaczyna kropić. Wróćmy przed deszczem. Nienawidzę deszczu. Przynajmniej do póki, kiedy nie jestem w przytulnym pokoju z kominkiem.
\qd

Gavin wstał z krzesła, po czym przeszedł przez odległy hol wieżowca w stronę okna. W prawej ręce trzymał nadgryzioną kanapkę, w drugiej butelkę wody.\\
Zaraz po skończeniu snu, o siódmej rano, Gavin przebrał się w zielony sweter i czarne, ocieplane spodnie. Nie zmienił jedynie skórzanych kapci, którymi szurał po podłodze. Była ósma rano, więc sporo pozostało mu do stanu ,,pełnego wybudzenia''? teraz wyglądał jak uzależniony od kofeiny, który rano nie dostał dwóch pełnych filiżanek kawy do wypicia. Poranek był szary, nieprzyjemny i pełen przenikliwego, zimnego oraz suchego powietrza.\\
Po dojściu do pokoju Teda, który opuścił je o szóstej rano, Gavin wyjrzał przez okno na niepowtarzalną panoramę Prypeci.\\
Bloki mieszkalne były w dużej mierze zasłaniane przez brązowe drzewa. Jedną z niewielu rzeczy widzianych z kwatery Teodora był daleki stąd komin elektrowni atomowej. Dumnie górował nad ciągnącymi się dalej lasami i polami. Był bardzo wyraźny i trudno go nie było zauważyć z racji niemal śnieżnobiałych chmur, które podkreślały i eksponowały jego kształt.\\
Co chwilę następował długi i mocny podmuch wiatru, który wyginał lekko rozsiane po Prypeci drzewa, wydając przy tym daleko roznoszący się szum liści. Wiele z nich opadało z koron, dołączywszy do wyblakłych i brązowych listków, które ścieliły ulice i chodniki. Jeśli podmuch wiatru był naprawdę silny, liście wraz z pyłem, kurzem i drobnym piaskiem sunęły razem po asfalcie, przypominając morski piasek podczas sztormu.\\
Gdyby nie wymienione okna, Gavin prawdopodobnie by zamarzł. Nie miał najmniejszej ochoty wychodzić dzisiaj na zewnątrz, chciał spędzić całą dobę w ciepłym pokoju, najlepiej oglądając telewizję. Nie czuł się dziś na siłach, przynajmniej takich, które pozwoliłyby mu wyruszyć w Zonę.\\
Tak, dziś będzie oglądał telewizję, i specjalnie w tym celu przytarga sobie najwygodniejszy fotel, który stał w tej chwili cztery piętra niżej. Po za tym będzie rozmyślał nad osobą, która kupiła od niego Vychlopa.
\sx Co tam? -- spytał ktoś. Gavin rozpoznał głos Huberta -- ochrypły, gruby i wyjątkowo spokojny. -- Mamy dziś specjalnego do roboty? -- dodał.
\qd
Gavin odpowiedział, wciąż oglądając widoki za oknem:
\sx Dziś? Nic. Ktoś w ogóle ma zamiar dzisiaj gdzieś? iść?
\xx Ted i ja.
\xx Dokąd?
\xx Siedziby rządu. Znasz Radka?
\xx Tak, a co? -- Gavin upił duży łyk herbaty, po czym ugryzł spory kawałek kanapki.
\xx A jako kogo go znasz?
\xx Hmm? dobry agent, stalker, który na szczęście się nami nie interesował. A co, zaczął?
\xx On nie, ale Mark powiedział mi, że Dymitr?
\xx Jaki Dymitr?
\xx Ich nowy szef. Barry'ego gdzieś wcięło, gdzieś w okolicach tego wybuchu. To nie było Zwarcie, prawda?
\qd
Gavin zachłysnął się herbatą. Odwrócił się w końcu, spuszczając oko z komina elektrowni, po czym uśmiechnął się do Huberta. Nie szyderczo, raczej uspokajająco. Hubert był wyjątkowo nerwowym i niepewnym człowiekiem, który by uspokoić swoje obawy na jakiś temat, musi przepytać i usłyszeć to, co chce, od co najmniej czterech osób. To rozwiewa jego obawy na jakieś pięć godzin -- potem, przez niekontrolowaną myśl, zaczyna martwić się jeszcze raz. Miał przez to (dziś co prawda śmieszne) -- problemy -- w dzieciństwie, zwłaszcza w przypadku zabiegów i inwazyjnych badań, które potrafiły doprowadzić do omdlenia i nawet przejściowych depresji.\\
Raz Gavin zakpił sobie z Huberta, gdy ten zadręczał go pytaniami na temat zabezpieczeń kwatery głównej przed mutantami -- po jego przykrej reakcji już nigdy więcej nie był w stosunku do niego złośliwy, do czego przekonał też resztę S.\\
Hubert był bardzo nietypowym mężczyzną, którego ilość fobii mogłaby posłużyć jako źródło informacji do napisania scenariusza filmowego -- hipochondryk, w pełnym tego słowa znaczeniu. Kilka jego zachowań uważano powszechnie za zabawne, jednak nie mówiono o tym otwarcie -- podkomendni Gavina zostali uświadomieni, że jest to zwyczajnie nie w porządku.\\
Najbardziej zapamiętał on dzień, w którym Hubert wziął kilka, blisko położonych pryszczy, za objaw choroby popromiennej. Przez nerwy i ciągłe rozmyślania o chorobie (o której zresztą, zaraz po odkryciu niewielkiej wysypki, zaczął czytać i jeszcze bardziej pogłębiać swe zakłopotanie) sprawdzał wykrywaczami promieniowania całe swoje ciało, jak i okolicę. Efekt był chyba do przewidzenia -- szukanie promieniowania w Prypeci jest jak szukanie czegoś mokrego w morzu -- Hubert nie mógł się przekonać do argumentów jego kolegów, że poziom promieniowania, który według Hipochondryka jest źródłem jego ,,choroby'', w Prypeci jest ,,poziomem typowym'', który panuje tu od kilku lat.\\
Kiedy Hubert wmówił sobie objawy ,,na dobre'' i domagał się antyradów, Ted rozwiązał jego problem.\\
Zastosował efekt placebo.
\sx Nie, to nie było Zwarcie. Ale nie masz się czego obawiać, od tego nawet nikt nie zginął.
\xx To dobrze?- Hubert kolejny raz odetchnął z ulgą.
\qd

Dawid spacerował wzdłuż silosów w północno-zachodniej części Złomowiska. Z typowym kombinezonem, AK-103 w lewym ręku i sporym, wypełnionym po brzegi plecakiem, zmierzał w kierunku obozu Wolności, w Wojskowych Magazynach. Było południe, słońce wisiało wysoko na niebie, zalewając wszystko słońcem. Spore przedmioty i konstrukcje, jak silosy, które mijał Dawid, rzucały długie, półprzezroczyste cienie.\\
Ciepłe promienie częściowo niwelowały nieprzyjemny, chłodny wiatr. Dawid wzdrygał się, szczękając zębami za każdym razem, kiedy zawiał.\\
Po piątym w ciągu minuty powiewie, Dawid usłyszał dźwięk telefonu. Wyjął go z prawej kieszeni zziębniętą, trzęsącą się od zimna dłonią, po czym przyjrzał się wyświetlaczowi. Zakaszlał, zasłaniając usta ramieniem; kiedy przestał, odebrał połączenie.
\sx Gdzie jesteś? -- spytał z słuchawki głos, należący do Michała.
\xx Na Wysypisku, w przemysłowych będę za jakieś dwadzieścia minut. Jak ci poszło?
\xx Dobrze. Widzisz jakieś chmury na północy?
\xx Tak.
\xx Zaraz zacznie padać, a one idą w twoją stronę, więc jak nie chcesz zmoknąć, to się pospiesz.
\xx Zdążę. -- zapewnił Dawid. -- Jacy są ci, no?
\xx Ci z urzędu?
\qd
Dawid mruknął na ,,tak''.
\sx Dziwni. Jeden z ich, Igor, to prawdziwy dziwak. Kiedy z nami wędrował, szeptał coś pod nosem?
\xx Prześlij mi jego zdjęcie.
\xx Co? A jak mam?
\xx Prześlij jego zdjęcie! -- krzyknął Dawid, kończąc rozmowę. Igor. sku*wysyn? Ten śmieć? 
\qd
Jeśli to ten, który w obozie dokonuje przesłuchań, Dawid będzie gotów sprowokować bombardowanie nuklearne Zony, byle by zabić tego gnoja. Igora otaczano opieką całego obozu -- nigdy z niego wychodził, rzadko też opuszcza tamtejszą piwnicę, w której spędza całe dnie. Ma tam nawet własną kuchnię i mieszkanie, a do zaspokajania osobistych zachcianek może korzystać z interkomu. Nazywano go Kretem -- siedząc w podziemiach niemal odzwyczaił się od widoku prawdziwego słońca i ich promieni.\\
Mijając betonowy, odrapany przystanek autobusowy, Dawid zauważył siedzącego w nim stalkera.
Jak porażony, wyjął prawą rękę z kieszeni, którą chwycił za lufę AK, teraz skierowanej w stronę nieznajomego. Ten rozłożył ręce, po czym uniósł je wysoko w górę. Był wyraźnie przestraszony -- na jego czole w ciągu trzech sekund pojawiło się kilka kropel potu. Jego oczy niemal wyszły na wierzch, a usta wygięły się w grymasie paniki.
\sx Tylko nie? -- wyjęczał, zmagając się z trudnościami z oddychaniem.
\qd
Karabin Dawida opadł w dół, uderzając o jego nogę ze stukotem. Dawid przetarł lewe oko, po czym sapnął z wysiłkiem.
\sx To? tylko odruch? -- rzekł z uśmiechem i ulgą, widząc znikające z twarzy nieznajomego obawy o swe życie.
\qd
Nie lubił robić innym przykrości, a tym bardziej wymachiwać im przed oczyma bronią.
\sx Mogę? wstać? -- spytał siedzący na przystanku stalker. 
\qd
Miał około trzydziestu lat, bladą, grubo ciosaną skórę i duże, zmęczone, czarne oczy. Był mały i chudy -- niczym typowe szkolne popychadło; jego palce długie i kościste, podobnie jak ubrane w czarne spodnie nogi i wąski tors, na który założono szary kombinezon. Dawid nie widział włosów spotkanego człowieka, gdyż były one zasłaniane w całości przez kaptur.
\sx Wstawaj i przepraszam, po prostu jestem przezorny.
\xx Nie szkodzi? -- chudy mężczyzna powoli się podniósł, po czym szeroko rozciągnął ręce, wzdychając, jakby dopiero co wstał z łóżka po dziesięciogodzinnym śnie. 
\qd
Takim, jaki z pewnością Dawid sobie uczyni, kiedy tylko wróci do Baru.\\
Najpierw jednak wolnym krokiem ruszył w stronę nowopoznanego stalkera, by uścisnąć mu dłoń. Idąc, zarzucił karabin na plecy. Wszedł z szeleszczącej trawy na betonową powierzchnię przystanku i uniósł w górę prawą dłoń. Stojąca naprzeciwko niego osoba zrobiła to samo, jednak nie doszło do uścisku, przynajmniej nie w tym sensie.\\
Kiedy Dawid uniósł rękę na wysokość około pół metra, stojący przed nim najemnik chwycił go za szyję i zaczął dusić.\\
Teraz to on wyglądał na przerażonego, czując uścisk, który był silny niczym stalowa obręcz. Nie zdążył nawet krzyknąć ani się ruszyć -- uniesiona przed chwilą w geście przyjaźni dłoń zesztywniała, podobnie jak lewa.\\
Napastnik odchylił lekko głowę Dawida w tył, po czym z całej siły pchnął go w przód, w stronę betonowego słupa. Dawid grzmotnął weń tyłem czaszki.\\
To uderzenie niemal pozbawiło go przytomności -- zaczął się powoli osuwać wzdłuż betonowej konstrukcji, na której, rozbitą głową, pozostawiał krwawy ślad. Takiego bólu nie czuł nawet, kiedy był przesłuchiwany przez Igora -- doprowadzał go on do szału, tym bardziej, że poczuł się kompletnie bezradny. Był zdany na łaskę nieznanego mu napastnika.\\
Kiedy Dawid głucho opadł na pośladki, jęcząc żałośnie, stojący nad nim stalker wyjął z tylnej kieszeni spodni mały, matowy rewolwer, który natychmiast skierował w stronę Dawida.
\sx Dlaczego?! -- krzyknął z przerażaniem, zanim pierwszy pocisk trafił go w żołądek.
\xx Igor. Nikt go do tej pory nie poznał, i niech tak pozostanie.
\qd
% 
\podro{Rok 1978}
% 
% 
Tego dnia czułem się ,,dziwnie''.\\
Miałem ,,dziwny'' sen tej nocy. W nim, śnie, stałem na najwyższym piętrze jakiejś niedokończonej budowli. Było to co najmniej czwarte piętro -- widziałem jedynie odległe, duże, blaszane konstrukcje skąpane w pomarańczowych promykach słońca jak i trzy duże, jakby stare, dźwigi.\\
Wszystko było zniszczone, jakby zdezelowane, w fazie rozkładu i ogromnego zaniedbania. Beton, na którym stałem, pokryte były grubą warstwą zakurzonych, brązowych i zielonych liści. Spośród, jak później spostrzegłem, zardzewiałych konstrukcji spostrzegłem kilka torów kolejowych -- również starych i zaniedbanych, których nie używano od wielu, wielu lat.\\
Spośród zardzewiałej stali rosły wysokie chwasty, który były na tym terenie jedyną roślinnością, nie licząc wyschłego, gołego drzewa, którego korona sięgała podłogi czwartego piętra.
Niebo było błękitne i bezchmurne, słońce dopiero wspinało się na szczyt, było więc przed południem. Oddychałem rześkim, ciepłym powietrzem, tak też się czułem. Ciepło, z powodu grzejącego słońca, rześko z powodu powietrza i samego poczucia, które w tym śnie miałem wyjątkowo dobre. Czułem się naprawdę dobrze -- jak w pierwszy dzień wakacji, zaczynający się długi weekend czy spotkanie ze znajomymi. Samopoczucie niwelowało nawet moją obecną niepewność.\\
W jakimkolwiek miejscu w tej chwili byłem, było to miejsce całkowicie wymarłe. Nie słychać było najmniejszego szmeru, odgłosu przyrody, zwierzęcia. Panowała martwa, nieprzerwana cisza. Nieprzerwana, nie licząc mego oddechu.\\
Rozejrzałem się dookoła.\\
Na tym piętrze nie wybudowano jeszcze żadnych ścian -- być może nie było to piętro, lecz dach budynku, gdyż nie wiedziałem jeszcze, co jest niżej. Zauważyłem schodzące w dół schody, dwa metry przede mną. Po wykonaniu pierwszego kroku, sceneria snu zmieniła się.\\
W mojej głowie rozległ się ogłuszający dźwięk walących się kamieni i osuwającego się gruzu. Towarzyszył im stukot odłamanych skałek i miękki odgłos osadzania się grubego pyłu na ziemi. Z sekundy na sekundę przeniosłem się do wąskiego korytarza.\\
Ten także należał do niedokończonej budowli, którą jednak wykonano w większej części niż budynek, w którym znajdowałem się jeszcze kilka chwil temu.\\
Korytarz miał kilka metrów długości, miał szerokość ponad trzech. Stałem mniej więcej pośrodku.\\
Ściany były już zbudowane i pomalowane śnieżnobiałą, nową farbą. Po lewej i prawej stronie korytarza znajdowały się framugi drzwi, których w pokoju po lewej jeszcze nie wstawiono. Generalny stan tego miejsca świadczył, że wystarczy jeszcze go ,,wyposażyć'' w odpowiedni sprzęt, do czegokolwiek miałby służyć. Pomieszczenia były gołe i niczego w sobie nie miały, poza wspominanymi drzwiami i przyczepioną do sufitu dużą, długą jarzeniówką.\\
W korytarzu było jasno od słonecznego światła, które wpadało przez progi dwóch pomieszczeń, w których zapewnie zamontowano spore okna.
Poza otoczeniem zmieniło się także moje samopoczucie.\\
Nigdy w życiu tak się nie bałem.

Obudziłem się nad ranem, jak wynikało z ściennego zegara, o piątej. Nie byłem ,,mokry'', nawet ani trochę się nie spociłem, jednak ciągle czułem powoli ulatniający się z mnie strach, którego napędził mi sen. Pierwsze pięć sekund po otwarciu oczu byłem z jego powodu lekko otępiały, później już tylko zaniepokojony. Przez minutę, może dwie, leżałem bez ruchu, tępo wbijając swój wzrok w kołyszące się, pozłacane wahadło, próbując dojść do siebie.\\
Dziesięć po piątej podniosłem się częściowo z łóżka i usiadłem, przecierając oczy. Wyjrzałem za okno, na prypeckie osiedle.\\ Huśtawka, blaszana zjeżdżalnia wraz z piaskownicami i ciągnącymi się wokół drzewkami tonęły w nieprzeniknionym mroku, któremu uległ nawet blask księżyca.\\
Ziewnąłem długo i głośno, podrapałem się po plecach i głowie, po czym podjąłem próby przypomnienia sobie ostatniego snu.\\
Coś w rodzaju szpitalnego korytarza, przedtem placu budowy?

Gomez podszedł do dużej, drewnianej szafki, otworzył ją szeroko, po czym wyjął z niej długi, biały lekarski kitel. Zdjął go z wieszaka, który włożył z powrotem do szafki, same ubranie zaś położył na wysłużonej desce. Wyrównał je w miarę możliwości własnymi dłońmi; dalszą pracę wykonywał żelazkiem po wcześniejszym spryskaniu fartucha wodą.\\
Pierwszy dzień w nowej, upragnionej pracy -- musiał więc zrobić dobre wrażenie, a jedną z rzeczy składających się na definicję tego pojęcia jest wygląd zewnętrzny. Gomez zaczął od ubrania. Spodnie, skórzany płaszcz na zimniejsze dni (było zero stopni, do tego obficie pruszył śnieg) wraz z butami były już gotowe, teraz wystarczyło już tylko przygotować koszulę i wspomniany lekarski ubiór.\\
Był wąski, jako, że Gomez był bardzo szczupłym i zdrowym mężczyzną -- ostatniego razu zważył się i zmierzył tydzień temu -- jego waga wynosiła 80 kilogramów, wzrost zaś metr osiemdziesiąt.\\
Niemal maniakalnie szukając zagięć, nierówności i innych niedoskonałości w fartuchu, Gomez słuchał radia, które stało na drugim końcu salonu, pod ścianą. Trochę dobre muzyki plus typowa, obowiązkowa dawka propagandy.
\sx pałek i milicji mrowie, Marks się przewraca w grobie! -- zanucił Gomez, jeżdżąc żelazkiem po ubraniu.
\qd
Był z siebie dumny (i miał do tego pełne prawo), iż dostał pracę w tutejszym szpitalu. Całe życie spędził na studiowaniu medycyny i niedawno dostał gwarancję, że jego trud nie poszedł na marne.
Wiele mądrych, znanych haseł pasowało by do tej sytuacji, jednak żadne nie wyraziło by w pełni szczęścia Doktora (tak, już był doktorem) Gomeza.
\sx Opłaciło się? -- mruknął, odwracając kitel na drugą stronę. 
\qd
Po wyprasowaniu i jej zabrał się za białą koszulę.\\
23 Grudzień 1978 roku, jeden z niewielu dni, podczas których Gomez tak zadbał o swój wygląd.
\dd
\sx A wiesz, co jest najgorsze? -- spytał Leon, przestępując nerwowo z nogi na nogę. W lewej ręce trzymał palącego się papierosa.
\xx Ta. -- odparł Radek pewnym tonem.
\qd
Siedział na ławce, za szpitalem, w cieniu wielkiego drzewa, który przysłaniał silne promienie słońca. Jego gałęzie nieprzerwanie kołysały się w prawą stronę, ulegając wiatru.
\sx Co? -- Leon przystanął na chwilę, obok kamiennego śmietnika, by mocno się zaciągnąć.
\xx To, że palisz.
\xx On jest zdenerwowany, zrozum to. -- mruknął siedzący obok Radka, Mikołaj. -- Próbuje się\3k -- zaczął, lecz przerwano mu brutalnie:
\xx Uspokoić? -- Radek nie przestawał atakować. -- Leon, ile czasu masz zamiar spędzić u Jonathana, kiedy już skończysz?
\xx Bo ja wiem? pół godziny i wracamy do siebie.
\xx Jak będziesz dalej kopcił i udawał dorosłego, spędzisz u jego boku kilka tygodni, bo wylądujesz z nim w jednej sali. Choć, bo zanosi się na burzę.
\xx A gdzie Kamil? -- zapytał Mikołaj, podnosząc się z ławki. Jęknął głośno, łapiąc się za biodra. -- W sklepie?
\xx Tak, mamy na niego zaczekać, na górze.
\xx Jak zareaguje?
\xx A skąd mam wiedzieć? -- oburzył się lekko Radek, także wstając na równe nogi. Podparł się lewą ręką, prawą zaś miał luźno zwieszoną u boku. -- Te zakwasy\3k
\qd
Po minucie, Radek z Leonem i Mikołajem u boku ruszyli w stronę głównego placu Prypeci, na front od szpitala.\\
O tej porze Prypeć była bardzo tłoczna -- ulice wręcz pęczniały od przechodniów; dzieci, kobiet, dorosłych i młodych mężczyzn, czasem osobno, czasem razem, całą rodziną. Od większości mieszkańców bił entuzjazm, niemal wszyscy wyglądali na pogodnych, szczęśliwych i zadowolonych. Szczególnie było widać to po plotkującej parze młodych dziewczyn z wózkami, które właśnie obgadywały miejscowego przystojniaka, czy też domową kuchnię nie lubianej sąsiadki.\\
Część dzieci szła u boku matki czy ojca, a to trzymając rodzica za spodnie, spódnicę, czy też zwyczajnie za rękę. Wiele roześmianych dzieci przemierzało plac na placach równie uśmiechniętych rodziców, którym sprawiało to nie mniejszą frajdę.\\
Cynik nazwałby ten widok ,,sielankową sieką, z której można zrobić by zdjęcie na pocztówkę'', jednak ktoś, komu pozytywne emocje takie jak te nie są obce, byłby zwyczajnie zadowolony, i obserwowałby mieszkańców małego ukraińskiego miasta z czystym zadowoleniem.\\
Trójka przyjaciół minęła rząd ławek, na których siedziało kilka starszych osób, czytających gazety i książki. Wielu z nich broniło się przed południowym słońcem, wkładając czarne okulary przeciwsłoneczne.\\
Radek zaproponował, by w oczekiwaniu na Kamila przysiąść na schodach prowadzących do głównego wejścia do kliniki.
\sx Myślicie, że jak zareaguje? -- spytał Radek po usadowieniu się na skraju betonowego stopnia. Odpowiedział mu Mikołaj:
\xx Pewnie się załamie. Wiesz, że oni są sobie najbliżsi.
\xx Ciekawe, dlaczego\3k
\xx Kiedyś\3k -- zaczął głos, który rozległ się za trójką siedzących na schodach nastolatków. Wszyscy trzej aż podskoczyli i odwrócili się wystraszeni.
\xx \3k może się dowiecie. -- skończył Kamil z uśmiechem. -- Chodźmy, bo im dłużej tu stoję, tym bardziej się denerwuję.
\qd
% 
\ro{30}
% 
% 
Idąc do sali, w której przebywał Jonathan, Kamil napotkał jego ojca -- Jana, niegdyś Ulissesa Mastertona. Wyszedł zza rogu korytarza prowadzącego na oddział pediatryczny, zmierzał w kierunku schodów, wyraźnie mu się spieszyło.

Chociaż znałem Jonathana niemal tak dobrze jak jego ojca, dopiero teraz, chyba ze względu na okoliczności, spostrzegłem pomiędzy nimi wielkie podobieństwo w wyglądzie.\\
Jan miał ponad metr osiemdziesiąt wzrostu i dobrą budowę ciała -- zaliczał się do osób pomiędzy ,,ludzi z sporą muskulaturą'' a tych, których określano mianem ,,dość bliskich do grubasa''.
Jego budowa twarzy -- wielkość nosa, ostre rysy, lekko rozdzielony podbródek i smukłe, zapadłe kości policzkowe dopiero dziś wyglądały dla mnie identycznie z twarzą Jonathana, która różniła się jedynie szerszym czołem, wielkością oraz umiejscowieniem oczu oraz szerokością ust, które u młodego Mastertona były znacznie węższe.\\
Siwiejący lekko zarost trzydziesto siedmio letniego Ulissesa lepił się od potu, podobnie jak włosy. Oczy wysokiego mężczyzny były przekrwione i potwornie zmęczone, kąciki zaś pełne powstałych w wyniku zmartwienia zmarszczek. Ne lewym, szorstkim policzku błyszczało kilka zaschniętych łez.
W nieładzie było także ubranie Jana -- spodnie oraz koszula były pogniecione i pomięte, jakby noszono je bez przerwy przez ostatni tydzień. Z pach ojca Jonathana donosił się silny smród potu, zaś z ust przykra woń alkoholu. Niezależnie, ile wypił ostatniej nocy, teraz był trzeźwy, choć w stanie zakrawającym o ,,opłakany''.\\
Nie zwrócił żadnej uwagi ani na mnie, ani na Leona, Radka czy Mikołaja. Mijając nas i mając tego doskonałą świadomość, mruknął tylko trzy słowa zmęczonym głosem:
\sx Reszta już czeka.
\qd
Po tych słowach zaczął szybko schodzić po schodach. Wyglądało na to, że miał ochotę wyskoczyć ze szpitala przez okno.\\
Chciał stąd jak najszybciej wyjść i, jak podejrzewałem, uspokoić się poprzez kolejne kilka spojrzeń w dno kielicha.
\sx Jak on wygląda? Jak się czuje? -- zawołałem w momencie, w którym Jan pokonał pierwszą część schodów i skręcał, by w końcu zejść na parter.
\qd
Słysząc moje zapytanie, stanął w miejscu, po czym nie odwracając spojrzenia (zapewne wlepiał je w drzwi wyjściowe ze szpitala) odpowiedział krótko i boleśnie:
\sx Gorzej ode mnie. -- nigdy w życiu nie powiedziano mi czegoś tak dobitnie, szczerze i krzywdząco. 
\qd
Ojciec młodego Mastertona najwyraźniej i z tego zdał sobie sprawę, gdyż po ostatnich stopniach zszedł jeszcze szybciej, wręcz z nich zbiegł.\\
Do goniącego do widma okaleczonego oka Jonathana dołączyła świadomość złego potraktowania jego najlepszego przyjaciela.\\
Byłem pewien, że od zaglądania w kielich oczy będą go boleć do jutra.\\
Oby tylko jego wątroba to przetrwała -- jeśli nie, Jonathan zostanie zupełnie sam.\\
Nawet ja nie wystarczałbym mu w takiej sytuacji.
\sx Zepsujmy sobie w końcu nastroje. Wolę świadomość tego, co mu się stało niż tą\3k niepewność. -- powiedziałem. -- Jestem niemal pewny, że to Adrian. Chyba go, ku*wa, zabiję! -- krzyknąłem, po czym wbiegłem na górę. 
\qd
Towarzysząca mi trójka podążyła za mną podobnym tempem.

Zaraz po opuszczeniu sali Jonathana, Kamil wraz ze znajomymi ruszył w stronę schodów prowadzących na parter szpitala. Był wściekły -- po trochu na Adriana, po trochu na samego siebie.\\
Nigdy otwarcie nie okazywał złości -- kiedy się denerwował, nie dał po sobie tego poznać zachowaniem; ruchem warg, oczyma czy nawet podniesionym tonem. Wyjątkiem była sytuacja sprzed kilkunastu minut. Kiedy był naprawdę zły, zdarzało mu się zgrzytać zębami, co robił w tej chwili; tak intensywnie, że rozbolała go szczęka.\\
Za to, gdyby zamienił uczucia na odznaki zewnętrzne, prawdopodobnie zmiótł by wszystko gołymi rękoma w zasięgu kilometra. Teraz jeździł jedynie po sobie stałymi siekaczami i milczał niczym głaz -- momentami bał się, że mimowolnie krzyknie, jeśli choć lekko otworzy usta.\\
Za to Leon, Mikołaj, a szczególnie Izaak byli pod tym względem niebywale wylewni. Od momentu przyznania przez Mastertona, iż oko rozciął mu Adrian, nieustannie klęli pod nosem i wymyślali różnorakie sposoby na zemstę -- od doniesienia na policję na zabójstwie skończywszy. Oczywiście by do tego nie doszło -- było to odruchowe zachowanie, poprzez które przyjaciele Jonathana rozładowywali nerwy. Świadomość rzekomej gotowości do podobnego czynu była jedynie objawem złości, która prawdopodobnie minie do dzisiejszego wieczora, kiedy wszyscy zaczną już myśleć ,,trzeźwo''.\\
Schodząc na dół, Leon przypadkiem potrącił młodą pielęgniarkę. Na początku zrobiła rozgniewaną minę, lecz już chwilę później uśmiechnęła się nieśmiało, słysząc przeprosiny. Zaraz po nich Leon zbiegł na dół, przecisnął się przez kolejki i grupy schorowanych ludzi, po czym uderzając barkiem, brutalnie otworzył drzwi na zewnątrz. Szybko skręcił w lewo, po pokonaniu trzech metrów siadł na środku zielonej ławki.\\
Kaszlnął, po chwili zaś zaczął masować czoło, próbując się uspokoić.

Po kolejnym głośnym i świetlistym grzmocie Mikołaj wstał z kanapy i zasłonił ostatnie okno w salonie swojego mieszkania. Od co kilkusekundowego jasnego rozbłysku w całym pokoju rozbolały go już oczy.
Zanim zasunął żaluzje, rozejrzał się po wyludnionym placu zabaw i okolicy jego osiedla -- blaszana zjeżdżalnia i wszystkie podobne powierzchnie lśniły metaliczno-księżycowym blaskiem, dodatkowo wzmocnionym przez krople czystego deszczu. Wpatrując się w ciemną noc, Mikołaj stukał palcami o szybę okna, rozmyślając o różnych sprawach.\\
Jonathan i jego ojciec. Nie może tak po prostu się zapić -- zdarzają się gorsze rzeczy od rozcięcia oka i gnębienia ,,przez starszych'', czego ofiarą padł młody Masterton. Po kilku miesiącach, maksymalnie dwóch latach, wszystko wróci do normy.
\sx Znowu myślisz o Jonathanie? -- spytał Kamil, który leżał na kanapie obok z przykrytą przez poduszkę połową twarzy. 
\qd
Tak jak się spodziewał, cała złość zdążyła już wyparować.
\sx Szósty zmysł? -- pomyślał sobie Kamil w duchu. -- Inaczej tego nazwać nie mogę.
\xx Pamiętaj, co ci powiedziałem. -- kontynuował rzekomy posiadacz owego zmysłu. -- Są gorsze tragedie, z których ludzie wychodzą cało. Raki, amputacje, śpiączki i choroby psychiczne -- jego oko to w porównaniu z tym istny pikuś. Więc nie myśl o tym -- stało się. Myśl, jak to naprawić, nie tylko jemu wyjdzie to na dobre.
\qd
% 
\podro{Rok 2001}
% 
% 
Lech rozejrzał się na boki, upewniając się, że nikt go nie obserwuje. Kiedy to zrobił, wyjął z kieszeni metalowy wytrych, który włożył do zamka drewnianych drzwi. Z pewnością, jedno z dziwniejszych zleceń, jakie dostał do tej pory. Dziwniejszych, nie trudnych, choć do łatwych też z pewnością nie należało.\\
Wykonując każde zlecone zadanie, Lech lubił w czasie tym przypominać sobie słowa jego zleceniodawcy. Pomagało mu to utrwalić ważne szczegóły (czasem, jeśli było to konieczne, wykonywał notatki) , niezbędne do wypełnienia kontraktu. Poza tym, po prostu ułatwiało mu to skupienie się, całkowitą koncentrację, uporządkowanie myśli i tym samym usunięcie tych zbędnych.\\
Osoba, której Lech miał się pozbyć, była znanym ,,wojownikiem'' na Barowej arenie, i właśnie z tego powodu stała się obiektem zlecenia dla Najemników. Kelvin, bo tak nazywał się obecny czempion, słynął ze swojej brutalności. Ponoć zdarzały się sytuacje, w których kierownik areny wyłudzał od obstawiających walki dodatkowe pieniądze, obiecując, że starcia będą wyjątkowo krwawe -- jako, że w Zonie nie brakowało tego typu popaprańców, organizator walk często podpowiadał Kelvinowi, w jaki sposób toczyć ma się walka. Niektóre miały trwać jedynie kilka sekund, inne zaś miały toczyć się kilkadziesiąt minut, co na tak małym polu walki jest nie lada wyzwaniem.
\sx Wiesz, Lechu\3k Z czasem walki na arenie zaczynały przypominać rynek porno -- tyle, że tutaj ludzie wyżywali się patrząc się na mordujących się nawzajem bliźnich. Rzecz jasna, ci bogatsi, mający więcej do zaoferowania mieli bardzo ,,wygórowane'' wymagania. Zabójstwo nożem, daną bronią czy nawet wybranym rodzajem amunicji to jeszcze nic. Jedni chcieli, by pokonany należał do którejś z frakcji, miał na sobie wybrany przez klienta kombinezon, z czasem zaś arena zaczęła być alternatywą dla takich jak my, czyli najemników.\\
Pomyśl tylko -- gość płaci kupę szmalu i chce w zamian, by porwano konkretną osobę, naszprycowano ją narkotykami i puszczono na arenę na pewną śmierć. Nasz klient zaś\3k cóż, jego brata spotkał właśnie taki los. Pewien stalker, którym też wkrótce się zajmiemy, zapłacił szefowi całej tej rzeźni, by zorganizował porwanie, a uprowadzoną osobę, czyli brata naszego klienta, posłać na arenę do walki z Kelvinem.\\
Cyprian zajął się tym, który zlecił porwanie i ludźmi, którzy w nim uczestniczyli. Niestety kierownik areny jest póki co nietykalny\3k Ty zaś zajmiesz się Kelvinem, a klient życzy sobie, by zginął w wyjątkowo haniebny sposób.
\qd
Wchodząc do magazynu areny, Lech ponownie rozwiał wszystkie możliwości co do słuszności zlecenia. Kelvin zasłużył na śmierć -- przez takich jak on, czyli ludzi gotowych zrobić wszystko za pieniądze, Zona stawała się coraz gorszym miejscem.\\
Sam Najemnik nie robił dla pieniędzy niczego bez powodu. Stalker, który zginął na arenie, pewnie nie zdając sobie nawet z tego sprawy, nie zrobił nic złego -- jego jedyną winą były niekorzystne okoliczności -- stał się jedną z losowych ofiar pogoni za pieniędzmi.\\
Lech zamknął za sobą drzwi, po czym wolnym krokiem skierował się w głąb wolnego, pokrytego starą, zielonawą farbą, pomieszczenia.
\dd\sx Gotowy? -- spytał Josh, odwracając się przez plecy. Otworzył skrzynkę z bronią i wyjął jej zawartość na stół obok.
\xx Jak zwykle. -- odparł Kelvin pewnym siebie tonem. 
\qd
Z niecierpliwością obserwował, jak kolejne sztuki jego broni lądowały na obrusie. Najpierw pojawiła się na nim Beretta, później sześć dodatkowych, załadowanych magazynków do niej. Następnie Josh wyłowił z głębokiej skrzynki Ingrama MAC10, obok którego następnie ułożył trzy pary złączonych ze sobą taśmą, wydłużonych magazynków.\\
Na koniec Kelvin dostał trzy granaty odłamkowe i jeden ogłuszający.\\
Podszedł do stolika z zadowoloną miną i dozbroił się -- 92-kę schował do udowej kabury, magazynki do niej w kieszeniach obok, zaraz przy kaburze. Ingrama zaś zawiesił na prawym ramieniu przy pomocy zamszowego, czarnego paska -- dodatkowe magazynki Kelvin wsadził do ładownic zawieszonych na klatce piersiowej. Granaty umocował przy pasie, u którego zwisał też duży, wojskowy nóz.
\sx Powodzenia. -- powiedział Josh. -- I pamiętaj, ta walka ma trwać co najwyżej dwie, trzy minuty. To pierwsze tego ranka, więc na dobrą rozgrzewkę w sam raz. Wywal co najwyżej jeden magazynek, bo widzowie dostaną euforii na sam początek, a nie o to chodzi, prawda?
\xx Się wie. -- zapewnił Kelvin, po czym wyszedł z pokoju Josha.
\qd
Zamknął za sobą drzwi i rozejrzał się po okolicy Rostoka. Na prawo od niego znajdowało się wejście do osławionego baru, nazwanego ,,Sto Radów'', dalej, jeśli pójdzie się na wprost, znajdzie się niewielką placówkę Powinności, frakcji niesłusznie oskarżanej o współpracę z rządem Ukrainy.\\
Cały ten teren był bezpiecznym miejscem na postój dla wszystkich stalkerów, swoistym przedsionkiem terenu zwanego Dziczą -- obszaru przemysłowego, który przed wybuchem w 86 roku pełnił ważną rolę w całym Czarnobylu. Roiło się na nim od przemysłowych budynków, dźwigów i magazynów, a także stacji kolejowych i rzecz jasna torów -- kiedy działały, pociągi codziennie rozwoziły węgiel i inne potrzebne surowce na okoliczne tereny.\\
Sama Dzicz była dość niebezpiecznym miejscem -- zamieszkiwało w nich wiele gatunków mutantów, z których, na nieszczęście stalkerów, większość stanowiły osławione Pijawki. Było to chyba najbardziej znienawidzone przez wszystkich mieszkańców Zony stworzenie, które skutecznie odstraszało odważnych i głupich samotników.\\
Od czasu do czasu dwie największe w Strefie frakcje -- Wolność i Powinność, wysyłały patrole i grupy swoich członków na jednodniowe wypady. Zwykle trwały one od rana do wieczora, zależnie od pory roku, jednak najważniejsze było wydostanie się z Dziczy przed zapadnięciem zmroku.\\
Spierano się, czy lepiej jest zginąć od śmiertelnej rany zadanej przez mutanta, czy też spędzić w Dziczy całą noc.\\
Tak czy siak -- w Rostoku każdy mógł czuć się bezpiecznie, przestać przejmować się otaczającymi go terenami. Tutaj nikt cię nie zastrzeli (chyba, że na arenie, do której zresztą nikt cię nie zmusi), nie grozi ci też tu żaden mutant. Tu byłeś pewien, że nic ci się nie stanie.\\
Raj dla tych, którzy lubili trzymać się na uboczu; mekka stalkerów, którzy po dniu niebezpiecznych łowów na artefakty marzyli o odpoczynku i rozmowie przy posiłku.\\
Kelvin przestał rozmyślać o Rostoku i wolnym krokiem skręcił w lewo, do wejścia na teren starć.\\
Euforia?\\
Duma?\\
Ciekawość, niepewność, zadowolenie, oczekiwanie, ekscytacja?\\
Które z tych uczuć tak naprawdę dominowały nad innymi, kiedy wchodziło się pokrytą piaskiem, pełną poustawianych beczek, kartonów i innego rodzaju barykad halę areny?\\
Według Kelvina, był to strach.\\
Ten zawsze i bez wyjątku, przez pierwsze dziesięć sekund, panował nad Kelvinem. Przez te dziesięć sekund musiał się wewnętrznie uspokoić i przygotować -- odgonić myśli o ewentualnej porażce, wypełnić głowę wizjami zwycięstwa, chwały i wiwatującymi widzami, których ryk wzrasta z każdym strzałem, który padł podczas walki.\\
Kelvin chwiejnie przekroczył blaszaną bramkę i z zamkniętymi oczyma, wsłuchiwał się w wiwaty, niczym dyrygent w kierowaną przez siebie orkiestrę. Czuł się jak ryba w wodzie, choć będzie to krótka walka.
Kiedy tłumy przycichły dość, by dało się usłyszeć upiorny świst wiatru wdzierający się w blaszany, rozklekotany dach areny, Kelvin wykonał pierwszy ruch.\\
Szybkim tempem pobiegł w prawo, z Ingramem w prawej ręce, zaś granatem ogłuszającym w drugiej. Minął dwie leżące na ziemi, czerwone beczki i przyparł do wysokich skrzynek, dwa metry dalej. Jego rywal w tym czasie pobiegł w lewo, lawirując pomiędzy blaszanymi osłonami, także na podobną odległość.\\
To będzie krótka walka.\\
Kelvin wyszedł zza skrzynki z pistoletem maszynowym uniesionym w górze, przebiegł bokiem sześć metrów w lewo, po czym kucnął przy zielonym kontenerze, wypełnionym po brzegi ciężkimi beczkami. Jego oponent wciąż czekał, przyczajony za kawałkiem blachy. Nie liczył, że go osłoni -- chciał się jedynie schować, by w pewnej chwili pokonać kolejne pięć metrów.\\
Obecny czempion areny powstał na równe nogi i odbezpieczył trzymany w ręku granat. Kiedy zawleczka upadła na brązowym piachu, Kelvin wychylił zza kontenera prawą dłoń, po czym błyskawicznie schował.\\
Dwa pociski z MP5 załomotały w blaszany róg, po czym odbiły się rykoszetem w lewo, lądując całkiem już zniekształcone na betonowej ścianie.\\
Tłum momentalnie ryknął, przepełniony euforią.\\
Kelvin, wciąż pozostając za pełnym żelastwa kontenerem, rzucił granat.\\
Ten po pokonaniu w powietrzu jednego metra odbił się od dachu blaszanego pojemnika, po czym wylądował za blachą, przy której czaił się stalker.\\
Huk i błysk były tak mocne, że nawet połowa widowni odruchowo schowała twarze w ramiona. Kelvin wybiegł zza osłony i sprintem zaczął zbliżać się do zdezorientowanego, miotającego się na wszystkie boki mężczyzny. Stanął metr przed nim.\\
Tłum zaczął skandować hasła typu ,,Zabij go!'', ,,Dobij!'' ,,Wykończ go!'', krzycząc coraz głośniej.\\
Kelvin nie miał powodu, by przedłużać pojedynek. Uniósł Ingrama przed siebie, wycelował w głowę przeciwnika i nacisnął spust.\\
Wydawało się, że wrzask tłumu był ponad dwukrotnie głośniejszy od wystrzału. Oba te dźwięki poniosły się szerokim echem po sali. Chwilę później rozległy się jęki niezadowolenia.\\
Choć Kelvin wystrzelił, jego wróg nie odniósł żadnej rany.
\sx Co do? -- szepnął czempion, po czym przyłożył pistolet do ramienia i począł opróżniać cały magazynek.
\qd
Mechaniczny, ogłuszający terkot zdawał się nie mieć końca. Kiedy w końcu setki ech opadło, tłum zawył jeszcze głośniej, niż dotychczas.\\
Zanim efekt grantu błyskowego ustał i Kelvin niechlubnie stracił swój cenny tytuł, usłyszał spośród wszystkich krzyków chyba najbardziej szokujące wyznanie w życiu.
\sx Masz ślepaki, idioto! -- krzyczał jeden ze stalkerów, bijąc rękoma o ścianę. Prawdopodobnie obstawił wszystkie swoje pieniądze. Przegrał.
\qd\dd\sx
Będziesz się tam gapić cały Boży dzień? -- spytał Ted, wchodząc do swojego pokoju na piętrze.
\qd
Gavin od godziny siedział w tym samym miejscu, wyglądając przez okno na okolice centrum Prypeci, zerkał też co jakiś czas na elektrownię atomową.
\sx Nie. -- odpowiedział Gavin. -- Za dwie godziny idziesz ze mną do super samu.
\xx Po kiego? -- oburzył się Ted, stanąwszy jak wryty w pół kroku. -- Miałeś dziś\3k
\xx Odechciało mi się, cokolwiek miałem dzisiaj robić. -- po tych słowach dowódca S. podniósł się na równe nogi, odwrócił się na pięcie i ruszył w stronę wyjścia z pomieszczenia.
\xx Oglądać telewizję, to miałeś zamiar robić dziś cały dzień. -- po tych słowach ręka Teda oparła się o framugę, zagradzając drogę.
\qd
Gavin zatrzymał się nagle i niespodziewanie, robiąc zażenowaną minę.
\sx Czego? -- spytał zrzędliwym i zrezygnowanym głosem.
\xx Sam. Czego tam szukasz?
\xx Nigdy nas tam nie było\3k
\xx I co z tego?
\xx Czas chyba w końcu zbadać Zonę do końca, nie? -- Gavin uśmiechnął się, zadając to retoryczne pytanie.
\qd
Zignorował wystawioną rękę swego kompana, pod którą przeszedł, nisko się pochylając. Schodząc po schodach na niższe piętro, gdzie miał zamiar się przygotować, usłyszał krzyk Teda.
\sx Co, może następny będzie szpital?
\qd
Gavin po pokonaniu pięciu stopni upił kolejny łyk napoju.
\sx Czemu nie? -- mruknął pod nosem, znowu chwyciwszy się barierki.
\qd

Omawiany supersam miał zostać największym tego typu sklepem w całej Prypeci -- miał oferować wiele produktów z wielu branż, między innymi żywność. Znajdował się kilkanaście metrów od 16-to piętrowego wieżowca, kwatery S, z którego dachu zresztą było idealnie widać sklep poniżej.\\
S. zajęła się, w co trudno uwierzyć, eksploracją Prypeci od całkiem niedawna, ledwie trzech miesięcy. Wcześniej próbowała zachować swoje istnienie w tajemnicy, przygotowywała też swoją obecną kwaterę, jak i zdobywała kontakty w całym Czarnobylu, od Wysypiska na Rostoku kończąc. Był to okres, podczas większość członków tej frakcji zginęła nie w ,,trudnych warunkach'' czy walce, lecz z powodu własnej głupoty i braku wiedzy na temat Strefy -- dwie osoby zwyczajnie weszły w anomalie, myśląc, że ,,Spalacz'' to gorące powietrze powodowane temperaturą.\\
Jakiś czas temu S zaczęło stosować najbardziej podstawową zasadę panującą w Zonie -- ,,Nie łaź gdzie popadnie.'' Te cztery słowa niejednemu ocaliły życie; one i rzecz jasna, kilogram śrub lub innego zbędnego żelastwa.\\
Grupie Gavina do całkowitego poznania Prypeci zostało niewiele -- jedynymi ulicami, których S nie znała na wylot, były ulice Ukrainki i Gorbaczowa, licząc tamtejsze budynki mieszkalne. Z tych ważniejszych ,,obiektów'' pozostał właśnie supersam, teatr, kino, kawiarnia ,,Prypeć'' i niesławny szpital. O tym ostatnim należało całkowicie zapomnieć, zaś z reszty wymienionych budowli najniebezpieczniejszą był teatr, w którym roiło się od mutantów.\\
Ostatnio nawet widziano tam Kontrolera.\\
Z obserwacji dokonywanych z wieżowca wynikało, że supersam jest całkowicie opustoszały -- od czasu do czasu jedynie ,,przewinął'' się przez niego pies, ew. zombie. Było to dość napromieniowane miejsce, zwłaszcza okolice metalowych wózków sklepowych, które wchłonęły promieniowanie niczym gąbka. Mimo to nie przekraczał od poziomu, który powstrzymałby członków S do dostania się do środka dyskontu -- dobór odpowiednich artefaktów i przede wszystkim ostrożność ochronią ich przed śmiertelnymi skutkami promieniowania.\\
Całej frakcji niezaprzeczalnie przydała by się książka Autorów. Między innymi Ted był niemal stuprocentowo pewny, że pierwsi eksploratorzy Strefy dokonali wielu ciekawych notatek na temat nie otwartego dyskontu.

Kilkadziesiąt minut później Gavin wrócił do swojej kwatery. Miała ona pięć metrów kwadratowych, odnowione ściany, sufit i podłogę. Nie było tu mowy o żadnym luksusie, jednak dowódcy grupy wystarczyło odróżnienie jego pomieszczenia od reszty bliźniaczych wnętrz bloków w Prypeci -- brudnych, zaniedbanych, pełnych pyłu i gruzu, nie wspominając o paskudnie złażącej farbie, grzybach ściennych i tych podobnych oznak ,,starości'' ścian.\\
Gavin przysiadł na obrotowym, biurowym krześle, odsuniętym na metr od biurka z komputerem, po czym chwycił w prawą dłoń słuchawkę interkomu. Zmienił częstotliwość na ,,ogólną'', odchrząknął i wcisnął duży, czarny przycisk.
\sx Ted, Hubert i Paweł -- za dziesięć minut zbierzcie się na drugim piętrze, gotowi do wyjścia. Mało amunicji, niech każdy ma przy sobie co najmniej 3 kilo żelastwa i po jednym ,,Krysztale''. Wrócimy do pierwszej na obiad, zbadamy tylko supersam. -- zakomunikował, po czym wsłuchując się w echo swych słów, które obiegły cały budynek, dodał: -- Marek, przynieś jego plany. Marcin, Kamil, idziecie na dach.
\qd
Po odłożeniu słuchawki na biurku, Gavin powstał z krzesła i ruszył do zbrojowni.

Marcin po pokonaniu kolejnych dwóch, kamiennych stopni, uderzył końcem, a dokładniej mówiąc, hamulcem wylotowym swego Barreta w stare, drewniane drzwi. Ustąpiły od razu, uchylając się z cichym skrzypnięciem, które zostało jednak w dużej mierze zagłuszone przez głośny wiatr. Tu, na dachu, poważnie dawał się we znaki, był głośny i nieprzyjemny.
\sx Nie rób tak\3k -- rzekł Kamil, podążając ze swym towarzyszem po schodkach. Kiedy postawił stopę na czarnej, śmierdzącą wciąż smołą, powierzchni, dodał: -- Nie po to wydaliśmy na to tyle kasy, byś sobie tym wyważał drzwi.
\xx Eee tam\3k -- Marin zlekceważył usłyszaną uwagę, po czym wolnym krokiem obszedł dach.
\qd
Przy bezchmurnym, czystym niebie, widoczność stąd była wręcz doskonała.\\
Sam dach w dużej mierze pokryty był gruzem i odpadłym tynkiem, głównie pochodzącym od wciąż niszczejącej betonowej klatki schodowej, przy której wybudowano dumny znak sierpa i młota oprawionego w kłos zboża, pośrodku którego wisiała radziecka, pięcioramienna gwiazda.\\
Niegdyś widoczny z daleka i wyróżniający się znak, dziś niemal całkiem zardzewiał, tym samym zlewając się z resztą podobnych, wyblakłych i niezauważalnych obiektów. Jego wygląd był niemal przygnębiający. Typowe widmo przeszłości.\\
Po za tym szczyt siedziby S ,,zdobiło'' kilka, blisko rozstawionych siebie, niskich kominów o kształcie prostokąta i niewiele wyższy słupek, z którego sterczało kilka metalowych linek.
\sx Gdzie to? -- zaczął Marcin, obracając się dookoła.
\xx Tam. -- podpowiedział Kamil, wystawiając przed siebie rękę.
\xx Ok?
\qd

Dziesięć minut później Gavin, Hubert, Paweł i Ted byli gotowi do wyjścia. Zebrali się na schodach, pomiędzy pierwszym a trzecim piętrem. Odprawę mieli już za sobą, teraz czekali tylko, aż dwójka na dachu się przygotuje.
\sx Ile jeszcze? -- spytał Ted, mówiąc do mikrofonu. Chwilę później na zniszczonej klatce schodowej rozbrzmiała odpowiedź, zniekształcona elektronicznym szumem.
\xx Pięć minut.
\qd
Hubert sięgnął do kieszeni SEVY i wyciągnął z niej paczkę Marlboro. Podobnie jak reszta, schował czarną, błyszczącą zasłonkę na twarz.
\sx Komu? -- spytał z uśmiechem.
\qd
Trzymając otwartą paczkę, z której ubyły trzy papierosy, wolną ręką dobył i zapalił srebrną zapalniczkę. Ted, Gavin i Paweł po kolei schylali się z fajką w ustach, by zapalić ją od Zippo Huberta. Kiedy każdy z nich w końcu pierwszy raz się zaciągnął, zapalniczka zatrzasnęła się z suchym dźwiękiem a jej posiadacz wszedł po schodach na wyższe piętro. Wierzył w bierne palenie.

Dwójka stalkerów położyła się na czarnym dachu wieżowca.\\
Kamil, z lornetką zawieszoną u szyi, rozglądał się po okolicy. Podziwiał, jeśli można było to tak ująć, bliźniaczo podobne, opustoszałe budynki mieszkalne i częściowo zasłaniające je drzewa. Jednak większość swojej uwagi skupiał na elektrowni. Był blisko niej, właściwie, gdyby miał przy sobie naprawdę dobry sprzęt, mógłby stąd zauważyć nawet muchę, która osiadła by na powierzchni sarkofagu.\\
Kamil odczuwał jej siłę, tajemniczość, odnosił wręcz mistyczne, nierealne wrażenie, wpatrując się w obity tonami blachy komin. Sarkofag przytłaczał go swym ogromem.
\sx Pięć minut, Marcin.
\qd
Operator karabinu skinął głową i począł ustawiać M82. Najpierw zdjął go z pleców i położył obok, na zaschniętej papie. Przesunął ciężki, wciąż złożony pod lufą dwójnóg w przód, odchylając go na 90 stopni w dół, po czym rozłączył go na dwie części. Po odpowiednim ,,rozszerzeniu'' dwóch oparć, Marcin postawił je na dachu, kierując lufę w kierunku domu kultury, przed którym zaś stał dyskont spożywczy.\\
Barret miał już magazynek na swoim miejscu, Marcin dostosował też wcześniej lunetę do odpowiedniej odległości. Kiedy sięgał palcem do bezpiecznika, zauważył coś kątem oka. Puścił broń, opierając kolbę o podłoże, po czym przewrócił się na plecy, po czym zaczął wpatrywać się w bezchmurne, bezkresne niebo.\\
Samolot.
\sx Patrz\3k -- wyszeptał cicho.
\qd
Kamil również obrócił się o sto osiemdziesiąt stopni. Szybko spostrzegł źródło ,,zachwytu'' Marcina.\\
Nad nimi, z północy na południe, wysoko nad ziemią, szybował potężny odrzutowiec pasażerski. Leciał na dużym pułapie, toteż nie było słychać jego potężnych silników. Pozostawiał za sobą białą smugę, mocno kontrastującą z błękitnym niebem.
\sx Szczęściarze\3k -- mruknął Kamil wzruszonym tonem. 
\qd
Każdorazowe spojrzenie na przelatujący samolot czy jakąkolwiek rzecz czysto związaną ze światem zewnętrznym, powodowała w większości stalkerów potężną nostalgię, tęsknotę ale i żal do siebie za to, że przekroczyło się Granicę do Strefy. Czasem był to żal do innej osoby, było tak w przypadku Marcina, Gavina i większości osób z S, na których wojsko poza Zoną wydało wyrok śmierci. Mieli wielkie szczęście, że rząd Ukrainy całą swoją uwagę skupia na zwykłych, prostych stalkerach. W Prypeć wojskowi zapuszczali się okresowo -- co trzy tygodnie ulicami Miasta Duchów przez trzy dni maszerowała spora grupa, nie raz wspierana przez śmigłowiec. Jeśli któryś ze stalkerów obraził wojsko czy zaszkodził mu w jakikolwiek inny sposób, w Zonie pojawiał się Specnaz -- prawdziwa zmora dla wszystkich mieszkańców Zony.
\sx Nie mogę dłużej\3k -- wysapał Kamil, po czym znowu przekręcił się na bok, lądując czołem na cuchnącym dachu.
\qd
Nie mógł już dłużej wpatrywać się w samolot -- zbytnio go to dołowało.
\sx Siadaj i powiedz im, że jesteśmy gotowi. -- dodał, próbując powstrzymać się od płaczu. Udało mu się, wynikiem czego było jedynie zaczerwienienie oczu i chwilowa utrata tchu.
\qd
Po rozwiązaniu problemu, jakim są tutejsi wojskowi, Kamil będzie mógł w końcu urealnić swoje wizje na temat ucieczki ze Strefy. Myśl ta motywowała go od dwóch lat do tak zwanego ,,stalkerstwa''.

Lenny leżał na sofie, z zagipsowaną nogą opartą o taboret. Czytał książkę, wsłuchując się w muzykę z gramofonu. Winylowa płyta Armstronga pochodziła z domu pewnego mieszkańca w Prypeci -- odnalazł ją znajomy Lenny'ego, Norbert, członek pierwszego oddziału wojska, który wkroczył do opustoszałego miasta.\\
Po przeczytaniu 233-ej strony powieści płyta skończyła się -- ostatnie dwie sekundy ,,What A Wonderful World'' zostały zakłócone przez mechaniczne zgrzyty, spowodowane rysą na powierzchni krążka.\\
Stalker włożył skórzaną zakładkę w książkę, po czym zamknął ją i odłożył na podłodze. Chwycił leżące obok kule i, wkładając w to sporo wysiłku, podniósł się z sofy. Idąc o kulach, podszedł do gramofonu, potem, utrzymując równowagę na zdrowej nodze, odsunął igłę, która wciąż przytrzymywała obracającą się płytę. Następnie zaś chwycił ją ostrożnie, po czym schował do tekturowego opakowania. Kiedy włożył je w rząd bliźniaczo podobnych pudełek, w pokoju zaczął dzwonić telefon. Lenny sięgnął słuchawkę lewą ręką i przyłożył ją sobie do ucha, z powrotem opierając się o kulach.
\sx Tak? -- zapytał, manewrując dłońmi tak, by nie wypuścić z nich metalowych podpór.
\xx Twój koleżka straszy mi klientów! -- krzyknął Mirek, tutejszy barman. Jego słowom towarzyszył typowy, ,,spelunowy'' gwar -- stuki szklanych butelek, głośne rozmowy i krzyki. Dziś towarzyszyło im pijackie zawodzenie.
\xx Kto? -- zapytał Lenny, po czym zaczął szykować się do wyjścia. Lewą, wolną ręką, otworzył szufladę i niedbale wyciągnął z niej płaszcz.
\xx Ten Michał! -- wrzasnął Mirek jeszcze głośniej, niż poprzednio. -- Upił się i wymachuje wszystkim nożem przed oczyma. Jak nie przyjdziesz tu w ciągu pięciu minut, to sam go ukatrupię i wrzucę do anomalii! Rusz dupę!
\xx Kretyn\3k -- mruknął Lenny, odkładając słuchawkę.
\qd
Schylił się niżej po płaszcz, narzucił go na siebie, po czym skierował się do drzwi wyjściowych. Zdjął klapki i zastąpił je luźnymi butami na rzepy. Docisnął obie, przejeździwszy nimi po framudze drzwi, po czym otworzył je, nacisnąwszy klamkę łokciem.\\
Po placu spacerowało kilkunastu stalkerów, co najmniej dwóch obserwowała okolicę z okien swych kwater, przed barem zaś, jak zwykle, stała grupka rozmawiających ze sobą agentów. Popołudnie tego dnia było słoneczne, bezchmurne i przyjemne. Lenny poczuł się nieco dziwnie, widząc nieruchome korony drzew okalające obóz -- przyzwyczaił się do ich szumu, dzisiejszy dzień był zaś całkowicie bezwietrzny.\\
Ignorując ból w nodze, popędził w kierunku baru tak szybko, jak pozwalał mu na to obecny stan piszczela. Z racji tego, że ziemia była zziębnięta, twarda i sucha, szło mu się łatwo i komfortowo, w przeciwieństwie do chodu po błocistej, śliskiej breji, która tworzyła się po każdym dłuższym deszczu.\\
Po dotarciu do progu pubu, Lenny usłyszał już wrzaski Michała, donoszące się w wnętrza pomieszczenia. Minął ustawionych w rzędzie stalkerów, przecisnął się przez utworzony przez nich krąg, po czym zobaczył zataczającego się w środku koła stalkera z nożem w ręku.\\
Michał miał na sobie zielony, wzmacniany kombinezon bez kaptura, z maską przeciwgazową, która obecnie zwisała luźno u szyi. Wisząca na prawym ramieniu kabura była pusta, podobnie, jak przyszyty do lewego barku uchwyt na nóż -- skórzane zapięcie kołysało się we wszystkie strony, w rytm kroków łowcy.\\
Miał mgliste, zapite i niewyraźne oczy, które obłąkańczo łypały na prawo i lewo, szukając potencjalnej ofiary. Co do stanu upojenia alkoholem ich właściciela nie było wątpliwości -- każdy jego krok był chwiejny, niepewny i niewiele brakowało mu do upadku. Michał z trudem utrzymywał równowagę, zaś każde jej zachwianie okazywał pijackim jękiem, połączonym z byle jakim, wręcz niedbałym machnięciem klingą, która była wolniejsza od ręki zamachującego się zombie.
\sx Dwie minuty! -- krzyknął ktoś do Lenny'ego.
\qd
Spojrzał w lewo i ujrzał Mirka, wesoło machającego ręką, w której trzymał pistolet, lewą ręką zaś pukając w zegarek. Ponaglił go, wybałuszając oczy.
\sx Michał! -- zawołał Lenny w stronę pijanego stalkera.
\xx Dawid? -- spytał, odwróciwszy się. 
\qd
Na chwilę stał w miejscu, mimo to dalej się chwiał i zdawało się, że za chwilę upadnie. Powoli uniósł prawą rękę, po czym spróbował trafić nożem w pochwę. Ostrze ześlizgnęło się po ramieniu, tak więc jego posiadacz spróbował jeszcze raz. Wziąwszy większy zamach, trafił myśliwskim nożem do specjalnego uchwytu. Zdjął palce z rękojeści, po czym zaczął męczyć się z zapięciem. Po czterech próbach udało mi się docisnąć je na tyle, by zabezpieczyć narzędzie, które w końcu pewnie siedziało w uchwycie.
\dd
\sx Stój! -- ostrzegł wartownik na wieży.
\qd
Do obozu zbliżało się dwoje nieuzbrojonych stalkerów. Janusz, obecny wartownik, nakazał asystentowi obserwację nieznajomych stalkerów przez PSG. Ten przyłożył oko do lunety, obserwował przez dwie sekundy, po czym z ulgą oznajmił:
\sx To Radek i Mikołaj.
\qd
Radek usiadł przy bramie obozu, oparł się o nią plecami, po czym luźno opuścił dłonie i zwiesił głowę. Był wykończony. Uniósł prawą rękę na wysokość dziesięciu centymetrów.\\
Natychmiast opadła z wysiłku, po czym załomotała w twardą ziemię.
\sx Jezu\3k wysapał, przecierając lewą ręką spocone, gorące czoło, używając ostatnich zapasów sił. Już dawno się tak nie zmęczył. -- Już dawno się tak nie zmachałem\3k -- wydusił z siebie wycieńczonym głosem.
\xx Ja też\3k -- odparł Mikołaj podobnym tonem, siadając obok Radka.
\qd
W prymitywny sposób schłodził głowę, jeżdżąc potylicą po lodowatej, metalowej bramie. Wpatrując się znudzonymi oczyma w przesłonięty drzewami horyzont, wymacał kieszeń kombinezonu, w której trzymał butelkę wody.\\
Po niezdarnej próbie chwycenia jej za korek, wypadła z kieszeni. Woda wewnątrz plastikowej butelki zalśniła w promieniach słońca. Odbijała się na przemian od krańców butli z charakterystycznym, stłumionym pluskiem.
\sx Podnieś\3k -- szepnął Mikołaj, nerwowo mrużąc oczy.
\xx Nie dam rady\3k -- odpowiedział Radek, przeciągając ostatnie słowo. -- Zastukaj do bramy.
\xx Nie chce mi się. Jak twoje gardło?
\xx A co?
\xx Kto krzyknie, poda kod, żeby nas wnieśli?
\qd
Radek otworzył usta, z których wydobyło się przeraźliwe chrypnięcie. Zapewne był to efekt próby podniesienia głosu.
\sx Na mnie nie licz\3k -- wykaszlał Radek. Mikołaj zmierzył go spojrzeniem, od którego ciarki przeszły mu po plecach. Westchnął ciężko, nabrał powietrza w usta, po czym krzyknął:
\xx Otwórzcie bramy! Siedem, siedem, dwa, jeden, pięć! Leniwa świnia\3k -- dodał po chwili, zwracając się do Radka.
\xx Od zawsze\3k
\qd
Zza blachy rozległ się dźwięk rozsuwanych rygli -- po trzech, metalicznych szczęknięciach, brama z piskiem poczęła obsuwać się w bok. Dwójka agentów pochyliła głowy w przód, by nie walić nimi w bramę. Kiedy w końcu się zatrzymała, Radek i Mikołaj, będąc wciąż odwróceni plecami do obozu, bezwładnie padli na ziemię. Obaj poczuli łoskot źdźbeł trawy na szyi i uszach. Radek, przewrażliwiony na tego typu dotyk, zachichotał cicho, po czym zakaszlał. Tak bardzo zmęczył się tylko dwa razy w życiu -- w 79-tym roku, na jednej z lekcji W-Fu, do której przystąpił bez zjedzenia śniadania oraz w roku 2000, dokładniej dwudziestego czwartego maja, kiedy to uciekał przed goniącą go grupką złodziei.\\
Ktoś przeskoczył nad leżącym bez ruchu Radkiem. Ubrany w cieńkie spodnie khaki i skórzaną kurtkę postać odwróciła się powoli, swą grzywą przesłaniając promienie słońca.
\sx Jezu! -- krzyknął Radek, przytomniejąc momentalnie.
\xx Nie bluźnij\3k -- szepnął Mikołaj, który wciąż nie odrywał oczu od nieba.
\xx Jonathan?
\xx Gdzie?!
\qd
Mikołaj zerwał się na równe nogi, odzyskawszy nagle całą energię i siły witalne. Ujrzał Jonathana podającego rękę Radkowi, który z jego pomocą wstał z ziemi, po czym wpadł mu w ramiona. Przedtem wziął niemal sporego rozbiegu, przez co dwójka zaczęła się chwiać we wszystkie strony, przystępując z nogi na nogę.
\sx Kiedy? -- spytał Radek zachwyconym głosem, stłumionym przez lewe ramię, do którego się przycisnął.
\xx Godzinę temu. -- odpowiedział Jonathan. W jego tonie dało się wyczuć coś niepokojącego.
\xx Co, nie cieszysz się?
\xx Jasne, że tak\3k -- szepnął Jonathan w mało przekonywujący sposób. Zanim Mikołaj zdążył do niego dobiec, zwolnił już uścisk, po czym szybkim krokiem skierował się do centrum obozu. Mikołaj stanął jak wryty, czując spore rozczarowanie.
\xx Wiem jak to wygląda. -- rzekł Masterton. -- Nie stój tak, właź! -- zachęcił Mikołaja, który wciąż stał w miejscu z zawiedzioną miną.
\qd\dd\sx
Zabił Dawida! -- krzyknął Michał niewyraźnie. -- Ten\3k jak mu\3k -- zatoczył się nagle, po odzyskaniu równowagi dodał. -- I? -- nie mogąc wymówić reszty imienia, tupnął wściekle nogą w podłogę.
\qd
Lenny zrobił krok w przód, stanowczo występując z tłumu.\\
Zaraz po nim, z naprzeciwka, z kręgu gapiów wyłonił się Igor. Ubrany w swój ,,krwawy'' fartuch i pochlapane czerwoną posoką spodnie i buty. Na skórzanym, mocno zapiętym pasku nosił siedem długich, błyszczących brzytw o drewnianych rękojeściach. Odczepił jedną prawą ręką, po czym kciukiem wcisnął metalowy przycisk. Ostrze wystrzeliło przed siebie, ze świstem przecinając powietrze. Uniosło się ono wysoko, wraz z trzymającą je ręką na wysokość oczu zdezorientowanego, bełkoczącego Michała. Uniósł prawą stopę, zamierzając się do kroku w bok.
Igor napiął mięśnie, pochylił się nisko, po czym wykonał błyskawiczny krok naprzód. W tym samym czasie zamachnął się ostrzem, tak mocno, że siła zamachu zmusiła go do obrotu przez plecy.\\
Opuścił brzytwę, po czym przejechał jej krańcem po fartuchu. Ku zaskoczeniu, a raczej zdumieniu obecnych w barze, pozostawiła ona za sobą krwawy ślad.\\
Michał poczuł na czole narastające pieczenie. Uniósł niepewnie prawą rękę, którą przejechał po miejscu swądu. Opuścił dłoń i skierował na nią swe oczy.\\
Ręka była cała czerwona od świeżej, jeszcze lśniącej krwi.\\
Obserwujący całe zajście Mark uśmiechnął się pod nosem, widząc padającego bezwładnie na podłogę Michała. Albo alkohol łowcy wydawało się, że dosłownie urwało mu głowę, albo Igor nasączył ostrze jakąś trucizną.\\
W ciągu niecałego tygodnia w Zonie Mark najwięcej nasłuchał się o Igorze, a raczej tym, czym się od lat zajmował.\\
Teraz miał zamiar opowiedzieć wszystko Gavinowi. Wychodząc z baru, Mark był niemal pewien, że to osoba zajmująca się przesłuchaniami więźniów, Igor, jest mordercą owego nieznajomego Dawida.\\
,,Bądź ostrożny, bo skończysz jeszcze gorzej, niż Olaf.'' -- powiedział do siebie w duchu, mijając dwójkę jeszcze nieznanych mu stalkerów, wynoszących z pomieszczenia Michała. Z płytkiej rany na czole popłynęło już znacznie więcej krwi, której jedna spływająca struga dotarła aż do podbródka.\\
,,Już my ci pokażemy, jak przeprowadza się przesłuchania\3k''\\
Ted, Gavin albo Hubert. Któryś z nich należycie zajmie się Igorem.

Henryk nacisnął świecący się zielonawym światłem przycisk elektrycznego czajnika. Zgasł.\\
Unoszącej się gorącej parze towarzyszyło głuche bulgotanie, dobywające się wnętrza grzejnika. Właściciel odczekał kilka sekund, by nie narazić chorobliwie delikatnej skóry na działanie oparów.\\
W końcu chwycił czajnik za pokryty antypoślizgową gumą uchwyt, po czym wrócił do swojego biurka. Nachylił się i zalał przezroczystą szklankę wrzątkiem -- torebka miętowych liści herbaty popłynęła w górę, zabarwiając wodę zielonym kolorem. Po zamieszaniu naparu, czajnik wylądował w drewnianej szafce, obok biurowego krzesła. Grzałkę odłączono od prądu, zaś w jej miejsce, do kontaktu, Henryk wetknął kabel zasilający komputer.\\
Po usadowieniu się na krześle schylił się lekko, by ręką dosięgnąć dużego plastikowego przycisku. Po jego naciśnięciu komputer zaczął cicho buczeć, na obudowie zaś zaczęły lśnić odblaski dwóch lampek -- zielonej i czerwonej.\\
Monitor jak zwykle zareagował z opóźnieniem -- wyświetlił obraz dopiero po pięciu sekundach.\\
Henryk znajdował się na 15-tym piętrze wieżowca-kwatery S. Poziom ten był najobficiej wyposażony w sprzęt elektroniczny -- sterownie kamer, masę komputerów osobistych jak i tych dostępnych dla każdego, na przykład Pecetów, przez które przeglądać można było całą ,,jawną'' bazę danych frakcji, której większość stanowiły mapy Zony i Jej okolic.\\
Henryk był marnym żołnierzem, poszukiwaczem artefaktów i, krótko mówiąc, stalkerem. Wszystkie jego umiejętności, zarówno te przydatne dla reszty jak i te, które były pomocne tylko dla niego, można było określić jednym słowem -- ,,wiedza''.\\
Był wykwalifikowanym fizykiem, chemikiem i biologiem, a przy okazji, badaczem Zony. Pod tym jedynym względem mógł nazwać się prawdziwym mieszkańcem Strefy -- jego znajomość anomalii, artefaktów była wręcz niebywała. Rozpoznawał wszystkie po dotyku, dźwięku, jakim wydają przy zetknięciu z jakąkolwiek powierzchnią, a ich działanie, na przykład zasklepianie ran, miał zapisane w głowie w postaci skomplikowanego, matematycznego wzoru.\\
Po wyświetleniu pulpitu (jako tapetę Henryk ustawił widok Czarnobyla z perspektywy szczytu komina elektrowni). Włączył internetowy komunikator, przeglądarkę i już przymierzał się do uruchomienia programu dokonującego obliczeń, kiedy zadzwonił telefon.\\
Obrócił się na krześle w lewo, podjechał do stojącej przy drewnianych drzwiach szafki i podniósł czarną słuchawkę.
\sx Halo? -- spytał, obserwując pojawiające się na ekranie komputera ikony.
\xx Znalazłem go. -- powiedział Mark.
\qd
Henryk prawie spadł z krzesła. Zmarszczył czoło, skupił się, po czym rzekł:
\sx Gdzie?
\xx Jest w obozie bezpieki. W ogóle nie zmienił wyglądu, od reszty dowiedziałem się, że w ogóle nie wychodzi z podziemi. Wygląda identycznie, mówię ci, jakby nawet nie przejmował się tym, że ktoś go pozna. On nie opuszcza nawet obozu, mieszka w tych katakumbach i\3k
\xx Już to mówiłeś.
\xx Wybacz, ale\3k
\xx Rozumiem, doskonale to rozumiem. Mów dalej.
\xx Wciąż zajmuje się przesłuchaniami i jest jeszcze bardziej nienormalny. Mam kilka zdjęć, udało mi się nawet nagrać jedno przesłuchanie. To prawdziwy wariat, Henry. Jego śmierć ocali życie wielu stalkerom.
\xx Święte słowa. Najpierw on, potem?
\xx Najlepiej cały obóz. Jutro opowiem ci o\3k
\xx O? -- ciągnął Henryk, oczekując odpowiedzi.
\xx Słyszałeś o Jonathanie Mastertonie?
\qd
% 
\ro{31}
% 
Igor Tomasz Gustowski.\\
Pracujący w Zonie syn członka NKWD. Człowiek znający na pamięć setek wzorów, substancji i sposobów mieszania poszczególnych składników w celu uzyskania danej trucizny. Pracujący, mający do czynienia z największymi brudami, jakie kiedykolwiek zawstydzały gatunek ludzki. ,,Kopał'' w brutalnych przesłuchaniach, wymuszeniach, porwaniach i morderstwach niczym kret, co doprowadziły go, niczym kreta, do konieczności wejścia pod ziemię.\\
Od trzech lat, od 1998 roku, Igor przebywa w Zonie, Strefie, w tamtejszej siedzibie ukraińskiego rządu, w podziemnym bunkrze. Ukrywał się przed rodzinami, znajomymi i przyjaciółmi wszystkich, których skrzywdził przez te wszystkie lata.\\
Ukrywał się, przebywał w podziemiach tyle czasu niczym kret -- każdorazowe wyjście na zewnątrz było jak wyjście z łona matki -- promienie słońca były oślepiające, powietrze nienaturalne świeże, zaś te kilkadziesiąt metrów placu były dla Gustowskiego niczym bezkresne morze.\\
W końcu wrażenie bezpieczeństwa i azylu, jakim był obóz bezpieki, legło w gruzach. Igor niestety dożył dnia, w którym poczuje się autentycznie zagrożony -- czuł, że nie pomoże mu nawet cała armia Iluzjonistów.
\swk[5em]
Zabił\3k\\
Michał\3k\\
Dawida!
\qwk
\sx W końcu? -- wyszeptał Igor, wpatrując się w lustro. Obserwując swe czoło, próbował zachować spokój, bowiem niepewność, która go zżerała, podsuwała mu momentami nawet wizje. Wyobrażał sobie, jak na jego czole pojawia się cięcie -- o wiele większe i głębsze, niż te, którym oszpecił dziś twarz Michała. Z nim będzie inaczej.
\xx Kiedy mnie dorwą? Użyją na mnie wszystkich moich metod?
\qd
Od jutra zacznie żyć jak oficer Gestapo. Z cyjankiem przyczepionym do zęba, by mógł go w każdej chwili rozgryźć i uwolnić się od przyszłych cierpień.
\sx Nie do wiary? W końcu dobrali mi się do dupy?
Ale kto?
\qd

Henryk odłożył słuchawkę.\\
W końcu go znaleźli. I w końcu go dorwą.\\
Igor.\\
Stalkerzy, którzy przeżyli spotkanie z nim dosłownie popuszczają w spodnie ze strachu na samo usłyszenie, wspomnienie imienia, a co gorsza osoby, do której ono należy.\\
Psychopata.\\
Takich akurat w Zonie jest bardzo, bardzo wielu, ale tacy jak Igor są o wiele gorsi. Dlaczego?\\
Ludzie pokroju Igora otrzymywali za to pieniądze i działali w imię kraju, dla rządu, pozującego na wybawcę Zony. Niczym Rosja i Afganistan, USA i Zatoka Perska. Samozwańczy zbawiciele, którym naprawdę zależy na czystym zysku, dominacji, władzy.\\
Tacy byli najgorsi, wzbudzali największą pogardę nie tylko w Henryku, który miałby więlką radochę z każdej sekundy cierpienia tego typu zwierząt. Tak, to nie byli ludzie, przynajmniej pod względem moralnym.\\
,,Cel uświęca środki.''
,,Jeśli chodzi o bezpieczeństwo Zony, tortury są dopuszczalne?''
\sx Wyrwę mu żywcem serce? -- wysapał Henryk, przypominając sobie losy jednej z ofiar Igora, szczególnie bliskiej jemu i całej S.
\qd
Michał.\\
Henryk podejrzewał, że ten wkrótce dołączy do jego organizacji. Niekoniecznie na stałe, najważniejszym i być może jedynym powodem będzie okazja do zemsty na Igorze, odpowiedzialnym za śmierć stalkera imieniem Dawid, który, jak doniósł Mark, był bliskim przyjacielem łowcy.\\
\sx Świetna robota, Mark\3k
\qd

Ktoś załomotał do ciężkich, blaszanych drzwi. Patryk, do tej pory obserwujący zza weneckiego lustra reakcja Maksa, wciąż bełkoczącego o Iluzjoniście, odwrócił się przez plecy.
\sx Idę! -- krzyknął, ruszając przed siebie.
\qd
Po drodze, po wyjściu z specjalnego sektora obserwacji, zmienił laczki na luźne półbuty -- zapiął odstającą rzepę lewego, po czym sięgnął ręką do wciąż uginających się od walenia w nie drzwi. Nacisnął potężną klamkę, po czym gwałtownie się cofnął pod naporem wrót.\\
Do punktu medycznego wbiegło dwóch mężczyzn niosących nieprzytomnego stalkera z długą, choć płytką raną ciętą na czole, z której krew płynęła po całej twarzy nieszczęśnika.
\sx A temu co? -- spytał Doktor, kucnąwszy, by przyjrzeć się ranie. Nie mógł jednak skupić swego wzroku na jednym konkretnym miejscu, bowiem ciało kiwało się w rytm kołyszących się przez ich ciężar Świstaka i Andrzeja. -- ku*wa, połóżcie go! -- krzyknął na nich Patryk. Ci, robiąc wielkie oczy, usłuchali -- nieprzytomny Michał wylądował na pryczy, jeszcze niedawno zajętej przez Maksa.
\xx I po co całe to zamieszanie? -- spytał Doktor, ocierając krew z czoła rannego mokrą ścierką.
\xx Masterton wyzdrowiał.
\xx No co ty! -- zdziwił się lekarz, odwróciwszy się nagle. -- Kiedy?
\xx Jakieś pół godziny temu?
\xx A co?
\xx Co ,,co?''
\xx Wyzdrowiał. W jakim sensie wyzdrowiał?
\xx Po prostu. Harlan przy nim siedział, a ten nagle wrzasnął coś i wstał.
\xx Czyli on ciągle?
\xx Nie rozśmieszaj mnie! -- zadrwił Świstak. -- Po tym co przeszedł, już nigdy nie będzie normalny.
\xx Dobrze chociaż, że może chodzić? -- pomyślał na głos Patryk, po czym znów zabrał się do ocierania i dezynfekowania rany łowcy. -- Co z nim? -- spytał, wyjmując z kieszeni buteleczkę wody utlenionej.
\xx Z kim? -- rzekł tym razem Andrzej, siadając na blaszanym stołku obok, pod szafką z medykamentami. -- Stalkerem? Jak on?
\xx Michałem. -- podpowiedział Świstak, zajmując krzesło obok kolegi.
\xx Ta? No więc? Nasza kochana paczka z Prypeci chce go przesłuchać.
\xx Z Igorem?
\xx Przeciwnie. Pogadają z nim, a potem go puszczą; potrzebny im jest chociaż do upolowania kolejnych stworów, do badań.
\xx Ma farta\3k -- mruknął Doktor. -- Podobnie jak cały ten gość od Iluzjonisty -- przyślą tu pięć ,,wielkich umysłów'' by go zbadać, a potem, uważaj, zabierają go poza Zonę! Zwykły stalker po trafieniu do obozu rządowego opuszcza Strefę? Wiesz, z czego czasem niezmiernie się cieszę?
\xx Z czego?
\xx Że nie działamy jak militarni\3k
\qd
% 
% 
\ro{32}
% 
\podro{Rok 1978}
% 
\sx Skoczę po coś do picia. -- oznajmił Kamil, wstając z fotela. -- Co wam?? -- zaczął, oczekując szybkiej odpowiedzi.
\xx Jakiejś herbaty? -- rzucił Izaak, przewróciwszy się na brzuch, wtapiając tym samym swą pulchną twarz głęboko w poduszkę.
\qd
Zawsze, podczas każdej wizyty u domu Kamila, zajmował on to samo miejsce -- Dużą, miękką kanapę pod lewą ścianą salonu. Co pięć minut przewracał się na plecy i na odwrót, by, jak sam ujął, ,,żadna pozycja mu się nie znudziła''.\\
Leon udowodnił wtedy, że w każdych okolicznościach zdolny jest rzucić świńskim kawałem.
\sx Mi wódki? -- rzekł stanowczo, stopując na chwilę stukanie palcami o blat stołu. Słysząc reakcję Kamila -- śmiech -- dodał. -- Nie, no, poważnie!
\qd
Gospodarz zrobił wielkie oczy.
\sx Nienormalny jesteś? -- spytał oskarżycielsko.
\xx Nie. To tylko mały łyk?
\xx Morda w kubeł! -- wrzasnął niespodziewanie Kamil, tak głośno, że Mikołaj prawie zleciał z taboretu, Izaak zaś momentalnie zerwał się z kanapy, po raz pierwszy dzisiaj, siadając na niej jak normalny człowiek.
\qd
Mina Leona mówiła, że nie rozumie on powodu wściekłości kolegi. Widząc to, ten zajął miejsce naprzeciwko, po drugiej stronie stołu. Rozsiadł się wygodnie, odchrząknął, po czym zapytał:
\sx Znasz Radka? Nie tego, ,,od nas'', tego drugiego.
\xx Taa\3k
\xx A Tomka?
\xx Też.
\xx Kulera?
\xx Jezu, po kiego wymieniasz mi swoich byłych\3k -- tu Leon zawiesił głos, by poszukać odpowiedniego słowa. Wypowiedział je z niekrytą dumą, drwiącym tonem. -- Kolegów??
\xx Wiesz, czemu z nimi ,,zerwałem''?
\xx Bo robiłeś u nich za frajera?
\qd
Ta szczera do bólu odpowiedź zauważalnie poruszyła Kamila. Mimowolnie zacisnął on usta, tłumiąc kolejny napad wściekłości. Zamknął powoli oczy i wewnętrznie się uspokoił, głośno biorąc oddech i jeszcze głośniej wypuszczając całe zabrane powietrze nosem. Otworzył oczy, po czym odpowiedział, wbijając je bezlitośnie w Leona.
\sx Nie tylko. To banda zwykłych skurwysynów, pijaków i ćpunów, a ty, ,,kolego''? -- Kamil idealnie sparodiował ton jednego z poprzednich słów Leona. -- Ty, takim nędznym\3k
\xx O co ci?
\xx Żałosnym, smętnym, ku*wa mać, żebraniem o wódkę, się do nich przybliżasz, rozumiesz?! -- wycedził Kamil, podnosząc ton głosu co każde słowo. -- Dlatego się od nich ,,odłączyłem'' -- ciągle tylko biegałem im po wódę, tanie wina i szlugi, bo sami wydalii pół pensji rodziców ,,na krechę'' albo wyglądali na nieletnich. Więc, do ch*ja, proszę, przestań się zachowywać jak oni! Przyszedłeś tutaj, by się napić, czy żeby spędzić ze mną i resztą czas?! -- wykrzyczał, wychylając się nad stołem. -- Pytam jeszcze raz -- Co dla ciebie?
\xx Wody\3k -- wyszeptał, przewracając pretensjonalnie oczyma.
\xx Nie rób tak! -- rozkazał Kamil, wstając z stolika, stawiając pierwszy krok w stronę kuchni.
\xx Czego?!
\xx Te oczy. Co, czujesz się zawiedziony, że wódki nie dostałeś?
\xx Przestań.
\xx Nie, nie ,,przestań''! Jeszcze jeden taki numer, a wylatujesz za drzwi, rozumiesz?!
\qd

Mikołaj zabrał głos zaraz po opuszczeniu domu przez Izaaka i Leona.
\sx Nie za ostro? -- zapytał, przekręcając klucze w zamku.
\xx W życiu! -- zawołał Kamil, skręcając w głąb kuchni. -- Teraz będzie siedział cicho\3k -- dodał już ciszej, widząc wchodzącego za nim do kuchni Mikołaja. Odsunął jeden z ustawionych pod stołkiem taboretów, po czym usadowił się na nim, plecami luźno opierając się o ścianę. Westchnął głośno.
\xx Co? -- zapytał Kamil, słysząc to. Posłodził swoją herbatę, po czym odwrócił się i oparł o aneks kuchenny.
\xx Chcesz do niego iść? -- spytał Mikołaj, majstrując przy radiu.
\xx Ba.
\xx A jak będzie pijany?
\xx Na pewno nie będę dzwonił na policję. W sumie, to sam nie wiem co zrobię. Nastraszę go, zgnoję? Szczerze mówiąc, nie wiem co pocznę, kiedy zobaczę go pijanego. Oby nie?
\dd
\xx Który to pokój? -- dopytywał przez większość drogi Mikołaj, rozglądając się dookoła po centrum miasta. Po pokonaniu ledwie stu metrów zdążył spytać o to trzy razy.
\xx Siódmy i zamknij się w końcu? -- mruknął Kamil z niechęcią.
\xx Szczęśliwe, co?
\xx Nie za bardzo?
\qd
Mieszkanie numer siedem.\\
To tam, z dotychczasowego miejsca pobytu Jonathana, szóstego pokoju bloku D, przeprowadził się Ulisses Masterton po rozwodzie. Od ponad roku mieszkał w hotelu Polisia, niedaleko centrum Prypeci, utrzymując się za zaoszczędzone pieniądze, o których pochodzeniu opowiadał innym bardzo niechętnie -- mówił ,,Ważne, że są, i nie sypiam pod mostem, nie?''. Syn jak ojciec -- Ulisses mimo wielu, momentami bardzo zawziętych namów kolegów nie zdradził, jak zarobił sumę, z której opłacał siódmy apartament Polisii. Zawsze starał się pomijać tą kwestię. Z zamierzonym skutkiem.\\
Kamil z Mikołajem skręcili w lewo, mijając pusty kwadrat ziemi. Dalej, po lewej, przy wąskim chodniku przecinającym trawnik, zaczynały się schody do domu kultury ,,Elektryk''. Tego dnia biel farby, którym pokryto budynek, świeciła momentami mocniej od samego słońca, które zalewało je swoimi promieniami. Jego blask nikł jedynie w chwilach zasnucia słońca przez chmury, których było jeszcze niewiele, mimo iż pod wieczór zapowiadano sporą burzę.\\
Na razie jednak, o czternastej dwadzieścia, czyli daleko od zapowiadanych opadów, pogoda była słoneczna i ciepła, momentami upalna oraz całkowicie bezwietrzna.\\
Kamil, idąc przed siebie szybkim krokiem, zapatrzył się w duże, prostokątne okna ,,Elektryka''. Miał ochotę porzucić obecny cel -- spotkanie z ojcem Jonathana -- by udać się do środka budynku. Chciał usiąść w miejscowej bibliotece i przejrzeć losowo wybraną książkę, zerkając co chwila przez jedno z wielkich okien na południe Prypeci. Marzyło mu też się kolejne przedstawienie -- nie obchodziłby go w tej chwili temat ani charakter sztuki -- po prostu czuł potrzebę zanurzenia się w miękkim fotelu i poczucia atmosfery spektaklu, skupienia, spokoju.
Wszystko, byle nie ten smutny obowiązek.\\
Szesnastolatek sprawdzający zachowanie ojca swego najlepszego przyjaciela? Zapewne, dla wielu brzmiało to śmiesznie. Tak naprawdę, to Kamil sam nie wiedział, czy postępował słusznie -- ciągle zmagał się z wątpliwościami na temat tego, jak Ulisses oceni jego postawę.\\
Z jednej strony, gdyby okazało się, że naprawdę dojrzał, Kamil mógłby w jakimś stopniu pomóc Mastertonom, o ile dobrze ocenił sytuację i okaże się, że Ulisses naprawdę weźmie się w garść, widząc zatroskanego przyjaciela swego syna.\\
Z drugiej strony Kamil mógł się przeliczyć i z jego szlachetnych zamiarów pozostałby jedynie wstyd -- usłyszał by gadkę w rodzaju ,,Ty, gówniarzu, mówisz mi, jak postępować z synem?!''
Drugi argument był dla Kamila o wiele mocniejszy. Dopiero przed chwilą uświadomił sobie, po co tak naprawdę chce odwiedzić Ulissesa.\\
Dla czystej pewności i spokoju.

\dd\sx Jezu\3k -- jęknął Mikołaj, zerkając na budynek. 
\qd
Osłonił oczy przed słońcem. Pomyślał, że musi sobie w końcu kupić okulary.\\
Szeroki, biały budynek miał siedem pięter i górował nad licznymi niewysokimi rosnącymi dookoła drzewkami. Każdemu z nich towarzyszył trawnik czerwonych kwiatów i krzewów, zasianych wzdłuż chodnika.\\
Na dachu hotelu znajdował się największy pokój z dwoma dużymi oknami -- choć można było z niego zapewne wyjść na taras z widokiem, Kamil nie wiedział jeszcze, czy pokój ten był do wynajęcia.
\sx Co? -- spytał Mikołaja, wpatrując się w dwie duże anteny telewizyjne sterczące z dachu hotelu. Zapewniały dość dobry odbiór większości kanałów telewizyjnych w kraju -- kanałów informacyjnych, publicystycznych i kilku stacji komercyjnych.
\xx Który pokój zajmuje Ulisses?
\xx Mówiłem ci, że?
\xx Które piętro?
\xx Trzecie?
\xx Naprawdę musi mieć dużo kasy.
\xx Na pewno.
\qd
Mikołaj z Kamilem skręcili w lewo, zbliżając się do wejścia do ,,Polisii''. Wspięli się po kilku małych, krótkich kamiennych stopniach, po czym weszli w chłodzący cień rzucany przez łuk łączący hotel z Domek Kultury kilkanaście metrów dalej. Kamil odsapnął z ulgą, żegnając się przynajmniej na kilka minut z prażącym słońcem. Zaczekał wraz ze swym kolegą aż grupka starszych osób wyjdzie z poczekalni -- pewien starzec po paru sekundach zmagania się z drzwiami w końcu uchylił je dość szeroko, by można było się przez nie przecisnąć.\\
Kamil przytrzymał je, pozwalając bezproblemowo wyjść reszcie grupy z holu recepcji, za co serdecznie mu podziękowano. Kamil w odpowiedzi uśmiechnął się leniwie, po czym, po wyminięciu ostatniej staruszki, przekroczył próg hotelu. Mikołaj wszedł do niego tuż za nim.

Leon otarł pot z czoła i sapnął z wysiłku.\\
Od dziesięciu minut siedział na zimnych, wąskich schodkach przy centralnym placu Prypeci. Co kilka minut mijały go dziesiątki ludzi, zerkających na niego krzywym spojrzeniem. Niemal słyszał ich myśli -- ?Zaraz dostanie wilka??, ?I gdzie on się gapi?? i tym podobne. Jemu było wszystko jedno -- był zdenerwowany, a ta nietypowa metoda wyjątkowo go uspokajała. Wpatrywanie się w niebo, tłumy ludzi na ulicach i innych wyglądających z okien i balkonów -- sam kontakt z ludźmi, nawet wzrokowy, był dla Leona przyjemnością.\\
Żałował tylko, że nie mógł sobie zapalić w tym miejscu. To też niewątpliwie by go uspokoiło.\\
Wbrew pozorom, dziesięciominutowe wylegiwanie się na betonie nie przyprawiało go o ból pleców -- przyzwyczaił się do tego i umiał się ułożyć na nim w komfortowy sposób. Mimo to, nie pogardził by miękkim łóżkiem. Kiedy w końcu się uspokoi, zapewne pójdzie się zdrzemnąć. Co doprowadziło go do tego stanu?\\
Kamil.\\
Nie był zły na Kamila, na to, co powiedział. Właśnie przez to, co powiedział, Leon był zły na samego siebie.\\
Od pewnego czasu czuł się dojrzały emocjonalnie -- uważał, że wiedział jak zachować się w każdej sytuacji i jak w nich postępować. Jednak, kiedy dotarło do niego, czego ,,czepiał'' się Kamil w jego prośbie o łyk alkoholu, dotarło do niego, że między okresem dojrzewania a prawdziwą dorosłością dzieli go prawdziwa przepaść.\\
Z drugiej strony, większość alkoholików stanowią właśnie dorośli.\\
Bał się.\\
Leon nie mógł stuprocentowo zapewnić się w duchu, że jego dzisiejsze ,,wybryki'' z ,,procentami'' wkrótce się skończą, jak to zachowania dojrzewających ludzi. Nie miał pewności, że to nie rozwinie się dalej -- autentycznie obawiał się, że skończy jako śmierdzący potem menel, pijący rozcieńczone wodą z kranu kosmetyki. Czuł lęk i niepewność, które jednak stopniowo go opuszczały, w miarę analizowania sprawy.\\
Myślał o specjalistach, którzy mogliby mu pomóc -- o kimś, kto nie zwróci się do niego ,,synu'', ,,skarbie'', ,,kochanie''. Nie był nieczuły i niechętny co do rodziny, wręcz przeciwnie, jednak tą jedną rzecz wyjątkowo chciałby ,,załatwić'' z kimś całkiem obcym -- kimś, kto dopiero go poznał i da mu szansę poznania go od nowa, nie wiedząc o poprzednich kilkunastu latach jego życia.\\
Musiał wziąć się w garść.
% 
% 
\podro{Rok 2001}
% 
% 
\sx Co masz? -- zawołał Graham z końca pokoju w stronę Johna, plądrującego sąsiedni pokój.
\qd
Podszedł do niskiej, drewnianej szafki, zapewne należącej do młodego dziecka. Była oblepiona kilkunastoma kolorowymi naklejkami i symbolami -- bohaterami komiksów, filmów, książek. Kilku młodocianych idoli, postacie z najbardziej znanych i kultowych bajek -- Kubusia Puchatka, czerwonego kapturka i wielu innych, których Graham nie znał.\\
Kucnął nisko, podginając palce stóp, po czym chwycił obiema rękoma za metalowy uchwyt szafki, po czym pociągnął go z całej siły, brutalnie patrosząc zawartość. Pusta szuflada wylądowała na końcu pokoju -- Graham rzucił ją tam niedbale, rozrzucając dookoła dziecięce ubranka, ręczniki i kilka zabawek.\\
Drewniana szuflada z hukiem rozpadła się w pół, rozrzucając dookoła drzazgi.\\
Graham przekopał się przez stosy ubrań, szukając czegoś wartego uwagi. W kieszeniach małych, pozornie nowych dżinsów znalazł monetę. Przyjrzał się jej możliwie dokładnie, na tyle, na ile pozwalała mu przesłonka kombinezonu. Uznał ją za cenną, po czym schował w specjalne opakowanie.\\
Stalkerowi przypomniały się zasłyszane gdzieś w okolicy Baru słowa.\\
,,Ludzie nie powinni trzymać kosztowności w takich oczywistych miejscach jak sejfy, skrytki, depozyty. Trzy razy okradli mi dom -- za każdym razem wszystkie karty kredytowe, dokumenty i tym podobne miałem w kieszeni spodni leżących na wielkim stosie innych. Jedyne, co wynieśli, to kilka wartościowych książek.''\\
Jak widać, mieszkańcu tego domu albo wierzyli w tą zasadę, albo któreś z ich dzieci dostało monetę w prezencie i schowało ją do kieszeni, po czym zapominając o niej, zapakowało spodnie do szafki.\\
Graham, łapiąc się za kolana, wyprostował się powoli, odwrócił się na pięcie, po czym wszedł do salonu. John siedział na kanapie i męczył się z jakimś kwadratowym przedmiotem -- chwytał go na wszystkie możliwe sposoby i oglądał dokładnie z każdej strony. Kiedy Graham zbliżył się do stolika z mlecznego szkła, zauważył, że Finn usiłuje otworzyć szkatułkę.\\
Zrobiono ją z ciemnego, niemal czarnego drewna, które dodatkowo ozdobiono ręcznie wykonanymi wzorami, które podkreślono złotem. W środku pudełka brzęczały jakieś małe przedmioty, prawdopodobnie biżuteria.
\sx Gdzie to było? -- spytał Graham, siadając po turecku na dywanie.
\xx Tam. -- odpowiedział John, wskazując wysoką szafę w rogu salonu.
\qd
Była opróżniona z jej zawartości, która ścieliła teraz niechlubnie podłogę, podobnie jak każda szafka, każda komoda i szuflada w apartamencie.\\
Finn przewrócił szklany stolik nogą, by w jego miejsce, na gołej podłodze, położyć szkatułkę, którą przed chwilą znalazł. Odczepił bagnet od leżącego na sofie karabinu, po czym chwycił go oburącz i klęknął na kolanach.\\
Wziął daleki zamach w tył i, niczym osoba składająca ofiarę, ugodził małą skrzyneczkę bagnetem prosto w zamek, wkładając w to całą swoją siłę.\\
Ostrze bagnetu zazgrzytało w momencie spotkania się z zamkiem, który puścił już po pierwszym uderzeniu -- blaszka wygięła się i odpadła, po ciągając za sobą kilka śrubek, które rozsypały się po całym salonie. Jedna z nich trafiła w stojące na komodzie lusterko, pozostawiając na nim wielką, szkaradną rysę.\\
John otworzył szkatułkę i wygrzebał z niej duży naszyjnik.\\
Zawiesił go na palcu wskazującym, uniósł przed siebie i dokładnie obejrzał. Był to naszyjnik ze srebra z wielką, pięknie oszlifowaną perłą w środku. John gwizdnął z zadowoleniem.
\sx Co jeszcze? -- spytał Graham, wyglądając przez okno na rynek dookoła kościoła.
\xx Pocztówki.
\xx Pocztówki?
\xx Ta, pocztówki i zdjęcia. Te elektrowni już mamy, zróbmy jeszcze kilka kościołowi i rynkowi.
\xx Niedaleko jest wyższa uczelnia?
\xx Chcesz sfotografować roczniki?
\qd
Graham uśmiechnął się pod nosem.
\sx Cały ten nikły upływ czasu, wrażenie, że jeszcze parę dni temu szkoła tętniła życiem i pękała w szwach od studentów -- niektórych to kręci. Wiesz, ile dawali za zdjęcia szkół w Prypeci?
\xx Nie musisz mi przypominać. Możemy iść do uczelni, ale najpierw kościół.
\xx Jak chcesz\3k -- Graham wyszedł z salonu do przedpokoju i chwycił klamkę od drzwi wyjściowych. 
\qd
John ruszył zaraz za nim, najpierw jednak zabezpieczył odpowiednio znaleziony naszyjnik. Wyszedł za swym towarzyszem na korytarz kamienicy -- starej, powojennej, której bardzo przydałby się chociaż powierzchowny remont, nie wspominając o zabiciu nieprzyjemnego zapachu i pozbyciu się wielu graffiti.\\
Schodząc po brudnych stopniach kamienicy, Graham zatrzymał się na chwilę i przyjrzał się metalowej, zniszczonej skrzynce pocztowej. Skrytki trzech z ośmiu mieszkań kamienicy zwisały bezwładnie ku ziemi -- dwie z nich były powyginane, jakby od uderzeń ciężkim przedmiotem, z ostatniej zaś została jedynie połowa -- druga została prawdopodobnie zerwana.\\
W skrytce siódmej znajdowała się biała koperta.
\sx Listy? -- mruknął Finn z zadowoleniem. -- Są tam jeszcze jakieś? -- zaciekawił się i zaniepokoił lekko, widząc Grahama dobierającego się do siódmej skrytki.
\xx W pierwszym. -- poinformował Graham po krótkiej obserwacji. Uderzył pięścią w włożony między zawiasy bagnet -- stary i zaniedbany zamek puścił, ukazując wnętrze. Koperta wylądowała w dłoniach stalkera -- przeczytał jedynie dane adresata, gdyż list był anonimowy.
\xx Michał Piotrowski, ulica Łesii Ukrainki, 3A, mieszkanie siódme, Mryńsk -- 81300. Lepiej zrób temu zdjęcie, bo nie uwierzą.
\qd
Finn chwycił w prawą rękę zwisający z szyi aparat i sfotografował skrzynkę pocztową.
\sx Ile czasu zostało nam na kamerze?
\xx Z dwie godziny?
\xx Ile nam wystarczy na jakąś anomalię?
\xx Zależy, jaką spotkamy.
\xx ,,Na oko''?
\xx Pięć minut. Myślisz, że jakaś już powstanie?
\xx Zapewne. -- rzekł pewnym tonem Graham. -- Ale najpierw idziemy do kościoła, znajdziemy w wieży jakieś okno i sfilmujemy panoramę miasta. Mgła opadła?
\xx Zobaczymy. -- rzucił Finn, kierując się do wyjścia z kamienicy. 
\qd
Minął jeszcze kilka grafików i tablicę ogłoszeń skierowaną dla mieszkańców okolicy, po czym wyszedł na podwórze przez framugę wyłamanych drzwi, które leżały obok. Omal się o je nie potknął.\\
Znajdował się z Grahamem z powrotem na rynku miasta -- jakieś dwieście metrów od kościoła. Wrażenie opuszczenia miasta było przeogromne, choć nie w tej dzielnicy Mryńska -- najbardziej sugestywne wrażenie wywierała jego południowa część, a dokładniej mówiąc, wielka stołówka, podczas której w czasie zbliżonym do wybuchu odbywało się wesele.\\
Ludzie zniknęli, pozostała jednak muzyka, wciąż grająca w pustej sali. Była głośna i roznosiła się po całym otaczającym stołówkę osiedlu.\\
Graham rozciągnął się, ziewnął, po czym, chrząkając, wolnym krokiem ruszył w stronę kościoła.
\sx Mam nadzieję, że znajdziemy ,,Czad''. -- mruknął, rozglądając się dookoła.
\qd
Opuszczone, ponad pięćdziesięciotysięczne miasto, nie przestawało go zadziwiać ogromem swojej pustki.
\dd\sx
Co się z tobą w ogóle działo? -- zaciekawił się Leon, nachylając się do przodu, niemal kładąc się na stole.
\xx Wiesz, co mi było? -- odpowiedział obojętnym tonem Masterton. Wpatrywał się w kuchenkę gazową i tańczący płomień grzejący czajnik z wodą. Niedługo powinien zapiszczeć.
\xx Ale jak się czułeś? Jeśli w ogóle coś czułeś?
\xx Jak podczas snu. W ogóle nie czuję tego upływu czasu -- jakbym położył się wczoraj w nocy, a dziś rano obudził.
\xx Eee\3k -- jęknął Leon, siadając z powrotem w normalnej pozycji na krześle.
\xx Co?
\xx Mówisz o tym\3k ze zbyt dużym spokojem. Co widziałeś?
\xx Jak to ,,co widziałem''? Po prostu miałem atak -- zachowujecie się, jakbym\3k lunatykując prawie kogoś zabił!
\xx Blisko.
\qd
Jonathan zrobił wielkie oczy.
\sx Podziękujesz mi później. -- powiedział Mark, pokazując zabandażowaną dłoń.
\xx Coś jest z tobą nie tak? -- oznajmił Mikołaj. -- Na pewno nic ci nie jest? Żadnych zmian?
\xx Mam problemy z lewą ręką.
\xx ,,Problemy''?
\xx Niedowład. -- Jonathan spróbował położyć na stole lewą rękę, którą do tej pory opierał na kolanie.
\qd
Z widocznym wysiłkiem uniósł ją na kilka centymetrów, skierowanie jej zaś nad stół sprawiło mu wyraźny ból. Nagle, zanim zdołał ją wyciągnąć jeszcze kilka centymetrów przed siebie, opadła ona w widocznie niekontrolowany sposób.
Masterton zasyczał głośno z bólu, w chwili, w której dłoń uderzyła o kant. Miał przerażony wyraz twarzy, który pogłębiał się z każdą sekundą poświęconą na wygodne ułożenie ręki, która bezwładnie zwisała pomiędzy kolanami.
\sx Jezu? -- jęknął błagalnie.
\qd
Zaciskając zęby, jakby spodziewając się kolejnej fali bólu, chwycił bezwładny nadgarstek prawą ręką, który ułożył z powrotem na kolanie, po czym odetchnął z wyraźną ulgą.\\
Chcąc zmienić temat, Jonathan powiedział:
\sx Musimy wrócić do Johnsona.
\xx Po cholerę? -- zapytał zdziwiony Leon.
\xx Muszę coś\3k sprawdzić.
\xx Jego ciało?
\xx Być może?
\xx Nie masz na co liczyć -- wtrącił się do tej pory milczący Lenny. -- Psy go zjadły.
\xx Na zdrowie. Ale i tak muszę coś tam sprawdzić. Za godzinę pod flagą.
\dd
\xx Co? -- spytał zdziwiony John, słysząc gwizdnięcie towarzyszącego mu stalkera.
\xx Ta szkoła. -- odpowiedział Graham, rozglądając się dookoła. -- Coś w niej jest.
\xx Graham?
\xx Tak?
\xx Teraz ja go poniosę.
\qd
Graham zwiesił ramiona i zrobił zażenowaną minę.
\sx To nie przez niego. -- począł się tłumaczyć. -- Taki już jestem i tyle. -- Graham, trzymając plecak za pasek lewą ręką, mruknął pytająco, mierząc Finna spojrzeniem. -- Więc??
\xx A noś go sobie\3k -- mruknął Finn. -- W sumie to i tak jest odizolowany, więc?
\xx Właśnie. Więc przestań się czepiać, to nie Ostatni Pierścień, żebyśmy się o niego zatłukli, czy tak?
\xx Taak.
\xx Drzwi\3k
\xx Zablokowane.
\xx To znaczy?
\xx Zamknięte na klucz. Wyjmij łom.
\qd
Graham zdjął z pleców wielką torbę, położył ją na chodniku, po czym usiadł przy niej niemal po turecku -- mimo tego, że kombinezon był lekki, ograniczał jego ruchy.\\
Graham i John Finn znajdowali się w północnej części Mryńska, kilometr od krańca jego centrum i otoczonego rynkiem kościoła. Po przejściu przez duży, miejski park, pokonali niewielki kilometr wzdłuż głównej ulicy miasta, po czym, skręcając na zachód, dotarli do uniwersytetu.
\sx Kurwa? -- przeklął Graham, grzebiąc w plecaku.
\xx Tylko nie mów, że\3k
\xx Nie\3k mam łom. -- poinformował stalker, na ironię wyciągając narzędzie obiema rękoma. Wysuwając je całkiem i kładąc kawał żelaza na ziemi, dodał. -- Ja miałem zbierać informacje, ty zdjęcia, tak?
\xx Do rzeczy!
\xx Trzy godziny przed wybuchem w szkole zastrajkowali nauczyciele.
\qd
John zrobił zdziwioną minę i pierwszy raz podczas tej rozmowy spojrzał na Grahama.
\sx Sprzeciwiali się dyrektorowi -- złe traktowanie, bezsensowne pomysły, jakieś próby zmiany funkcjonowania szkoły. -- kontynuował stalker. -- Ustawili się przed szkołą i zażądali dymisji -- a, że nie posłuchał, to wpadli do szkoły i ją zdemolowali, przy okazji tłucząc kilka osób -- uczniów, dyrektora, po drodze staranowali nawet woźnego.
\qd
John gwizdnął z podziwem.
\sx Nie ma co, potrafią walczyć o swoje? -- rzucił ironicznie. -- I trzeciego dnia, w dniu wybuchu?
\xx Zabarykadowali się, a kiedy w elektrowni walnęło, zostali w środku.
\xx I?
\xx Gdy wojskowi wparowali do środka, nikogo nie znaleźli. Pusto.
\xx A propos wojska, jakim cudem nie znaleźli nas? Gdzie oni w ogóle są?
\xx Zasługa Sussaro. -- Graham, z łomem w lewym ręku, podniósł się z chodnika, po czym stanął na równe nogi, zarzucając plecak z powrotem na plecy. -- Co? -- zaniepokoił się, widząc zaniepokojony wyraz twarzy Finna, zniekształcony przez przesłonkę kombinezonu.
\xx Mieliśmy tylko go naładować i wrócić, tak?
\xx Ta, ale on nam nie zapłaci za to tylu pieniędzy. A stalkerzy za zdjęcia stąd, owszem. Chwila\3k wiem o co ci chodzi.
\xx No, pochwal się.
\xx Czy skarci nas za to, że nadużywamy jego ,,wkładu''?
\xx O tym właśnie mówię. -- zmartwił się John. -- Jeśli się dowie?
\xx Nie obrazi się, uwierz mi. Jemu, o dziwo, pieniądze też są potrzebne. Wchodzimy? -- jęknął, opierając łom na ramieniu. -- Robimy kilka zdjęć. -- dodał, robiąc pierwszy krok w stronę drzwi. -- I już nas tu nie ma.
\xx No dobra?
\qd

Siedziałem na starej, szpitalnej kozetce w podziemnej, medycznej części podziemi obozu rządu. Znajdowałem się w odizolowanym pomieszczeniu, z ustawionymi na lewo i prawo ode mnie dwoma łóżkami, którym ,,towarzyszyła'' blaszana półka. W kącie długiego na dziesięć, a szerokiego na około pięć metrów pomieszczenia stało pięć wysokich słupków na kółkach, służących do wywieszenia na niej kroplówki oraz oparta na krześle zmięta, zielona zasłonka.\\
Śnieżnobiałe kafelki w połączeniu z silnym światłem jarzeniówki biły mnie w oczy bardziej, niż poranne słońce -- co chwila mrużyłem przewrażliwione na światło oczy, ocierając je z łez. Panujące światło było dla mnie wręcz nienaturalne, wszechobecne, tak silne, że odbiło się na szklanych szybach, naprzeciwko mnie, uniemożliwiając mi zobaczenie, czegokolwiek na korytarzu.\\
Słysząc metaliczny zgrzyt, obejrzałem się w stronę źródła dźwięku -- wielkich, blaszanych drzwi, które ktoś najwyraźniej zaczął otwierać.\\
Kiedy otworzyły się na oścież, przez ich próg przeszedł wysoki mężczyzna w stroju, który określiłem jako ,,nie pasujący do sytuacji''. Jednak nie garnitur zwrócił moją uwagę, a białe, nie, trupio blade i naznaczone szkarłatną szramą lewe oko. Zdawało się funkcjonować równie dobrze jak zdrowe, prawe -- podążało ono w rytm drugiego, nie zezowało i zdawało się wciąż pracować w pełni sprawnie. Mimo to wyglądało okropnie, na dodatek było ono podkreślone przez bliznę idącą od górnej części policzka aż po brew -- był to dokładny przebieg rany, powstałej najprawdopodobniej w wyniku cięcia.\\
Gdybym był w lepszym nastroju, gwizdnąłbym z niejakim podziwem, widząc ranę stalkera, jednak w obecnych okolicznościach jedynie mnie ona zniesmaczyła.\\
Za agentem w garniturze do pokoju wkroczyło jeszcze sześć osób -- trzech postawnych mężczyzn około czterdziestu lat, jeden wyraźnie niższy, grubszy i starszy -- oceniałem jego wiek na ponad pięćdziesiąt, któremu towarzyszył dobrze znany mi Mark.\\
Spojrzał na mnie zdziwiony -- próbując nie dać po sobie poznać, że niejednokrotnie go spotkałem, zwiesiłem głowę i zacząłem wpatrywać się w podłogę.\\
Jeśli się wygadał, to obaj najpewniej skończymy w anomalii.\\
A Igor uniknie kary.
\sx Gdzie Lenny? -- spytał stalker z rozciętym okiem.
\xx Zmienia gips. -- odpowiedział mu zwalisty, siwiejący agent, kierując się w stronę najbliższego krzesła. -- Zgaś to? -- rzucił, łapiąc krzesło za oparcie, po czym przeciągnął je pod szybę. -- No i? -- mruknął, usadowiwszy się z wyraźną ulgą na oparciu. -- Co z nim robimy? -- rzekł, kierując na mnie spojrzenie.
\qd
Poczułem się dość nieswojo.
\sx Do anomalii? -- mruknął jeden z opierających się o ścianę stalkerów, sięgając ręką pod pazuchę.
\qd
Sparaliżowany strachem, zdołałem tylko wlepić w niego gały, spodziewając się widoku pistoletu, po którym nastąpiła by całkowita ciemność.
\sx Czekaj! -- zawołał stojący obok zamierzającego mnie zabić agenta mężczyzna. Wpatrując się we mnie, z niemałym rozbawieniem z powodu przerażonego wyrazu mojej twarzy, po omacku przytrzymał dłoń sięgającą po broń. -- Pogadajmy z nim. -- zaproponował. Podszedł do mnie i wyciągnął przed siebie rękę, oczekując powitalnego uścisku. -- Leon. -- przedstawił się. Widząc, że wciąż jestem przestraszony, zapewnił. -- Nic ci nie zrobimy\3k
\xx Michał? -- wymamrotałem, trąc piekącą bliznę na czole. Zabije sku*wiela\3k -- Na pewno nic mi nie zrobicie?
\xx Na pewno? -- zapewnił mnie Leon.
\qd
Mimo niewątpliwie dużej szczerości włożone w te słowa nie przekonały mnie one -- brzmiały jak wymówka nastolatka, który kłamie, że nigdy w życiu nie palił papierosów. Sam w ten sposób kłamałem, było więc to mi aż za dobrze znane.
Dopóki nie znajdę się u siebie, w magazynach, nie przestanę się denerwować, choćby nie wiem co.\\
Nie zważając na dręczącą mnie niepewność, uścisnąłem dłoń Leona. Ten odwrócił się przez plecy i zaczął po kolei przedstawiać mi pozostałe osoby.
\sx Sznyta na czole to Jonathan, ten, który najwidoczniej bardzo lubi rozmawiać to Radek, ten na lewo od niego to Mikołaj, a na prawo, Mark. Mnie znasz, a ten staruch\3k
\xx Pierdol się! -- warknął Izaak, wystawiając w stronę Leona środkowy palec.
\xx To Izaak.
\xx Jeśli nie chcecie mnie zabić\3k -- zacząłem.
\xx Pomyślimy o tym. -- wtrącił się Radek.
\qd
Sku*wiel.
\sx Jeśli się nie zamkniesz, sam się zabiję. -- wycedziłem ze złości. Widząc jego zdziwiony wyraz twarzy, zwróciłem się do reszty, tym samym całkowicie go lekceważąc. -- To o czym chcecie? porozmawiać? -- zapytałem, wciąż nie będąc pewny swojego losu.
\xx O Igorze. -- odezwał się Jonathan, siadając na szpitalnym łóżku po mojej lewej stronie. Kiedy kładł ręce na pościeli, zauważyłem na jego prawej dłoni błyszczący srebrzystym blaskiem przedmiot -- swoiste połączenie kastetu ze sztyletem, uchwyt na dwa palce, zakończony ostrzem.
\qd
Spojrzałem znowu na jego rozcięte oko. Zauważył to.
\sx Sam sobie tego nie zrobiłem. -- zapewnił, uśmiechając się.
\xx Co z tym Igorem?! -- krzyknąłem niemal, ku zdumieniu obecnych stalkerów.
\xx Spokojnie\3k -- uspokoił mnie Leon. -- Mówiłem, że nic ci nie grozi.
\xx Podejdź tu. -- rozkazałem.
\xx Wedle życzenia.
\qd
Kiedy Leon zbliżył się do mnie na dwa metry, przyjrzałem się jego palcom.\\
Były przyżółkłe od wieloletniego palenia papierosów.
\sx Od dawna palisz? -- spytałem. Leon, w pewnym sensie zbity z tropu tym pytaniem, odpowiedział po chwili namysłu.
\xx Od trzynastego roku życia, a co to ma do rzeczy?
\xx Ile razy kłamałeś mamusi, że nie palisz?
\xx Wiele, wiele razy.
\xx Teraz też kłamiesz?
\xx Jezu\3k -- mruknął Leon błagalnie. -- Zabierzcie mnie od niego, bo chyba coś mu zrobię. Mówię ci po raz ostatni -- nic ci się nie stanie, rozumiesz?
\xx To o co chodzi wam z tym Igorem, co? Tak bardzo się interesujecie tym kretem?
\qd
Radek gwizdnął.
\sx Ktoś tu jest dobrze poinformowany. -- powiedział. -- Ale do sedna. Nie tylko ty chcesz zrobić porządek z Igorem.
\xx Serio? -- jęknąłem. -- A któż to będzie mnie wspierał w dorwaniu tego ch*ja?
\xx My. -- oznajmił Jonathan.
\qd
Nigdy nie znalazłbym odpowiednich słów na opisanie mojego zdziwienia. Zdołałem z siebie wydusić jedynie krótkie, urwane ,,Co?''.
\sx Właśnie to. -- rzekł Leon. -- Święty, podaj listę. -- zwrócił się do Mikołaja. Ten wyciągnął z kieszeni zmiętą kartkę i podał ją kompanowi.
\qd
Leon rozłożył kartkę, odwrócił ją do góry nogami, po czym zaczął czytać.
\sx To lista rzeczy, których mamy się pozbyć w najbliższym czasie, sporządzona przez, jak mu tam było?
\xx Dymitra. -- podpowiedział Jonathan, kładąc się ,,plackiem'' na łóżku. Z sekundy na sekundę zrobił się wyjątkowo senny. -- Czytaj, chętnie posłucham.
\xx No więc\3k nie będę czytał słowo w słowo, za dużo tu tego. ,,Jako wasz nowy szef i przywódca, na początek nakazuję wam zrobić dwie rzeczy -- zmienić nasz wizerunek i wzmocnić ochronę. Musimy się jak najbardziej odróżnić od militarnych, bo wszyscy wiemy, za co tak nienawidzą stalkerów. My musimy być inni -- musimy trzymać pieczę nad Zoną a ze stalkerami się pogodzić; zapewnić ich, że też potępiamy wojskowych. Co najważniejsze, musimy znaleźć dowody na to, że masakra w grudniu dwutysięcznego nie jest naszą sprawką. No dobry początek, pozbywamy się dwóch rzeczy. Patroli i Igora''\\
Czyli, krótko mówiąc, mamy się nim zająć, nieważne w jaki sposób, Igor ma, powiedzmy, opuścić to miejsce.
\xx Jak? -- spytałem.
\xx Bo ja wiem? -- mruknął Jonathan, wpatrując się w sufit. -- Jednego już załatwiliśmy, a teraz mamy po naszej stronie Dymitra.
\xx Z nim będzie o wiele łatwiej. -- zauważył Radek.
\xx Możliwe. Ale to będzie prawdziwie brudna robota.
\qd
% 
\ro{33}
% 
\sx No, no\3k -- mruknął Graham, rozglądając się dookoła.
\xx Nieźle ich musiał wku*wić, co? -- rzekł John, zamykając drzwi wejściowe uczelni. Odwrócił się przez plecy, oparł ręce o biodra, po czym dodał. -- Pozdrapywać farbę i mielibyśmy Prypeć.
\xx Sussarowi by się to spodobało. -- zauważył stalker. -- Zawsze jest jakaś różnica, ale w końcu\3k -- nie kończąc zdania, uśmiechnął się lekko.
\xx Fakt, spodobałoby się mu.
\qd
Długi, ciągnący się kilkanaście metrów w obie strony korytarz z lekko zaokrąglonym sufitem zaścielony był połamanymi ławkami, porozszarpywanymi książkami wszelakiej maści oraz róznymi przedmiotami codziennego użytku. Białe i czarne, poukładane na wzór szachownicy poprzykrywane były butelkami, (zarówno szklanymi jak i plastikowymi) linijkami, ekierkami, kątomierzami, cyrklami, zerwanymi ze ścian obrazami i kilkoma przyrządami z pracowni fizyczno-chemicznej.\\
Na końcu korytarza po prawej stronie, na parapecie przy oknie stał wielki model globusa, odznaczający się na strumieniu intensywnego, słonecznego światła wylewającego się zza nowych, plastikowych szyb. Czarny kształt odznaczał się na tle niczym kleks na kartce papieru.\\
Naprzeciwko dwójki stalkerów zaczynały się szerokie, dość niskie schody prowadzące na piętro uczelni, również ,,przyozdobione'' porozrzucanym wyposażeniem budynku. Ściany na piętrze miały, w przeciwieństwie do kremowych na parterze, niebieski, soczysty kolor, mocno kontrastujący z identycznym, białym sufitem. Kolorystyka wewnątrz szkoły była niewątpliwie dziwna, bardziej jednak prawdopodobna była wersja, że z racji wybuchu w elektrowni nie skończono przemalowywać reszty ścian na biały/niebieski kolor.\\
Drewniana, zakończona elegancką gałką poręcz była podrapana i zniszczona, pokryta cięciami zrobionymi nożem lub scyzorykiem. Nawet na stopniach leżało kilka kupek wiórów i drzazg, jeszcze suchych.
\sx Odkąd zaczynamy? -- spytał Graham, odbezpieczając pistolet.
\xx Korytarz na lewo. -- oznajmił John, wskazując palcem za siebie. -- Sprawdzamy każdy pokój po kolei. I staraj się nie hałasować -- pamiętaj, że ktoś wciąż może tu być.
\xx Zobaczymy\3k
\dd
\xx Jak twoja ręka? -- spytał już drugi raz w ciągu minuty Leon Jonathana, bez przerwy wyglądając za okno swej kwatery. Spojrzał za zegarek. Widząc to, odezwał się Mikołaj:
\xx Która godzina?
\xx Siódma. Jeszcze pięć.
\qd
O wspomnianej siódmej pięć Jonathan, Radek, Lenny, Leon, Mikołaj, Izaak i Michał mieli się spotkać z Dymitrem, obecnie, jak to określił, ,,porządkując swoje nowe biuro''. Mieli wiele do omówienia -- sposób na dalsze rozpracowywanie łowców organów, kwestię samego Mastertona i wojska, grudniowej masakry lecz przede wszystkim -- mieli się tam udać by porozmawiać o Igorze.\\
Nie zdobycie broni, zabicie kogoś, upolowanie mutanta czy zdobycie któregoś z rejonów Zony.\\
To zmiana wizerunku miała się okazać najtrudniejszym zadaniem dla przedstawicieli rządu w Strefie. Oczywiście, nie uniknie się kilku zabójstw, jednak wiele rzeczy takich jak niechęć stalkerów do agentów po grudniu dwutysięcznego żadne morderstwo nie załatwi.\\
Podobno Dymitr miał plan co do tego ostatniego. W wstępnej rozmowie przez telefon wydawał się bardzo podekscytowany i niezwykle pewny siebie. Patrząc na jego dotychczasowe sukcesy i generalnie, jego ,,referencje'' można się było spodziewać naprawdę sensownego pomysłu. Nie wchodząc w szczegóły, Dymitr był w ścisłej czołówce wszystkich działaczy rządu Ukrainy, a Zona to idealny teren, w którym mógłby dożyć emerytury, jakkolwiek dziwnie to by nie brzmiało.
\sx Jak moja ręka? -- zadrwił lekko Jonathan. -- Bez tego i łykania prochów co godzinę się raczej nie obędzie. -- wyszeptał, wkładając na lewe ramię stary pas nośny, w który po chwili, z pomocą Mikołaja i ,,zachętą'' widocznego bólu, umieścił bezwładne a mimo to bolące ramię. -- A pogoda?
\qd
Leon jeszcze raz odsłonił firanę okna i wyjrzał przez nie na dziedziniec i wystające za nim drzewa, dość znacznie przysłaniające niebo.
\sx Wietrznie, ciemno, zimno i nieprzyjemnie a do tego burzowe chmury.
\xx Ujdzie\3k -- Jonathan z pomocą prawej dłoni wstał z kanapy, po czym otworzył stojącą na końcu długiego pokoju szafkę i począł wyciągać z niej płaszcz.
\qd
Zdołał wciągnąć go na prawe ramie, którym, po włożeniu w rękaw, nasunął pozostałą część ubrania na lewe ramię. Sam uporał się też z guzikami, pozostawiając nie zapięty jedynie ten najwyższy, przy samej szyi.
\sx Już? -- zaciekawił się.
\xx Ta. -- odpowiedział Izaak, sprawdzając godzinę w telefonie. -- Chodźmy.
\qd

Stałem na środku obozowego placu, opierając się o maszt flagi już od ponad dziesięciu minut. Co chwilę wyciągałem ręce z kieszeni i energicznie nimi pocierałem, próbując choć trochę się rozgrzać. Cienki kombinezon, w dodatku bez żadnego cieplejszego ubrania pod spodem, stanowił marną ochronę przed zimnem, który dziś było wyjątkowo dotkliwe. Suche powietrze, wiatr i całkowity brak słońca -- prawdziwa zgroza. Będę miał szczęście, jeśli jutro nie obudzę się z przeziębieniem czy, co gorsza zapaleniem.\\
Każdy podmuch wiatru nasilał uporczywe szczypanie zadanej mi wczoraj przez Igora rany. Ona, Dawid i przyjaciel Gavina. Do trzech razy sztuka, skurwielu.
\sx Już ja cię dorwę? -- wyszeptałem mimowolnie rozgoryczonym głosem.
\qd
Zrobiłem to dość głośno, zwracając na siebie znowuż uwagę mijających mnie agentów. W sumie, to czułem na sobie spojrzenia całego obozu. Patrzyli na mnie to ze skrzywieniem, to z rozbawieniem czy pogardą. Ale najbardziej ze zdziwieniem -- myśleli pewnie coś w stylu ,,Czemu jego szczątki nie są jeszcze rozrzucone dookoła przez anomalię?''.\\
Próbując nie myśleć o komentarzach i niemiłym podejściu mieszkańców co do mojej osoby, pomyślałem przez chwilę o Jonathanie Mastertonie.\\
Słyszałem o nim wcześniej, głównie od Gavina, wiele opowieści i historii, jednak dopiero po spotkaniu w cztery oczy osoba ta mnie naprawdę zaintrygowała. Co oczywiste, największą uwagę zwróciłem na jego rozcięte oko -- prawdziwie paskudny i przeszywający widok. Nawet nie próbowałem sobie wyobrazić bólu, który musiał wtedy czuć. Wolałem zastanowić się nad tym, jak wpłynął on na Mastertona, wreszcie, kiedy okaleczenie to go spotkało.\\
Tym właśnie zajmę się po opuszczeniu Zony -- będę szukał osób mieszkających w Prypeci i rozmawiał z nimi o Czarnobylu przed wybuchem w 86-tym. Jako, że Zona niewątpliwie będzie istnieć prawdopodobnie do końca świata, informacje o ,,rdzennych'' jej mieszkańcach mogą być niezwykle użyteczne i cenne. Akta S. -- marzeniem Gavina jest, by stały się one owianą legendą bazą informacji, o której będzie się mówić, że ,,znajdzie się w niej wszystko''. Ja dopilnuję zaś, by tak się stało, a pierwszą osobą, o którą zacznę wypytywać będzie właśnie Jonathan.\\
Między innymi po prostu z czystej ciekawości. Od dziecka miałem chorobliwe chęci jej zaspokajania, tym razem nie będzie inaczej.\\
Idą.\\
Prowadził Jonathan, przy jego lewym boku krzątał się Leon i Mikołaj, za których plecami z pomocą Izaaka, Radka i Marka o kulach kuśtykał Lenny z zagipsowaną nogą.\\
Cała zgraja psycholi, z których największym był Masterton.
\sx Już? -- zawołałem, gdy ten zbliżył się do mnie na cztery metry.
\qd
W odpowiedzi kiwnął głową i ruszył dalej, w stronę dużego, jasnego drewnianego domku, w którym od dziś miał rezydować cały ten Dymitr.\\
Westchnąłem, po czym podążyłem za nim, zachowując metrowy odstęp od ostatniego w ,,szyku'' Lenny'ego.\\
Teraz to dopiero byłem ciekaw.
\dd\sx I jak im idzie?
\qd
Sussaro sięgnął pamięcią od ostatniej rozmowy z Grahamem. Odtworzył ją sobie w głowie, pomyślał jeszcze kilka sekund, po czym odpowiedział Gomezowi.
\sx Byli już pod reaktorem, obeszli miasto? Myślę, że fotografują i nagrywają wszystko dookoła, by sprzedać to tutejszym ''kolekcjonerom''. -- wyrecytował Sussaro. -- Nam, o dziwo, też przydadzą się pieniądze. -- dodał.
\xx Zapewne.
\qd
Idąc dalej, wzdłuż szarej, kamienistej ścieżki, Gomeza znowu naszła ochota na wspomnienia. Na całą Prypeć przed wybuchem, tętniące życiem okolice Czarnobyla i tym podobnym rzeczom, które ludzie pokroju Finna nazwaliby ,,sentymentalną bzdurą''. Prawdę powiedziawszy, uważał go za osiłka, nikogo więcej. Graham był już nieco bardziej wrażliwy, zaś najlepszym rozmówcą w tego typu trudnych tematach był sam Sussaro. Przez te wszystkie lata znajomości, urozmaicane tego typu pogawędkami, Gomez i Sussaro prawdę powiedziawszy stali się bliskimi przyjaciółmi, których łączyły nie tylko wspólne cele.
\sx Nienawidzę takiego wiatru? -- syknął doktor, ściskając kaptur kurtki.
\xx Już prawie jesteśmy.
\xx Jak promieniowanie?
\xx Ujdzie?
\qd
Gomeza nagle naszło dość zastanawiające i niespodziewane pytanie.\\
,,Kim jest Sussaro --Człowiekiem''?
\sx No! -- nagły okrzyk zadowolenia obiektu refleksji doktora zupełnie wytrącił go z rytmu.
\qd
Zmrużył oczy, próbując zamaskować zdezorientowanie, po czym odgarnął kaptur z głowy, odsłaniając go dla przenikliwego wiatru. Rozejrzał się dookoła, próbując w pewnym sensie przywołać każdy szczegół widziany kilkanaście lat temu. Pierwszym, oczywistym wnioskiem był brak ludzi. A co z resztą?\\
Cóż, brak ludzi miał w gruncie rzeczy bezpośredni wpływ na ową ,,resztę''.\\
Niemal cały asfalt i chodnik dookoła zarośnięty był gęstym, żółtawo-brunatnym, gęstym mchem, gromadzącym na sobie spadłe liście i gałęzie. Mech ów wyglądał niczym spadły na nierównej, zaniedbanej drodze -- ,,powlewał'' się on w najgłębsze fragmenty szosy, upodabniając niezarośnięte fragmenty do małych wysepek.\\
Osiadła kilka metrów dalej kałuża aż tak bardzo nie różniła się od ,,pola'' mchu; różniła się jedynie kolorem, wielkością (miała ponad metr średnicy) i faktem odbijania w swojej brudnej, pełnej zanieczyszczeń tafli szarego nieba i przesłaniających go gałęzi.\\
Same drzewa nie zmieniły się przez cały ten czas -- wciąż stały dumnie przed budynkiem, patrząc na wszystko z wyższością. Inaczej miały się krzewy i trawa dookoła kompleksu -- zapuszczona, zachwaszczona, nie sprzątnięta z rozwianych rzeczy, w tym gnijących liści.\\
Budynek zaś\3k\\
Budynek stracił wiele ze swojej ,,świeżości'' i istotnie, wyglądał jak ruina. Jedyną ratującą go rzeczą było kilka okien, które przetrwały próbę czasu.\\
W oczy najbardziej rzucały się zacieki po deszczu -- wymieszały się z całą masą różnych rzeczy, przyjmując zielonawą barwę, która mocno odznaczała się na wypłowiałej, kremowych kafelkach.
Większość okien była zalana zielonawą mazią i brudna czy też zaparowana od środka, kilka jednak było zadziwiająco czystych, zupełnie, jakby niedawno je umyto.\\
W porównaniu do stanu przed kilkunastoma laty, prypecki szpital wyglądał dziś wręcz nienaturalnie.
Czyste, przyjemne drogi i okolica stały się niemal lasem, biel kafelków zniknęła, tak samo jak błysk szyb i wyblakłe światło rzucane przez odbijające promienie słońca drzwi wprawiane w ruch przez pacjentów, członków ich rodzin czy pracowników kliniki.
\sx Nienawidzę tego uczucia. -- jęknął Gomez.
\xx Nikt go nie lubi. -- uspokoił doktora Sussaro. -- I tak masz dobrze.
\xx To znaczy?
\xx Wyobrażasz sobie, co będzie przeżywał Masterton, jak tu przyjdzie?
\qd

Drzwi otworzyły się szeroko, ukazując dobrze znane wszystkim wnętrze. Dymitr raczej nie miał zamiaru przemeblowywać parteru, prawdopodobnie postąpi tak tylko ze swoim nowym biurem.
Za drzwiami stał Harlan -- wyjrzał przez nie, zbadał wzrokiem cały ganek, po czym dał obecnym znak, by weszli do środka. Pierwszy z miejsca ruszył się Jonathan, który poczuł się wyjątkowo nieswojo, przekraczając próg. Leon zaś niemal jęknął, przypominając sobie dzień pojmania Barry'ego.
\sx Banda inwalidów\3k -- zażartował Harlan, widząc pokaleczonych agentów.
\qd
Lenny ze złamaną nogą, Leon z raną brzucha i Mark z obwiązaną ręką, nie wspominając o bliźnie po brzytwie na czole Michała. Grupka stalkerów wyglądała jak zbiorowisko barowych zabijak, które dopiero co ,,rozwaliły'' kolejny bar.
\sx To jeszcze nic. -- odparł Masterton, zamykając za sobą drzwi. -- Bez pukania? -- spytał, wskazując kciukiem schody prowadzące na górę.
\xx Bez.
\xx Nareszcie\3k
\xx Hej! -- zawołał niespodziewanie Michał w stronę Jonathana. Ten, pokonując już pierwszy stopień schodów, zatrzymał się nagle i spojrzał w stronę łowcy przez ramię.
\xx Ta? -- spytał zirytowany.
\xx Mam przy tym być.
\xx Przy czym?
\xx Jego śmierci.
\qd
Jonathan uśmiechnął się pod nosem.
\sx Dobra\3k psycholu. -- odpowiedział, ,,strzelając'' ironicznie okiem. -- Czy to jest zaraźliwe? -- mruknął, wbiegając po schodach.
\qd
Michał prychnął na dźwięk ostatnich słów Mastertona. Odwrócił się powoli przez plecy, po czym spojrzał błagalnie na resztę stalkerów i, jak słyszał, bliskich przyjaciół Jonathana.
\sx Nie pytaj. -- rzucił rozkazująco Radek. -- My znosimy to przez czterdzieści lat\3k
\qd

Nacisnąłem kryształową (pałacu mu się zachiało?) klamkę jednym, pewnym ruchem i niedbale otwierając drzwi na oścież, wpadłem do środka. Ciekawe, jak?
\sx Jezu! -- krzyknął zaskoczony Dymitr, podskakując na krześle i łamiąc ołówek, którym wypełniał na biurku pewien dokument. -- poj*bało? -- krzyknął po raz drugi, stukając się w czoło. W komiczny sposób zsunęły mu się okulary -- dyndały bezwładnie, zwisając z wnętrza ucha.
\xx I to ostro. -- odpowiedziałem, zdając sobie sprawę, że w pewnym sensie zwyczajnie błaznuję.
\qd
Musiałem jednak odreagować wszystkie godziny spędzone w łóżku i pieluchach, ?wyładować się?, by znów zachowywać się jak na co dzień. Czyli nieprzewidywalnie.\\
Miałem dobry dzień.
\sx Jonathan, tak? -- spytał nowy dowódca, temperując urwany koniec ołówka ,,korbką''. Nie poprawił nawet okularów, które w końcu spadły na ziemię, niemal się tłukąc. -- A gdzie reszta?
\xx Zaraz przyjdą, ale ten cały Michał chce dotrzymywać nam kroku każdego dnia i być przy tym, jak w końcu go wykończymy.
\xx ,,Go''?
\xx No, Igora.
\qd
Dymitr temperował dalej, wbijając swe spojrzenie w kurczący się ołówek.
\sx A sam nie chce go zabić? -- rzekł po chwili.
\xx Myślę, że byłby wniebowzięty. -- powiedziałem z dumą w głosie.
\xx Masz. -- rzekł Dymitr, wyciągając w końcu ołówek z temperówki. Był cienki niczym igła, zapewnie niemal równie ostry. Wcisnął mi go w dłoń. -- Niech załatwi go tym.
\xx Serio? -- spytałem z niedowierzaniem.
\xx Nie. Ale przyznaj, że to by było niezłe. -- odparł z uśmiechem, zabierając mi ołówek.
\qd
Pochylił się nisko nad biurkiem, po czym wrócił do pisania na kartce. Podszedłem bliżej i zerknąłem na papier -- wyglądał mi na regulamin albo instrukcje -- każdy akapit czystej kartki A5 zaczynał się cyfrą, od jednego do, na razie, dziewięciu. Każdy punkt składał się ze średnio trzech zdań.
\sx Co to?
\qd Jutro ci powiem? -- mruknął Dymitr, kończąc zdanie zapisane przy cyfrze dziesięć. Schował papier do szuflady, zamknął ją na klucz, który schował w lewej kieszeni spodni i spojrzał na mnie, mówiąc:
\sx Siadaj.
\qd
Usłuchałem -- złapałem krzesło stojące w kącie pokoju, ustawiłem je przy biurku, po czym usiadłem wygodnie, zakładając nogę na nogę. Lewą rękę trzymałem w kieszeni, prawą zaś ,,stukałem'' w czubek kolana. Mój beztroski nastrój wskazywałby na to, że dopiero jutro będę w stanie zrobić coś ,,złego'' -- dziś miałbym, sam nie wiem czemu, opory by chociażby kogoś uderzyć, a co dopiero zabić.\\
Durna psychologia\3k
\sx Mówiłeś coś? -- spytał Dymitr.
\xx Kiedy mamy się nim zająć?
\xx Bo ja wiem? Kiedy chcecie?
\xx W sobotę?
\xx Może być.
\xx A jak mamy to zrobić?
\qd
Dymitr zakaszlał głośno, niemal lecąc w tył razem z krzesłem. Typowy kaszel palacza, do którego przyzwyczaił mnie Leon. Wolałbym nie wiedzieć, jak wyglądają ich płuca.
\sx Jak chcecie -- wysapał dowódca, łapiąc łapczywie oddech. Każdej próbie towarzyszyło głośne uderzenie pięścią w stół. -- Byle by\3k -- urwał po raz drugi, znów kaszląc.
\xx Skończył w anomalii?
\xx T\3k ta! -- sapnął, rzucając się we wszystkie strony.
\qd
Po trzech sekundach przestał, wziął pięć razy głęboki oddech. Jego zaczerwieniona twarz rozbłysła kilkoma kropelkami potu. Odetchnął z ulgą, przygładził włosy, po czym kiwnął głową.
\sx Tak, ma skończyć w anomalii. Ale przed Igorem macie się zająć kimś jeszcze. Zawołaj ich.
Kur\3k
\qd
Wstając przeszła mi przez głowę zabawna myśl.
,,Nie lubię zabójstw na zlecenie.'' Dziwnie to zabrzmiało, biorąc pod uwagę, kim jestem. Tego typu zlecenia wykonywałem chętnie, owszem, ale kilka lat temu. Znudziły mi się, do tego uznałem je za zbyt ryzykowne.\\
Ale cóż\3k Są momenty, w których trzeba się przemóc.
\sx A kogo chodzi? -- zagadnąłem, zbliżając się do drzwi.
\qd
Uchyliłem je lekko, wyjrzałem przez nie, po czym dałem znać reszcie, by weszła na górę. Najszybszy, co wcale mnie nie dziwiło, był Michał, który dosłownie wyrwał się z tłumu i jak dziki pomknął w moją stronę.
\sx Graham Kellerman. -- rzekł Dymitr sekundę przed uniknięciem przeze mnie zderzenia z pędzącym Michałem, który wleciał do biura jeszcze gwałtowniej ode mnie. Zanim zdążył cokolwiek powiedzieć, głos zabrał Dymitr.
\xx Tak, będziesz mógł go zabić, ale najpierw zajmiemy się kim innym, dobra?
\qd
Byłem bliski parsknięcia śmiechem, widząc ulatniający się entuzjazm Michała. Po dwóch sekundach upokorzenia w końcu kiwnął nieznacząco głową, po czym oparł się o ścianę, czekając na innych, zwłaszcza wleczącego się Lenny’ego.\\
Kiedy w końcu dokuśtykał do pokoju i wraz z resztą usadowili się na miejscach, Dymitr zaczął wprowadzać nas w szczegóły.\\
Zakręcił się dość mocno na krześle, obracając się dookoła co najmniej dwa razy.
\sx Graham Kellerman\3k -- mruknął, wciąż się kręcąc. Zahamował prawą nogą o stolik, po czym kontynuował. -- Graham był w Mryńsku przez kilkanaście godzin, ledwie kilka godzin po wybuchu. Z pewną osobą, której nie znamy\3k
\xx A jego skąd znacie? -- zapytał Radek.
\xx Stalker -- mało, prawie w ogóle nieznany, ale na tyle jednak, że wojskowi dali radę go ,,spisać''. W Mryńsku zrobiliśmy mu zdjęcie i\3k Nie spodziewałbym się tego po nim. Wiadomo o nim, że w życiu zdobył jedynie dwa artefakty i to, że całkowicie stroni od ,,życia publicznego'', w sumie to widzieliśmy go tylko raz, w ,,tej'' Zonie. Rok temu był w Rostoku przez tydzień, potem się nie pokazywał. Do wczoraj, kiedy wraz z jakimś gościem zestrzelili nam śmigłowiec\3k
\xx Zwolnij. -- rzuciłem nagle. -- Kiedy, jak, gdzie?
\xx Trzecia grupa badaczy, znaczy się, rządowa, która miała zbadać elektrownię została zestrzelona nad wieżą kościelną wczoraj w nocy. Więc, choćby z zemsty, macie go zabić. A najlepiej by było, gdybyście go złapali.
\xx Gdzie szukać? -- spytałem.
\xx W Prypeci rzecz jasna.
\qd
I tak oto prysnął cały mój dobry nastrój. Czyli jego mogłem już mieć z głowy.\\
Szybkim krokiem, ku zdziwieniu mych przyjaciół, Dymitra i Michała, wyszedłem z pokoju i zbiegłem po schodach. Sięgając dłonią do klamki drzwi wyjściowych z chatki usłyszałem kilka dudniących kroków za mną. Obstawiałem, że pobiegł za mną Radek, Leon i Mikołaj. Izaak był na to za ciężki.
Nie zwróciłem uwagi na lodowaty powiew wiatru, który niemal wbił mnie w ziemię swym chłodem, kiedy wyszedłem na plac. Schowałem prawą rękę w kieszeń i udałem się do ,,nory'' Igora. Nie było sensu tego dłużej odkładać -- sobotę spędzę przed telewizorem.


Był bliski popadnięcia w autentyczny obłęd.\\
Czuł się zagrożony -- pierwszy raz od ponad dwudziestu lat. Nigdy nie wyobrażał sobie takiej sytuacji -- takiej, w której w końcu ktoś będzie chciał się do niego dobrać.
Zastanawiał się, gdzie popełnił błąd, przez co lub przez kogo został ujawniony. Wszystkie okropieństwa, których się dopuszczał\3k\\
Nie zdziwiłby się, gdyby armia zjednoczonych stalkerów z całej Zony zaszarżowała na obóz rządu, gotów zrobić wszystko, byleby go zlinczować.\\
Igor od trzech godzin siedział na kozetce w bezruchu, próbując poukładać sobie w głowie wszystkie myśli, uspokoić się, chociaż trochę. Wszystko na nic.
\sx Ile mu dajesz?
\xx Bo ja wiem\3k trzy dni?
\xx Eee! Ja góra półtora. Pewnie zakradną się do niego w nocy i\3k
\xx A co z wywiezieniem?
\xx Rozpuszczą go.
\xx Wątpię. Ja na ich miejscu rozłożyłbym folię i go pokroił, a potem wyrzucał kawałek po kawałku, żeby nawet zęba nie znaleźli.
\xx Ząb?! Nie zdziwię się, jak któryś z nich zawiesi go sobie u szyi!
\xx Mimo to uważam, że nie będzie im się chciało go kroić. Obstawiam, że poproszą go o upolowanie mutanta i\3k
\xx \3kw tył głowy.
\xx Może. Ale gdyby Jonathan miał odpał, pochlastałby go tym swoim nożykiem.
\xx Nie musi chlastać. Jedno precyzyjne cięcie i\3k
\qd
I tak od dwóch godzin.\\
Armia halucynacji i omamów otoczyła Igora i nieustannie się nad nim znęcała, podsuwając wizje i wyobrażenia jego rychłej śmierci. Był prawdziwie zszokowany -- do tej pory nie zdążył się nawet przyzwyczaić czegoś bać, teraz zaś ,,kazano mu'' przygotowywać się do zejścia z tego świata w niezbyt miły sposób.\\
Może lepiej samemu się wykończyć? Oszczędzić cierpień? Igor i tak nie wierzył w Boga, więc nie przejmował się prawdą wiary, jako by samobójca miał trafić do piekła.\\
Otruć się czy zastrzelić? Raczej to drugie. Szybsze i mniej męczące.\\
Poza tym, pod ręką Igor miał jedynie pankuronium, a śmierć poprzez uduszenie po wcześniejszym paskudnym zwiotczeniu mięśni nie należała do przyjemnych. Dobrze to wiedział, bo już niejednokrotnie używał pavulonu, jak brzmiała ,,handlowa'' nazwa owego specyfiku.\\
Pistolet Igora, Glock 20, leżał na biurku, metr od szpitalnej, skórzanej kozetki. Pięć sekund i było by po wszystkim. Chociażby mieliby wywiesić jego ciało ku przestrodze, jemu by to nie przeszkadzało. Nie czułby już kompletnie niczego.\\
Cóż, trudno.\\
Mimo męczącej go niechęci, Igor jednym pewnym ruchem zsunął się z oparcia i wyprostował, po czym wykonał pierwszy krok w stronę pistoletu. Zatrzymał się chwilowo, by na niego spojrzeć. Tak, to będzie najlepsze wyjście. Szybko i bezboleśnie.\\
Nagle w głowie Igora odezwała się kolejna myśl.
\sx Trzy dni? Półtora? W życiu. Ja daję mu trzy minuty.
\qd
Stanął jak wryty, słysząc kroki. Szybkie, pewne i ciężkie, przypominały raczej bieg. Co oczywiste i dla Igora niestety nieszczęśliwe, kroki zbliżały się w stronę jego pomieszczenia.
\sx Idą po mnie\3k -- odezwał się po raz pierwszy od trzech godzin Igor.
\qd
Drzwi.\\
Przez całe te zamieszanie zapomniał nawet o tym, żeby zamknąć blaszane drzwi, wchodząc do środka. Niczym porażony prądem rzucił się w ich stronę, te jednak zdążyły już pomknąć do przodu, prawdopodobnie uderzone barkiem. Igor uniknął zderzenia z kawałkiem zardzewiałej blachy, nie zdążył jednak uchylić się przez żelaznym uściskiem dłoni, która chwyciła go za szyję.
\sx Master\3k! -- sapnął, wierzgając się we wszystkie strony.
\qd
Udało mu się raz kopnąć Jonathana w lewą nogę, pozbawiając go tym samym równowagi, jednak na zaledwie pół sekundy, której oprawca praktycznie nie odczuł. Parł wciąż naprzód, z długo wyciągniętą przed siebie lewą ręką, którą dusił Igora. Miał zamiar zdążyć, zanim przybiegnie za nim Radek, który pewnie by go zatrzymał.
\sx Skurwiel! -- wydyszał Igor, zderzając się ze ścianą. -- Wiedziałem, że to będziesz\3k -- urwał w wyniku ciosu zadanego przez Jonathana.
\qd
Uderzył go z główki prosto w czoło, niemal pozbawiając go przytomności. W tym czasie Masterton zwolnił lekko uścisk by przesunąć kastet z ostrzem -- wąską klingą do przodu. Widząc odzyskującego świadomość Igora, Jonathan ,,doprawił'' go dodatkowym kopnięciem w brzuch. Skulił się odruchowo, łapiąc się za bolące mięśnie. Jonathan złapał jęczącego Igora za włosy, wyprostował go do pionu, po czym zamachnął się lewą dłonią. Ściągnął ją w dół, po czym pchnął w stronę brody agenta, trafiając go ostrzem za podbródek, odchylając tym samym głowę nienaturalnie w tył.\\
Jonathan niemal się zawahał, słysząc przeraźliwy krzyk i krew powoli wyciekającą z pod brody. Musiał jednak ,,kontynuować''. Rzadko używał tego artefaktu, teraz zaś miał do tego idealną okazję.\\
Z pomocą ,,Sprężyny'' i niewątpliwie silnej lewej ręki, Masterton zdołał unieść ją wysoko w górę. Razem z wciąż nabitym na ostrze Igorem, który słabł z sekundy na sekundę, nie mogąc pewnie uwierzyć w to, co się właśnie dzieje.\\
,,Wisiał'', przybity do ściany, ponad pół metra nad Mastertonem, nie mogąc nic zrobić.
Nie mogąc wydusić słowa, jęknął jedynie ze zdziwienia widząc niespodziewany grymas na twarzy Jonathana. Próbował poruszyć prawą ręką, co sprawiało mu widoczne trudności. Do tej pory bezwładnie zwisająca, zdołała unieść się na wysokość pasa, przesunąć w stronę pleców i w końcu wyciągnąć pistolet.\\
,,Goły'' Colt należący do Radka, jeden z dwóch posiadanych przez niego modeli M1911A1. Ten był bardziej ,,klasyczny'' -- nie modyfikowano w nim niczego (nie licząc kilku wymienionych części), miał typowy, matowy, ciemnosrebrny kolor i okładzinę z ciemnego drewna, prawdopodobnie dębu.\\
Jonathana zaczęły opuszczać siły -- trzęsący się Igor, teraz już cały we krwi opadł kilka centymetrów w dół, poważnie przeciążając dłoń Jonathana. Problem stanowiła teraz jednak prawa, w której obecnie dzierżył pistolet.\\
W nagłym przypływie siły i adrenaliny Masterton poderwał broń do góry i przyłożył ją do klatki piersiowej Igora, celując w serce. Wcisnął się, niemal wgryzł lufą w żebro Igora.\\
Samemu czując wielki ból -- mięśni, głowy i przede wszystkim ścięgien, zdołał jakimś cudem pociągnąć za spust. Ciało Igora, teraz już bezwładne, podobnie jak grube mury i metry ziemi skutecznie zdusiły odgłos wystrzału.\\
Obie ręce Jonathana opadły bezwładnie z sił, wypuszczając tym samym ciało, które załomotało o kamienną podłogę, lądując na brzuchu.\\
Wypuszczony z bezwładnej dłoni Mastertona Colt wylądował obok prawej dłoni zwłok Igora. Palce wyglądały, jakby próbowały coś chwycić -- pozostały w panicznym rozwarciu.\\
Masterton uśmiechnął się lekko, widząc tą dość nietypową scenę. Zaraz potem krzyknął znowu, czując napływ bólu w prawej ręce.\\
Czuł się, jakby po tygodniu poruszono mu w niej ogromny skrzep krwi, który zaczął krążyć po żyłach.
\sx Szybciej! -- ponaglił Radek, zeskakując z pięciu stopni na raz.
\qd
Chwilę później zrobił to pędzący Mikołaj, który z trudem zachował równowagę po lądowaniu; upadłby, gdyby nie zderzył się z kamienną ścianą. Wyprostował się i miał zamiar już biec dalej, stanął jednak w miejscu, słysząc pewien odgłos.\\
Z końca betonowego korytarza rozległ się suchy trzask, jakby grubej, suchej gałęzi. Był jednak zbyt donośny i, choć wytłumiony między innymi przez ściany, łatwy do rozpoznania.\\
Mikołaj chwycił wciąż biegnącego przed siebie Radka za ramię i pociągnął go do tyłu. Widząc jego oburzone spojrzenie, rzekł:
\sx Już po nim.
\xx Jak\3k?
\xx Nie słyszałeś? Załatwił go.
\xx Kurwa\3k -- jęknął Radek, wyraźnie mniej podekscytowanym tonem. -- I co teraz?
\xx Na początek\3k zamknij drzwi.
\xx Jestem ciekaw\3k -- zaczął stalker, podchodząc do blaszanej klapy. -- Co mu zrobił. -- dokończył, chwytając obie klapy, po czym pociągnął je do siebie i zamknął, robiąc przy tym spory hałas. Mikołaj aż zgrzytnął zębami, słysząc metaliczny huk.
\xx Już?! -- zapytał zniecierpliwiony.
\xx Ta\3k -- mruknął Radek, blokując uchwyty blachy, spinając je łańcuchem. -- Już\3k
\xx Myślicie, że ma ta jakiś kwas, albo coś w tym stylu? -- odezwał się milczący do tej pory Leon, który stał od pozostałej dwójki o ponad dwa metry, nerwowo wpatrując się w głąb korytarza.
\xx To nie Jefferson. -- rzekł Mikołaj, robiąc pierwszy krok naprzód. Chwilę później ruszył za nim Radek.
\xx To znaczy? -- zapytał po kilku sekundach milczenia, podczas których trójka agentów pokonała ledwie trzy metry. Szli bardzo wolno i spokojnie.
\xx Do jutra, no, może góra za dwa dni, Dymitr ogłosi wszystkim, że mieliśmy załatwić Igora, a wszystkich, którym się to nie podoba, spotka zapewne to samo\3k
\xx W sumie\3k -- zamyślił się Radek. -- Przydałoby się zrobić tu nieco porządku.
\xx Z wojskiem będzie mu trudniej. -- zauważył Leon, idący powoli bokiem wzdłuż ściany. -- Jak ty sądzisz, Ra\3k
\xx Nie zaczynaj! -- krzyknął nagle Radek, mierząc Leona zabójczym spojrzeniem. Ręka odruchowo wylądowała mu na rękojeści Colta.
\xx Pytam poważnie\3k -- kontynuował wciąż spokojny Leon. -- Byłeś z nimi, widziałeś co potrafią, więc\3k
\xx Nigdy\3k -- wysypał Radek, łapiąc Leona za kark i przystawiwszy go do ściany, boleśnie trafiając jego nosem w mur. -- Nie mów nikomu o tamtym dniu, rozumiesz?! -- tym razem już krzycząc, wyjął pistolet z kabury. Widząc jego wściekły wyraz twarzy Mikołaj aż zdziwił się, że lufa M1911A1 nie wylądowała na potylicy Leona.
\xx Niby czemu? -- spytał Leon, wciąż w miarę spokojny, choć już lekko przestraszony, z niepewnie brzmiącym głosem.
\xx Jest mi wstyd, po prostu wstyd, rozumiesz?
\xx Wstyd za co?
\xx Nie udawaj, że nie\3k
\xx Jasne, że wiem! Trzech stalkerów i możliwość załatwienia reszty, ale coś ty im\3k
\xx Zaraz zrobię to, ku*wa tobie! -- wykrzyczał Radek, tym razem wyjmując pistolet z kabury i dociskając jego lufę tak mocno do głowy Leona, że tej jęknął z bólu. -- A raczej by ci się to nie spodobało.
\xx Daj mu spokój\3k -- szepnął ktoś zrezygnowanie z zacienionej części podziemi, idąc powoli w stronę trójki agentów.
\qd
Z cienia wyłoniła się zmęczona i spocona twarz Jonathana, dwa kroki później odsłonił on swój garnitur -- widniała na nim spora plama jasnej krwi; ciemna, która niemalże już skrzepła, pokrywała jedynie biały kołnierzyk koszuli, którą Masterton niósł pod marynarką.\\
Jej prawy rękaw zwisał bezwładnie -- był pusty, prawa ręka Jonathana oparta była o zwisający z szyi pas. W lewej dłoni, najobficiej pokrytej krwią, Masterton jak zwykle trzymał swój kastet zakończony ostrzem. O dziwo klinga była całkowicie czysta, wypolerowana ,,na połysk''; lśniła metaliczno niczym firmowa, nowo kupiona zapalniczka.\\
Masterton miał wiele dziwnych zachowań, ale do ścisłej czołówki zaliczało się polerowane owego ostrza tuż po jego ,,użyciu''. Zrobił tak nawet w przypadku\3k
\sx Słuchaj. -- powiedział Radek stanowczo w stronę świeżego mordercy Igora. -- Jeśli on\3k -- w tej chwili wskazał kiwnięciem głowy Leona. -- Jeszcze raz wspomni o\3k
\xx Wybiciu tamtego obozu? -- domyślił się Jonathan. Widząc zdziwione spojrzenie kolegi, uśmiechnął się lekko pod nosem. -- Lepiej o tym zapomnieć.
\xx Raz na zawsze. -- szepnął Mikołaj.
\xx Smuci mnie to. -- mruknął po nosem Jonathan, prostując się.
\xx ,,To'' czyli\3k? -- dociekł Leon.
\xx Pamiętacie te wszystkie nasze ,,przysięgi przyjaźni'', jeszcze z Prypeci?
\qd
W ciemnym korytarzu zapadła głucha cisza. Każdy na swój sposób zaczął rozmyślać o dzieciństwie, konkretnie zaś o przypomnianej przez Mastertona rzeczy -- ,,przysięgach''. Mikołaj, mimo swego przekonania o własnej dorosłości i prób nie myślenia o przeszłości z chęcią zaczął rozmyślać o tamtych czasach i dniach, w których poznawał swych pierwszych przyjaciół. Radek, pierwszy raz od ponad trzech tygodni wrócił myślami do swej młodości, dokładnej zaś, do pierwszego w życiu pobytu w szpitalu; tego, podczas którego ,,oficjalnie'' uznał za swego najlepszego przyjaciela Leona. Ten z kolei nie myślał o niczym konkretnym -- zaczął sobie jedynie wyobrażać siebie w wciąż ,,żywej'' Prypeci, leżącego na kamiennych schodach w centrum, wpatrującego się w niebo. Już wtedy go to uspokajało, teraz również był to jeden z jego lepszych sposobów na wszelkie przejawy złości. Zaraz po papierosach i alkoholu, których niestety od dłuższego czasu nadużywał, co gorsza, zaczął odczuwać tego skutki.\\
Masterton zaś, jak zwykle, nie mógł się skupić na jednej, konkretnej rzeczy. Z jednej strony myślał o Kamilu, z drugiej wciąż przeszkadzały mu w tym wspomnienia związane z Adrianem i pobytem w szpitalu po bolesnym ranieniu oka. W cały ten i tak już poważny ,,konflikt'' mieszał się ojciec Jonathana, Ullises -- zarówno te dobre jak i złe, tych pierwszych jednak, na szczęście, było znacznie więcej.\\
O dziwo, po tak wielu zmianach w swoim życiu i niewątpliwego częściowego wyprania z emocji, Jonathan pozostał optymistą.
\sx Więc\3k -- kontynuował, mimo tego, że nie doczekał się odpowiedzi. -- Robi mi się przykro z powodu tego, co dziś robimy\3k
\xx Mogło być gorzej. -- uspokoił się Radek.
\xx Jasne. Zawsze mogliśmy się pozabijać nawzajem za byle co, ale to\3k smutne.
\xx To przepadło. -- Radek schował w końcu pistolet do kabury, po czym podszedł do Mikołaja, wciąż kierując jednak słowa do Jonathana. -- Jedyne, co zostało z tamtych czasów to to, że wciąż jakoś trzymamy się razem, a to już naprawdę wiele.
\xx Szkoda, że nie wszyscy\3k -- szepnął Masterton, mając na myśli Kamila.
\qd
Mimo wszystko, dziś nie miał odpowiedniego nastroju na to, by wyłuskać wszystkie pozytywne związane z nim wspomnienia. Nie mógł odpędzić od siebie wydarzeń z końcówki kwietnia 1986 roku. Po prostu nie mógł. Zmienił temat.\\
Radek i jego współpraca z wojskiem.\\
Powód, dla którego mające kilkanaście lat więzi mogły w ciągu kilku sekund całkowicie stracić ważność. Był to chyba najbardziej drażliwy dla Radka temat; wątek, który dołował go bardziej niż Jonathana dołował brak Kamila. By uwolnić się od tych drugich i tym samym potwierdzić pewne słowa zasłyszane niegdyś w prypeckim szpitalu (,,Zawsze łatwiej jest mówić nie o swoim, ale cudzym nieszczęściu.''), Masterton myślami cofnął się do ubiegłorocznej gwiazdki.\\
Dnia, w którym stalkerzy tak naprawdę zaczęli prawdziwie nienawidzić stacjonujących w Zonie przedstawicielu rządu. Fakt, nigdy się nie lubili, ale była to raczej niechęć, drobne uprzedzenie, tłumaczone najczęściej konfliktem interesów.\\
Jednak od tamtego dnia, stalkerzy wypowiedzieli wojsku otwartą wojnę.\\
Biorąc pod uwagę okoliczności, mieli do tego całkowite prawo.
% 
\ro{34}
% 
\podro{Rok 2000}
% 
Zjadł zdecydowanie za dużo.\\
Karp, kilkanaście pierogów, ogromny zraz -- wszystko to popite wielką, wręcz przyprawiającą o mdłości ilością czerwonego barszczu. Zawsze objadał się niczym dziki wieprz, ale nigdy nie jadł tak dużo jak w Boże Narodzenie. Ilość potraw, samo wyczekiwanie na sposobność spożycia ich w tą niezwykłą okoliczność, jaką były święta, zwiększały nienasycony apetyt Grahama. Zwykle, tak jak teraz, odkupywał to godzinami spędzonymi w toalecie, nie były one jednak widocznie na tyle złe, by zapobiec przyszłorocznemu obżarstwu. Tym razem dopadła go dość silna niestrawność i jeszcze silniejszy ból brzucha; był on jednak do wytrzymania.\\
Kiedy skończy, wróci na swoje miejsce przy wigilijnym stole i pełnienia grzechu obżarstwa. Ku zdumieniu wszystkich jego znajomych, Graham nie był gruby, wręcz przeciwnie -- był doskonałej kondycji fizycznej, silny, wysportowany i dobrze zbudowany. Zawdzięczał to nie tyle ćwiczeniom, co wyjątkowo dobrej przemianie materii -- już od dziecka pamiętał, jak pomagała ona mu zachować odpowiednią wagę.
\sx Święta w Zonie\3k -- mruknął z lekkim niedowierzaniem, wyjrzawszy przez okno.
\qd
Chyba najbardziej przydatne wykorzystanie dawnej zabudowy, jakie Graham w życiu widział.\\
Do czterech chat wioski niedaleko Prypeci ponad tydzień temu wniesiono stoły, krzesła, sprzęt grający oraz dekoracje, które jeszcze do wczoraj leżały w pudłach, dziś zaś spoczęły na lampach, oknach i drzwiach drewnianych budynków.\\
Dwa dni temu umieszczono w każdej z nich przenośny grzejnik, który juz uruchomiono ubiegłej nocy tak, by zdążyć ,,nagrzać'' każdą chatę; na zewnątrz było wyjątkowo zimno i nieprzyjemnie, a prognoza z dnia na dzień się pogarszała. W miarę możliwości polokowano i pouszczelniano okna każdej ocieplanej chaty tak, by zachować w środku gorące powietrze. Co do drzwi, o dziwo, były one sprawne w każdej z chat, co było ,,zasługą'' mieszkających tu ponad rok temu handlarzy, którzy w pewnym stopniu wyremontowali okoliczne pomieszczenia i urządzili sobie tu coś w rodzaju targu. Po paru miesiącach jego niezwykle owocnego funkcjonowania handlarze postanowili jednak przenieść się bliżej Zony i jej niebezpieczniejszych, bardziej znanych terenów; nie każdy miał ochotę na kilku kilometrowy spacer nieznanymi ścieżkami, często naszpikowanymi anomaliami. Fakt miało to swoją zaletę; do Targu zjeżdżała się sama elita, która wydawała u kupców prawdziwy majątek, mimo to handlowcy postawili na powszechność i łatwy dostęp swoich usług.\\
Wynieśli cały sprzęt, wymontowali nawet okna, ale drzwi zostawili, ułatwiając tym samym w znacznym stopniu zadanie stalkerom, którzy podjęli się zorganizowania wigilii.\\
Na pierwszy rzut oka wydawało się to niemożliwe -- wystarczyło wypowiedzieć sobie w duszy słowo ,,Zona'', pomyśleć o wszystkich Jej ,,urokach'' i tym, z czym najczęściej się kojarzy, by się zniechęcić. Jednak nawet w Zonie było kilku ludzi dobrych woli; nie tyle mądrych, do odważnych, by choć zaprezentować innym pomysł urządzenia wigilii w takim miejscu jak Strefa. Po oczywistym zdziwieniu i niewątpliwie ironicznym parsknięciu śmiechem stalkerzy zastanowili się jednak głębiej nad tą kwestią -- była niezwykle kusząca, mogła w pewnym stopniu pomóc mieszkańcom Zony i\3k
\sx Nie\3k -- syknął Graham, zaciskając pięść ze złości.
\qd
Nienawidził, kiedy jego rozmyślania szły w złą stronę i traciły swój początkowy rytm. -- Inaczej\3k\\
Wigilia była po prostu dobra.\\
Tak, użycie tak prostego i wręcz prymitywnego słowa jak ,,dobra'' nieco zdziwiło Grahama, ale uznał on to za lepsze niż wielominutowe wymienianie różnorakich epitetów. Ta ,,obiektywna'' strona pojęcia ,,dobre'' (dla niektórych, przykładowo, oznacza to, że będzie miał okazję kogoś zabić) zawierała w sobie wszystkie pozytywy związane z wigilią; choć nikłe, to zawsze -- poczucie wspólnoty większe niż to odczuwane podczas wypadu, rozmowy w barze czy\3k\\
Po prostu niepowtarzalne.\\
Te święta były po prostu potrzebne, a w szczególności w Zonie; mimo wszystkich dziejących się w niej okropieństw, a może właśnie dlatego -- by dać żyjącym w niej ludziom choć trochę normalności. Potem? Niech robią co chcą; wrócą do tego, czym się na co dzień zajmują -- wypraw, morderstw czy polowań.\\
Ale ten jeden dzień Zona miała być nie do poznania -- widząc ją w święta ludzie mieli sobie zadawać pytanie ,,I to ta niesławna Strefa?'', mieli uznać nazywanie skażonego Czarnobyla piekłem za niedorzeczne.\\
Przez społeczność stalkerów przewijało się setki różnorakich pomysłów, w których nie brakowało dobrego zamysłu i szczytnych intencji. Ale tym razem w parze razem z intencjami miało iść wcielenie w życie i dotrzymanie obietnic ,,złożonych'' w teorii.\\
Miał to być spokojny, zwyczajny dzień, tak bardzo różniący się od każdego innego -- tych, które wypracowały opinię Zonie i stalkerom.
\sx No i się udało\3k -- mruknął Graham, znowu patrząc przez okno (jedno z dwóch w całej ,,wiosce'') na, uwaga, bogato przyozdobioną choinkę.
\qd
Niezwykły, za to bardzo krzepiący widok.

\sx Nie uważasz, że to lekka przesada? -- spytał Marek, obserwując gotujący się barszcz.
\qd
Bulgotał, aż miło, wypełniając pokój swym aromatem -- mimo tradycyjnego przepisu dominował w nim uwielbiony przez Marka czosnek. Nie mógł się doczekać, by go spróbować, mimo, że musiał poczekać jeszcze ledwie dwie minuty, wydawało się to dla niego za dużo.\\
,,Głodny jestem!'' -- pomyślał błagalnie.\\
\sx Może trochę\3k -- odpowiedział siedzący przy kuchennym stoliku Daniel.
\qd
Od godziny rozmawiał z Markiem, podczas gdy ten szykował barszcz, teraz zaś zeszli na temat ,,zakupów''. Otóż Daniel od ponad dwóch tygodni przymierzał się do zakupu nowej broni -- odkąd w wyniku spotkania z pewnym stalkerem, prawdopodobnie pracującym dla wojska, stracił swój karabin, nie mógł nawet zapolować na obłażącego ze skóry psa. Przesiadywanie w bazie, barze czy jakimkolwiek innym obozowisku nie tyle go dołowało, co zwyczajnie irytowało -- od zawsze rozpierała go energia, a polowanie na mutanty\3k było najlepszym sposobem na rozładowanie się.
\sx Znam jednego gościa\3k Adama. -- Marek powąchał barszczu. Był prawie gotowy.
\xx I co z nim? -- spytał Daniel.
\xx Załatwi ci tego samego Remingtona za osiemdziesiąt procent tego, za ile chce ci sprzedać ten twój znajomy.
\xx Ale nówkę?
\xx Oczywiście, że tak. On sprzedaje same nówki, chyba że chcesz jakiś kolekcjonerski egzemplarz.
\xx Kolega poszukuje rządowego Colta.
\xx Jedenastkę? -- domyślił się Marek, skręcając gaz w kuchence.
\xx Ta, jeden z pięciuset egzemplarzy na cały świat, Sing coś-tam\3k Mam to gdzieś zapisane, jutro ci powiem konkretnie i pogadasz z tym Adamem.
\xx W porządku\3k
\xx Słyszałem, ba, widziałem kiedyś ten model. Chyba nawet ma go jakiś stalker\3k
\xx Gdybyś mi go załatwił, byłbym bardzo wdzięczny. -- zapewnił Daniel. -- I jak? -- spytał po paru sekundach milczenia.
\xx Chwila\3k -- Marek chwycił w prawą dłoń dużą łyżkę, zanurzył ją w zupie, po czym spróbował sto\3k sto\3k sto dwudziestego trzeciego barszczu zrobionego w swoim trzydziestoletnim życiu. Raz, dwa, trzy -- miła liczba.
\xx Jak zwykle\3k -- mruknął z zadowoleniem. -- Idealny. Idź do\3k Kur\3k! -- Marek urwał nagle, łapiąc się za brzuch. Po dwóch sekundach koszmarnego bólu nastąpiła niespodziewana chęć na\3k
\xx Co ci? -- zaniepokoił się Daniel, widząc chwiejącego się Marka.
\xx Muszę\3k do kibla! -- wysapał Marek, biegnąc w kierunku drzwi.
\xx Zajęty!
\xx Co?! -- stalker zdążył otworzyć już drzwi. Stanął we framudze, gorączkowo łypiąc oczyma. -- Przez kogo?
\xx Graham\3k
\xx Kurwa! -- Marek wypadł z chaty tak gwałtownie i szybko, że stłukł szybkę w drzwiach wejściowych.
\qd
Już w przedpokoju poczuł chłód, na zewnątrz jednak był on tak nieprzyjemny, że stalker miał chęć zawrócić do środka. Robić w krzaki w jakimś w lesie przez tego zasranego debila-obżartucha! To się nazywa upokorzenie\3k\\
Wciąż trzymając się za brzuch, ominął przerobioną na kuchnię chatę dookoła i skierował się w stronę najbliższego, większego skupiska drzew, którego nie oświetlały nawet lampki wielkiej choinki. Mimo wszystkich niedogodności wsłuchiwał się w przyjemny dla jego ucha suchy dźwięk gniecionego śniegu.\\
Biegnąc obok niej, kołysząc nerwowo głową i zmagając się z setkami jaskrawych światełek poczuł się jak w pędzącym pociągu w słoneczny dzień -- choć zdarzyło mu się to ostatnio dopiero kilka lat temu, poczuł, że zaraz dopadnie go atak padaczki. Zamknął oczy i pobiegł kilka metrów na oślep, lewą, wolną ręką osłaniając się przed ewentualnymi gałęziami. Po czterech sekundach przestał odczuwać mdłości, oddaliły się one wraz ze światełkami choinki oraz ozdób, podobnie jak widmo ataku epilepsji. Brakowało mu tylko papieru do pełni szczęścia w tym nieszczęściu.\\
W końcu.\\
Zacieniony, wolny od czyjegokolwiek wzroku zakątek.

Otworzyłem powoli drzwi i wszedłem do przedpokoju. Poczułem zapach świeżego barszczu i kilku innych potraw -- ciekaw byłem, czy Marek zdążył już którejś z nich spróbować. Poczułem nieprzyjemny ścisk w żołądku.\\
,,To śmiecie.''\\
Po czterech krokach dotarłem do otwartych drzwi prowadzących do kuchni. Tutaj czosnkowo-ziołowy aromat osiągnął już apogeum, poczułem się jak w ekskluzywnej restauracji podającej tradycyjne potrawy. Znowu ten ścisk\3k\\
,,\3kgnoje!''\\
W kuchni, przy okrytym ceratą stoliku siedział młody mężczyzna, wyraźnie zaniepokojony. Nie wyglądało na to, aby miał przy sobie jakąkolwiek broń -- nie widziałem żadnej wypukłości ni w okolicy pasa czy klatki piersiowej. Jeśli w ogóle miał jakiś pistolet, to trzymał go najpewniej za plecami, wepchnięty w spodnie.
\sx Gdzie Marek? -- spytałem go.
\qd
Mężczyzna spojrzał się w moją stronę nieco zaskoczony, jakby dopiero zauważył moją obecność. Obadał mnie spojrzeniem, po czym odpowiedział.
\sx Wybiegł za potrzebą. Masz do niego jakąś sprawę? -- zaciekawił się po chwili.
\qd
Nie mogłem się powstrzymać od ironicznego uśmieszku, który szybko jednak zniknął z mojej twarzy, zmyty przez kolejną wątpliwość, po której zaś z kolei znowu w mej głowie rozbrzmiało propagandowe hasełko.\\
,,sku*wysyny!''\\
Nigdy w życiu nie czułem się tak niepewnie. Ale byłem już za daleko, żeby się wycofać.
\sx Martwi nie mówią\3k -- mruknąłem, wyciągając pistolet z kabury, po czym wymierzyłem go w młodego stalkera.
\qd
Singer mfg Co., jeden z pięciuset egzemplarzy na świecie, wart ponad trzydzieści tysięcy dolarów, nigdy nie miałem zaś ochoty przeliczać tego na ruble czy hrywny. Moje samouwielbienie miało pewne granice.\\
Przebłysk sceny z wczorajszego prania mózgu wywołał u mnie mimowolny skurcz mięśni, przez co kula Colta zamiast trafić stalkera w głowę wwierciła się w jego szyję.\\
Okropieństwa tego widoku nie dało się opisać słowami, toteż nie próbowałem. Zamiast tego skupiłem się i wycelowałem jeszcze raz, tym razem prosto w czoło.\\
Stalker umilkł zanim łuska zdążyła drugi raz odbić się od podłogi. Zakrwawiona ręka zsunęła się z szyi i opadła bezwładnie, chlapiąc wszystko dookoła krwią. Niepotrzebny bajzel.\\
I na co to wszystko?\\
Przełożyłem pistolet do lewej ręki, prawą zaś złapałem za nogę trupa, po czym ściągnąłem go z krzesła i powlokłem do przeciwległego pokoju, w którym panowała kompletna ciemność. Pod oknami postawiono kilka wysokich szaf, pomiędzy którymi wepchano szmaty i różne tkaniny, chcąc ocieplić budynek, jak widać, skutecznie. Mimo uszczelnienia słyszałem przez cienkie deski, co działo się na zewnątrz, a raczej, co działo się w pozostałych trzech chatach. Głośne rozmowy wymieszane ze śpiewami i odgłosami rodem z baru -- dźwięki sztućców, talerzy i szklanek. Zabawa trwała w najlepsze, póki co.\\
Znowu poczułem wstyd, że brałem w tym udział. Jak oni będą na mnie po tym patrzeć? Jak na typowego rzeźnika, który za pieniądze zrobi wszystko. Cała nasza przyjaźń pójdzie się zwyczajnie je*ać, bo\3k\\
Poczułem wibrujący w górnej kieszeni kurtki telefon. Zaciągnąłem zwłoki w kąt, przymknąłem drzwi, po czym wytargałem aparat z dna i odebrałem, nie patrząc nawet na to, kto dzwoni.\\
Gdybym wiedział, prawdopodobnie cisnąłbym nim w kąt.
\sx Radek?! -- krzyknął do słuchawki Masterton. -- Gdzie ty kurw\3k
\xx Nie teraz! -- rozkazałem stanowczo.
\xx Gdzie jesteś?
\xx Nieważne\3k
\xx Więc jednak.
\qd
Wiedziałem, że się domyśli. pierdo*ony wróżbita.
\sx Jednak co? -- udałem, mizernie tłumiąc w mym głosie wyraźnie słyszalny strach.
\xx Nie bój się. Choć nawet ja nie robię takich rzeczy, to wraz z resztą wybaczymy ci ten jeden raz.
\xx Wybaczyć co?! -- wrzasnąłem, uderzywszy czubkiem tłumika pistoletu w ścianę, omal nie robiąc dziury w starej desce.
\xx To, że dałeś się na to namówić. -- rzekł Jonathan, kończąc rozmowę.
\qd
Stałem przez chwilę wpatrzony w wyświetlacz telefonu -- tapetę, którą stanowiło zdjęcie kontrabasu, stan baterii i godzinę -- 21:42. Zerknąłem w najciemniejszy kąt pokoju, w którym leżało ciało zabitego przeze mnie stalkera -- nie było go widać, jak się zresztą spodziewałem. Poza tym\3k jeszcze trzy minuty i już nie będę się musiał niczym przejmować. Ponoć Masterton i reszta mi wybaczy.\\
21:43. Dwie minuty.\\
Wybrałem prędko numer Igora. Odezwał się po pierwszym sygnale.
\sx I\3k? -- zamyślił się pytająco.
\xx Ślijcie ich. -- rozkazałem. -- Dwóch do mnie, reszta tam gdzie ustaliliśmy. Przedłużam czas o dwie minuty.
\xx Będą u ciebie za dwadzieścia.
\xx Marek?
\xx Po nim.
\qd
21:47 i będzie po wszystkim.
% 
\dd\sx Jezu\3k -- sapnął Graham, czując kolejny napływ bólu w żołądku.
\qd
Wyjrzał jeszcze raz przez okno, tak, jakby widok ośnieżonej, różnokolorowej choinki miał go jakoś uspokoić. Zauważył przemykających obok niej szybkim krokiem dwóch mężczyzn. Nie dziwił się, widząc tempo ich chodu, zważywszy uwagę na chłód, który dosłownie wpędzał do najbliższego ciepłego pomieszczenia niczym gorąc zachęcał do zimnej kąpieli.\\
Ile on już tu\3k\\
Za siedemnaście dziesiąta. Ponad pół godziny spędzonej na sraczu przez głupi nawyk obżarstwa. Na szczęście Graham czuł, że nadchodzi kres jego niestrawności i zdąży przynajmniej skosztować barszczu Marka.\\
Obstawiał jeszcze góra dziesięć minut.

Dwójka ,,cyngli'' przekroczyła próg pomieszczenia. Na sam ich widok aż mnie zmroziło.\\
Nie, wyglądali całkiem normalnie. Ten po lewej miał wręcz przyjazną, okrągłą twarz, można by rzec, grubą, nie pasowała ona jednak do zwartego ciała poniżej. Nawet ciemne, gęste włosy wyglądały sympatycznie. Całości przyjemnego wrażenia dopełniał czarny, gęsty zarost, któremu niewiele brakowało do brody. Oczy, małe, czarne niczym węgiel i głęboko osadzone w ironiczny sposób mówiły ,,Tak, ja też określam siebie mianem >>do rany przytul<<''\\
Mężczyzna po prawej wyglądał już nieco bardziej poważnie -- miał pociągłą, a zarazem szeroką, kanciastą twarz, był także wyraźnie starszy od swego towarzysza -- oceniłem jego wiek na około trzydzieści pięć lat. Zamiast zrostu dookoła podbródka po nosem nosił niewielki, lecz równie czarny wąsik, mocno kontrastujący z bladą skórą. Nie widziałem jego włosów, ponieważ nosił na sobie dość nisko opuszczoną czarną, wełnianą czapkę.\\
Cóż\3k zawsze, kiedy wiem o kimś, że ma za sobą lata służby w Specnazie, mrozi mnie na jego widok, choćby wyglądał na najsympatyczniejszego i najmilszego człowieka na świecie i tak też się zachowywał, wciąż czułbym do niego pewne uprzedzenie, zupełnie jak teraz. Pozostawało mi cieszyć się, że są po mojej, a nie innej stronie.\\
Jeszcze bardziej zmroził mnie widok dwóch wyciągniętych przez nich AKMS’ów, z których każdy doczepiony miał wielki, bębnowy magazynek. Póki co, zawiesili je sobie na plecach, a w dłoniach chwycili po tłumionym Tokariewie.
\sx Ilu jest na zewnątrz, znaczy się, ,,dla nas''? -- spytał wyższy z żołnierzy.
\qd
Zastanowiłem się przez chwilę.
\sx My bierzemy chatę obok. -- odrzekłem, przeładowując Colta.
\qd
Schowałem go niedbale za pazuchę i wyszedłem z ,,gastronomicznej'' chaty, po czym skierowałem się w stronę tej pełniejszej. Robiła ona za tutejsze miejsce do pogawędek, tam zebrali się wszyscy stalkerzy oczekujący na przybycie do dwóch pozostałych chat, w których równo o dziesiątej miała się zacząć wigilia. Jako, że zostało już niewiele czasu, większość ze stalkerów usiadła już przy stole i tam wyczekiwała dziesiątej -- w ,,barze'' pozostało, wedle przebywającego tam współpracownika Igora, ledwie dziewięciu ludzi.
Obejrzałem się przez plecy. Dwójka członków Specnazu szła za mną, zerkając nerwowo na boki. Popatrzyłem w lewo i zauważyłem, że kryjący się w lesie oddział Igora przesunął się kilka metrów do przodu, wstępując na skraj gaju.\\
Niejednego mógł zastanawiać brak ochrony całej wioski -- brak ogrodzeń, budek wartowniczych, nawet pojedynczego strażnika. Jednak do tej pory stalkerzy przejmowali się jedynie mutantami; nie do pomyślenia było, aby wojsko bez żadnego ostrzeżenia zaatakowało obóz samotników -- była to abstrakcja, rzecz nie do pomyślenia, niczym hitlerowskie obozy zagłady za czasów drugiej wojny światowej -- choć poglądy faszystów były znane, ludziom nie mieściło się w głowie, by pogląd wcielić w życie i tym samym dziennie mordować kilku tysięcy ludzi. Tu sprawa miała się podobnie -- choć między stalkerami a rządem Ukrainy iskrzyło od dłuższego czasu i dochodziło do pewnych ,,incydentów'', ani jedna, ani druga strona nawet nie myślała o otwartej wojnie, ataku któregoś z należących do nich obozów, baz. Po prostu trzymali się od siebie z daleka, nie kryjąc wzajemnej niechęci.\\
Jednak stało się coś, przez co rząd podjął dzisiejszą decyzję. Nie znałem szczegółów i prawdę mówiąc, nie miałem zamiaru ich znać. ,,Wystarczyło'' mi to, że uczestniczyłem w pierwszej zbrodni wojska wobec stalkerów -- zwykłym skurwysyństwie, wymordowania kilkunastu Bogu ducha winnego ludzi, których nawet nie znałem, uczestniczących w pierwszej w Zonie wigilii.\\
Tkwiłem w błędnym kole -- choć dotarło już do mnie w końcu, że robię źle, musiałem doprowadzić tą rzeź do końca, inaczej skończyłbym jak Marek -- zadźgany ukradkiem i porzucony w lesie.
Są chwile, w których trzeba się przemóc\3k

Usłyszałem dziwny, przytłumiony odgłos, po którym coś załomotało w podłogę. Sekundę później zza drzwi łazienki rozległ się krzyk, któremu również towarzyszył łomot. Pękające szkło, jakby ktoś pociągnął pełen kuchennych zastaw obrus i zerwał go ze stołu. Cichy, przytłumiony syk, coś, co przypominało z brzmienia wtapiający się w kawał mięsa nóż, kroki, kilkanaście chaotycznych, panicznych kroków, kolejny łomot.\\
Podciągnąłem i zapiąłem spodnie, błagalnie wyjrzałem przez okno.\\
Choinka, chyba najoczywistszy zaraz po Świętym Mikołaju i prezencie symbol Bożego Narodzenia naśmiewał się ze mnie. Manifestując swym niewzruszonym, śmiałym obrazem ducha świąt naigrywał się z tego, że za drzwiami obok jakieś gnoje mordowały moich kolegów.\\
Po tej dziwnej myśli natychmiast w końcu dopadł mnie strach. Niepokojące odgłosy za drzwiami stawały się coraz głośniejsze, podobnie jak kroki. Prawdopodobnie nie miałem szans z oprawcami, ale przynajmniej miałem okazję zabić któregoś z nich.\\
Zanim zdążyłem złapać leżącego przy umywalce Uzi ktoś wyważył drzwi ubikacji -- w progu stanął wysoki mężczyzna, mierzący do mnie z zakończonego tłumikiem Colta. Dobrze zapamiętałem jego twarz -- niewzruszoną, zimną, ale mimo to jakby niepewną.\\
Czyżby tego gnoja targała jakaś wątpliwość?\\
Trafił mnie w szyję, po czym odwrócił się przez plecy i oddalił. Upadając, widziałem strzępki dużego pokoju, w którym jeszcze niedawno bawili się moi przyjaciele -- zanim boleśnie wylądowałem na plecach, ujrzałem fragment zakrwawionej, sztywnej dłoni, ściskającej pokryty kapustą widelec.\\
Z każdą sekundą brakło mi tchu, a ból stawał się coraz bardziej nieznośny. Wierzgając się we wszystkie strony, próbowałem wymacać swój pistolet, by chociaż przyspieszyć swoją nieuchronną śmierć.\\
,,Kula w szyję to koniec.''

Iście hipnotyzujący widok. Często spotykany, wręcz pospolity, ale ludzi pokroju Grahama niezwykle ujmujący.\\
Wzbijające się od niskiej świeczki aż po sufit barwne kłęby dymu, wijące się w każdą możliwą stronę, z sekundy na sekundę zwiększające objętość, zalewając pokój tysiącami przezroczystych, podświetlanych blaskiem kominka smolistych wzorów, setki różnych, nakładających się nawzajem kształtów, kłęb dymu przypominających litery, przedmioty, nawet twarze, wszystko to było dodatkowe przeplatane tysiącami elastycznych, dymnych linii, przewijając się przez resztę niczym pnącza.
Kiedy obłok dymu wzniosły się po zgaszeniu świecy przybrał kształt przypominający atomowego grzyba, Sussaro przerwał panującą w pokoju ciszę.
\sx Niby zwyczajne, a\3k -- zagadnął do Grahama, słodząc herbatę.
\xx Coś w tym jest\3k -- przyznał świeżo upieczony współpracownik. -- No, to o czym chciałeś pogadać.
\xx Ty mi powiedz\3k
R\3k
\xx O Radku, tak? -- ubiegł Grahama Sussaro. -- Pokażę ci coś\3k Mówiłem ci o dyktafonie? -- zapytał.
\xx Nie\3k Ale kiedy wtedy mnie znalazłeś i doszło do tej\3k ,,wymiany'', coś mi o nim świtało. To jakiś artefakt?
\qd
Sussaro uśmiechnął się chytrze. Zaraz zacznie swój zwyczajowy monolog.
\sx Coś więcej. -- rzekł z otwartą dumą w głosie. -- To coś w rodzaju tej kłęby dymu -- niby zwyczajne, ale\3k Coś w tym jest. -- po tych słowach wyciągnął zza pazuchy pewien przedmiot.
\qd
Zakrywając go palcami, postawił na stole, przytrzymał przez chwilę, po czym zabrał rękę.\\
Staromodny, czarny dyktafon na klasyczne, dwustronne kasety. Z boku ciemnej obudowy sterczało kilka szarawych przycisków. Pomiędzy dwoma ostatnimi włożona została poskładana kilkakrotnie karteczka. Sussaro po paru sekundach, w trakcie których dał czas Grahamowi na ponowne poważne nastawienie się do całej sytuacji (widok zwykłego dyktafonu mocno go zdezorientował) chwycił ów papier, rozwinął go (okazało się, że jest to kartka A5, w linie) i położył na stoliku. Graham ujrzał prześwitujące z drugiej strony zapisane linijki -- wyjątkowo drobnym, eleganckim drukiem, prawdę powiedziawszy, była to rasowa i dokładnie wyćwiczona kaligrafia.
\sx Jeden dyktafon, jedna kaseta, jedna kartka -- i tak od siedemnastu lat. -- powiedział Sussaro, kładąc rękę na stole, drugą drapiąc się w głowę. Miał bardzo poważne oczy -- jakby szykował się do dłuższej wypowiedzi i tylko czekał na to, aż ktoś mu przerwie. -- Od tylu lat na tym oto dyktafonie nagrywane są bardzo, ale to bardzo ważne, choć najczęściej kilkosekundowe, wypowiedzi. Różne osoby nagrywały -- czasem ja, czasem Gomez\3k
\qd
Gomez.\\
Ta postać naprawdę zainteresowała Grahama, zaczynając od dnia, w którym Sussaro przedstawił mu pracującego niegdyś w prypeckim szpitalu doktora. Opowiedział o nim niewiele -- właściwie to Graham wiedział o Gomezie tyle, że ma czterdzieści dziewięć lat, sporo lat doświadczenia w medycynie, wliczając w to długie studia i liczne lata pracy lekarza oraz fakt posiadania niebywale zasobnej wiedzy na temat wszelkich rodzajów artefaktów -- w tej ostatniej dziedzinie był największym znanym Grahamowi autorytetem, a widział już niejednego specjalistę. Gomez o artefaktach wiedział po prostu wszystko.
\sx \3kraz nawet nagrywał jeden z nauczycieli z podstawówki, wyobrażasz to sobie?
\xx Co tam jest? -- zaciekawił się Graham.
\xx Sam zobacz. -- zachęcił Sussaro, spoglądając w stronę rozłożonej kartki zeszytu. Widząc ruszającego się stalkera dodał pospiesznie. -- Nie zalej jej herbatą!
\xx Spokojnie\3k -- zapewnił Graham, wyciągając dłoń przez całą długość stołu.
\qd
Chwycił delikatnie kartkę między palec wskazujący, a kciuk, po czym, obchodząc się z nią jak z relikwią, ułożył przed sobą, tak, by mógł siedząc wyprostowany, bezproblemowo ją przeczytać.\\
Miejsce.\\
Rok.\\
Miesiąc.\\
Dzień.\\
Godzina.\\
Minuta.\\
Sekunda.\\
Okoliczności.\\
Osoby mówiące w nagraniu.\\
,,Rejestrator''.\\
Po trzech linijkach oczy wyszły mu na wierzch, po trzech następnych serce zaczęło głośno dudnić, kiedy dotarł do połowy zapisanych dat i miejsc, sapnął głośno, jakby z wycieńczenia, po czym z sercem, tym razem walącym jak młot, omiótł gorączkowo ostatnie dwa wersy, po czym otarł czoło -- w ciągu tych kilku sekund zdążył się obficie spocić. Westchnął i skierował spojrzenie na Sussaro.
\sx Nie wierzę. -- oznajmił mu tonem niedowiarka. W odpowiedzi otrzymał jedynie pewny siebie uśmiech. -- Naprawdę? -- tym razem skinienie głową, któremu również towarzyszył uśmiech, jeszcze szerszy, niż poprzednio. -- Jezu\3k
\xx Chcesz posłuchać?
\qd
% 
% 
\podro{Rok 2001}
% 
% 
\sx Jeszcze jedna wzmianka\3k -- wysyczał Radek. -- I po tobie.
\xx Spokój! -- zarządził niemal Jonathan, poprawiając marynarkę. -- Wynieście go w piz*u, bo zdąży nam zgnić.
\xx Na widok? -- zapytał Izaak.
-Na widok -- to Dymitr będzie się tłumaczył, a pewnie i tak będą siedzieć cicho, Igor nie miał tu nawet prawdziwego kolegi.
\xx No dobra\3k -- mruknął Leon, wzruszając ramionami, po czym niechętnie skierował się w stronę pokoju, w którym zginął Igor. Mijając jego oprawcę, zapytał:
\xx Byle jak, czy ostrożnie?d
\xx Zdecydowanie ostrożnie. -- odpowiedział Jonathan z uśmiechem. -- Bo krwią się zachlapiecie. Ta\3k -- chrypnął z zadowoleniem, kiedy Radek z Leonem zniknęli w ciemnym, kamiennym korytarzu. -- Idźcie, grabarze\3k -- szepnął z pogardą.
\qd
Słysząc głośne ,,Jezu!'' wykrzyknięte przez Radka, roześmiał się cicho pod nosem. Ból w ręce zaczął powoli przemijać, a jej właściciel czuł się jak ryba w wodzie.
\sx Musiałeś? -- powtórzył już drugi raz w ciągu dziesięciu minut Radek, spoglądając z rozpaczą na zakrwawiony rękaw. -- Nie lepiej było ci go udusić? Strzelić Rugerem? Musiałeś, go ku*wa, chlastać?!
\qd
Masterton w odpowiedzi kiwnął zadowolony głową, dodając do tego uśmiech.
\sx Lepiej byście go już wrzucili. -- zmienił temat. -- Zimno, jak skur\3k
\xx Dobra. -- przerwał Radek. -- Izaak, bierz go za nogi. Ta\3k wyżej\3k gotowy?
\xx Raz\3k
\xx Dwa\3k
\xx Trzy\3k
\qd
Zwłoki Igora pokonały w powietrzu dwa metry, po czym załomotały o podłoże i zaczęły się głośno staczać w dół, wzbijając przy tym tumany kurzu. W pewnym momencie, zanim jeszcze się zatrzymały, u zboczu urwiska zaczął się tworzyć powietrzny wir, zgarniający okoliczne liście, grudki ziemi i pył w promieniu co najmniej trzech metrów. Po sekundzie urósł do niebotycznych rozmiarów i zaczął ,,ciągnąć'' do siebie bezwładnego Igora -- wir uniósł się wraz z nim na wysokość dwóch metrów, zbierając jeszcze więcej pyłu i kurzu, zaczął wydawać też charakterystyczny, narastający świst.
\sx Nie wiem jak wy! -- krzyknął schowany za drzewem Jonathan do stojących nad urwiskiem dwójki agentów. -- Ale ja bym się na waszym miejscu schował. -- dodał jeszcze głośniej, ledwie przekrzykując niemal rozsadzający bębenki szum anomalii.
\qd
Radek i Izaak, jakby wyrwani z głębokiego zamyślenia runęli w stronę drzewa, w ostatniej chwili stając w bezpiecznym miejscu, tuż obok Mastertona

Szybko oddalił się o kilka metrów do tyłu, z dala od promieniujących wózków. Próbował jednocześnie wypatrzyć źródło hałasu -- coś szeleściło i wiło się w długim, blaszanym, pospinanym ze sobą łańcuchami rzędzie. Samo w sobie, cokolwiek to było, nie wydawało żadnych dźwięków -- nie oddychało, nie sapało, czy ryczało, nie był więc to prawdopodobnie żaden z tych bardziej znanych mutantów.\\
Gavin aż odskoczył ze strachu w tył na widok kilkunastu iskier, które wystrzeliły spośród sklepowych wózków, wydając przy tym ogłuszający, metaliczny zgrzyt. Kiedy iskry opadły, wśród setek metalowych pospawanych drutów po raz pierwszy przemknęło coś, co można było nazwać jako takim kształtem -- iskrzący, niematerialny, jaskrawy obłok, plama nieznanej wielkości.
\sx Nie\3k -- jęknął idący obok Gavina Hubert. W jego głosie zabrzmiał prawdziwy, namacalny strach. -- Tylko nie on\3k -- przełknął nerwowo ślinę, bezradnie opuszczając karabin lufą w dół.
\qd
Wózki zaszeleściły i zazgrzytały przeraźliwie kolejny raz, niewyraźny do tej pory obłok zaś zwiększył niebotycznie swoje rozmiary, w ciągu sekundy osiągając ponad metr średnicy. Przeistoczył się w coś wyglądającego na kulę żywej elektryczności, kulisty piorun, setki iskrzących się łuków elektryczności, z których każdy zdawał się żyć i co gorsza, stanowić część myślącej samowolnie całości.\\
Kiedy po budynku nieotwartego z powodu katastrofy Super Samu przetoczył się niewyraźny, wszechobecny ryk, Ted i Gavin nie mieli już żadnych wątpliwości; Hubert był za to pewny co do swojego zdania od dłuższego czasu. Wszyscy trzej stali za to osłupiali, wpatrując się w wciąż rosnącą kulę elektryczności. Pierwszy zareagował Hubert.
\sx Polter! -- wrzasnął głośno, zrywając się do biegu. –Polter!!!
\qd
Nie trudząc się próbami trafienia ducha, odwrócił się przez plecy i pognał przed siebie, z karabinem w lewym ręku. Po trzech sekundach biegu, podczas których znacznie zbliżył się do wyjścia ze sklepu, ujrzał obok siebie Gavina, biegnącego zygzakiem, z komicznie fruwającym na boki automatem, zawieszonym na lewym ramieniu. Po ułamku sekundy przed dwójkę uciekających stalkerów znienacka wybił się Ted, który ,,wpadł'' w swoje zwyczajowe tempo. Dosłownie -- biegł, aż się za nim kurzyło; stukot butów swą częstotliwością przypominał serię z karabinu maszynowego, kończyny poruszały się nienaturalnie szybko, rytmicznie, jak u profesjonalnego biegacza, którym zresztą Ted pewnego czasu był.\\
Nie używał żadnych ,,wzmacniaczy'' -- artefaktów, proszków, sterydów; niczego, jedynym i wystarczającym atutem było ciało Teda samo w sobie -- rozwinięte do granic możliwości, wyrobione latami treningu, o niesamowitej kondycji.\\
Mimo hałaśliwego coraz bardziej sapania Hubert wciąż słyszał za tobą elektryczne trzaski Poltergeista, które na szczęście zdawały się powoli milknąć, pozostawać w tyle. Mimo to narastało w nim wciąż (zresztą pewnie nie tylko w nim) charakterystyczne, rozchodzące się po całym ciele, mrowienie, które, jeśli nie ucieknie się od ducha w ciągu kilkunastu sekund przerodzi się w jeden wielki, wręcz paraliżujący skurcz. U Huberta zawsze zaczynało się od palców u rąk -- mrowienie z okolic paznokcia szybko obejmowało cały palec, potem zaś, szczypiąc już niemiłosiernie, obejmowało całą dłoń.\\
Zwolnił nieco, czując nieprzyjemne rozpieranie w ramieniu, przyspieszył znowuż po wyjściu ze skrętu w lewo -- znajdował się teraz wraz z Hubertem i Tedem przed wyjściem z Supersamu, tym południowym, stanowiącym zjazd dla samochodów. Zbiegli w dół, nabierając dodatkowej prędkości -- Gavin z Hubertem sapali już głośno, Ted zaś biegł, nie dość że kilka metrów dalej, to jeszcze bez najmniejszego śladu zmęczenia.\\
Trzy kolejne metry -- Biegacz bez zmrużenia okiem wybiegł spod cienia sklepu, znajdując się na prażącym słońcu. Powietrze nie cuchnęło już pyłem i zwietrzeliną -- Ted, zwalniając lekko, po raz pierwszy odetchnął pełną piersią, aż wyginając się do tyłu, po czym zerknął ukradkiem przez ramię.
Gavin, który został nieco w tyle za Hubertem, znacznie zwolnił, a po Poltergeiście nie było już śladu. Ted zatrzymał się gwałtownie, obrócił się i pomachał ręką.
\sx Już po nim. -- zawołał. -- Nie ma go, stójcie!
\xx Jezu\3k -- sapnął Gavin, opierając się na kolanach, sapiąc ciężko. -- I tak mieliśmy szczęście. Pamiętacie\3k
\xx Nawet mi nie przypominaj! -- wtrącił się Hubert, siadając na chodniku. -- Nie powinniśmy jeszcze się trochę, no\3k oddalić?
\xx To był słaby Polter. -- rzekł Ted. -- Mógłby ci co najwyżej rzucić kamieniem w twarz albo skaleczyć kawałkiem szkła. Stąd oczywiście, bo gdyby się do nas zbliżył\3k Jezu\3k -- jęknął nagle przerażonym głosem.
\xx Co? -- zaniepokoił się Gavin, prostując się. -- Ted?
\xx On tu jest\3k -- Sprinter rozejrzał się nerwowo dookoła, zatrzymał swoje spojrzenie na wjeździe dla samochodów. -- Pusto, ale on gdzieś jest\3k Mrowi\3k spierda*am stąd! -- krzyknął panicznie, zrywając się do biegu.
\qd
Zrobił trzy, coraz szybsze kroki, po czym jakby trafiony obuchem, padł bezwładnie na ziemię, boleśnie przejeździwszy się policzkiem po betonie. Zasyczał z bólu, po czym przewrócił szybko na plecy, kurczowo przekładając karabin do ręki -- choć nie mógł nim skrzywdzić ducha, dodatkowe kilogramy w ręku zwiększały jego poczucie bezpieczeństwa.\\
Pośród trójki członków S. rozbłysło oślepiające światło, któremu towarzyszył ogłuszający, rozdzierający powietrze huk.\\
Był ponad dwukrotnie większy, niż poprzednio, złożony z jeszcze większej ilości jeszcze bardziej poplątanych iskier i piorunów. Biło od niego gorące, cuchnące spalenizną powietrze, które zdawało się rozsadzać nozdrza, kiedy się nim oddychało. Gavin odskoczył do tyłu i boleśnie wylądował na plecach, bezradnie wpatrując się w ducha, Hubert zaś stał bez ruchu z założonymi na kolanach rękoma, mierząc go wzrokiem. Beznadziejnie obojętnym spojrzeniem, jakby przestało mu na czymkolwiek zależeć i w jednej chwili przestał się przejmować groźbą utraty życia. Skrajna apatia.
Jedna z wijących się iskier urosła do niebotycznych rozmiarów i trafiła go w twarz, zrywając go z nóg. Upadł bez słowa ani wydania z siebie nawet cichego krzyku, rozrzucając na boki bezwładne, wiotkie ręce.

\sx Oho! -- mruknął zadowolony Jonathan, odrywając oczy od lornetki. -- Nareszcie. Ej! -- klepnął drzemiącego Mikołaja w czoło. Poczekał, aż ten przestał się awanturować i klnąć, po czym przypomniał, ,,Co on tutaj robi.'' -- Adam jedzie.
\xx Dziesięć minut\3k Który bierze łopatę? -- Masterton podniósł się na równe nogi bez użycia rąk, rzucił niedbale lornetkę na wciąż leżącego na trawie Mikołaja, który odpowiedział na to wiązanką, której nie powstydziłby się zawodowy komik.
\xx Wstań, bo rzucimy cię na asfalt i Adam przejedzie ci oponą po łbie! -- warknął Jonathan, zerkając przez ramię.
\qd
Raz jeszcze rozejrzał się po okolicy.\\
Wielkie, pagórkowate, trawiaste pole, przecięto pośrodku popękaną szosą otoczoną przez drzewa. Tych było tu niezbyt wiele -- jedyne większe skupisko, poza majaczącym na północy, kilkadziesiąt kilometrów dalej lasem, stanowiło obecną kryjówkę stalkerów -- kilkanaście wysokich, liściastych, wilgotnych drzew na najwyższym w okolicy, aż bardzo łagodnym, wzniesieniu, doskonale nadającym się na miejsce do obserwacji. Niebo było szare i zachmurzone, powietrze zimne, rześkie. Rzadką, słabą mgłę dało się zauważyć dopiero nad horyzontem -- wypełniała zaczynające się tam rozległe lasy.
\sx Co robimy? -- spytałem, zmagając się z targającymi mną dreszczami, od których aż drżał mi głos.
\qd
Stałem skulony, oparty o pień z dłońmi głęboko wciśniętymi w kieszenie. Mdliło mnie; tej nocy spałem zdecydowanie zbyt krótko. Nie dla mnie czterogodzinny odpoczynek.
\sx Schowacie się za drzewami, przy zakręcie i wyskoczycie jakieś dwadzieścia metrów przed nim, żeby zdążył wyhamować. Strzelcie, ale w powietrze albo w bok maski, byle nie w niego na Jezusa!
\xx Od kiedy zrobiłeś się taki religijny?
\xx Nieważne. Bierzcie kominiarki i złaźcie na dół. -- zwrócił się Jonathan do Leona i Marka. -- My zaczniemy schodzić na dół nieco wcześniej, niech się przestraszy\3k W tym czasie\3k szkoda, że nie ma z nami reszty.
\xx Dopatrują Radka, jakby mu rękę, a nie palec urwało. -- zdenerwował się Mikołaj, z wielką niechęcią wstając.
\qd
Porozciągał się we wszystkie strony, odruchowo sprawdził pistolet, podniósł leżącą obok kominiarkę i schował ją do kieszeni. Spojrzał na opartą o drzewo łopatę. -- Ja ją wezmę.
\sx A będzie konieczna? -- zmartwił się Mark. Obrońca\3k
\xx Jak to ,,konieczna''? -- zdziwił się Jonathan, chichocząc wrednie. -- Masz opory, żeby raz walnąć komuś łopatą? Gdzieś ty był, zanim cię tu przysłali, w AI?
\xx Nie, w narkotykowym kartelu.
\qd
Cisza. Zdawało się, że wyznanie Marka wprawiło w milczenie nawet otoczenie -- wiatr umilkł, źdźbła trawy przestały szeleścić. Mikołaj z Leonem zrobili wielkie oczy, Jonathanowi zaś kompletnie opadła szczęka. Uniósł teatralnie lewą brew w niemym akcie zdumienia, po czym wciąż nie mogąc uwierzyć, w to, co usłyszał, zapytał:
\sx Rosyjskim?
\qd
Mark, zyskując nagle na pewności siebie prychnął śmiechem, w którym dało się słyszeć nutkę urazy.
\sx Nie, ku*wa, amerykańskim i to ja zabiłem Tony’ego Montanę! -- zdenerwował się. -- Jasne, że rosyjskim, do cholery! I powiem wam\3k -- tu zmierzył pobliskich stalkerów zabójczo poważnym wzrokiem. -- To, co tam się wyprawiało w porównaniu z waszymi metodami\3k wypadacie bardzo blado. Bardzo.
\xx Taa\3k -- mruknął Masterton, uśmiechając się chytrze. -- Założysz się, kto tu jest większym\3k
\xx Zapomnij. Idę na dół\3k -- zniechęcony Mark założył kominiarkę na twarz, po raz pierwszy przyjmując groźny, odpychający wygląd, po czym zaczął powoli zsuwać się ze zbocza w stronę grubego drzewa rosnącego przy asfalcie.
\qd
Zajęło mu to kilka sekund -- po dwóch uznał powolne kroczenie za bezcelowe i zbiegł z niego szybko, lądując pośrodku jezdni. Oparł się o plecy i zapatrzył w ziemię.
\sx Teraz można rzec, że wygląda posępnie. -- zauważył Jonathan, kiwając z zadowolenia głową. -- Myślicie, że zrobili tam coś, od czego mu się poprzestawiało we łbie i zaczął być taki świętoszkowaty?
\xx Zapewne. -- zgodził się Mikołaj, bawiąc się trzymaną w rękach łopatą. -- Powinieneś mu zazdrościć.
\xx Niby czego?
\xx U niego poszło to w dobrą stronę.
\xx Gdybyśmy nie znali się tyle lat\3k odciąłbym ci palec, wiesz o tym?
\xx No i właśnie\3k -- Mikołaj zarzucił łopatę na prawe ramię. -- Ale na serio, nie zrób niczego głupiego, Adam ma\3k
\xx Wiem! -- warknął Jonathan rozgoryczony. -- Szykuję się na kogoś innego\3k
\xx Na Grahama? -- zgadłem, co lekko rozkojarzyło Mastertona.
\qd
Zamrugał gniewnie oczyma, przełożył nogi, kaszlnął, jednym słowem, wykonał całą gamę zbędnych ruchów, które miały odwrócić uwagę od jego zakłopotania. Do tej pory wydawał się bardzo opanowanym rozmówcą, widocznie jednak temat Grahama to jego słaby punkt.\\
Jakiż to wielki pożytek niosło za sobą czytanie akt Autorów.
\sx Tak. Jemu już naprawdę konkretnie dobiorę się do dupy. Będzie żałował, że się urodził. Tak\3k
\qd
,,Urodził\3k''

Autorzy\3k\\
Handlarz, który mi dostarczył ich zapiski, właśnie miał być przeze mnie przesłuchany. Wszystko przez tego sku*wiela Igora\3k Na szczęście już gryzie glebę, przynajmniej jego rozerwane przez wir anomalii szczątki to robią. Tak zwana ,,seria niefortunnych zdarzeń'' doprowadziła mnie tutaj, zmuszonego do współpracy z miejskimi służbami\3k nieźle trafiłem.\\
Dobrze, że jest tu ten cały Mark.\\
Wydawał się w miarę\3k ,,normalny'', choć niedawna rewelacja na temat jego przeszłości dość skutecznie pogorszyła wizerunek byłego\3k właśnie, kogo? Zajmował ,,wyższą funkcję'', zarządzał innymi, wydawał, czy wykonywał rozkazy i polecenia -- był, czy podlegał tamtejszym wielkim osobistościom? Nie dało się tego przewidzieć -- tam każdy, nawet przewoźnik małych działek może doświadczyć szoku, byś świadkiem najgorszego okropieństwa -- nawet nie próbowałem zgadywać. A co z resztą? Jonathan -- skończony wariat z huśtawkami nastroju, a tacy są najgorszy. Izaak zachowywał się do tej pory\3k zwyczajnie. Nie zrobił niczego nadzwyczajnego, strasznie zapadającego w pamięć, podobnie jak Lenny. Leon i Mikołaj zdawali się szczególnie ze sobą związani -- widać to było w ich wzajemnym stosunku, sposobie prowadzenia rozmów, gestach, dosłownie wszystkim. Jakby jeden drugiemu kiedyś ocalił życie i do dziś starał się okazywać wdzięczność. Mikołaj był szczery, zawsze w dobrym humorze, posiadał tak zwaną ,,duszę dziecka'' -- 
zachowywał się
luźnie,
nieco obojętnie, ale mimo wszystko jednak stanowczo, wszystko jednak, każde jego słowo i gest przepełnione były zwykle beztroską, całkowitym brakiem zmartwień. Albo naprawdę tak było, albo świetnie to udawał. A Radek\3k Bił od niego jakiś chłód, surowy i nieprzyjemny, mimo paru serdecznych zachować z jego strony, których byłem świadkiem, zdawał się wyprany z emocji i beznamiętny -- czy lata rozpracowywania organizacji płatnych zabójców doprowadziły go do chłodnego patrzenia na życie, czy coś złego spotkało go już wcześniej?\\
Odwinąłem rękaw i spojrzałem na zegarek. Należałoby już\3k
\sx Złaźcie. -- rozkazał Jonathan, zasłaniając twarz.
\xx I coście tam na przykład robili? -- spytał po trzech minutach, siedząc skulonym przy drzewie, czekając na nadjeżdżającego Adama. Mark, przyczajony trzy metry od niego, na drugim krańcu wąskiej szosy, także przy drzewie, zamyślił się na chwilę przed odpowiedzią.
\xx Nie mam na to dzisiaj ochoty\3k -- jęknął niedbale.
\xx Kochanie, boli mnie głowa! -- zawył w prześmiewczy sposób Jonathan. -- Co, ja mam zacząć? No?!
\xx Dalej chcesz się w to bawić?
\xx Jasne! W końcu jakaś wymiana doświadczeń -- ci tutaj\3k -- Masterton wskazał palcem pozostałych stalkerów. -- Nie palą się zbytnio do rozmów na ten temat.
\qd
Mikołaj chrząknął, jakby się czymś dławiąc.
\sx Po ostatnim prawie puściłem pawia. -- zwrócił się do Marka. -- To z Johnsonem\3k
\xx Johnsonem?! To twój najlepszy przykład?
\xx Póki co\3k
\xx Minęło mało czasu, rozumiem. Ale zastanów się -- co uważasz za mój najgorszy wybryk?
\xx I po kiego ci to?\3k
\xx Żeby go przekonać. -- tu Jonathan skinął głową na Marka. -- No, dawaj! -- zachęcił. -- Kto? Wy też! -- krzyknął.
\qd
Zapanowała chwila ciszy, podczas której przetrawiłem ostatnie słowa Mastertona. Nie wierzę -- ten wariat chce, żeby w wręcz demokratyczny sposób osądzono o nim, który z jego dotychczasowych czynów był najpodlejszy, jakby miał z tego satysfakcję, radość. Mógł też się po prostu tym nie przejmować -- przyjąć, że ktoś musi wykonywać brudną robotę, a samemu się temu bez odrobiny sprzeciwu podporządkować. Z takim podejściem mógł by używać brudów ze swojej przeszłości jako argumentów na rozmowę i przechwałki w byle spelunie -- ale robił to i bez tego. Był po prostu nienormalny -- w jego życiu stało się coś, co kompletnie go złamało, nieważne, czy do tej pory był dobrym człowiekiem, czy zwykłą mendą -- spotkało go coś, po czym załamałby się każdy, bez wyjątku. Pewnego dnia coś lub ktoś po prostu go zniszczyło.
\sx Tamten podrzędny szpicel z tej organizacji, K\3k
\qd
Wszyscy prócz mnie nagle zawołali głębokim ,,Aaa!'', jakby właśnie przypomnieli sobie odpowiedź na ostateczne pytanie w jakimś teleturnieju.
\sx Przegiąłeś. -- rzucił Mikołaj. -- Nigdy ci tego nie zapomnę\3k
\xx A ja twojej miny. -- Leon zaniósł się śmiechem. -- Z każdą kroplą coraz bardziej opadała ci szczena, aż dziw, że go nie zabiłeś! Miałeś takie oczy\3k
\xx Dobra, starczy!
\xx To był jeden z pierwszych, jeszcze się nie przyzwyczaiłem\3k -- powiedział Izaak. -- Od haftowania przez dwa dni paliło mnie w przełyku!
\xx To co on w końcu zrobił? -- spytałem, niecierpliwiąc się.
\xx Kroił tym swoim nożykiem milimetr po milimetrze, sumiennie polewając wszystko najdroższą whisky Mikołaja.
\qd
Tamten płakał z bólu, a ten tu płakał za każdym rublem wydanym na ten flakon.
\sx Skurwiel! Jakby nie mógł wziąć rozpuszczalnika! -- krzyknął Mikołaj. -- Ile ci jeszcze zostało do oddania, cwaniaku?
\xx Sto.
\xx I kiedy masz zamiar mi je oddać?
\xx Bo ja wiem\3k -- Jonathan spojrzał na Marka wzrokiem biedaka widzącego wybawcę, dobroczyńcę. -- Ty mi pożyczysz, przyjacielu! W końcu sto w tę czy we w tę nie zrobi ci zbytniej różnicy, prawda?
\xx Nie robi\3k Kiedy wrócimy to dam ci je jako zapłata za wzbogacenie mojego życia o nowe, ciekawe doświadczenia, pasuje? -- Mark wychylił się niemrawie zza drzewa. -- Jedzie.
\qd
Nadstawiłem uważnie uszu.\\
Przez pierwsze pięć sekund nie słyszałem nic -- po nich dobiegł mnie cichu pomruk silnika spalinowego, połączony z odgłosem ścierających się z asfaltem opon. Adam jechał powoli -- góra czterdzieści kilometrów na godzinę, zaczął się widocznie przymierzać do ostrego zakrętu, przy którym właśnie byłem schowany z resztą stalkerów. W sumie to z trudnością nazywałem ich stalkerami\3k\\
Ostatnia rzecz, której bym się spodziewał.\\
Kiedy samochód był już jakieś trzydzieści metrów przede mną i s\3k agentami, z oddali, z bardzo daleka, rozległ się potężny, niemal basowy huk. Chwilę później usłyszałem odgłos przebijanej blachy, szkła i\3k coś, co wyglądało, a raczej brzmiało jak pocisk nagle umilkło -- widocznie utkwił w ciele Adama, na czym skończył swój przebijający wszystko po drodze lot. Nie milkł za to odgłos silnika i kół auta -- zdawało się wręcz, że przyspieszył. Coś targnęło drzewem, do którego byłem przyparty -- Adam wbił się w nie po skręceniu z dotychczasowego kursu -- zanim przetoczyłem się w tył z zasłoniętymi uszyma nerwowo zacisnąłem zęby, słysząc przeraźliwy dźwięk gniecionej blachy i rozlatujących się do reszty szyb, których odłamki poszybowały dookoła.\\
Wciąż lekko oszołomiony, przewróciłem się na brzuch, nie miałem siły, by wstać.\\
Stanął na de mną Leon, krzycząc coś wściekle dookoła. Wyciągnął rękę w moją stronę, by pomóc mi wstać. Podniosłem swoją.\\
Drugi, ogłuszający huk wystrzału.\\
Leon wygiął się do przodu, a w jego brzuchu rozkwitła potworna, ziejąca, krwawa dziura. Poczułem kilka kropel krwi na twarzy, potem ciężar upadającego bezwładnie Leona -- momentalnie oślepłem, albo nieświadomie zacisnąłem z całej siły oczy, chcąc uchronić się przed paskudnym widokiem.\\ Straciłem kontakt z rzeczywistością -- ciemność, którą widziałem, zaczęła mienić się niczym kalejdoskop, po czym znowu poczerniała, wzbudzając we mnie potworną falę mdłości. Plamki, nieprzyjemny pisk, stłumione odgłosy czyichś rąk próbujących zdjąć ze mnie zwłoki. Zemdlałem, nie mogąc wytrzymać narastających drgawek.

Nie czuł tego od kilkunastu lat.\\
Przez ten czas doświadczył wiele -- upokorzenia, ból, poniżenia, rozczarowania, jednym słowem, życie prostego człowieka ,,przefiltrowane'' przez Jonathana, co dawało serię zdarzeń, od których mało kto nie chciałby ze sobą skończyć. On jednak się do nich przyzwyczaił; nauczył z nimi żyć, przyjmując, że kogoś Bóg musi sobie wybrać na ofiarnego kozła, który spełni jego chore poczucie humoru, czyniąc to obrazem swych cierpień. Czego nie czuł do tak długiego czasu?\\
Kompletnej, aż odbierającej resztki chęci do życia, bezsilności.\\
Pierwsze, co zrobił, słysząc wystrzał, to obejrzał się w stronę reszty -- czy nikt z nich nie upadł, chwytając się kurczowo w miejsce trafienia, czy nikomu z nich nie urwało czy zmiotło głowy -- upewniał się, że jego jedyni najbliżsi są cali i zdrowi. Potem dopiero wykonywał ,,mechaniczną'' resztę.\\
Rozglądając się dookoła, nad horyzont, na próżno szukając strzelca, odbezpieczając za to granat dymny, usłyszał drugi odległy huk. Znowu obejrzał się w stronę towarzyszy -- tym razem jednak widząc to, czego zawsze najbardziej się obawiał.\\
W ciągu sekundy, podczas której obracał głowę, czas dla niego zwolnił, a raczej zrobił to specjalnie dla uczucia niepewności -- dał mu chwilę na wzrośnięcie i osiągnięcie poziomu tak wielkiego, że niemal paraliżującego myśli. Jonathan podczas jednej zaledwie sekundy poczuł niepewność i obawę przerastającą uczucia podczas dowiadywania się o nowotworze.\\
Widząc rozkwitającą, szkarłatną dziurę w plecach stojącego nad oszołomionym Michałem Leona kompletnie się załamał. Zapomniał o trzymanej broni, która wyleciała mu z rąk, o dymiącym, wbitym w drzewo aucie z nieżywym Adamem we wnętrzu, o reszcie przyjaciół i strzelcu, który właśnie uśmiercił jednego z nich i wciąż mógł trafić kogoś jeszcze. Mając wciąż w głowie, jakby na złość powtarzający się widok ranionego Leona, pognał w jego stronę, czując, że zaraz kompletnie się rozklei. Tak, wyczuł wewnętrzny konflikt, nabierając ochoty na płacz -- po 86 roku stwierdził, że nic już go nie zmartwi, ,,ruszy'' na tyle, by doprowadzić go do łez. Widocznie się mylił.
Runął na ziemię, dopadając rękoma zakrwawione ciało, po czym po panicznej szamotaninie zdołał przewrócić je na plecy.\\
Palce wykrzywiły mu się do granic możliwości, ręce podjechały do góry, wrzynając się łokciami pod pachy, jakby wykonywały w ten sposób błagalny gest skierowany do samych niebios. Po tym jakże wyczerpującym akcie bezsilności ręce Mastertona opadły na ramionach Leona, przygniatając go do ziemi. Zawył głośno, zdzierając gardło, wypełniając je palącym bólem. Zwiesił bezwładnie szyję, czując narastający, dławiący ścisk w gardle, słony smak łez i niewyobrażalny żal. Obraz nieżyjącego Leona zniekształciły łzy -- był rozmazany, niewyraźny, przesuwał się w różne strony i rozmazywał niczym podczas wizji narkomana.\\
Granat wybuchnął smolistym obłokiem, lecz Jonathan nie zwrócił na niego najmniejszej uwagi.\\ Wkrótce dym dotarł w jego stronę i całkiem go zasłonił, pozostawiając go Leona i siebie samym sobie, okrywając wstyd, który właśnie czuł.
%
\podro{Rok 2000}
%
Leon stał oparty o drewniany słupek przed obozowym pubem i, paląc, rozmawiał z Mickiem. Z nim zawsze rozmawiało mu się dobrze -- kiedykolwiek by na niego nie trafił, z paru wzajemnych uwag, każdego razu, bez wyjątku, rodziła się długa konwersacja na wszystkie możliwe tematy. Było tak też tego, niewątpliwie burzliwego dnia -- trzydziestego grudnia, niemal tuż przed sylwestrem, kiedy cała Zona żyła jeszcze wydarzeniami sprzed paru dni -- zorganizowaną przez Igora masakrą w Boże Narodzenie, w której niechlubny udział brał Radek.\\
To jego oczekiwał Leon, dlatego też odnalazł wśród zatłoczonego dziś placu Micka, by móc zamienić z nim parę słów, ot, dla zabicia czasu. Kilka razy, kiedy jego rozmówca rozgadywał się na dobre, nie słuchał go, by całkowicie skupić się na Radku -- Leon zdecydowanie nie należał do osób z tak dobrze rozwiniętą podzielną uwagą. Udając, że słucha, potakując po omacku i kiwając na oślep głową, myślał o Radku i jego niedawnym ,,dokonaniu''. Znowu się na niego wściekł.\\
Ledwie ponad tydzień temu morduje kilka osób, ma z tego powodu wyrzuty sumienia, a jednocześnie przeżywa fakt, że sprawiło mu to zwyczajną radość. Teraz, po upływie tego czasu, jak gdyby nigdy nic zdarza mu się kolejny wyskok -- wraz z trzema żołnierzami ostrzeliwuje granatnikami chatę znanego handlarza bronią, zaopatrującego stalkerów. Chatę, z jej właścicielem i czterema innymi, nieznanymi stalkerami. Wszyscy niegdyś współpracowali z wojskiem i zginęli prawdopodobnie za to, że się od nich odwrócili. Zwyczajne marno\3k\\
Leon wyrwał się z zamyślenia. Brama obozu szczeknęła sucho i rozchyliła się ze zgrzytem, ukazując wycieńczonego Radka, ubranego w długi płaszcz, z zawieszonym przez ramię cylindrowym granatnikiem. Ręce miał niemal bezwładnie, z wysiłku, spuszczone wzdłuż nóg. Tymi poruszał niemrawo, sztywno, jakby męczyły go ostre zakwasy. Minę miał, jakby się nie wyspał -- zmrużonymi oczyma ledwie omiatał grunt, po którym chodził, usta miał wydęte, twarz zmęczoną i pomarszczoną bardziej, niż zwykle.\\
Leon poczekał na niego -- aż zbliży się w stronę baru. Kiedy zorientował się, że ten ominął go szerokim łukiem i właśnie mija słup pośrodku placu, odwrócił się gwałtownie przez plecy i zawołał go.\\
Na pewno go słyszał, a mimo to nie zwrócił na krzyk najmniejszej uwagi.\\
,,Sku*wiel'' -- pomyślał z pogardą Leon. Machnął na Micka przepraszająco ręką, po czym ruszył w stronę Radka, który był już parę metrów od swej kwatery. Przyspieszył kroku, w końcu przeszedł w trucht. Ostatnie dwa metry pokonał trzema skokami, dopadając ramienia Radka, który zdążył już wyciągnąć z kieszeni klucze.\\
Odwrócił się zlękniony, klnąc pod nosem -- zawieszony na plecach granatnik uderzył kolbą w ścianę budynku.
\sx Co ty\3k -- zdążył powiedzieć oburzony, nim Leon w mało bolesny fizycznie, za to upokarzający i pogardliwy sposób trzasnął go otwartą dłonią w skroń.
\xx Pojebało cię? -- krzyknął Leon? Przymierzał się do drugiego ,,kuksańca'', lecz Radek zadziwiająco szybko strącił jego dłoń, uderzył w ramię, doszło do niewielkiej, pełnej wyzwisk szamotaniny. -- Co ty wyczyniasz, człowieku?!
\xx Odpierdol się\3k -- syknął Radek, odwracając się w stronę drzwi. Leon brutalnie odwrócił go z powrotem, po czym przycisnął do ściany.
\xx Co się z tobą dzieje?! -- niemal krzyknął. -- Rzeź w święta, mija ledwie kilka dni, a ty znowu ciągniesz do tych gnoi? Po co?! Po co się pytam! Czy ty nad sobą, ku*wa, nie panujesz?! Musisz wykonywać z nimi brudną robotę, akurat ty?! Pytam się, po co?! Czujesz tak wielką potrzebę pozbawienia kogoś życia?! Musisz to robić akurat z nimi? Daje ci to większą radochę, popaprańcu? Że robisz to z kimś działającym nielegalnie, dla własnych interesów, niż dla dobra innych?
\xx Tymi metodami\3k -- Radek widocznie spuścił z tonu. Leon spodziewał się, że ten zaraz znowu się rozpłacze i przeżyje kolejne dwa dni w stanie ciężkiej depresji.
\xx Tak, nimi też można zrobić coś dobrego! A ta banda płatnych zabójców, którą rozpracowałeś? Nikt nie mówi o tym, że jednemu z nich odrąbałeś siekierą głowę dla przestrogi, wszyscy mówią o tym, że nikt nie zleca już innym zabójstw za marne dwadzieścia monet!
\qd
Radek spojrzał na Leona bezradnie, po czym znowu się rozpłakał, co ten drugi skwitował machnięciem ręką i wyrazem mówiącym ,,Skończ już''. Widząc, że nie daje to żadnego efektu, kontynuował.
\sx Weź się w garść, rozumiesz? Mam cię przywiązać na smyczy przybitej do kołka?! Weź się za siebie, nie daj im się więcej namówić, rozumiesz? Coś w tobie siedzi, ale nie możesz się temu dać, postaw się, do cholery! Ej, słuchaj mnie. Słuchaj! Jeśli nie poradzisz sobie z samym sobą, każdy inny, nawet najmarniejszy wyrostek kiedyś cię dorwie. Nie bierz się za nic, nawet za byle pierdołę, jeśli najpierw nie zajmiesz się samym sobą, rozumiesz?
\xx Rzucasz hasłami\3k -- Radek zbył słowa machnięciem ręką, znowuż robiąc krok w stronę drzwi. Leon złapał go za ramię, tym razem mocniej, zaciskając na nim palce do granic możliwości, niemal je paraliżując, odwrócił i przycisnął do ściany.
\xx Nabawiłeś się, ku*wa, agorafobii, że tak ci się spieszy do środka?! Chcesz się wypłakać w poduszkę? Psujesz nam wizerunek, rozumiesz? Wszyscy są ci wdzięczni za tamtych gnoi, ale powoli przeginasz! Barry\3k
\xx Co Barry?! Co Barry?! Chce się mnie pozbyć?! Proszę bardzo, ja to wszystko dookoła pier*olę, i ciebie też! -- krzycząc, Radek ponownie strącił rękę ze swojego ramienia. Splunął Leonowi pod buty, odbierając mu resztki zapału, po czym w końcu otworzył drzwi szeroko i wszedł do środka, trzaskając głośno.
\qd
Leon włożył ręce do kieszeni, bezradnie wsłuchiwał się w szczęk zamka, zwiesił luźno głowę i zamknął oczy.
\sx Niedobrze\3k -- mruknął. -- Spojrzał na zamknięte drzwi kwatery Radka. -- Bardzo niedobrze.
\qd
Odwrócił się na pięcie i ruszył w stronę baru. Mijał rozmawiających dookoła stalkerów i agentów, zarówno tych regularnie udających się w różne części Zony jak i członków tutejszej administracji -- osób zarządzających i zajmujących się magazynami, wydatkami przy zamówieniach broni i sprzętu i podobnymi czynnościami związanymi z chociażby ,,papierkową robotą''.\\
Nie zwracając na unoszone w powitalnym geście i uprzejme skinienia głową, Leon wpadł w obłoki dymu, alkoholowy gwar i rozmawiające ze sobą grupki ludzi. Stanął na palcach, szukając pożądanego stolika.
\sx Ej! -- krzyknął ktoś z lewego kąta pomieszczenia. Leon spojrzał się w tamtą stronę.
\qd
Jonathan energicznym ruchem dłoni przywoływał go do siebie, mówiąc bezgłośnie ,,Dawaj, dawaj!''. Ucieszył się na jego widok, przecisnął ominął kilka lepkich stolików i zajętych krzeseł, po czym zajął miejsce obok Mastertona. Od razu, Leon nie wiedział jak, wyczuł w nim niepokój, niepewność, jakiś smutek, zatroskanie spowodowane niedawną rozmową z Radkiem. Być może wyczytał to z jego niemrawego kroku, zmartwionego spojrzenia, czy też przygnębionego wyrazu twarzy -- jakkolwiek to zrobił, Leon czuł się tym mocno zaskoczony -- wydawało mu się, że dobrze ukrywa targające nim uczucia.
\sx Co cię trapi? -- zapytał, bawiąc się kieliszkiem z nalaną w nim wódką.
\xx Radek\3k -- mruknął Leon, siadając przy stole. -- Źle z nim.
\xx Jak bardzo?
\xx Bardzo. Kompletnie nad sobą nie panuje, do jasnej cholery, idzie na akcję w pełni tego świadom, potem ryczy i żałuje, do tego ledwie tydzień po tamtym w święta. Tragedia\3k
\xx Trzeba by go podreperować. -- zaproponował Jonathan.
\xx Trzeba by. I to ostro. Jakieś propozycje?
\qd
Jonathan zamyślił się, opróżniając kolejny kieliszek. Pomasował czoło, wpatrując się tępo w stolik. Czknął.
%
\podro{Rok 2001}
%
Zatrzasnął drzwiczki, po czym znowu pociągnął ostro nosem. Spojrzał się przez lewę ramię. Jonathan klęczał ciągle w tym samym miejscu, wokół rozlanej przez Leona plamy krwi. Nieruchomo, choć pewnie nogi od dawna mu zdrętwiały. Mikołaj przeniósł spojrzenie na Izaaka, który tylko wzruszył bezradnie ramionami. Jego wyraz twarzy mówił ,,Ja zająłem się wtedy Radkiem, to ty bierz teraz Jonathana.''.
\sx Jona\3k -- zaczął Mikołaj, lecz Masterton przerwał mu, unosząc dłoń, wykonując tym samym pierwszy ruch od ponad paru minut.
\xx Jedźcie. -- powiedział zachrypniętym od krzyku i płaczu głosem. Opuścił rękę z powrotem na kolano. -- Wrócę sam.
\qd
Mikołaj z trudem umieścił kluczyk w stacyjce, z wielkimi trudnościami odpalił silnik, zwolnił hamulec i bieg, nacisnął lekko sprzęgło. Zanim jednak ruszył z miejsca, usłyszał energicznie pukanie w szybę. Jonathan najwyraźniej czegoś chciał. Po opuszczeniu szyby na kilka centymetrów rzekł krótko:
\sx Łopata.
\xx W bagażniku. -- mruknął Mikołaj, schylając się do otwierającego bagażnik przycisku.
\qd
Z tyłu zgrzytnął zwalniany zamek, chwilę później Jonathan obszedł samochód i lewą ręką chwycił wielki szpadel i postawił obok siebie, po czym prawą ręką przymknął zatrzasnął klapę bagażnika. Odwrócił się przez plecy i ruszył przed siebie, ciągnąc łopatę za sobą, jakby nie miał siły, by ją unieść.\\
Wszyscy siedzący w aucie na moment odwrócili się przez ramię, by móc oprowadzić Mastertona wzrokiem. Ten po dojściu do dymiącego auta Adama odwrócił się nagle, zmierzył samochód Mikołaja rozdrażnionym spojrzeniem, uniósł prawą rękę, upuszczając tym samym łopatę, po czym machnął nią niechlujnie, każąc tym samym Mikołajowi odjechać.
\sx Nie dobijajmy go. -- szepnął obecny kierowca, naciskając pedał gazu.
\xx Chyba nie powinniśmy go tak zostawiać. -- zmartwił się Izaak.
\xx Poradzi sobie\3k -- zapewnił Mikołaj.
\qd
\end{document}
