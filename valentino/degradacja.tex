\documentclass[../MAIN.tex]{subfiles}
\begin{document}
\ro{1}
%\\
Zbudził mnie ogłuszający huk.\\
Półprzytomny poderwałem się z materaca i jednym susem dopadłem leżącego pod kaloryferem rewolweru.\\
Czając się pod oknem próbowałem ustalić, co spowodowało ów tępy grzmot, który wyrwał mnie ze snu. Początkowo myślałem, że to bomba lub granat, z czasem jednak, w miarę powracania z otępienia do pełnej sprawności wszystkich zmysłów, byłem niemal pewny, że był to huk anomalii. W przypadku zaatakowania obozu czy chociażby rozlegnięcia się w dalekim sąsiedztwie pojedynczego strzału natychmiast rozległby się w nim alarm.\\
To nie mógł być wybuch granatu\3k\\
Podświadomość dopuściła do mnie jednak jedną potworną wizję -- a jeśli była to dywersja? Co, jeśli strażnicy magazynu leżą z poderżniętymi gardłami, a mający przed chwilą miejsce wybuch miał jedynie rozpocząć serię eksplozji podłożonych przez skrytobójców ładunków, które w konsekwencji zmiotłyby z powierzchni ziemi cały obóz?\\
Jeśli jeden jest niedaleko, dom się zawali, a na mnie runie gruz -- czeka mnie więc powolna śmierć w męczarniach\3k\\
Uderzając pięścią w ścianę, co było pewnego rodzaju zrywem, wyrwałem się z ciągu czarnych, absurdalnych myśli o swoim losie. Traciłem nad sobą kontrolę -- jeszcze tydzień temu nawet nie przyjąłbym się jedną silnie negatywną myślą, teraz, co gorsza nieświadomie, pogrążałem się w niemalże trans polegający na wyobrażaniu sobie siebie cierpiącego okropne katusze kończące się wybawczą, bo uwalniającą od bólu śmiercią. Tracąc nad sobą kontrolę, wpadałem w wir samo upodlenia i autodestrukcji.\\
Traciłem wolę przetrwania.\\
Sama egzystencja przestała być dla mnie specjalnie pociągająca i sensowna -- to zaś, w połączeniu z ciągłym stresem wiążącym się z przebywaniem w miejscu takim jak Zona, powoli wpędzało mnie w obłęd.\\
-Ludzie! Zbierajcie szmal!!! -- za oknem rozległ się radosny, dziki wrzask.\\
Przestałem myśleć o moich ostatnich niepokojących zachowaniach, notując sobie jednocześnie w myślach, bym później zajął się tym problemem. Choć blisko było mi do całkowitej obojętności wobec swojego losu, jeszcze zależało mi na tym, żeby żyć.\\
Jeszcze się nie poddałem.\\
Jeszcze nie.\\
Lewy bark, którym przywarłem do ściany, zdążył mi już zdrętwieć, wyprostowanie się więc sprawiło mi nieco kłopotów. Kiedy w końcu to zrobiłem, zabezpieczyłem rewolwer, wciąż jednak trzymałem go w ręku.\\
Stojąc plecami do okna, zauważyłem cienie tańczące na wpadających przez nie promieniach słońca. Miały drobny, postrzępiony kształt -- przypominały spopielone gazety lub ich pourywane strony, unoszone chwilowo przez wiatr, po czym swobodnie opadające w dół, niczym spadające z drzew liście.\\
Odwróciłem się o sto osiemdziesiąt stopni i natychmiast pożałowałem tej decyzji.\\
Szybę znaczyło kilkanaście małych kropel krwi, nad którymi górował szkarłatny, przypominający swoim kształtem ranę ciętą, rozprysk. Krwawe drobiny zmieniały co chwila swoją barwę, będąc zasłanianymi przez migoczące cienie.\\
Cienie prawdziwego deszczu, niemalże wiszącej w powietrzu, chmury postrzępionych banknotów.\\
Dziesiątki, nie, setki banknotów o identycznym nominale pięciuset hrywien wirowały w górze -- o zwęglonych bokach, zakrwawionych powierzchniach, niektóre stanowiące jedynie połowę, czasem nawet nie ćwierć, całego banknotu. Nieliczne w całości, z błogim spokojem, opadały na ziemię, inne zaś zostały zmielone w drobne kawałeczki, niektóre nawet w pył. Rój pieniędzy, który zdawał się brać udział w góry zaplanowanym układzie, majestatycznie przesłaniał świecące mocno słońce.\\
Widok ten był przerażający i niesamowity -- przerażała ilość pieniędzy, jaka uległa zmarnowaniu, ale i ta część, która nie uległa zniszczeniu i była na wyciągnięcie ręki. Mogłem otworzyć okno i spróbować wyłowić niektóre z banknotów wirujących w powietrzu, korzystając z zamieszania, które miało zaraz nastąpić, z tłumi stalkerów, który zaraz zbiegnie się przy chmarze pieniędzy.\\
W jakby centrum owego niezwykłego układu znajdował się cały jego autor, sprawca -- leżący na boku, nieżyjący już stalker z urwanymi przy samym pasie kończynami, które przerobione przez anomalię w krwawą miazgę, leżały metr dalej. Niedoszły rabuś oburącz kurczowo ściskał rozerwaną w pół skórzaną torbę, z której niczym wnętrzności wypatroszonego zwierzęcia, wysypało się jeszcze całkiem sporo pieniędzy. Te, co dziwne, miały inny nominał niż wirujące nad trupem hrywny -- poznałem wśród nich między innymi euro, ruble, polskie złote i dolary.\\
Kolejny złodziej, niedoszły sprawca udanego rabunku na skarbcu frakcji.\\
Uchyliłem okno i wyjrzałem przez nie w lewo, gdzie znajdowało się centrum obozu.\\
Tłum stalkerów z prawdziwym obłędem w oczach, dziką chęcią łatwego zysku, rzucenia się niczym zwierzęta, na pieniądze, nacierał w moją stronę. Żądza ta widocznie przyćmiła ich umysły na fakt, iż złodziej wpadł w anomalię, która urwała mu nogi. Nie obchodziło ich, że sami mogą w nią wdepnąć i podzielić los martwego stalkera.\\
Łatwo dostępne pieniądze, sumy, które dla zwykłych mieszkańców Zony mogły pozwolić na przekupienie granicznych strażników i jej opuszczenie, pieniądze, na których nawet nie jedną dziesiątą część sami muszą pracować, ryzykując życie, były ważniejsze. Desperacja osiągnęła szczytowy poziom -- każdy ze stalkerów wolał zostać bezboleśnie rozerwany na strzępy przez anomalię i z tą świadomością zaryzykować, chapnąć trochę grosza, niż później żałować zaniechania, braku podjęcia chociażby próby odwrócenia swojego losu.\\
Kiedy tłum znalazł się około dziesięciu metrów przed pieniędzmi, z lewej nadbiegło kilkunastu wojskowych, którzy oficjalnie stanowili jedno z ramion zbrojnych frakcji. Kilku z nich stanęło obok, pod ścianą jednego z baraków, czterech z liczącego kilkunastu ludzi oddziału, wkroczyła w chmurę pieniędzy, która zdawała się zwisnąć w powietrzu, niczym pszczelarze w rój owadów. Stojący przy rogu baraku żołnierz rzucił mi wyzywające spojrzenie, gestem ręki pokazując jednocześnie, żebym zamknął okno.\\
Natychmiast usłuchałem -- zamknąłem okno wolną ręką tak gwałtownie, że jego szyba aż zadygotała i przez chwilę byłem pewien, że w końcu pęknie. Mimo, iż okno było zamknięte, miałem doskonały obraz całej sytuacji -- tłum stalkerów i wojskowych dzieliła bardzo niewielka odległość.\\
Jeden z ubezpieczającej ciało złodzieja żołnierzy uniósł karabin w górę i oddał z niego krótką serię.\\
Tłum stanął jak wryty. W ich oczach, zaślepionych chęcią łatwego zysku, na chwilę pojawił się strach.\\
Około dziesięciu wojskowych, tych stojących pod barakiem, wyszło naprzód, swoją postawą i zachowaniem jednoznacznie dając do zrozumienia tłumowi, że nie zawaha się go wystrzelać -- każdy z żołnierzy miał teraz za zadanie dostarczyć z powrotem do sejfu jak największą sumę, która ostała się ze zrabowanych pieniędzy.\\
Trwała próba sił -- tłum mierzył się spojrzeniem z oddziałem wojska; niczym biorący udział w pojedynku gladiator czekał na pierwszy ruch swego rywala.\\
Ku mojej uciesze, tłum zrezygnował -- wizja, iż zostanie wybity w pień przez wojsko była dużo okrutniejsza w swojej prostocie niż ewentualność wpadnięcia w anomalię i przemówiła im do rozsądku. Zrezygnowany, z wyrazem gorzkiego rozczarowania, powoli zaczął się rozchodzić.\\
Nagle jeden, jedyny stalker odwrócił się na pięcie i sprintem ruszył w stronę okaleczonych zwłok. Były niemal całkowicie przykryte przez banknoty, które zdążyły już co do ostatniego opaść w dół.\\
Dowodzący oddziałem zorientował się, co jest w zamiarze stalkera. Wyjął z kabury wielki, czarny pistolet, wycelował nim przed siebie i jednym, nieludzko celnym strzałem uśmiercił zmierzającego ku zrabowanym pieniądzom mężczyznę. Ten padł głucho na ziemię, huk wystrzału zaś zwrócił uwagę tłumu.\\
Mierzył wojskowych morderczym, wyzywającym od największych oprawców spojrzeniem. Żołnierze, całkowicie się tym nie przejmując, szybko ustalili położenie anomalii, po czym przy wzajemnej asyście w obserwowaniu zachowania tłumu, zebrali pieniądze do foliowego worka. Pozostawili po sobie jedynie kilkanaście, nic nie wartych strzępów -- szanse, że wśród całego tego bałaganu znajdowały się całe banknoty, była nikła. Dowódca oddziału wydał rozkaz, po którym powoli, wraz ze swoimi podkomendnymi, oddalił się w stronę kwatery dowodzenia frakcji.\\
Noszony przez jednego z żołnierzy worek pieniędzy skupiał na sobie uwagę zgromadzonych stalkerów. Ostatecznie żaden z nich nie podjął jednak próby na jego odbicie -- każdy, co do jednego, pozostawiając jednocześnie pasożytom ciało swojego kolegi, ze smutno spuszczoną głową poszedł w swoją stronę.\\
Cały spocony i zszokowany, z rewolwerem ślizgającym się w mokrej dłoni, usiadłem na materacu.\\
Po raz pierwszy w całym moim stalkerskim życiu, w ciągu dziesięciu lat przebywania w Zonie, zostałem zniesmaczony jej okropieństwem. Zdałem sobie po raz pierwszy sprawę, że tak być nie może, że coś jest tu nie tak.\\
Przede wszystkim jednak poczułem, że nie jest to miejsce dla mnie.\\
Muszę się stąd wydostać.
%
\ro{2}
%
\sx On biję cię na głowę. Ty jesteś na prawie samym końcu łańcucha ewolucji, on osiągnął niemal szczytu. Ty byś granatnikiem przeciwpancernym szczura nie zabił, on podrzynał gardła rozwścieczonym, wygłodniałym psom! Ty\3k 
\qd
Adam urwał, czknął, po czym zwalił się pod stół. Towarzystwo w barze rzuciło na niego spojrzeniem, po czym mrucząc coś pod nosem o jego złych alkoholowych nawykach, wróciło do prowadzonych rozmów.
%
\sx Podnieść go? -- spytał mnie Mikołaj.
\xx Nie\3k -- rzuciłem. -- Niech leży.
\xx Często się upija?
\qd
Sięgnąłem pamięcią wstecz.
%
\sx Odkąd go okradli, z dwa razy na tydzień.
\xx A czy\3k -- Leon zdawał się ugryźć się w język. -- A co mnie obchodzi jakiś pijak? -- rzekł gorzko, zapowiadając zmianę tematu. -- Zajmijmy się czymś naprawdę ważnym.\\
\xx A więc\3k -- odłożyłem kieliszek na bok. Ta rozmowa wymagała pełnej trzeźwości umysłu.
\qd
%
Mikołaj postawił na stoliku wielką, skórzaną teczkę, odpiął oba jej zatrzaski i zaczął szukać czegoś wśród mnóstwa teczek, aktówek i luźno zbitych ze sobą dokumentów.\\
Obserwowałem go przez chwilę i nagle zdałem sobie sprawę z tego, co przed zapadnięciem w pijacki sen, mówił Adam. Mikołaj zaiste, bił mnie na głowę pod każdym względem.\\
Będąc stalkerem przed dziesięć lat czułem się prawdziwym weteranem Zony -- myślałem, że niewielu wśród ogólnej społeczności Strefy cieszy się takim uznaniem, jak ja; ludzie mówili o mnie, o moich wyczynach, często niezwykłych, zwykle jednak chwalono mnie za to, jak radziłem sobie w tym „państwie w państwie”.\\
Podziwiali moją wytrwałość i wolę walki -- fakt, iż wybierałem się w niebezpieczne zakątki nie raz dwukrotnie w ciągu dnia, często wracając z niczym, tracąc zaś bardzo wiele. Niektóre z takich sytuacji trwale utkwiło mi w psychice -- przykładowo, wypad do dzielnicy przemysłowej dwa lata temu, podczas którego miałem zamiar zapolować na mutanta. Nim jakiegokolwiek znalazłem, zostałem otoczony przez bandytów, którzy okradli mnie z absolutnie wszystkiego, poważnie pokiereszowali, kazali mi rozebrać się do naga, by w końcu pozbawić mnie przytomności i zostawić w jednym z napromieniowanych hangarów.\\
Nie wiem jakim cudem, nie zauważyli jednak szpetnego symbolu zdobiącego moje prawe ramię. Jeśli tak by się stało, prawdopodobnie ze wściekłości zatłukliby mnie na śmierć.\\
Zbliżała się zima, było zimno i wietrznie -- kiedy się ocknąłem, miałem wrażenie że skuł mnie lód -- byłem sztywny niczym głaz, bezwładny i odrętwiały.\\
Ten jeden jedyny raz byłem całkowicie bezsilny i we władzy losu, który tego razu się do mnie uśmiechnął. Zostałem uratowany przez dwóch rodzonych braci, razem przebywających w Zonie od niedawna. Sposób, w jaki mnie potraktowali, na krótki czas przypomniał mi o istnieniu rzeczy zwanej dobrem. W tydzień doszedłem do siebie.\\
Przez następny tydzień kolejno lokalizowałem i eliminowałem napastników -- każdy z nich skończył z obciętą ręką i kulką w głowie.\\
To był naprawdę trudny miesiąc.\\
-- Już prawie to mam\3k\\
Ten wyczyn jednak, podobnie jak cały mój stalkerski dorobek, był niczym w porównaniu do rzeczy, jakich dokonał Mikołaj. Od wielu lat pracował dla rządu Ukrainy, był jednym z czołowych przedstawicieli ich głównego obozu w Zonie, w 2001 roku startego z powierzchni ziemi, później jednak odbudowanego, funkcjonującego do dzisiaj. Mikołaj był człowiekiem od wszystkiego -- strzelał i walczył wręcz nie gorzej niż członek elitarnej jednostki wojska, był wyjątkowo inteligentny, sprawny fizycznie, znał się na informatyce, materiałach wybuchowych, technikach przesłuchań -- swoją osobą mógł zastąpić tuzin ludzi o różnych specjalizacjach. Przy całej tej mądrości i umiejętnościach był jednocześnie wielkim stalkerem.\\
Nie była to wiedza dla każdego, ja znalazłem się jednak wśród tej wąskiej grupy, która znała skład pierwszej drużyny stalkerów, którzy zdołali przetrwać okropieństwa prypeckiego szpitala. W Zonie szpital był mitem -- wystarczyło rzucić o nim słowo, by milkły rozjuszone tłumy; miejsce to urosło do rangi niemal legendy, potwierdzenie zaś, że kilku stalkerów przebyła w nim kilka godzin czasu i zdołała się z niego wydostać, tchnęło nowe życie w stalkerską społeczność.\\
Nikt nie powtórzył jednak heroicznego wyczynu grupy, w której znajdował się między innymi Mikołaj i jeden z najgorszych w historii Zony zwyrodnialców, Łazarz Niemierow.\\
Patrząc na przeszukującego teczkę Mikołaja czułem się jak w obecności Jeżozwierza -- legendy, osoby kultowej, prawdziwej osobowości.\\
% 
-- Mam. -- powiedział z ulgą, kładąc na stole kilka odwróconych wierzchem do góry kartek. Zamknął teczkę i postawił ją obok siebie, po czym przeciągnął stos kartek ku sobie, odwrócił je w sposób, w jaki pokerzyści umieszczają w talii nowe karty.\\
-Łukasz Aryndejski\3k\\
Sposób, w jaki wypowiedział moje imię i nazwisko, przypomniał mi szkolne lata.
% 
\sx Nudzi mnie czytanie takich pierdoł jak twoja dotychczasowa kariera w Zonie. Nie każdy zasłuży sobie u nas na kartotekę jako stalkera -- ty widniejesz raczej w ogólnym spisie, nie mamy o tobie zbyt wielu informacji. Twoje akta osobiste, które niejako rozsławiłeś przez pięcioletnią odsiadkę, są w tej sytuacji bezużyteczne. Granica między człowiekiem przed, a po przebywaniu w Zonie jest zbyt płynna. Niektórzy po przeżyciu w niej kilku lat zmieniają się tak diametralnie, że nie można poznać ich nawet na dawnych zdjęciach. Pewne rzeczy się jednak nie zmieniają -- wiesz, co mam na myśli?
\qd
Pokręciłem przecząco głową, czując jednak niepokojący ścisk w dołku.\\
Mikołaj nachylił się do przodu i spojrzał mi prosto w oczy.
% 
\sx Zona jest anonimowa. Wielu próbuje załapać się na legalny wyjazd, wielu podszywało się pod sławnych stalkerów, żeby uciec z Zony tak, jak ty usiłujesz w tej chwili. Dowiadywało się przez to rzeczy, których nie powinny wiedzieć -- jeśli typowy stalker, który trafił do Zony przez swoją głupotę, podszyje się pod Łowcę i dowiaduje się przykładowo o działalności Gomeza, my, ośrodek rządowy, mamy przez to poważne problemy. Zweryfikuj się. Tatuaż.
\qd
Moje ciało przebiegł silny, niepokojący dreszcz.\\
-A jeśli ktoś go zobaczy?\\
Mikołaj wstał z krzesła i stanął po mojej prawej stronie, skutecznie zasłaniając mnie przed większością klientów baru.\\
-Pokazuj. Widzę w tobie dobry materiał na kogoś, kto miałby się zająć tą sprawą, poza tym słyszałem, że naprawdę ciągnie cię do domu i\3k\\
W gniewny i agresywny sposób zacząłem podwijać prawy rękaw swetra. Zatrzymałem się w zgięciu łokcia, rozejrzałem się dookoła chcąc się upewnić, że nikt nie zobaczy tatuażu. W końcu zdobyłem się na odwagę i podwinąłem sweter pod same ramię. Poczułem wstyd.\\
W skórę na moim prawym bicepsie, niczym wypalony przez hodowcę na byczym cielsku symbol, na stałe wryta była czarna swastyka.\\
Mikołaj spojrzał na nią z zaciekawieniem, sprawdził, czy nie jest tymczasowym malunkiem wykonanym na potrzeby mistyfikacji. W końcu dał mi znak, żebym zasłonił tatuaż, co uczyniłem z niewyobrażalną ulgą.\\
-Nieprzyjemną część mamy z głowy. -- odetchnął Mikołaj, zajmując swoje miejsce. -- Jak każdy stalker podejmowałeś się pewnie niejednej roboty. Zapytam wprost -- zabijałeś na zlecenie?\\
W głowie przemknęło mi kilka obrazów. Bardzo nieprzyjemnych i teraz, kiedy żywiłem obrzydzenie do Zony, wstydliwych. O tak, miałem bardzo wiele grzechów na sumieniu.\\
Ten jeden raz będę się musiał po części wyrzec siebie.\\
-Tak. -- odpowiedziałem.\\
Odpowiedział mi nieprzyjemnym, niepokojącym uśmiechem.\\
-Nic specjalnego, nic za bardzo trudnego, przynajmniej dla ciebie, a nie wolno ci w to wtajemniczyć nikogo, choćby miał załatwić ci broń. Działasz absolutnie sam.\\
Umilkł, po czym przybrał wyraz twarzy, jakby przypominał sobie ważne szczegóły, instrukcje, którymi chciałby się ze mną podzielić.\\
Ostatecznie nie odezwał się słowem -- zmielił kartkę z moimi danymi, schował ją do kieszeni, wstał, zasunął za sobą krzesło i podniósł teczkę.\\
Ruszył do wyjścia. Już miałem zamiar zagrodzić mu drogę i nawrzucać za jego dziwne zachowanie, Mikołaj jednak zatrzymał się tuż obok mnie, położył rękę na ramieniu, po czym nisko nachylił i szepnął do ucha:\\
-Za dwa dni o drugiej w nocy w szkole muzycznej. Wszystkich.
% 
\ro{3}
%
Po wyjściu z baru przeszył mnie wyjątkowo chłodny wiatr , tak mocny, że zgrzytnąłem zębami z zimna. W okolicy nie było śladu Mikołaja -- prawdopodobnie oddalił się z obozu samochodem , mógł też zajść do któregoś z pobliskich domów -- bardzo możliwe, że miał jeszcze inne sprawy do załatwienia.\\
Nie dał mi żadnych konkretów. Szedłem w ciemno -- znałem tylko datę i miejsce, o którym wiem bardzo niewiele, gdyż na przestrzeni lat spędzonych w Zonie starałem się unikać Prypeci. To miejsce mnie przerażało i naprawdę nigdy nie przyszedłby mi do głowy pomysł, by dokładnie przeszukać któryś z tamtejszych budynków. Radioaktywne lasy, podziemia, jaskinie -- wszystko to nie stanowiło dla mnie większej przeszkody, Prypeć jednak była dla mnie zbyt dużym wyzwaniem. Wewnętrzny opór nie pozwalał mi przebywać tam zbyt długo.\\
Teraz jednak nie miałem wyjścia. Moja determinacja, chęć wyrwania się z tego piekła była tak duża, że gotów byłbym nagi i bez żadnej broni udać się do szpitala.\\
„Piekła”\\
Zaskoczyłem sam siebie, używając tego określenia. Jeszcze nie tak dawno śmiałem się z ludzi, którzy określali Zonę takim mianem -- nazywałem ich tchórzami, ludźmi nie potrafiącymi się przystosować do otaczających ich realiów, nie potrafiącymi wziąć się w garść. Teraz, samemu używając tego określenia, poczułem silny wstyd, ale i ulgę. Choć moja wizja Zony od dzisiejszego poranka zaczęła się zmieniać i musiałem „dojrzeć” do jej nowego obrazu, ostatnie dziesięć lat życia wydało mi się pewnego rodzaju wegetacją. Cieszyłem się z wizji porzucenia stalkerskiego rzemiosła, opuszczenia Zony; czułem się jak skazaniec po długim, ciężkim wyroku -- na pewien czas przymuszony do zmiany swoich poglądów i zachowań, spuszczony na dno moralności, by przeżyć pewnego rodzaju szok i móc się od tego dna odbić tak wysoko, by znaleźć się na szczycie. To, co jeszcze niedawno uważałem za piękne, cała moja egzystencja w przeciągu ostatnich lat, począwszy od znalezienia się w Zonie, z której byłem dumny, teraz zaczęła mnie dołować, 
wręcz 
przyprawiała mnie o poczucie winy, moralnego kaca.\\
Jak mogłem zmarnować taki kawał życia?\\
Kiedy pewne etyczne i moralne bariery, których odnowę rozpocząłem od początku dzisiejszego dnia, zaczęły pękać i pozwoliły mi na tak diametralną zmianę swojego wnętrza? Który wniosek przyczynił się do zmiany mego nastawienia -- od kiedy zacząłem zabijać bez żadnych skrupułów, często w iście ohydny sposób i dla równie ohydnych celów?\\
Teraz, z upływem czasu, całkowicie straciłem w tym rachubę, w końcu uświadomiłem sobie, że nie na tym powinienem się skupić -- ważne jest, że ta paskudna zmiana zaczęła się odwracać, „odkręcać”, że wracam do poprzedniego stanu ducha.\\
Zabicie nieznanej mi jeszcze ilości osób w prypeckiej szkole muzycznej.\\
Jeszcze wczoraj powiedziałbym o tym „Kolejne zadanie, kolejna porcja satysfakcji z jego wykonania i kolejne wynagrodzenie, które pozwoli mi godziwie pożyć do czasu znalezienia następnego zlecenia”.\\
Dziś mówię „Ostatnie w życiu wyrzeknięcie się siebie i długo oczekiwany koniec tego koszmaru”.\\
Udałem się do północnej części obozu -- w schowanym tam pomiędzy dwoma warsztatami budynku znajdował się mój osobisty sejf, w którym trzymałem wszystko -- od broni poprzez artefakty i pieniądze, na osobistych pamiątkach skończywszy.\\
Zdjęcia.\\
Byłem pewny, że teraz, w trakcie następującej we mnie przemiany, będę patrzeć na nie z zupełnie innej perspektywy.
%
\ro{\Huge\texttt{4}}
%
W budynku było wyczuwalnie cieplej, niż na zewnątrz -- byłą to zasługa stojącego w rogo pokoju parowego grzejnika.\\
Zamknąłem drzwi i nie odpowiadając na przywitanie Aleksego, bez słowa skręciłem w prawy korytarz. Był wąski na niecałe dwa metry i ciągnął się na ponad dwadzieścia -- przejście dalej było zablokowane przez anomalię. Ściany miał pomalowane w kolorze zielono-białym, z podłogi zerwano wszystkie niegdyś zdobiące je płyty linoleum, na suficie zaś ciągnął się długi pas lamp, z których zapalona była obecnie jedna, ta najbliżej anomalii.\\
Mijałem umieszczone po lewej stronie korytarza drzwi -- każde oznaczone były innym numerem i prowadziły do odrębnego magazynu -- wszystkie z nich mogły zostać za odpowiednią opłatą wynajęte na dany okres czasu. Osoby, do których tymczasowo należały mogły korzystać z nich na wszelkie sposoby -- trzymać tam swój dobytek, broń, sejfy, sprzęty elektroniczne i tym podobne, a nawet korzystać z pomieszczenia w ramach „pokoju” -- przebywać w nim ile się chciało i nocować. Wielu nazywało magazyn Aleksego „drugą Polisią”.\\
Zatrzymałem się przy drzwiach opatrzonych numerem 10.\\
Budziły we mnie bardzo nieprzyjemne wspomnienia.\\
Klucz do nich, który znajdował się w tylnej kieszeni spodni, zdawał się nagle nieprzyjemnie ciążyć, jakby przybrał na wadze cały kilogram. Miałem również wrażenie, że bije od niego fala gorąca, która zaraz dosłownie mnie poparzy.\\
Jakimś cudem zdołałem wyrwać się z otępienia i uciec od „dziesiątki”.\\
Mimo chwiejnego kroku i zawrotów głowy dotarłem do mojego „apartamentu”, magazynu z numerem 15. Włożyłem rękę do kieszeni kurtki i zacząłem szukać odpowiedniego klucza spośród wielu, wypychających kieszeń po brzegi, śmieci. W końcu go znalazłem -- stary, lodowato zimny i błyszczący mimo wielu szpecących jego powierzchnię rys. Wepchnąłem go w zamek drzwi, przekręciłem dwa razy w prawo, wszedłem do środka, po czym rzuciłem go na materac.\\
Zamknąłem za sobą drzwi i zapaliłem światło, uderzając machinalnie w znajdujący się na wysokości klamki przełącznik.\\
Nic. W pomieszczeniu nadal było nieludzko ciemno -- z chwilą zamknięcia drzwi nie docierało do niego żadne wypełniające korytarz na zewnątrz światło.\\
Poirytowany uderzyłem w przełącznik jeszcze dwa razy. Nadal żadnego skutku.\\
Westchnąłem gniewnie, położyłem lewą dłoń na przycisku i zacząłem go non stop wciskać, w górę i w dół, z nadzieją, że żarówka w końcu zaskoczy.\\
Nie zaskoczyła.\\
Wściekły grzmotnąłem pięścią w przełącznik.\\
Żarówka huknęła jak przy krótkim spięciu, rozświetlając na ułamek sekundy pokój mdłym, niebieskim, elektrycznym blaskiem.\\
Zdołałem ujrzeć wnętrze pokoju i widok ten przeraził mnie tak bardzo, że zemdlałem.\\
Był pusty -- nie znajdowała się w nim żadna z moich rzeczy, nie było szafy, sejfu, śpiwora i materaca. Ściany pełne były zaschniętych, osobliwie błyszczących w świetle wielkiej iskry, krwawych skrzepów. Na każdej z nich wisiało krystalicznie czyste lustro, odbijające w swojej gładkiej tafli ten sam widok, ukazany z różnych perspektyw.\\
Dokładnie pośrodku trzech luster ze sufitu na grubym, trzeszczącym sznurze zwisały zwłoki -- gołe, poznaczone ranami ciało, odwrócone do mnie plecami. Nie zapamiętałem niczego poza wzrostem i kolorem skóry ciała oraz faktu, iż nieszczęśnik miał całkowicie łysą, naznaczoną guzami, głowę.\\
Z lustra naprzeciwko mnie spoglądało na mnie nienaturalnie białe, przekręcone w pionie w wyrazie agonii i bólu, oko.

Ocknąłem się.\\
Czułem wewnętrzny strach przed tym, żeby otworzyć oczy. Bałem się czegoś.\\
Lęk wzmógł się, kiedy zdałem sobie sprawę, że coś uciska mi szyję.\\
Wpijający się w skórę, gruby, cuchnący potem i skrzepłą krwią, sznur. Suchy, twardy jak z drewna bezlitośnie cisnął mi szyję.\\
Wstrząsnęły mną silne dreszcze. Stałem na czymś niestabilnym.\\
Zebrałem w sobie dość siły, by w końcu rozszerzyć powieki.\\
Byłem w śmieszne wąskim\3k pokoju? Nie mogłem nazwać tego pokojem -- pomieszczenie miało niecałe pół metra, przypominało raczej wnętrze rury, pionowego kanału. Panował w nim półmrok rozświetlany przez mdłą poświatę księżyca -- nie wiedziałem, jak wpadała do środka, prawdopodobnie jednak za moimi plecami znajdowało się okno lub spora dziura w murze.\\
Ściany były wilgotne i wiało od nich chłodem, sprawiały wrażenie niesamowitej wilgoci, która parowała z ich powierzchni i czyniła w środku zaduch porównywalny z wnętrzem łaźni parowej. Drobny pył zmieszany z kurzem wisiał w powietrzu, majestatycznie uwydatniany przez światło księżyca.\\
Te oświetliło również niewielki kołek, na którym stałem. Miał wysokość dziesięciu centymetrów i choć zrobiony z wyglądającego na solidne, drewna, nie wydawał się zbyt stabilny. Wręcz czułem w stopach, jak powoli próchnieje i zbliża się ku całkowitemu załamaniu pod moim ciężarem, skazując mnie na upokarzającą śmierć. Za wszelką cenę próbował bym oprzeć się końcami butów o podłogę, zapierałbym się rękoma i nogami o tak blisko położone mnie ściany, walcząc o przetrwanie.\\
W końcu jednak straciłbym siły i powoli się udusił.\\
Ręce. Nie skrępowano mi ich.\\
Dramatyzm sytuacji na kilka sekund wyostrzył mi wszystkie zmysły. Myślałem szybciej, niż kiedykolwiek. Kołek zaczął głośno trzeszczeć, początek mojej śmierci był bliski.\\
Po omacku wyrwałem z zapiętej na udzie kabury nóż i machnąłem nim w stronę sznura, z którego zwisałem.\\
Moją twarz oblała rzadka, oleista ciecz, cały jej strumień wylał się na mnie, jak z przechylonego gwałtownie wiadra.\\
Spadając w dół, znów czując, że tracę przytomność lub wchodzę w następną fazę traumy, o ile wszystko było to koszmarnym snem, ujrzałem drobny, biały kwadrat w górze. Kawałek nieba widoczny u wylotu tego wąskiego pomieszczenia.\\
Stały nad nim trzy osoby -- trzy owalne głowy były nisko pochylone w dół i najwidoczniej bacznie mnie obserwowały.\\
%
\ro{5}
%
Pianino.\\
Ktoś grał na pianinie.\\
W młodości interesowałem się muzyką klasyczną, pianino zaś było moim ulubionym instrumentem -- jeździłem po kraju na rozmaite koncerty, miałem półki pełne płyt z muzyką Chopina, Mozarta i innych mistrzów światowej sławy. Znałem się na tym, miałem na koncie nawet trzy artykuły pewnym specjalistycznym, zajmującym się muzyką poważną, piśmie.\\
Osoba, która zbudziła mnie uderzaniem w klawisze instrumentu, była niewątpliwie wielkim wirtuozem. Piękno gry, jaką tworzył, docierało do mnie stopniowo -- taką miałem „muzyczną duszę”, niejednokrotnie musiałem usłyszeć coś kilkanaście razy, by móc to docenić, tak było też tym razem -- muzyka stawała się piękniejsza z sekundy na sekundę, wprawiała mnie stopniowe w jakieś duchowe uniesienie.\\
Zdając sobie sprawę z tego, że leżę na podłodze, wciąż nie otwierając oczu, dźwignąłem się na lewym ramieniu.\\
Kiedy pianista wykonał pauzę, przetarłem oczy, otworzyłem je i rozejrzałem się dookoła.\\
Księżyc musiał być skryty za chmurami, gdyż znajdowałem się w absolutnej w ciemności -- czerń była nieprzenikniona aż do framug pokoju, z którego na korytarz, jak się okazało, wąski na jakieś trzy metry, wylewało się mdłe światło. Ogień -- w pokoju tym musiało się coś palić.\\
Miałem wrażenie, że to z tego pokoju rozgrywała się cudowna gra na pianinie.\\
Zdałem sobie sprawę, że jestem w szkole muzycznej. Nawet nie próbowałem dociec, jakim cudem się w niej znalazłem, jak się domyślałem, dokładnie o drugiej w nocy zaraz po tym, jak zemdlałem w wynajętym przeze mnie magazynie.\\
Coś we mnie pękło. Jak zahipnotyzowany czarodziejską melodią, wstałem i machinalnie sięgnąłem lewą ręką za plecy, wymacałem kolbę broni, później pas nośny -- chwilę później trzymałem skróconego SIG’a w rękach. Nie obchodziło mnie, jak znalazł się na moich plecach ani czy jest chociażby załadowany -- panowało we mnie jakieś przeświadczenie, potężna, wciąć nie odkryta przez moją świadomość pewność, która kierowała mnie niczym nadnaturalne moce kierowały ciałem opętanego.\\
A jeśli byłem opętany?\\
Stąpałem powoli, wbrew własnej woli -- czułem, że jedyne, nad czym mam teraz panowanie, to wczuwanie się w wypełniającą pachnący pleśnią i kurzem korytarz, muzykę. Zdawała się trzymać mnie w ostatnich ryzach zdrowych zmysłów, działając niczym ostatnia lina asekuracyjna podczas długiego upadku w dół. Gdybym nie mógł skupić się na pięknie tejże muzyki, tajemnicza siła, która właśnie mną kierowała pozbawiłaby mnie wszelakiej świadomości. Byłbym jak zombie.\\
Zbliżyłem się do przejścia w lewo i przywarłem doń, nasłuchując. Trzaski ognia, płonących w ich gałęzi i wyskakujących na zewnątrz iskier. Co dziwne, muzyka nie była głośniejsza, jakby miała swoje źródło gdzie indziej; mogłem się też wczuć w nią tak bardzo, że niektóre granice jej odczuwania uległy zatarciu. Brzmiała bezpośrednio w mojej głowie.\\
Odkleiłem się od ściany i susem wpadłem do środka. Gdyby nie to, że nie panowałem nad swoim ciałem, byłem jedynie „obserwatorem samego siebie”, krzyknąłbym z przerażenia.\\
Przy prawej ścianie starej szkolnej klasy rzędem stało około sześciu mężczyzn -- wszyscy mieli związane ręce za plecami, nogi rozstawione szeroko, czoła zaś bezwładnie oparte, niemal wbite w ścianę, jakby stali w takiej pozycji od dłuższego czasu. Wyglądali jak jeńcy czekający na rozstrzelanie.\\
Z lewej, na środku całego pomieszczenia, znajdowało się stare, spróchniałe, pozbawione wierzchniej pokrywy pianino przy którym siedział tajemniczy wirtuoz, który wręcz przywołał mnie do siebie swoją muzyką. Siedział bokiem do mnie, na zniszczonym obrotowym krzesełku, ubrany w skórzaną, czarną kurtkę z wysokim kołnierzem i powycierane, brązowe sztruksy. Widziałem jego prawy profil -- grubo ciosaną twarz, płaski podbródek, zarośnięte krótkim, sztywnym, siwym włosiem policzki, wysokie czoło i zaczerwienione oko ze spuchnięta powieką. Niewątpliwie wkładał w grę całą swoją duszę, całkowicie się dla niej poświęcał; mimo to zdawał się być zmęczony i zniechęcony, niczym robotnik po wielogodzinnej harówce bez ani minuty na odsapnięcie.\\
Cała ta sytuacja mnie przerosła -- omamy i nagłe utraty przytomności, koszmarny sen i niespodziewane znalezienie się w szkole muzycznej, fakt iż poruszałem się mimo woli i znalazłem się w tak niesamowitej sytuacji; obok ludzi czekających niewątpliwie na śmierć, którym przygrywał niezwykle utalentowany pianista sprawił, że coś we mnie pękło, zerwałem ostatnie więzy łączące mnie ze zdrowymi zmysłami i normalnością.\\
Moja dusza została zdegradowana. Najbardziej bolało mnie jednak to, iż nie wiedziałem dlaczego ani z jakich konkretnych przyczyn -- co symbolizował wisielec otoczony przez lustra, ja bliski takiej samej śmierci w wąskim pokoju, niemalże kominie, obserwowany z góry przez nieznajomych\3k ludzi? Dlaczego Zona przyszykowała dla mnie wszystkie swoje największe wynaturzenia, postanowiła „obdarować” mnie nimi akurat w dniu, w którym postanowiłem ją opuścić? Mściła się? Za to, że ją znienawidziłem, że dałem Jej poczucie, iż jestem Jej własnością by następnie spróbować z Niej uciec? Zdradziłem Ją?\\
Moja rozważania ustały w chwili, w której nacisnąłem spust karabinu -- nawet nie zauważyłem momentu, w którym wycelowałem w rząd stojących pod ścianą stalkerów. Kanonada, która się rozległa była ogłuszająca, a jej odbijane przez wąskie ściany echo dosłownie wstrząsało moimi wewnętrznymi organami, mimo to muzyka pianisty była wciąż słyszalna, co dziwne, dopiero teraz jej piękno zdawało się rozkwitnąć w pełni.\\
Dźwięki były czystsze i lepiej słyszalne, każda nuta przybliżała mnie do ekstazy, cała melodia zaś przerażająco wpasowywała się w masakrę, której byłem sprawcą -- czas zdawał się zwolnić, by móc uwydatnić i podkreślić każdy pojedynczy wystrzał, stukot każdej z łusek, mdlący odgłos kuli wrzynającej się w ciało, łomot opadających, zalanych krwią zwłok nieznajomych, którzy osuwali się po zabryzganej juchą ścianie jeden po drugim. Trwała okropna symfonia, przerażające połączenie muzyki z mordem, które uzupełniały się wzajemnie i zdawały się tworzyć jedną, nierozłączną całość. Powoli rosło we mnie poczucie ulgi, z każdym zabitym nieznajomym, zgodnie z poleceniem Mikołaja swoją drogą, zbliżałem się do ostatniego samo upokorzenia siebie, tego, po którym miał dla mnie nastąpić nowy etap w życiu.\\
Ostatni z milczących skazańców padł martwy. Wnętrze wypełniło się dymem, tynkiem i zapachem prochu. Ogień pięknie podkreślał długie, tłuste smugi dymu i wzbity w powietrze kurz. Nagle poczułem, że odzyskuję nad sobą kontrolę. Muzyka przestawała mnie hipnotyzować, a jedynie zachwycać, zacząłem odczuwać wszystkie mięśnie i kończyny, słuch znów sprawiał wrażenie przestrzennego i zależnego od bliskości źródła dźwięków; muzyka brzmiała tuż obok mnie, nie w mym samym. Wykonując pierwszy od kilku minut, w pełni świadomy ruch, odwróciłem się za plecy. Nieznajomy grał dalej, jak gdyby nigdy nic.\\
„Wszystkich”.\\
Ostatkiem sił wycelowałem w pianistę i nacisnąłem spust. Zamiast huku wystrzału rozległ się nikły trzask iglicy, na równi z wydaniem przez wirtuoza dźwięku, który zepsuł całą harmonijną kompozycję granej do tej pory melodii -- uderzył w ostatni klawisz po lewej, wydając złowieszczą, przejmującą nutę, której echo zdawało się nieść bez końca. Przerwał.\\
Zerwał się z krzesła, podbiegł do mnie i uderzył w brzuch.\\
Nie widziałem, aby trzymał coś w ręku, mimo to ból był tak wielki, jakby ugodzono mnie haczykowatym, postrzępionym ostrzem potrafiącym zahaczyć o jelita.\\
Na zewnątrz rozległ się potężny grzmot, któremu towarzyszyła eksplozja zielonawego światła. Zbliżało się Zwarcie.\\
Drugi cios sięgnął mnie pod lewą pachę, trzeci i ostatni roztrzaskał mi prawe biodro. Tracąc dech, nie mając siły na nic ponad spazmatyczne jęki i sapania, osunąłem się na dół. Na zewnątrz błyskało już w wszystkich tak bardzo przywodzących na myśl Zwarcie kolorach -- purpurze, czerwieni, mdłym błękicie. Zostałem uderzony gołą pięścią, a czułem, że umieram. Czułem się tym niezwykle upokorzony. Pozbawiony życia, kiedy te zaczęło zapowiadać jakąś zmianę, dawało o sobie znać, że jest w nim coś więcej, niż do tej pory mi się zdawało. Zdołałem jedynie liznąć Prawdy na jego temat -- nie zdążyłem odkryć jej nawet w najmniejszym stopniu, jedynie zdać sobie sprawę z jej istnienia. Ogrom gorejącej we mnie z tego powodu rozpaczy przyspieszył mój zgon, zniszczył we mnie jakąkolwiek wolę walki, uświadomił wielkość klęski, jaką właśnie poniosłem. Chłód ogarnął całe moje ciało, podobnie jak ciemności, które przysłoniły obserwującą mnie na tle sufitu szkoły twarz nieznajomego.\\
Przez pięć sekund słyszałem jeszcze milknące echo emisji, później nie było już niczego.\\
Zdążyłem zdać sobie sprawę, że ostateczną formą wyzwolenia jest śmierć. Gdybym był tego świadom wcześniej, przywitałbym ją z otwartymi ramionami.\\
\end{document}