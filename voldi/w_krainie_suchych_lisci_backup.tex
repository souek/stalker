\documentclass[../MAIN.tex]{subfiles}
\begin{document}
\section*{WOLNY CZŁOWIEK}
\mm Był piątek, wczesne popołudnie. Dawid Kart siedział w~fotelu popijając piwo po
ciężkim dniu pracy. Imał się różnej roboty. Od kiedy w~wypadku zginęła jego żona
i trzyletnia córka, wszystko straciło sens. Żył, aby żyć, nie mając konkretnego
celu.\\
W pewnym momencie zadzwonił jego telefon. Na wyświetlaczu pojawiło się „Rudy”.
Rudy, czyli Andrzej Rync, był najlepszym kumplem Karta jeszcze ze szkoły. Dawid
niezwłocznie odebrał połączenie.
\dd
\sd
\xx Siema, Olo, co u Ciebie? --~spytał wesoło Rudy.
\xx A, po robocie jestem. A co?
\xx Mam dla ciebie robotę. Niezłą. Nadasz się.
\xx Co, gdzie, kiedy i\3k za ile? --~na to ostatnie Dawid zwrócił uwagę
najbardziej.
\xx Kasy, ile zechcesz. Wymaga krzepy i~trochę wojskowego doświadczenia.
Pojechałbyś do Strefy wokół elektrowni w~Czarnobylu i\3k
\xx Pojebało Cię?! --~krzyknął w~słuchawkę Olo. A głos, jako były żołnierz,
miał mocny.~--~na śmierć mnie wysyłasz!
\xx Propaganda. Ludzie tam normalnie żyją i~jakoś nic im się nie dzieje.
\xx Ech\3k --~zawahał się. --~Mów dalej, kasa to kasa.
\xx Pojedziesz poszukać paru rzeczy. Kurde, to nie rozmowa na telefon. Jutro w~Koźle?
\xx Dobra, o osiemnastej.
\xx Okej. Na razie. --~Rudy się rozłączył.
\qd
\hspace{17.5em} Kart posiedział jeszcze trochę przed telewizorem, poszedł zjeść kolację i
udał się do łóżka. Rano wstał, wykonał parę ćwiczeń, wypił poranną kawę i~usiadł
przed komputerem. Chciał się dowiedzieć więcej o tej tajemniczej Strefie:
trzydziestokilometrowym obszarze  Po
wybuchu reaktora atomowego w~elektrowni   w~Czarnobylu utworzono
trzydziestokilometrowy obszar, szczelnie otoczony przez wojsko.
Wśród wielu
krzykliwych artykułów znalazł jeden ciekawy odnośnik do forum poświęconego
ludziom, którzy „włamywali się” do Strefy, wynosili z niej cenne znaleziska
zwane artefaktami i~w świecie zewnętrznym zarabiali na nich krocie. Poczytał
trochę i~nie zauważył, kiedy poranek przeobraził się w~wieczór. Dochodziła
siedemnasta trzydzieści. Czas iść do pubu, na spotkanie z Rudym. Gdy Olo tam
dotarł, jego kolega czekał już przy barze.
\dd
\sd
\xx Siema Rudy.
\xx Cześć.
\xx I co z tą robotą?
\xx Poczekaj\3k Barman! Dwa razy to, co zwykle.
\xx Więc? --~niecierpliwił się Olo.
\xx Jest tak: pewien typek z Ukrainy przejeżdża tędy wymieniać towar. Zapoznałem
się z nim. Człowieku, co on ze sobą miał! A mówił, że to można znaleźć w~byle
krzakach koło bazy, że jest tak popularne jak piwo na Ukrainie, ale nosi, bo się
ładnie błyszczy. Mówił, że może cię zabrać ze sobą, za drobną opłatą.
\xx Hmm\3k Ile?
\xx Nie tak prędko. Najpierw musisz kupić to, co w~takim miejscu potrzebne.
Prowiant, ekwipunek. Broń załatwi ci ten ruski, ale nie za darmo przecież.
\xx Dobra, ile?
\xx Dwa i~pół kafla za broń, przejazd i, jak to określił, zakwaterowanie. Za
tydzień o trzynastej mamy czekać gotowi pod Kozłem.
\xx Kurde, sporo\3k --~zmartwił się Olo.
\xx Tak, ale jakie zyski! Powiedział, że przez trzy miesiące zarobił osiem
tysięcy. Dolarów.
\xx O cholera. --~wybałuszył oczy Olo.
\xx Właśnie\3k
\xx Dobra, pogrzebię się w~tym, widzimy się za tydzień w~tym miejscu.
\xx To jedziesz?
\xx No pewnie. Tylko trochę poczytam w~sieci na ten temat.
\qd
\hspace{26em} Jak powiedział, tak i~zrobił. Znalazł forum, które przeglądał przed wyjściem dobaru i~jeszcze wnikliwiej zagłębiał się w~lekturę. Nim się obejrzał, była
dwudziesta trzecia. Dawid położył się spać nawet nie biorąc prysznicu.
\section*{DROGA DO NIKĄD}
\mm Rano poszedł do banku. Miał na koncie 2283 złote, odłożone na czarną godzinę.
Wyciągnął wszystkie pieniądze i~wrócił do domu. Spisał z forum wszystko, co
niezbędne do przeżycia w~Strefie i~ruszył na zakupy.

Po południu zorientował się, że Ukrainiec chciał 2500 złotych. Musiał zdobyć te
pieniądze. Pogadał z jedyną osobą, jaka mu została: z Rudym. Ten pożyczył mu
tysiąc złotych, pod warunkiem, że podzieli się z nim przynajmniej początkowymi
zyskami.

Po trzech dniach miał już prawie wszystko --~zapas jedzenia na miesiąc,
latarkę, zestaw baterii, trochę śrubek pomocnych przy wykrywaniu anomalii.
Zostało mu do kupienia tylko jedno --~licznik Geigera. Problem polegał na tym,
że dobre urządzenia są nie tylko rzadkie, ale jeszcze drogie jak~cholera. Olo
doszedł do wniosku, że znajdzie sobie jakiś w~Strefie.
\looseness-1
Nadszedł oczekiwany dzień. O dwunastej trzydzieści w~drzwi jego mieszkania ktoś
zapukał. Lekko podenerwowany podszedł i~spojrzał przez wizjer. To był Rudy, po
drodze do baru zaszedł po swojego kolegę. Pomógł mu zebrać graty i~razem udali
się w~okolice knajpy Waleczny Kozioł. W~pewnym momencie zza rogu wyjechała lekko
rozklekotana Łada Niwa.
\dd
\sd
\xx \textit{Prywiet!} --~z okna wychylił się pulchny, wąsaty Ukrainiec --~To
on?
\xx Tak. Nazywa się Olo.
\xx \textit{Prywiet}, Olo. Ja nazywam się Wania --~przywitał się z Dawidem
\xx \textit{Prywiet}.
\xx Masz całe wyposażenie?
\xx Mam. --~zapewnił Olo.
\xx Pieniądze?
\xx Mam.
\xx \textit{To pajechali! Dawaj!} --~krzyknął Ukrainiec.
\qd
\hspace{20.4em}Olo włożył wszystkie graty do samochodu, sam też zajął w~nim
miejsce. Pożegnał
się ze swoim przyjacielem i~auto z trudem ruszyło. Całą drogę spędzili próbując
się dogadać, śpiewając radzieckie i~polskie przyśpiewki. Jechali długo, prawie
dziesięć godziny, łącznie z postojem na granicy. W~końcu dojechali od miejsca z
czerwonym szlabanem i~strażnikami. Ściemniało się, zbliżała się noc. Gdzieś
daleko było widać tworzące się mgły.
\dd
\sd
\xx
\textit{Ooo! Wania! Dobryj deń!} --~podszedł strażnik. Widać, znali się z
nowym kolegą Ola.
\xx \textit{Prywiet!} Co słychać?
\qd
% \vspace*{.5em}
% \hspace{11em}
\mm Rozmowa przebiegła sprawnie i~po chwili byli już w~Strefie. Znaleźli się w~innym
świecie. Na mapach te tereny nie istnieją, są jakby wymazane. Nikt nie bierze
ich pod uwagę. Jechali, oficjalnie, drogą do nikąd. Po pół godzinie jazdy na
horyzoncie zaczęły rysować się jakieś zabudowania. Ukrainiec wyjaśnił, że to
pierwsza baza wypadowa stalkerów, że tu lądują wszyscy nowi. Kazał Olowi
zapoznać się z jej mieszkańcami.
Podjechali pod coś, co wyglądało jak kiosk. W~środku było dwoje ludzi. Wania
zajrzał przez szybę i~powiedział:
\sd
\xx Żora! Masz tą zabawkę za pięćset rubli?
\xx Mam, mam. --~sprzedawca schylił się pod ladę i~wyjął starą berettę.
\xx Zgłupiałeś?! --~oburzył się Wania
\xx Mam za ten złom płacić pięćset rubli?!
\qd
Ich kłótnia trwała chwilę. Ostatecznie stargowali cenę broni do 350 rubli.
Ukrainiec podszedł do Ola, wręczył mu pistolet i~powiedział, że od dziś ma sobie
radzić sam. Wyjął z samochodu wszystkie jego rzeczy i~odjechał w~nieznanym
kierunku. Chyba gdzieś w~głąb Strefy. Dawid podniósł graty i~zaczął je nieść,
chyba sam nie wiedział, gdzie dokładnie. Wydało mu się dziwnie cicho. Mijał
właśnie dogasające ognisko, gdy ktoś za jego plecami przeładował pistolet.
Zatrzymał się i~nie wiedział, czy się bronić, czy tylko powoli odwrócić, czy
stać nieruchomo. Wtem osoba stojąca za nim spytała:
\dd
\sd
\xx Ktoś ty?!
\xx Dawid Kart, mówią na mnie Olo. Przywiózł mnie taki gruby, wąsaty Ukrainiec,
Wania. Powiedział, że to obóz stalkerów.
\xx Wania? Ten cham? A ile sobie zaśpiewał?
\xx Dwa i~pół tysiąca złotych.
\xx Ty Polak, da? Skąd jesteś?
\xx Z małego miasta na Podlasiu.
\qd
\hspace{14.1em}
Usłyszał, że mężczyzna stojący za nim chowa broń i~powoli się odwrócił. Zobaczył
za nim jeszcze około 15 osób, z czego trzy próbowały na nowo rozpalić ognisko.
Facet, który był teraz tuż przed nim, przeprosił. Powiedział, że wrócili właśnie
z pobliskiego gospodarstwa, bo poszli wytępić paru bandytów, którzy zrobili tam
sobie kryjówkę. Chwilę potem zawołał gościa imieniem Witek i~kazał mu
odprowadzić Ola do szopy. Nie był to pięciogwiazdkowy hotel, ale przynajmniej na
głowę nie kapało.\looseness=-1
\section*{W MROKU DNIA}
\mm O piątej nad ranem ktoś przybiegł i~obudził Dawida.
\dd
\sd
\xx Ty, Nowy, umiesz strzelać?!
\xx Co? --~spytał zaspanym głosem Olo.
\xx Umiesz strzelać?!
\xx Ee.. Słyszę, nie drzyj się tak. Umiem, kiedyś służyłem w~wojsku.
\xx To łap za gnata i~chodź pomóc! --~rzucił mu pod nogi starego AKS74-U.
\qd
\hspace{33.5em}
Olo włożył buty i~w samych bokserkach wyskoczył na dwór, w~sam środek
strzelaniny. Szybko stanął za zaimprowizowaną osłoną i~na spokojnie przyjrzał
się sytuacji. Po swojej lewej miał kilkunastu bandytów biegnących na szczyt
wzgórza, a~po prawej trzech lub czterech żołnierzy w~mundurach. Okazało się, że
rozbójnicy nabroili coś koło posterunku wojskowego, ale stchórzyli i~zaczęli
uciekać, przypadkiem trafiając na obóz. Żołnierze usłyszeli wystrzały, odgłosy
walki między zbójami, a~stalkerami i~wysłali kilka osób do sprawdzenia sytuacji.
\pp
Olo bez mrugnięcia okiem wyskoczył zza osłony, zdjął jednego żołnierza strzałem
w głowę, drugiemu przedziurawił kolano, a~potem obrócił się i~oddał serię w
kierunku uciekających bandytów, raniąc jednego czy dwóch z nich. Chwilę potem,
gdy stalkerzy ruszyli za liczniejszą grupą, Olo z Witkiem zostali, by rozprawić
się z pozostałymi wojskowymi. Poszli ostrożnie w~ich kierunku, gdy nagle
mundurowi wyskoczyli zza drzew i~zaczęli strzelać. Olo instynktownie padł na
trawę, ale Witek został postrzelony. Zaklął i~przewrócił się w~krzaki rosnące
przy płocie. Chwilę potem Kart wstał i~zaczął strzelać do żołnierzy. Jednego
udało mu się zastrzelić, przy rozprawianiu się z drugim jego karabinek się
zaciął. Rzucił go więc w~kierunku przeciwnika, by chwilę potem podbiec do niego,
wybić broń z ręki, a~następnie skręcić mu kark.
\pp
Ciała leżały niedaleko drogi. Dawid postanowił zatrzeć ślady. Ciągnął już
ostatniego trupa, gdy nagle przypomniał sobie o swoim rannym koledze. Krzyknął:
,,Witek! Żyjesz?!''. Po chwili w~odpowiedzi dostał stos wulgaryzmów i~prośbę o
pomoc. Przybiegł, przerzucił kompana przez ramię i~zaniósł do obozu. Akurat
pozostali stalkerzy wrócili po gonitwie, byli w~bardzo dobrych humorach jak na
ludzi, którzy właśnie mogli pozbawić życia całą grupę.
Miny im trochę zrzedły,
gdy zobaczyli zakrwawionego Ola i~Witka trzymającego się za ramię, z którego
powoli kapała czerwona ciecz. Na szczęście okazało się, że krew na Dawidzie to
efekt dźwigania martwych żołnierzy i~pomocy koledze, a~i~on sam został w
strzelaninie tylko draśnięty.
\pp
Po ochłonięciu Kart zauważył, że jest mu dziwnie zimno. Dopiero wtedy
zorientował się, że ma na sobie tylko bokserki i~pośpiesznie założone,
niezasznurowane buty. Poszedł do swojej kwatery i~ubrał się adekwatnie do pory
roku. Wyszedł na zewnątrz. Była wczesna jesień, godzina szósta rano. Słońce
nieśmiało wychylało się zza drzew, biało-różowe obłoczki płynęły spokojnie
po niebie, liście w~okolicznych sadach i~lasach mieniły się wszelkimi kolorami,
powoli opadając na ziemię, ostatnie gwiazdy ginęły w~jasnym błękicie. Widok
zapierający dech w~piersiach.

I tylko lufy kałasznikowów połyskujące w~blasku
niewielkiego ogniska i~świeża krew zastygająca na płocie zdradzały, że to nie
jest jeden z obrazów malarza impresjonisty, a~miejsce, gdzie co chwilę ktoś
ginie od gradu pocisków. I nie tylko pocisków. Wbrew pozorom, broń palna, nawet
w rękach najlepszego żołnierza nie była najgroźniejszą rzeczą, jaką można było
znaleźć w~Strefie.
\pp
Stalkerzy posiedzieli jeszcze chwilę przy ognisku, po czym poszli zająć się
swoimi sprawami --~jedni szukali w~okolicy anomalii, które znaczyli i
artefaktów, którymi handlowali, inni poszli zakopać ciała żołnierzy, a
pozostali, w~tym Olo, wrócili do swoich kwater.
Koło ósmej rano Dawid postanowił
rozejrzeć się po obozie. Widok okazał się nieszczególny --~stara wioska
położona wzdłuż drogi, maksymalnie na dwadzieścia rodzin. Teraz opuszczona od
wielu lat. Niektóre domy legły w~gruzach, inne miały podziurawione ściany, a~z
innych tylko lekko poschodziła farba.
Nigdzie nie było okien. Dachy były
wykonane, jak to żartobliwie określano, metodą półprzepuszczalną --~w
niektórych miejscach leżały betonowe dachówki, bądź eternit, w~innych były
dziury. Na szczęście, dachy okazały się mocniejsze niż same budynki, bo dziur
było niewiele.

Na końcu wsi stał opuszczony, zardzewiały samochód. Wewnątrz
niego, od podłogi po sam przeżarty przez rdzę i~pociski dach, rosło sobie w
najlepsze młode drzewko. Może wyglądało to
tak,
jakby było w~tym wraku uwięzione, jednak tak naprawdę to pojazd był więźniem tej
rośliny. Natura zawsze wygra z człowiekiem.
\pp
Później Olo ruszył w~kierunku obozowego „sklepu”. Kiedyś była to chyba jakaś
letnia kuchnia. Teraz robiła za skład amunicji, depozyt, bank i~osiedlowy
spożywczy. Można tam było kupić prawie wszystko --~od bułek i~kiełbasy, przez
leki na promieniowanie, po amunicję do granatników i~miny piechotne.
Nikt nie wiedział, czy produkty spożywcze są świeże, czy leki same nie są skażone, ani
skąd jest ta amunicja, ale woleli nie pytać. Sklep ten prowadził na oko
sześćdziesięcioletni facet, z wyglądu trochę podobny do Wani, ale z okazałą,
zadbaną, siwą brodą. No i~z tego co widział Olo poprzedniego dnia --~zdzierał
jeszcze bardziej niż jego wąsaty odpowiednik.
\pp
Około godziny dziesiątej stalker ubrany w~szary, lekko podarty płaszcz, zapytał,
czy ktoś idzie posprawdzać szopy w~terenie. Miał lepsze wyposażenie niż
pozostali. Nosił duży wojskowy plecak typu kostka, wojskowe buty, a~przez ramię
miał przewieszony karabinek AK74/2 z doczepionym granatnikiem i~lunetą. Wstało
trzech z siedmiu zgromadzonych przy ognisku.
Mężczyzna w~płaszczu podszedł do
Ola, spytał go jak się nazywa i~poprosił, by poszedł z nim. Dawid zgodził się,
jednak poprosił o jakąś lepszą broń, bo z tą starą berettą mógł najwyżej
strzelać do tarczy i~to z odległości metra. Pewnie jego prośba zostałaby
zignorowana, gdyby nie fakt, że sam rozprawił się rano z czterema żołnierzami.
Dostał pistolet maszynowy Viper 5 i~trochę jedzenia.
Zaniósł prowiant do siebie, schował pod materac, zajrzał do kiosku, chcąc
sprzedać tą berettę. Facet w~budynku, o imieniu Żora, mógł go przyjąć, ale za
śmieszną kwotę --~sto rubli, choć sam sprzedał pistolet dzień wcześniej za
ponad trzysta. Olowi udało się podbić cenę do dwustu pięćdziesięciu, gdy
spostrzegł, że jego grupa przygotowuje się do wyjścia z bazy. Szybko wziął
pieniądze i~dołączył do kompanów.
\pp
Po drodze mężczyzna w~szarym płaszczu uczył go, jak powinien się poruszać w
Strefie, jak bezpiecznie sprawdzać, czy nigdzie nie czai się żadna anomalia.
Dawid dostał tylko szczątkową wiedzę, taką, jaka była wystarczająca by przejść
te kilkadziesiąt metrów z wioski do gospodarstwa.
Dostał przykazanie, by trzymać
się któregoś z kolegów, najlepiej Andreja. Andrej był dobrze zbudowanym, na oko
czterdziestokilkuletnim mężczyzną, z twarzą pokrytą kilkoma bliznami. Był w
Zonie już długo, ale, jak sam twierdził, nie chciał ruszać się dalej, bo już raz
tam był i~to co widział, było straszne.
\pp
Po niecałym kwadransie czteroosobowa grupa dotarła do granic gospodarstwa. Za
płotem było słychać nerwowe rozmowy, pojękiwania i~co jakiś czas wykrzyczane
wulgaryzmy.
\dd
\sd
\xx Dasz sobie radę? Ich z tej strony jest góra czterech, z czego dwóch rannych.\x spytał dowódca.
\xx Jeśli broń mi się nie zatnie, to żaden problem. --~potwierdził Olo.
\xx Dobra, to robimy tak: Olo zostaje tutaj, wpada przez dziurę, Andrej, ty idź
na północ, tam jest brama główna, powinno stać dwóch, a~trzeci na podwórku,
Awdan, na zachód, przez ten wyrwany kawałek płotu. A ja\3k ja wejdę przez tą
rozwaloną ścianę.
\qd
\dd
Olo czuł, że jest najbardziej doświadczonym strzelcem w~grupie. Zaczekał, aż
wszyscy się ustawią i~wtedy dopiero miał zaatakować. Jednak okazało się inaczej.
Strzały rozległy się już w~kilka sekund po rozdzieleniu grupy. Dowódca pewnie
spotkał kogoś w~tym rozwalonym domku.%\looseness-1

\mm Kart odbezpieczył swojego Vipera, splunął i~ruszył przez płot. Wpadł na
podwórze, gdzie powitała go seria kul z bliźniaczej broni. Na szczęście strzelec
miał problemy z celnością, wszystkie pociski wbiły się w~deski za Olem. Ten
wychylił się zza ściany i~kilkoma strzałami powalił przeciwnika, a~następnie
jeszcze jednego, który biegł w~kierunku bramy. Potem sam ruszył do przodu i~na
lewo, przez drzwi do budynku gospodarczego.
Tam zastał trzy osoby, jeden
mężczyzna z bandażami w~ręku, drugi siedzący, trzymający się za nogę i~trzeci,
leżący półprzytomnie na podłodze. Szybko zabił dwóch pierwszych, ale przy
ostatnim coś go tknęło.
Może to stara, wojskowa szkoła. Chciał podejść i
opatrzyć przeciwnika, gdy usłyszał świst kuli, tuż nad swoją głową. Szybko
przyległ do ściany obok okna, wychylił się z niego, rozejrzał i~dwoma celnymi
strzałami zrzucił stojącego na dachu, odzianego w~czarną, skórzaną kurtkę
bandytę na ziemię, prosto na stary, nieużywany traktor.\looseness-1
\pp
Chwilę potem strzały ucichły. Stalkerzy zaczęli się nawoływać. Jednak Olo nie
dawał znaku życia. Pozostali weszli do szopy i~zobaczyli, jak ten pochyla się
nad rannym wrogiem.
\dd
\sd
\xx
Co tobie? --~zdziwił się dowódca i~wyjął pistolet. --~Gdyby dał radę,
wbiłby ci nóż w~plecy.\looseness-1
\qd
\dd
\mm Dawid usłyszał wybuch, jakby w~swej głowie. Chwilę potem cienka strużka krwi
popłynęła z rany w~głowie na podłogę. Mężczyzna w~ciemnej kurtce leżał teraz bez
życia. Olo zamknął mu oczy, wstał i~ruszył nieco smętnie do obozu. Gdy mijał
Andreja, ten poklepał go w~ramię.
\pp
Tego poranka zrozumiał, że Strefa to nie tylko przygoda i~bogactwo, ale również,
a może przede wszystkim, śmierć czekająca na niego tuż za plecami. Rozumiał to,
choć jeszcze tak naprawdę nie spotkał najgroźniejszego wroga w~Zonie --~jej
samej.
\pp
Dotarli do obozu chwilę po dwunastej, w~porze obiadowej. Już z daleka było
słychać głośne rozmowy stalkerów. Olo poszedł do swojej kwatery, po jedzenie.
Jednak, części zapasów, które kładł tam rano, nie było.
Leżała karteczka z
przeprosinami i~pół litra wódki. Wychodząc na zewnątrz minął dowódcę, który dał
mu pięćset rubli i~trochę amunicji do jego broni.
\pp
Dawid ruszył do sklepu, kupić jakieś żarcie. Poszedł potem do ogniska i~usiadł
wraz z
innymi. Zaczął się z nimi powoli zapoznawać. To byli ciepli ludzie, można się tu
było czuć jak w~domu. Opowiedział, kim jest, skąd pochodzi, jak i~po co się tu
znalazł.
\pp
Zaczęło się powoli ściemniać. Była godzina szesnasta. Olo leżał na materacu i
czytał książkę, gdy usłyszał czyjeś kroki. Podniósł wzrok i~spostrzegł Witka.
Ten spytał go, czy chce iść z nim poszukać trochę artefaktów. Dawid odłożył
książkę, wziął karabin, latarkę, plecak i~wstał z łóżka.
\section*{OBJĘCIA STREFY}
\mm Przy wychodzeniu z obozu minęli Sieńkę. Młodzika, który był w~Strefie dopiero od dwóch tygodni. Stał na warcie przed obozem.
\dd
\sd
\xx Cześć chłopaki, gdzie idziecie?
\xx A, poszwendać się trochę po okolicy. Może coś ciekawego wskoczy.
\xx Aha, jasne. Udanych łowów!
\xx No, dzięki.
\qd
\hspace{1ex}Nie spieszyli się. Przeszli wzdłuż drogi około trzystu metrów, by skręcić przed małym mostkiem. Schodząc w~dół spostrzegli ciemny kształt leżący przy wraku ciężarówki. Gdy zbliżyli się na odległość ok. piętnastu metrów, usłyszeli pojękiwania. Jak na rozkaz włączyli swoje latarki. O koło przewróconego na bok wozu siedział oparty niewysoki, długowłosy mężczyzna w~zielonym swetrze i~trzymał się za brzuch, z którego powoli sączyła się krew. Stalkerzy dostrzegli też całkowicie podarte spodnie. Człowiek ten jednak stracił już za dużo krwi, by być świadomym. Nie zauważył Ola i~Witka. Jęczał tylko \textit{„W lewo niet, w~lewo niet\3k”.}
\pp
Witek przeszedł obojętnie obok niego i~skinął na swojego kolegę, by też nie zwracał uwagi na umierającego samotnika.
\dd
\sd
\xx Teraz ostrożnie, tu zaczyna się Strefa. --~szepnął Witek. --~Wytęż zmysły\\ i~trzymaj palec na spuście.
\qd
\hspace{7em}Szli powoli około dwudziestu metrów, gdy mężczyzna nagle się zatrzymał.
\dd
\sd
\xx Olo, patrz. --~szepnął --~widzisz te liście i~ciemną plamę pod nimi?
\xx Widzę. Co to?
\xx Anomalia, zwana Wirem. Odsuń się trochę.
\qd
\hspace{20.6em}Kart posłusznie cofnął się kilka kroków do tyłu. W~tym momencie jego towarzysz wyjął z kieszeni śrubkę i~rzucił w~kierunku unoszących się liści. Powietrze przed nimi gwałtownie zasyczało i~rozbłysło jasnym światłem. Sekundę później po metalowym elemencie pozostało tylko trochę dymu.
\dd
\sd
\xx Widzisz? Jakbyś w~to wsadził rękę, albo nogę, to by twoich resztek szukali przez tydzień. Chodź, obejdziemy tą gadzinę. --~machnął ręką Witek i~rzucanymi śrubkami zaczął oznaczać drogę, jak obejść anomalię.
\qd
\hspace{20.5em} Mężczyźni szli jeszcze kawałek, gdy natrafili na starą szopę. Przez małe okienko było widać zielonożółtą, lekko migotającą poświatę.
\dd
\sd
\xx A to, widzisz? Słyszałem, że tam w~środku może być Ślimak.
\xx Ślimak? Co to takiego? --~zdziwił się Olo.
\xx Ślimak to artefakt tworzony przez Galaretę, anomalię podobną do kałuży żrącego kwasu. Ani anomalie, ani te artefakty nie są zbyt częste, można więc sporo zarobić.
\xx Jakiś haczyk?
\xx Nikt, kto wszedł do tej szopy drzwiami, żywy z niej nie wyszedł.
\xx Aha\3k To jak zabierzemy Ślimaka?
\xx Coś wymyślę. --~w oczach Witka można było dostrzec dwie małe iskierki. --~Wiem! Wejdziemy po drzewie i~przez dach.
\qd
\hspace{17em} Zaczęli wdrapywać się na rosnącą tuż przy szopie sosnę. Weszli ostrożnie na eternitowe sklepienie budowli, po czym Witek spojrzał przez małą dziurę w~dachówkach.
\dd
\sd
\xx Jest! Tam! --~uradował się i~wskazał palcem małą kulkę leżącą wśród drewna. --~Olo, jesteś silny?
\xx Zależy ile tej siły potrzeba.
\xx Utrzymałbyś mój ciężar, gdy zsunę się trochę na dół?
\xx Nie wiem, pewnie tak.
\xx Dobra. Ryzykujemy. Złap mnie za nogi.
\qd
% \hspace{19.1em}
\mm Gdy Witek miał artefakt niemal na wyciągnięcie ręki, złowrogo zapiszczało jego PDA. To był alarm przed nadchodzącą emisją.
\dd
\sd
\xx Cholera, nie teraz! \x krzyknął gorączkowo.
\xx Co jest?
\xx Emisja. Szybko, zabieramy Ślimaka i~spierdalamy do obozu.
\qd
\hspace{27.65em} Złapał mocno artefakt i~wydał Dawidowi polecenie, by go wyciągnął. Następnie zeskoczył z szopy, wrzucił Ślimaka do plecaka, złapał swoją strzelbę i~ruszył pędem przez chaszcze, lawirując wśród anomalii grawitacyjnych. W~pewnym momencie niebo gwałtownie pociemniało, usłyszeli dźwięk, jakby w~oddali nastąpił wybuch, a~ziemia lekko się zatrzęsła.
\dd
\sd
\xx Kurwa, kurwa, kurwa! --~klął Witek, łamiąc rękami gałęzie przed sobą.
\qd
\hspace{32.5em} Gdy dobiegli do drogi, na horyzoncie zaczęły pojawiać się pierwsze błyskawice. Biegli dalej, nie zważając na zmęczenie. Olo nie wiedział, co się dzieje, ale skoro Witek był taki zdenerwowany, to na pewno nie działo się nic dobrego. W~pewnym momencie pierwszego mężczyznę coś podrzuciło do góry. Dawid usłyszał znajomy syk powietrza. W~biegu podskoczył, złapał towarzysza w~pasie i~wyciągnął ze śmiercionośnego objęcia anomalii. Teraz to on biegł pierwszy. Zobaczył niedaleko obóz. Wbiegł do niego i~dosłownie wleciał do podziemnego pomieszczenia, wykopanego przez stalkerów. Spadł po schodach i~wylądował u stóp Żory. Chwilę po nim w~podobnej pozycji znalazł się Witek. Z plecaka wytoczył mu się ten zielonkawy, okrągły obiekt. W~tej chwili tunelem zatrzęsło, wszyscy upadli na podłogę i~zaczęli łapać się za głowy.
\pp
Po chwili było już po wszystkim. Stalkerzy wstali i~powoli wyszli na powierzchnię. Ściemniło się, jakby była noc, choć zegarki wskazywały dopiero godzinę osiemnastą. Światło jednak z każdą chwilą przebijało się przez chmury coraz mocniej, aż w~końcu było na poziomie takim, na jakim być powinno o tej porze dnia. Idąc do ogniska mężczyźni mijali dziesiątki martwych ptaków, zabitych w~czasie emisji.
\dd
\sd
\xx Co to było? --~zapytał Olo, gdy usiedli przy ognisku.
\xx Emisja psioniczna. Nikt nie wie, czym one są, ale są niezwykle groźne dla ludzi i~wszystkich istot żywych w~Strefie. Osoby, które się przed nimi odpowiednio nie ukryją giną, bądź, co gorsze, są zamieniane w~bezmózgie zombie. --~wyjaśnił ktoś z kręgu.\looseness-1
\dd
\qd
\mm
Witek jeszcze raz pokazał grupie artefakt, wszyscy się nim zachwycali, Żora pytał telefonicznie po swoich znajomych co to może być, ale nikt nie był pewien, czy to faktycznie Ślimak. Niedługo potem każdy poszedł do swojego łóżka i~zasnął, we względnym spokoju.\looseness+1
\pp
Rano, tak jak co dzień, wszyscy zebrali się przy ognisku, na porannej kawie, flaszce i~śniadaniu. Któryś z przebywających tam chłopaków wyjął gitarę i~zaczął grać rosyjskie piosenki. Stalkerzy żartowali między sobą, opowiadali historie o tym, że komuś udało się dotrzeć do centrum i~wrócić, niektórzy chwalili się, że znają współrzędne skrytek z artefaktami. Olo opowiadał o strzelaninie w~gospodarstwie. W~obozie tego ranka panowała bardzo luźna atmosfera.
\pp
W pewnej chwili do Dawida podszedł mężczyzna w~lekkiej brązowej kurtce. Poprosił go na bok i~spytał, czy mogą porozmawiać. Z rozmowy Olo dowiedział się, że o świcie jeden z obozowiczów wyruszył, by znaleźć przejście pozwalające ominąć posterunek wojskowy przy zawalonym moście. Jakieś dziesięć minut temu przysłał jednak wiadomość, w~której widniało tylko jedno słowo: „Pomocy”. Mężczyzna w~kurtce, o pseudonimie Siwy, stwierdził, że nie chce niepokoić obozu, dlatego prosi Ola o pomoc. Ten zgodził się. Dokończył jeść kiełbasę, zawiesił na ramieniu swojego Vipera, zabrał trochę amunicji, znaczonych śrubek, dwie bułki i~odszedł od ogniska. Ktoś na odchodne machnął do niego ręką i~krzyknął „Niech cię Zona pochłonie!”, ale Olo nie zwrócił na niego uwagi, nie było mu w~tej chwili do śmiechu.
\pp
Szedł krótko, około dziesięciu minut. Skręcił niecałe sto metrów przed posterunkiem. Przeszedł przez stary elewator zbożowy i~usłyszał dźwięki podobne do tych, jakie wydają błyskawice. Jednak tego poranka, poza kilkoma małymi chmurkami, niebo było czyste. Tylko przy ziemi snuła się niezbyt gęsta, wilgotna mgła, która jednak nie przeszkadzała w~swobodnym poruszaniu się po Strefie.
\pp
Klucząc między anomaliami dotarł do małego tunelu pod nasypem kolejowym. Wewnątrz, za skrzyniami widział jakąś niebieską poświatę. Zapalił latarkę i~ostrożnie wszedł do środka. W~tym momencie, jego stopa o coś zawadziła. Skierował swój wzrok w~dół i~spostrzegł coś, co wyglądało jak wrzucony do ognia kawał świniaka. Przez myśl przebiegło mu, że to jakieś spalone zwierzę, leżące tu dłuższy czas, więc nie skupił na nim swojej uwagi. Przeszedł jeszcze dwa metry i~stanął. Teraz widział wyraźnie, skąd brała się ta poświata. Po powierzchni ziemi ślizgały się małe, jasnoniebieskie iskry. Olo doszedł do wniosku, że to jakiś rodzaj anomalii elektrycznej. Rzucił w~jej kierunku śrubkę i~nagle znalazł się pięć metrów dalej, rażony działaniem tego tworu. Otrząsnął się, sięgnął do plecaka po butelkę wódki, przemył nią twarz i~trochę się napił. Potem schował flaszkę, znalazł karabin, który wypadł mu z ręki i~powoli wyszedł z tunelu. Postanowił przyjrzeć się zwęglonym zwłokom.
\pp
Wśród resztek stopionej kurtki znalazł PDA w~całkiem dobrym stanie. Uruchomił urządzenie. Należało do Aleksieja Jenkowa, pseudonim Jeniec. Na ekranie pokazały mu się też notatki. Okazało się, że to raport odnoszący się do działania tej anomalii, skutków jej działania na przedmioty drewniane i~kamienie oraz częstotliwości, z jaką się ona pojawia. Olo wyczytał, że po każdym uruchomieniu formacji „elektro” robi się ona niebezpieczna ponownie po dziesięciu sekundach. Zatem czas, w~jakim można bezpiecznie przedostać się na drugą stronę, to tylko dziesięć sekund. Wiadomość kończyła się słowami „Spróbuję przejść. Cześć!”. Wszystko było opatrzone datą 26.09.10, a~więc zostało utworzone dzisiaj. Dawid odnalazł zatem zaginionego stalkera.
\pp
Nie chciał zostawać tam dłużej. Zabrał przydatne przedmioty, nakreślił na czole Jeńca znak krzyża i~wyszedł z tunelu. Po śladach śrubek wrócił do elewatora, gdzie akurat zebrało się kilku bandytów. Dawid miał do wyboru trzy ścieżki: albo przejdzie obok posterunku i~narazi się żołnierzom, albo przejdzie przez spichlerz, gdzie dostanie się w~ręce zbirów, albo będzie kluczył nie wiadomo jak długo między anomaliami i~w końcu nieostrożnie trafi na jedną z nich. Z trojga złego, jako były żołnierz, wybrał walkę z bandytami. Przeczołgał się niepostrzeżenie do ściany, sprawdził stan swojego karabinu, wstał i~ostrożnie wychylił się zza winkla. Zbóje nawet nie zauważyły, kiedy otworzył ogień. Po chwili było już po wszystkim. Podszedł i~zaczął grabić ciała, gdy nagle jeden z przeciwników otworzył oczy. Olo chciał mu pomóc, ale w~jego głowie zabrzmiały słowa któregoś stalkera, wypowiedziane doń po wczorajszej strzelaninie w~gospodarstwie. Wyciągnął z kieszeni nóż, przystawił go bandycie do klatki piersiowej,
wypowiedział słowa: „To jest Zona, przyjacielu. Przepraszam.” i~z wielkim wysiłkiem wbił ostrze, odwracając jednocześnie wzrok. Nieco krwi trysnęło na jego dłoń. Szybko wytarł ją w~kurtkę, zabrał wszystko, co bandzior miał przy sobie i~ruszył z powrotem do obozu.
\pp
Szedł powoli drogą, podziwiając piękną, ukraińską jesień. Mgła powoli ustępowała, było widać coraz więcej. Powoli narysował się obóz, niedaleko za nim jakiś posterunek wojskowy. Wszystko otoczone złoto-czerwonym lasem. Na szlaku przed nim majaczyły sylwetki dwóch osób, jednak z tej odległości nie potrafił rozpoznać, kim byli ci ludzie. Zbliżając się dostrzegł po odzieży, że są to stalkerzy. Za chwilę, gdy byli już przy nim, rozpoznał w~nich Andreja i~dowódcę grupy, która zaatakowała gospodarstwo. To dziwne. Olo nadal nie wiedział jak ten gość ma na imię.
\pp
\sd
\xx Coś ty taki przybity? --~zatrzymał się Andrej.
\xx Aj\3k Wspomnienia. --~odparł smętnie Olo.
\xx Stary, coś o tym wiem. Też byłem w~woju, też musiałem się przyzwyczaić, że to nie jest wojna.
\qd
\hspace{6em} Dawid spojrzał na niego, widocznie zaintrygowany, jednak jego kolega sięgnął tylko do kieszeni, wyjął z niej setkę wódki, wcisnął ją na siłę Olowi do ręki, uśmiechnął się i~odszedł. Kart wzruszył ramionami i~posnuł się do obozu. Droga trwała dziwnie długo. Jednak po jakimś czasie dotarł na miejsce. Znalazł człowieka, który zlecił mu poszukiwania.
\pp
\sd
\xx I jak, masz coś?
\xx Tak, to. --~Olo wyjął z kieszeni PDA. --~Jeśli miałem znaleźć Jeńca, to go znalazłem. Upieczonego. --~dodał po chwili milczenia.
\xx Co?! Pokaż mi to!
\qd
\hspace{9.5em} Facet w~jasnej kurtce uruchomił sprzęt i~czytał. Raz, drugi, trzeci, aż w~końcu zrozumiał. Wbijając tępo wzrok w~ziemię oddał urządzenie Dawidowi i~wyciągnął z kieszeni trochę gotówki. Nie patrzył, ile tego jest. Wcisnął mu w~rękę i~odszedł, nadal patrząc w~dół. Olo przeliczył pieniądze. Wyszło sześćset rubli. Poszedł do swojego pomieszczenia, wyjął cały ekwipunek i~popatrzył, czym może zahandlować. Najpierw jednak postanowił się napić. Odkręcił małą buteleczkę i~szybko wypił całą jej zawartość. Stracił ochotę na robienie czegokolwiek. Położył się i~wpatrywał w~sufit. Może szukał tam jakiejś odpowiedzi. Nie wiadomo. Chwilę później poczuł w~organizmie przyjemny dreszcz. To alkohol zaczął krążyć we krwi. Nie patrzył już na nic. Uśmiechnął się, zamknął oczy i~usnął.\looseness-1
\section*{ZŁY SEN}
/mm Jednak, jak się potem przekonał, zapadnięcie w~sen czasem może być gorsze od przebywania w~Strefie. Najpierw ukazywały mu się pojedyncze obrazy. Scena wybuchu elektrowni, eksplodujący samochód, człowiek rozrywany przez anomalię z charakterystycznym sykiem powietrza. Mary nawiedzały go jak złe duchy. Potem przyszła kolej na dłuższe sekwencje: Witek wrzucający śrubkę w~anomalię, walka z żołnierzami pierwszego dnia pobytu w~obozie, ucieczka przed emisją, zwęglone zwłoki Jeńca przy tunelu, zabójstwo rannego bandyty, wtedy, w~gospodarstwie. Olo miał ochotę uciec od tych koszmarów, wstać i~pobiec. Jednak nie mógł, coś powstrzymywało go przed wyrwaniem się z objęć Morfeusza.

\hspace{1ex}Nagle, jak przez mgłę zaczął oglądać to, czego widzieć już nigdy nie chciał --~siebie, pijanego, wsiadającego za kierownicę samochodu.
Był 1 listopada 2003 roku. Razem z żoną i~córką wybrali się na święto zmarłych do krewnych, mieszkających niecałe dwadzieścia kilometrów od miasta, w~którym mieszkał Olo. Tam, oczywiście, wybrali się na groby, a~po powrocie zaczął się rodzinny obiad. Mnóstwo jedzenia, pogaduchy, muzyka, alkohol. Około szesnastej postanowił, że pora wracać do domu. Zabrał żonę i~dziecko, pożegnał się, wsiadł do samochodu i~odjechał. Nie był jakoś mocno wypity.
\dd
\sd
\xx Dwa, czy trzy kieliszki, co to było? --~pytał bardziej sam siebie niż swoją żonę.
\qd
\dd
Jechał ostrożnie, za oknem powoli przesuwał mu się świat układający się do zimowego snu. Złoto-brązowe liście, skąpane w~czerwonym blasku zachodzącego słońca, co rusz rozpłaszczały się o przednią szybę samochodu. Polska złota jesień, jak to mówią. Całą trójką podziwiali krajobraz. Nagle Dawid zorientował się, że prawa strona samochodu zaczyna zjeżdżać na pobocze. By uniknąć zderzenia z drzewem z całej siły skręcił kierownicą w~lewo. Samochód wpadł w~poślizg i~wylądował na przeciwnym pasie ruchu. W~drugą stronę jechało auto dostawcze. Na ominięcie pojazdu nie było już czasu. Ciężarówka wjechała w~bok samochodu osobowego z pełną prędkością.
\pp
W tym miejscu sen jakby się urywa. Słychać tylko niewyraźne krzyki, ciężkie posapywanie, syreny karetek, a~przed oczami co chwila błysk niebieskawego koguta. Po chwili słychać bardzo wyraźną eksplozję, a~niebieskawy kogut przechodzi w~czerwoną łunę ognia. Łuna ta stopniowo przeobraża się w~salę szpitalną. Olo leży na jednym z łóżek, otwiera oczy, stoi nad nim mężczyzna odziany w~fartuch. Dowiaduje się, że jego córka zginęła na miejscu, żona godzinę po przewiezieniu do szpitala, a~on sam leży z niegroźnymi obrażeniami na obserwacji.\\
Tak wściekły, jak w~tamtej chwili, nie był ani nigdy wcześniej, ani później. Zerwał się z łóżka, przewrócił je wraz z całą stojącą wokół aparaturą, lekarzem próbującym go zatrzymać zwyczajnie rzucił o ścianę. Drzwi popchnął tak mocno, że wypadły razem z futryną. Wszyscy w~budynku zaczęli schodzić mu z drogi. Wściekłość w~jego oczach było widać chyba ze stu metrów. Wyszedł ze szpitala prosto na ulicę. Padała mżawka. Liście, które dzień wcześniej tak podziwiał dziś były tylko
szarobure, tak samo jak wszystko dookoła. Czuł, jak cała złość z niego uchodzi.
Chwilę później usłyszał grzmot i~natychmiast się obudził.
\section*{EMISJA}
\sd
\xx Wstawaj, do kurwy nędzy, chcesz zginąć?!
\xx Ee? --~zdziwił się półprzytomnym głosem Olo.
\xx Emisja, spierdalaj do bunkra! --~krzyknął Witek i~próbował postawić go na nogi.
\qd
\dd
Dawid podniósł się, choć z niemałym trudem. Wódka wciąż krążyła w~jego żyłach. Prowadzony przez swego wybawcę dotarł do podziemnego pomieszczenia.
\dd
\sd
\xx Wybacz, że tak niegrzecznie, ale\3k sam rozumiesz powagę sytuacji. --~zagaił Witek, gdy czekali na zakończenie emisji.
\xx Nie ma sprawy. Jeszcze ci flaszkę za to postawię! Znów ratujesz mi życie --~westchnął jego podpity kolega.
\xx E tam, ratuję. Zwykła ludzka pomoc. Nie oczekuję za to medalu --~odparł skromnie.~--~A co ci się tak śniło, jeśli wolno mi spytać?
\xx Eh, wspomnienia. Zaraz cztery lata jak ich nie ma\3k
\xx Kogo, stary? --~pytał Witek, widać było, że dobrzy z nich byli kumple.
\xx Mojej żony i~córki\3k To przeze mnie\3k
\xx Rozu\3k
\qd
\hspace{4.8em} Witek próbował położyć rękę na ramieniu swojego kolegi, gdy tunelem wstrząsnęło mocno jak nigdy wcześniej, tak jakby ziemia zatrzymała się i~ruszyła z powrotem w~ułamku sekundy. Stalkerzy spadli z ławek ustawionych pod ścianami bunkra i~powpadali jedni na drugich. Wstrząsy powtarzały się. Towarzyszyło im złowrogie, niskie, tubalne buczenie jakby rozwiercające mózg od wewnątrz. Nie zrozumie tego nikt, kto tego nie przeżył. Mężczyźni leżeli teraz bezwładnie na ziemi i~ostatkami sił próbowali złapać się za głowy, by powstrzymać ich, jak im się wydawało, puchnięcie. Po jakimś czasie, usłyszeli głuche trzaśnięcie, poczuli kolejny wstrząs, jednak nie tak silny jak pierwszy. Skończyło się. Organizmy stalkerów powoli zaczęły wracać do życia. Tylko Olo wciąż leżał nieruchomo, zwinięty w~kłębek. Ktoś pochylił się nad nim i~sprawdzał, czy żyje. Żył.
\dd
\sd
\xx Witek, zostań z nim, my sprawdzimy, co na zewnątrz. --~rozkazał jeden z mężczyzn.
\xx Dobra, idźcie, jak coś, wołajcie mnie\3k Dawid, wstawaj! --~potrząsnął kompanem.
\qd
\dd
\mm
Gdy próbował go obudzić, pozostali wyszli na zewnątrz bunkra. Otoczenie nie zmieniło się znacznie. Jak zwykle, pod nogami walały się martwe wrony i~dzikie gołębie, ale to był typowy widok po emisji. O wiele bardziej stalkerów zaciekawiły dwie sylwetki błądzące na wzgórzu. Sieńko już ruszył w~ich kierunku, ale przywódca jednym ruchem zatrzymał go w~miejscu. Złapał chłopaka za ramię, obrócił go przodem do siebie i~z ukosa spojrzał mu w~oczy. Ten natychmiast zrozumiał, spuścił wzrok i~się cofnął. Po chwili starszy mężczyzna sięgnął za pazuchę swojego płaszcza i~wyciągnął dużą, radziecką lornetkę. Przyłożył ją sobie do oczu i~ustawił ostrość. Po chwili zwrócił się do Sieńki:
\dd
\sd
\xx Młodyś i~głupiś. Pamiętaj, że w~Strefie nie wolno ufać nikomu. Nawet sobie. Nawet tym pieprzonym śrubkom do oznaczania drogi. A tym bardziej jakimś dwóm błądzącym postaciom! Zobacz sobie, byś wlazł na dwóch zombie. --~podetknął Młodemu lornetkę pod nos.
\xx Zabijmy ich. --~ten odpowiedział niepewnie.
\xx A po co? Pchać się w~anomalie, ryzykować trafienie, marnować pestki? Te bezmózgi same się wykończą.
\qd
\hspace{10.5em}Przywódca schował lornetkę do wewnętrznej kieszeni płaszcza, ze swojego plecaka-kostki wyjął paczkę papierosów, wyciągnął jednego, podpalił go, po czym oparł się o ścianę, spod której doskonale było widać sylwetki na wzgórzu. Obserwował ze znudzoną miną. Nie musieli czekać nawet minuty, gdy jeden z zombie wlazł w~Wir. Nawet tutaj, osiemdziesiąt metrów dalej było słychać syk powietrza, jaki towarzyszył uaktywnieniu anomalii. Postać uniosła się do góry, okręciła wokół własnej osi i~w błysku jasnego światła została rozerwana na niezliczone fragmenty.
W tej chwili, do przyglądających się tej scenie stalkerów przyłączyli się Witek i~Olo.
\dd
\sd
\xx Już Ci lepiej? --~spytał przywódca.
\xx Tak. Nie wiem, co się stało. Może to przez tą fla\3k
\qd
\hspace{24em} Lecz niedane mu było tego dokończyć. Właśnie druga z bezmózgich istot wpakowała się na anomalię i~została zabita w~równie spektakularny sposób.
\dd
\sd
\xx Przez co? Nie dokończyłeś. --~nadal obojętnie mówił facet palący papierosa.
\xx Przez flaszkę. Rozpiłem setkę na lepszy sen, a~potem była emisja.
\xx Chłopie, ty jeszcze żyjesz?! --~wrzasnął nagle Andrej.
\qd
\hspace{24.5em} Dosłownie po sekundzie wokół Dawida pojawił się spory tłumek ludzi.
\dd
\sd
\xx Szybko, do kiosku! --~powiedział, już z emocjami, dowódca.
\xx O co chodzi? --~dziwił się Olo.
\xx Ostatnio jednego gościa zakopaliśmy, bo się schlał i~musieliśmy go zanieść do schronu\3k W~środku, w~czasie emisji, zaczął rzygać. Zahaftował nam cały bunkier, a~potem wyrzygał swój własny żołądek. Dosłownie, kurwa! Rozumiesz to?! Własny żołądek! --~zaczął krzyczeć Andrej na przemian z Awdanem. --~Nikt nigdy czegoś takiego nie widział. Ty byłeś blisko. To dlatego tak długo nie mogłeś się ocknąć. Cholerny farciarzu.
\qd
\hspace{9.6em} Zaszli do kantorku, Olo zdjął koszulę i~położył się na zimnym, dębowym stole, na którym zwykle leżały produkty spożywcze. Jeden ze stalkerów, Bitel, z wykształcenia pielęgniarz, próbował zbadać narządy wewnętrzne Karta. Z zestawem zaimprowizowanych narzędzi nie było to łatwe, jednak nie wykryto niczego nieprawidłowego, żadnych większych zmian nie było.
\pp
Tłum zmalał. Ludzie rozeszli się, dalej wykonywali swoje codzienne czynności --~czyścili broń, zbierali artefakty, znaczyli anomalie, polowali na zmutowane dziki i~inne niezbyt groźne zwierzęta. Olo nie potrafił zrozumieć tego całego zamieszania wokół siebie. Naprawdę chciał wiedzieć, czemu wszyscy tak zareagowali na fakt mówiący o tym, że sobie golnął, potem nastąpiła emisja, a~on wciąż żyje. Jednak w~głębi ducha czuł, że to pytanie do niczego sensownego go nie doprowadzi. A na pewno nie do obiecanego majątku. Teraz już będzie wiedział, kiedy może pozwolić sobie łyknąć gorzałki, a~kiedy nie. O ile nauczy się wyczuwać emisje własnym organizmem. Podobno niektórzy stalkerzy nabyli tą umiejętność, co jest im niezwykle przydatne. Ostrzeżenie o zwarciu, jak też nazywa się te anomalie pogodowe, zwykle nadchodzi w~ostatniej chwili, lub dopiero, gdy samo zjawisko już się rozpocznie i~jest właściwie za późno na reakcję. Odczuwanie zbliżającej się emisji ciałem, według zasłyszanych opowieści, jest zauważalne nieraz
dziesięć, a~nawet i~dwadzieścia minut przed punktem kulminacyjnym.
\dd
\sd
\xx Tym warto się zainteresować. --~powiedział sobie w~duchu.
\qd
\hspace{26.5em} Poszukał w~obozie Andreja. On był długo w~Strefie. Powinien wiedzieć coś na ten temat. Może nawet ma tą intrygującą umiejętność. Obozowicze, na pytania o niego, odpowiadali, że poszedł badać faunę i~florę. Nie byłoby w~tym nic niezwykłego, gdyby nie to, że Andrej badał je w~sposób, delikatnie mówiąc, makabryczny --~szukał w~krzakach jakiegoś zwierza, zabijał go za pomocą śrutówki, a~potem, wojskowym nożykiem odcinał jakiś element ciała. Czasem był to ogon, czasem ucho. Niekiedy przynosił kopyta bądź nosy. Mówił, że wysyła je instytutom, ale i~tak każdy zdawał sobie sprawę z tego, że to osobliwe trofea.
\pp
Ponadto, Andrej był kiedyś wojskowym. Było to dawno, prawie piętnaście lat temu, jednak podstawy i~doświadczenie nabyte w~czasie wojny pozostało, może nieco rozmyte, aż do dziś. Olo zaczął zastanawiać się, jak wyglądałaby jego podróż po Strefie, gdyby trafił najpierw nie na Witka, a~właśnie na Andreja. W~pewnym momencie w~głowie Karta zahuczał donośnie głos rozsądku. Nie, to nie było żadne zjawisko paranormalne ani występujące na tych terenach wynaturzenie. Po prostu w~trakcie zastanawiania się zaświtała mu pewna myśl: „Hej! Nie ma czasu na pierdoły. Miałeś pogadać z nim o odczuwaniu emisji ciałem. Śpiesz się!”
\pp
Szedł tak jeszcze przez chwilę. Nagle, przed starym, poradzieckim przystankiem, ukazała się znajoma osoba. To był Awdan. Wracał w~przykulonej pozycji, trzymając się za brzuch z lewej strony. Z jego nosa ciekła krew, a~czoło miał całe podrapane.
\dd
\sd
\xx Co tobie?! --~wykrzyknął Olo.
\xx Cholera, wybijaliśmy dziki, tak jak zwykle, aż tu z zagajnika wychodzi sześciu skurwysynów. Nawet nie krzyknęli, od razu zaczęli napierdalać!
\xx Kto taki? Uspokój się trochę.
\xx Ci pieprzeni bandyci! Postrzelili mnie, potem skopali, zabrali plecak i~zostawili dziesięć metrów od Grawi! Dobrze, że szybko się ocknąłem.
\xx Był ktoś z tobą?
\xx Andrej. Nie wiem, chyba go zabrali.
\xx Kurwa! Gdzie oni teraz mogą być?
\xx O tutaj pewnie siedzą te gnidy. --~Awdan wskazał na stare gospodarstwo, na którym Dawid miał wątpliwą przyjemność kiedyś być.
\xx Dasz sobie radę?
\xx No pewnie, nie z takimi gównami sobie radziłem.
\xx Dobra, trzymaj się! Idę po Andreja, przyślij mi kogoś jak dojdziesz do obozu.
\qd
\dd
\mm  Olo postał jeszcze chwilę, patrząc w~kierunku oddalającego się powoli kolegi. Następnie ruszył pod drzewko, by tam odpocząć sobie i~poczekać na kompana. Też był kiedyś żołnierzem. Wiedział, że pośpiech to w~takich sytuacjach najgorszy wróg. Zdjął plecak, rzucił go przy samym pniu. Z ramienia ściągnął karabin i~oparł obok siebie. Przykucnął, wyjął termos, bułkę i~jakąś kiełbasę, po czym usadowił się cicho na swojej torbie. Nalał herbaty, połamał bułkę na trzy części, ugryzł trochę mięsa. Po chwili wziął w~rękę kawałek pieczywa i~włożył sobie do ust. Wszystko popił ciepłą herbatą. Wtem spojrzał dokładnie na gospodarstwo i, uwzględniając każdą szczelinę w~płocie, zaczął obmyślać plan odbicia kolegi. Miał trzy opcje: czekać do nocy i~zakraść się tam, nie powodując hałasu, siedzieć pod drzewem i~czekać na kolegę, który pomógłby mu zabrać Andreja do obozu oraz iść na żywioł, samemu.
\pp
Pierwsza opcja odpadła właściwie od razu, uwięziony mógł do nocy po prostu nie dożyć. Ci bandyci to totalni psychole. Potrafią dorwać człowieka, pozbawić go wszystkich dóbr. Dosłownie wszystkich. Czasem przy wieczornym ognisku można usłyszeć historię o jednym ze stalkerów, który natrafił na małą, trzyosobową grupę. Dorwali go, pobili, złamali rękę, czy wybili bark, krąży kilka wersji tej opowieści. Ponadto zabrali plecak, artefakty, pieniądze, broń, kurtkę, buty. Już mieli go odesłać do swoich, gdy ten coś do nich pysknął. Przywódca szajki ponoć tak się wściekł, że rzucił samotnikiem o asfalt, a~następnie twarzą stalkera zaczął z furią trzeć o chropowatą nawierzchnię. Potem zdarli z gościa koszulę i~spodnie. Nawet bokserki i~skarpetki kazali mu wyrzucić! Chwilę po tym jeden z nich wycelował karabin w~stronę stalkera i~kazał spierdalać w~podskokach. Gość przebiegł jakieś dwadzieścia metrów, gdy zbir zaczął zwyczajnie strzelać mu w~nogi. Jednak nie trafił ani razu. Jeśli potrafią zrobić coś takiego, to jakim
problemem byłoby dla nich zakatowanie Andreja na śmierć?
\pp
Druga propozycja uratowania kolegi, to oczekiwanie na przybycie kogoś z obozu. Jednak ten plan też mógł w~bardzo łatwy sposób spalić na panewce. Po pierwsze --~skąd pewność, że Awdan da sobie radę i~dojdzie szybko do bazy? Po drugie --~nawet jeśli uda mu się tam dotrzeć, skąd wiadomo, że będzie pamiętał o zaleceniu Ola, albo czy ktoś faktycznie wyruszy mu na pomoc? Nie, to było zbyt ryzykowne. Została trzecia opcja.
\section*{GOSPODARSTWO}
\mm Kart odetchnął głęboko i~jeszcze raz uważnie obejrzał budowle i~stary, trochę przegnity płot. Ostatnio z bliska przyglądał się temu miejscu miesiąc temu, drugiego dnia pobytu w~Strefie. Data wskazywana przez PDA tego dnia to 15.09.10. Dzisiaj, gdy jest tutaj ponownie, jest 13.10.10, na szczęście, w~kalendarzu wypada sobota, a~nie pechowy piątek.
\pp
Dawid znał i~pamiętał ten teren dobrze. Dwadzieścia metrów z przodu jest wyrwa w~płocie, akurat przy ścianie stodoły. Ostatnim razem to właśnie od tej strony przeprowadzał atak. Trochę w~lewo od wyrwy był róg gospodarstwa, dziesięć metrów za nim była kolejna dziura w~ogrodzeniu. Wchodząc tamtędy, miało się przed sobą zachodni brzeg stodoły, a~po lewej długą halę, w~której kiedyś zostawiano na noc maszyny rolnicze. Obecnie jest to ulubione miejsce bandytów, którzy jakimś cudem posiadają inną broń, lub zmodyfikowanego samodzielnie Vipera, z przyczepioną lunetą. Halę przykryto poziomym dachem, z którego było widać niemal całą okolicę. Na północ dało się zauważyć most, z posterunkiem wojskowym. Nieco dalej, po drugiej stronie drogi stał rozwalający się młyn i~dwa magazyny. Za elewatorem, przy pomocy lornetki, albo jeśli ktoś posiada wprawione oko, można dostrzec wejście do tunelu pod nasypem kolejowym. W~trawie, na początku podziemnego przejścia straszy i~ostrzega ciało Jeńca, który próbował przedostać się na
drugą stronę. Spoglądając w~przeciwnym kierunku widać obóz stalkerów, a~przynajmniej kominy chat. Przy drodze stoi stary przystanek autobusowy, a~na lewo od niego, zagajnik, w~którym w~tej chwili przebywał Olo. Na wschód od hali znajduje się główna i~jedyna brama prowadząca do gospodarstwa. Za nią znajduje się mały budynek mieszkalny, w~którym, od wschodniej strony, brakuje kawałka ściany. Na niewielkim placu oddzielającym budynki, pod dachem hali, stoi jeszcze stary, przerdzewiały traktor, któremu brakuje jednego koła.
Andrej jest przetrzymywany w~którymś z tych budynków. Teraz tylko należało wybić bandytów, a~potem przeszukać pomieszczenia.
\dd
\sd
\xx Pestka. --~powiedział Olo sam do siebie, z wyczuwalną ironią w~głosie.
\qd
\dd
\hspace{1ex}Tym, co odróżniało tę wyprawę od poprzedniej była przewaga przeciwnika. We wrześniu stalkerzy byli w~czteroosobowej grupie i~mieli przeciw sobie tyle samo bandytów, w~czym jeden ranny, a~drugi niemal nieprzytomny. Teraz Dawid jest jeden, na sześciu przeciwników. Jednak, przy odrobinie szczęścia, powinien dać sobie z nimi radę.
\pp
Wstał powoli i~rozejrzał się. Następnie z plecaka wyciągnął butelczynę. Odkręcił, przysunął do ust, przymknął oczy i~przechylił się całym ciałem do tyłu. Połknął płyn, który wlał sobie do gardła. Po chwili, od stóp do głów przeszedł go zimny dreszcz, tak jak lubił. Zakręcił dokładnie flaszkę i~włożył ostrożnie do plecaka, który zaraz potem zarzucił sobie na ramię. Przewiesił przez nie również karabin. Ugiął lekko nogi i~powoli, wyostrzając zmysły ruszył w~stronę zabudowań. Wszedł przez tą samą wyrwę, przez którą wchodził wcześniej. Sprawdzone pomysły podobno nigdy nie zawodzą.
\pp
Pierwszym zadaniem było sprawdzenie ile osób faktycznie ma przeciwko sobie. Na prawo, przy ceglanym domu z dziurawą ścianą klęczał człowiek. Na szczęście, nie spostrzegł Ola. Był odwrócony doń plecami i~chyba rozpalał ognisko. Dawid skierował się na zachód, idąc ostrożnie przy elewacji. Na końcu wychylił się zza rogu, jednak nikogo tam nie widział. Ruszył dalej. Minął drugą wyrwę i~za chwilę ponownie był na skraju stodoły. Wychylił się jednym okiem i~zauważył trzech bandytów. Jeden stał na traktorze i~chyba ukrywał coś w~schowku pod siedzeniem, drugi rozmawiał z nim i~przytrzymywał jakąś blachę, trzeci kręcił się po dachu hali i~chyba budował z worków z ziemią jakieś umocnienia. To już cztery osoby, które trzeba wyeliminować. Jeśli wierzyć w~to, co mówi Awdan, w~domu powinno być ich jeszcze dwóch. No i~oczywiście, Andrej.
\pp
Olo przylgnął do ściany. Odbezpieczył karabinek, sprawdził czy ma pełny magazynek, chrząknął i~wychylił się zza budynku. Czekał na dogodny moment. Chciał załatwić wszystkich naraz, jednak mężczyzna na dachu gdzieś zniknął. Kart stał tak kilka sekund, z okiem przyłożonym do celownika. Opłaciło się, zauważył na krawędzi czuprynę tamtego gościa. Przesunął palec na spust, a~lufę skierował na mężczyzn przy traktorze. Zacisnął palec, a~po chwili chłopak znajdujący się wyżej runął w~dół z maszyny, wprost na swojego kolegę. Nie minęła sekunda, a~ten także leżał już bez życia. Typ na dachu był chyba jakimś młodzikiem, bo, pewnie ze strachu, zeskoczył wprost w~stertę pudeł i~jakichś poobcinanych blach. Widok był tak zły, że nawet Ola, trepa z doświadczeniem, trochę to zniesmaczyło i~odwrócił wzrok. No i~dobrze się stało. Od strony stodoły biegł na niego jakiś facet z nożem. To pewnie ten, który wcześniej próbował rozpalić ogień. Dawid bez mrugnięcia okiem uderzył go kolbą, a~gdy ten padł na ziemię, Olo złapał go za
włosy, przydusił do ściany i~kilkukrotnie, z całej siły, uderzył nim o elewację. Następnie złapał nóż i~wbił swemu przeciwnikowi w~klatkę piersiową.
\pp
Złapał ponownie do ręki swojego Vipera, odczekał chwilę i~ruszył w~kierunku, z którego wybiegł przed chwilą typ z nożem. Ocierając się o ścianę dotarł do przeciwległego rogu budynku. Stąd miał widok na stary dom wzniesiony z cegły. Podejrzewał, że to właśnie w~nim przetrzymują Andreja. Stał chwilę, celując w~drzwi frontowe. Czekał na jeszcze dwóch bandytów, o których wspominał Awdan. Istotnie, po paru sekundach na zewnątrz wychyliła się głowa gościa w~czarnej skórzanej kurtce. Typ natychmiast spostrzegł Ola celującego w~jego stronę i~próbował oddać strzał. Jednak stalker był szybszy. Przeciwnik zsunął się bezwładnie z framugi, o którą wcześniej się opierał i~masą całego ciała, spadł na ziemię z wielkim impetem. Olo stał i~czekał na ostatniego. Jednak po minucie doszedł do wniosku, że jego ranny kolega pomylił się licząc wrogów. Śmiało, aczkolwiek nadal z wyostrzonymi zmysłami ruszył do mieszkania. Był już przed drzwiami, gdy nagle pociemniało mu w~oczach, poczuł na plecach przenikliwy ból i~padł nieprzytomny.
 W~gospodarstwie jednak była jeszcze jedna osoba. Awdan wcale się nie pomylił, zapamiętując wrogów. To zbyt pewny siebie w~tej chwili Dawid stracił cierpliwość. I Strefa go za to ukarała.
\section*{STARY ERNEST}
\mm Ocknął się po pewnym czasie. Nie był w~stanie określić ani, która jest godzina, ani gdzie się teraz znajduje. Wiedział jedno --~jechał na pace jakiejś zdezelowanej ciężarówki, zakneblowany, ze związanymi rękami. Przed sobą widział jakąś postać. Był jednak zbyt otumaniony, by dostrzec, kto to jest. Wiedział jedynie, że mężczyzna leżący obok ma zasłonięte oczy i, podobnie jak on sam, związane z tyłu ręce. Olo z trudem przechylił głowę w~drugą stronę i, przez powiewającą płachtę przykrywającą pakę spostrzegł, że zmierzcha. Lub świta, tego nie był jeszcze pewien.
Jechali długo, prawie godzinę. Nie tak łatwo poruszać się po Strefie pieszo, a~co dopiero taką wielką ciężarówką. W~pewnej chwili samochód zatrzymał się. Dawid postanowił grać nadal nieprzytomnego. Usłyszał, że drzwi szoferki otwierają się, a~zaraz po tym na asfalcie zastukały ciężkie, wojskowe buty. Najpierw jedna para, potem druga, potem trzecia. Co chwila ktoś zaglądał do ciężarówki i~głośno się śmiał. Wtem, wśród zgiełku, Olo usłyszał przytłumioną rozmowę.
\dd
\sd
\xx Ilu ich? --~zapytał ktoś grubym basem.
\xx Dwóch.
\xx Skąd?
\xx Obóz kotów.
\xx Doskonale\3k Kto konkretnie?
\xx Nazwisk nie ustalono, wiemy tylko, że nazywają się Witek i~Olo. Jeden z nich jest Polakiem, ale nie udało nam się dotrzeć do informacji, który.
\xx \textit{Palak, da?} Zaprowadźcie ich do C-23, sam z nimi porozmawiam.
\xx Tak jest!
\qd
\hspace{5.5em},,Kurwa, żołnierze?!'' --~zaklął w~duchu Olo. Zaczął się obwiniać. To przez niego teraz Witek znalazł się w~tarapatach. A mógł zaczekać, rozegrać to spokojniej, we dwójkę\3k Przecież Awdan na pewno dałby sobie radę. Po co mu był ten pośpiech? Pogadać mu się o emisjach zachciało.
\pp
W tym momencie drzwiczki szoferki głośno zatrzasnęły się. Samochodem zakołysało, z rur wydechowych popłynął niski odgłos silnika i~pojazd przejechał przez jakiś krawężnik lub bramę. Przemieścili się może o sto metrów, co chwila gdzieś skręcając. Wtem maszyna zatrzymała się, a~do środka wszedł mężczyzna. Żołnierz. Trzasnął Ola kilka razy w~twarz, ale ten nie reagował. Podszedł do Witka i~szturchnął go nogą. Ten też nie dał żadnego znaku życia. Za chwilę na pakę wgramoliło się jeszcze trzech wojskowych, dwóch zabrało Ola, a~dwóch pozostałych Witka.
\pp
Żołnierze nie obchodzili się z nimi zbyt ostrożnie. Obu wyrzucono z samochodu jak worki pełne ziemniaków. Dopiero, gdy leżeli na wilgotnym asfalcie zostali podniesieni i~przeniesieni do jakiegoś budynku, jednak Olo nie mógł określić ani w~którym kierunku względem pojazdu się poruszają, ani na jakim piętrze obecnie się znajdują. Wiedział natomiast jedno --~musiał szybko oswobodzić siebie oraz Witka i~dać nogę z tego miejsca. Jeśli to naprawdę wojskowi, to stalkerzy nie będą mieli tutaj łatwo. Niezależnie od miejsca, w~którym się ich spotkało, armia nie patyczkowała się z napotkanymi osobami, szczególnie, jeśli były one w~Strefie nielegalnie.
\pp
Po paru chwilach usłyszał, że pod ciężkimi butami chlupie woda. Był niemal pewien, że zeszli w~piwnicy. Więźniów niesiono jeszcze przez kilka metrów, po czym rzucono na mokrą posadzkę. Niedługo potem szczęknął zamek w~drzwiach. Stalkerów wciągnięto do pomieszczenia za ręce i~usadzono na jakichś starych krzesłach. Chwilę potem zostali doń przywiązani. Olo jednak wiedział, jak należy postępować w~takiej sytuacji. Ułożył dłonie w~wyuczony sposób, tak by móc szybko się oswobodzić i~wyciągnąć stąd swojego kompana
\pp
Mundurowi wyszli, zostawiając stalkerów samych sobie. Kart wreszcie otworzył oczy. W~pomieszczeniu panował półmrok. Po chwili jednak wzrok stalkera przyzwyczaił się do ciemności i~ten mógł ocenić swoją sytuację. Miał przed sobą gołą ścianę, do której pozaczepiane były jakieś rury i~coś w~rodzaju systemu wentylacyjnego. Wszystko było zardzewiałe, na betonowej powierzchni porobiły się rudobrązowe zacieki. W~kącie, pod samym sufitem była ponadto mała kratka wentylacyjna. Spojrzał w~dół. Brudna woda sięgała nieco ponad kostkę. Szukał wzrokiem swojego kolegi, jednak nie mógł go dostrzec. Szepnął. Za plecami usłyszał ciche jęknięcie. Był odwrócony do Witka tyłem.\looseness+1
\pp
Próbował oswobodzić ręce. Udało mu się, jednak w~tej chwili ktoś pociągnął za klamkę. Do pomieszczenia weszły dwie osoby. Jeden z mężczyzn był niski i~dość pulchny, z twarzą pokrytą bruzdami, przypominającą trochę starego ziemniaka. Na głowę miał wciśniętą czapkę z radziecką gwiazdą. Za nim wszedł wysoki gostek, w~wieku zbliżonym do Ola. Ten był w~typowym mundurze wojskowym. Na czoło miał naciągnięty hełm, a~w~ręku trzymał Kałasznikowa. Po chwili grubszy mężczyzna nachylił się nad samotnikiem.
\dd
\sd
\xx \textit{Prywiet, stalker!} --~zawołał basem. To był ten sam człowiek, który rozmawiał z żołnierzami, gdy więźniów przywieziono do bazy.
\xx Odczep się, ruska gnido. --~odpowiedział wściekły Olo.
\xx Tak do oficera? Nieładnie, stalker, nieładnie. --~pokiwał przecząco głową.
\xx Jurij! --~zawołał.
\qd
\hspace{8.3em} W~tej chwili drugi żołnierz podszedł do Karta, złapał go za brodę, a~następnie uderzył z całej siły twarz. Więzień omal nie spadł z krzesła.
\dd
\sd
\xx I co, stalker? Nadal takiś hardy? --~spytał oficer.
\xx Pieprz się.
\xx Przyjdę później, stalker. Widzę, że jesteś nieskory do rozmów. Jurij, spróbuj nakłonić więźnia do współpracy.
\qd
\hspace{11.4em} Gruby mężczyzna wytoczył się z pokoju i~zamknął drzwi. Olo obrzucił go pogardliwym spojrzeniem, a~następnie w~podobny sposób popatrzył na żołnierza. Ten jednak tylko się uśmiechnął, powiesił broń na haku wystającym ze ściany i~wyciągnął papierosa.
\dd
\sd
\xx Chcesz? Trochę tu posiedzisz. --~zagaił.
\xx Goń się. Nie gadam z trepami.
\xx Chciałem być miły. --~wzruszył ramionami.
\qd
\hspace{20em} Odpalił fajkę i, paląc, delektował się dymem. Co jakiś czas, wypuszczając z ust siwą smugę drapał się leniwie po brodzie. Wreszcie wyrzucił wypalonego szluga do wody.
\dd
\sd
\xx Ech, stalker\3k Powiesz coś ciekawego panu oficerowi, albo będzie źle.
\xx Zobaczymy, dla kogo.
\qd
\hspace{11em}Żołnierz zaśmiał się głośno, a~potem uderzył Ola. W~każdym razie, próbował. Kart bowiem bardzo szybko zablokował cios oswobodzonymi dłońmi, a~potem błyskawicznie pociągnął przeciwnikowi z główki. Ten zaklął siarczyście i~złapał się za nos. Stalker nie przestawał szarżować. Rzucił go w~wodę i~zaczął tłuc. Bił, póki gość w~mundurze nie przestał oddychać. „Albo Ty, albo oni” wołał głos w~jego głowie. Następnie ruszył do swojego przyjaciela, rozwiązał go i~zdjął opaskę z oczu. Witek nie wyglądał dobrze. Półprzytomny, ze złamanym nosem, wybitymi kilkoma zębami, wyczerpany.
\dd
\sd
\xx Trzymasz się jakoś, stary? --~zapytał Olo.
\xx Bywało lepiej. --~podniósł się i~jęknął.
\xx Co jest?
\xx Kurwa, chyba złamali mi żebro --~poskarżył się Witek.
\xx Chuje! Już ja im dam! --~Olo zaczął wpadać w~furię.
\xx Uspokój się, na razie to musimy stąd zwiać.
\xx Racja. --~odparł Kart. --~Jak myślisz, da się tędy? --~wskazał głową kanały wentylacyjne.
\xx Można spróbować, choć bardzo ostrożnie, żeby się nie urwało.
\xx No to dawaj, dasz radę mnie podsadzić?
\xx Nie bardzo\3k
\xx Dobra, podstawię sobie krzesło.
\qd
% \hspace{15.3em}
Wziął zbutwiałe siedzenie i~podstawił pod rurę. Rękami zdjął kratkę i~położył ją ostrożnie na ziemi. Wczołgał się do prostokątnego otworu.
% \dd
\sd
\xx Witek, podaj karabin, może się przydać.
\xx Dobra. Ja wezmę pistolet i~nóż.
\qd
\hspace{15em} Olo wyczuł metal uwierający go w~plecy, złapał lufę ręką i~przesunął sobie przed głowę.
\dd
\sd
\xx Masz latarkę?
\xx Coś się chy\3k --~zawiesił na chwilę głos. --~O, tu jest.
\xx Dawaj. --~rozkazał Kart.
\qd
\hspace{11.8em} Wziął również urządzenie, a~potem kazał kompanowi wdrapać się za nim do kanału. Czołgali się, oświetlając sobie drogę latarką. Wewnątrz konstrukcji biegało kilka szczurów, ale uciekały widząc oślepiające światło.
Po kilkunastu metrach znaleźli się na końcu systemu. Olo poświecił, by mieć jakieś rozeznanie, jednak to co zobaczył to nie było to, czego się spodziewał. Pod sobą miał Galaretę, musiał wybić się z tunelu bardzo daleko, by do niej nie wpaść. Pod ścianami widać było zielone łuny, to nie była jedyna anomalia tutaj. Ponadto, zamiast betonowej podłogi zastał gołą ziemię.
% \dd
\sd
\xx Kurwa\3k --~zaklął cicho Witek.
\xx Co jest?
\xx Jak mnie pamięć nie myli, to jest Stary Ernest.
\xx Co to jest Stary Ernest?
\xx Tunel, nazwany tak na cześć jego odkrywcy. Później ci opowiem. Teraz ważne jest to, że w~środku może być pijawka. I to nie jedna.
\xx Co to pijawka? --~Olo jeszcze ani razu nie spotkał się z mutacjami w~Strefie, pomijając dziki i~psy, które nie różniły się znacząco od ich odpowiedników z innych miejsc na Ziemi.
\xx Stwór, który zamiast zębów ma macki, którymi przysysa się ci do szyi i~wysysa całą krew. W~łapach i~na stopach ma pazury, a~w~dodatku te kurwy potrafią znikać.
\xx Co ty pierdolisz?! Da się to to zabić? --~Kart poczuł ciarki na plecach.
\xx Da, o ile będziesz trzymał dystans, miał dużo pestek i~spluwa ci się nie zatnie. Ewentualnie, możesz liczyć na fart i~rzucić nożem, ale suka szybko ucieka.
\xx Cholera jasna\3k Obyś się mylił. --~powiedział Olo i~wyskoczył z tunelu.
\qd
\mm Po chwili również Witek wygramolił się z kanału powietrznego. Kart złapał Kałasznikowa zdobytego od żołnierza i~powoli ruszył do przodu. Kolega poszedł za nim. Szli uważnie rozglądając się na wszystkie strony. W~pewnym momencie Witek odezwał się do swojego wybawiciela:
\dd
\sd
\xx Pamiętaj, w~Strefie jest o wiele więcej zła niż widziałeś do tej pory. Będę cię ostrzegać.
\qd
\dd
\mm
Zaczął tłumaczyć kumplowi o rozmaitych anomaliach, takich jak spalacz, spalony puch, czy karuzela. Sposobu działania tej pierwszej Olo domyślił się, karuzela była bardzo podobna do widzianego już niejednokrotnie Wiru. Inaczej było ze spalonym puchem. Ten rodzaj wynaturzenia nie był zbyt często spotykany w~Strefie. Najczęściej pojawiał się w~porzuconych budynkach, starych bunkrach, laboratoriach czy podziemnych halach. W~istocie była to zmutowana roślina, która atakowała ruchome obiekty wystrzeliwując w~ich kierunku chmarę drobnych igiełek. Te zaś mocno wbijały się w~słabo lub niechronione części ciała. Spalony puch dało się obejść jedynie szukając innej drogi lub czołgając się bardzo powoli, uważając by nie zahaczyć o któryś z licznych pędów.
\pp
Witek wyjaśniał też działanie spalacza. Gdy jest nieaktywny, to jest widoczny jako obłok rozgrzanego, falującego powietrza. Jeśli jednak zostanie uaktywniony przez kontakt z czymkolwiek materialnym natychmiast rozgrzewa się do temperatury wielu tysięcy stopni Celsjusza, a~ciała, które znajdą się w~tej fali gorąca zajmują się żywym ogniem. W~pobliżu anomalii słychać syk, jakby stała tam kuchenka gazowa z uruchomionym palnikiem.\looseness-1

Stalker będący w~Strefie dłużej zabierał się właśnie za opisywanie artefaktów, które w~ów anomaliach powstają, jednak niedane było mu dokończyć. W~korytarzu przed nimi coś ryknęło i~upadło. Brzmiało, jakby ktoś przewracał drewniane skrzynie bądź beczki. Olowi przebiegło tylko przez myśl: „Pijawka”. Kucnął, położył latarkę na ziemi, skierował ją w~stronę, z której było słychać odgłosy, przyłożył oko do lufy i~zastygł w~bezruchu. Witek stanął nad nim, celując z pistoletu.
\pp
Nagle z zagłębienia w~korytarzu wyszła jakaś zniekształcona, humanoidalna postać. Zwróciła swe białe ślepia w~kierunku stalkerów, ryknęła i~ruszyła w~ich stronę. Chwilę potem zniknęła, choć na ziemi było widać pojawiające się ślady.
\dd
\sd
\xx Napierdalaj! --~rozdarł się Witek i~otworzył ogień. Chwilę za nim Olo.
\qd
\hspace{32.55em} Strzelali jak oszalali, jednak bestia była coraz bliżej. Gdy zostały jej może dwa susy, Witek rzucił pistoletem w~jej kierunku, a~następnie wydobył z kieszeni nóż i~zamachnął się w~to samo miejsce. Pijawka pojawiła się tak szybko, jak zniknęła, po czym bezwładnie runęła na ziemię, zaczepiając głową o bark Ola. Stalkerzy spojrzeli po sobie. Obu ręce trzęsły się jakby chorowali na Parkinsona.
\pp
Witek obrócił bestię na plecy i~przyjrzał się jej uważnie. Włochate ciało, białe, przekrwione, pozbawione tęczówek oczy, wielka, czarna jama zamiast ust, z brody wystaje szereg macek. Nieproporcjonalne długie ręce zakończone szponami i~silnie umięśnione nogi. Straszna rzecz. Jednak, według stalkera, pijawki to jedne z popularniejszych mutantów w~Strefie. Gorsze były snorki, kryjące się w~laboratoriach i~burery, zamieszkujące jakieś jaskinie i~naziemne bunkry.
\pp
Po chwili wyciągnął ostrze wystające z czoła pijawki i~odciął jej wszystkie macki, po czym wrzucił je do jakiejś torby, którą miał w~kieszeni.
\dd
\sd
\xx Ze stówa za to będzie. Chuje wszystko mi zabrali.
\xx To tak jak mi. --~odpowiedział Olo.
\xx Choć, poszukajmy artefaktów. Trzeba jakoś zarobić na pukawki.
\qd
\mm Zostawili martwego mutanta tak jak leżał i~ruszyli do przodu, pilnując, by nie wleźć w~żadną anomalię, nasłuchując podejrzanych dźwięków i~rozglądając się za naturalnym bogactwem Strefy. Wreszcie, po kilkunastu minutach błądzenia udało się. Gdzieś wśród zielonej papki świecącej lekkim światłem Witek dojrzał małą, skaczącą w~górę i~w dół plamę, jakby rozlanego oleju. Jednak plama ta była w~stanie stałym. Wyciągnął gumową rękawiczkę i~ostrożnie złapał artefakt. Wyrzucił go do góry najdalej od ściany, by nie poparzyć się w~Galarecie. Następnie szybko zerwał kawałek gumy z dłoni. Jego zabezpieczenie stopiło się trochę.
\pp
Odczekał kilka chwil, by cały kwas bezpiecznie ściekł na ziemię, a~następnie podniósł bladozieloną taflę do góry. Ręka, również nieco poparzona, natychmiast się zagoiła i~rozchodził się z niej po całym ciele przyjemny chłód. Nagle z anomalii dało się słyszeć odgłos. Coś, jakby burczenie w~brzuchu. Parę sekund potem wyskoczył z niej artefakt zwany przez stalkerów Ślimakiem, a~zaraz po nim dwa o nazwie Śluz. Oba miały tą samą funkcję, wspomagały krzepnięcie krwi, jednak kosztem mniejszej odporności na różnego rodzaju poparzenia. Śluz był najpopularniejszy, jeśli chodzi o Galarety. Anomalie, zwłaszcza niedługo po emisjach wyrzucały ich dziesiątki. Były bardzo tanie. Przy dobrych umiejętnościach dało się za nie wytargować około tysiąca rubli, jednak umowną ceną rynkową była połowa tej kwoty. Ślimak pojawiał się nieco rzadziej, dobry stalker potrafił za jednego Ślimaka ściągnąć nawet trzy tysiące.
\pp
Co do pierwszego artefaktu, tej tafli, to Witek nie potrafił określić, co to takiego. Domyślał się, że formacja ta jest rzadka i~pewnie ktoś jest za nią gotów słono zapłacić, ale bez PDA nie potrafił podać ani jej dokładnych właściwości, ani ceny. Przeszło mu przez myśl nawet, że taki artefakt nigdy wcześniej nie był widziany i~warto by się udać do naukowców. Ci gotowi byli wydać równowartość nowego samochodu za kilka wartościowych znajdziek.

Pozbierali jeszcze kilka formacji, żeby poza tym, co w~Strefie niezbędne, kupić także coś do jedzenia. Pochowali wszystko po kieszeniach, ponieważ torby i~wszystko inne zostało odebrane w~bazie przez wojskowych. Wyszli do małego korytarza o zaokrąglonych ścianach. Na jego końcu, przy błyskach starej żarówki wiszącej tu chyba od czasów katastrofy, widać było właz prowadzący na zewnątrz systemu tuneli. Stalkerzy ruszyli w~jego kierunku, ale w~pewnym momencie ich uwagę przykuł płacz dziecka dobiegający zza zamkniętych drzwi. Olo pociągnął za klamkę i~włożył głowę do pomieszczenia. Nagle, z ciemnego kąta wyskoczyła karłowata postać w~czarnym, podartym płaszczu. Uciekinierzy ponownie musieli otworzyć ogień. Strzelali, jednak pociski odbijały się od postaci, która powoli się doń zbliżała. W~pewnym momencie, upiór wyrwał Kartowi broń, używając jedynie siły umysłu. Witek w~tym momencie ruszył do mutanta i, osiągając pełną prędkość, zagłębił nóż w~jego czaszce. Wszedł cały, nawet z kawałkiem rękojeści. Bestia ryknęła
i padła na ziemię.
\dd
\sd
\xx Co to, kurwa, było?!
\xx Burer. Kurwa, do tej pory tylko o nich słyszałem. Ktoś w~obozie mówił, że załatwił jednego nożem, to też spróbowałem.
\xx Co on, telekineza? --~dziwił się Dawid.
\xx Na to wygląda. Chodź, nie wiadomo, czy dziad nie zmartwychwstanie.
\xx Masz rację, spadajmy stąd.
\qd
\hspace{13.2em} Wyszli po starej drabinie na powierzchnię i~natychmiast ruszyli w~poszukiwaniu jakiegoś dachu nad głową. Padał deszcz, poza tym, byli wyczerpani. Szli po omacku, co kilka chwil włączając latarki, by upewnić się, że idą w~dobrym kierunku. Znaleźli jakąś ceglaną chatkę, weszli do niej i, używając fragmentów drewnianej podłogi, rozpalili sobie ogień. Zasnęli szybko.
\section*{100 RADÓW}
% \vspace*{1em}
\mm Obudzili się. Na zewnątrz było już całkowicie widno. Chmur po wczorajszej ulewie zostało tylko kilka. Porobiło się mnóstwo kałuży i~błota. Kumple pogadali trochę o wieczornej przygodzie i~wyszli z budynku, ustalić, gdzie tak właściwie są.

Po kilku minutach błądzenia po bagnach doszli do jakiejś drogi.
Na horyzoncie widzieli poruszającą się w~ich kierunku postać. Postanowili zaczekać i~spytać ją o drogę. Nie mieli jednak pewności, czy to sprzymierzeniec, czy wróg.%\looseness+1
\pp
Po chwili człowiek zbliżył się do nich wystarczająco, by mogli go rozpoznać. Typowy samotnik, w~ręku obrzyn, stara, podarta kurtka, z kieszeni wystaje zniszczona maska gazowa, na plecach zarzucona jakaś własnoręcznie wykonana torba.
\dd
\sd
\xx Cześć. Słuchaj, wracaliśmy wczoraj z łupów, ale dorwali nas jacyś bandyci.
Skurwiele nas skroiły i~wywlekły gdzieś tutaj. --~zagaił Witek. --~Jest tu w~okolicy jakiś obóz?
\xx Hoho\3k To jacyś łaskawi dla was byli. --~zdziwił się samotnik --~Niecały kilometr stąd, za tym laskiem, jest bar „100 Radów”. Pewnie słyszeliście o nim, no chyba, że jesteście kompletnie nowi.
\xx Nie, nie, no coś ty. Dzięki. Damy sobie radę. A, w~ogóle, ja jestem Witek, a~to jest Olo.
\xx Janka Tytan. Miło mi was poznać.
\xx Słuchaj, bądź jutro o tej samej porze w~barze, dasz radę?
\xx Postaram się.
\xx Okej, na razie. --~machnął ręką Witek i~ruszył w~kierunku wskazanym przez samotnika.
\xx Tak właściwie, która jest teraz godzina? --~wtrącił się Olo.
\xx Haha, dobrze was skroili! --~zaśmiał się Janka --~9.36, proszę państwa.
\xx Dobra, dzięki! Powodzenia, stalkerze! --~odpowiedział Olo i~ruszył za kompanem.
\qm
Po dziesięciu, może dwudziestu minutach dwójka kumpli dotarła wreszcie do bram obozowiska. Na workach z ziemią siedziało trzech stalkerów w~dziwnych mundurach. Nie byli to wojskowi, wyglądali też zbyt dobrze, jak na typowych samotników. Witek wyjaśnił, że to Powinnościowcy.

Powinność była jedną z frakcji. Ich celem była ochrona świata zewnętrznego przed niebezpieczeństwami, z jakimi można się było spotkać w~Strefie. Znalezione artefakty sprzedawali tylko naukowcom, zajmowali się odstrzeliwaniem mutującej zwierzyny. Jej członkowie słyną z dyscypliny, zawsze postępują według określonego kodeksu.
\pp
Przeciwieństwem Powinności była Wolność. W~swoje szeregi gromadziła głównie bandytów, anarchistów, śmiałków i~wszystkich innych, którzy walczyli o swobodny dostęp do Strefy i~zniesienie monopolu Ukrainy na jej cuda i~sekrety.
\sx Nie ociągać się! --~zawołał mężczyzna z Obokanem z lunetą.
\xx Spokojnie, jesteśmy samotnikami. Przyszliśmy do Barmana. --~odpowiedział Witek i~lekko wyszczerzył zęby.
\xx Nie właź mi w~dupę, tylko idź szybciej. --~odparł tamten.
\qd
Stalkerzy odeszli kilkanaście metrów, po czym Olo zapytał:
\sx Powinnościowcy zawsze są tacy mili?
\xx Nie, ten jest jakiś narwany, albo świeżak. Jeszcze paru takich znajdziesz. Większość jest milej nastawiona do samotników, choć też bez przesady.
\qd
\hspace{28.1em}Właśnie weszli do jednej z hal wybudowanych w~tym miejscu. W~środku niemal nie było widać betonu, wszędzie zalegało błoto i~trawa. W~kątach rosły paprocie. Wtem, nieco ponad nimi, na antresoli usłyszeli szczęknięcie zamka karabinowego.
\sx Stać! Kto idzie? --~spytał kolejny napotkany członek organizacji.
\xx Samotnicy. Witek i~Olo, z obozu kotów w~Kordonie. Bandyci nas przetrzepali i~wyrzucili niedaleko stąd.
\xx Jeszcze raz, kto? --~gość z karabinem zszedł z konstrukcji.
\xx Witek i~Olo, z obozu kotów. Idziemy uzupełnić brak gotówki u Barmana.
\qm
Gość zawiesił broń na plecach, wyjął notesik i~coś zapisał.
\sx Dobra, możecie iść. Ale tylko spróbujcie coś odwalić, to szybko was uspokoimy. \textit{Nu, bystro!} --~kiwnął głową i~wrócił na antresolę, wyciągając po drodze papierosa.
\qm
Witek klepnął swojego przyjaciela w~ramię i~mężczyźni ruszyli przed siebie. Po kilkunastu metrach ukazał im się wielki szyld „Bar 100 Radów”, pod którym narysowano strzałkę w~lewo. Skręcili do małego garażu, po betonowych płytach przeszli do drugiego wyjścia, a~następnie skierowali się do piwnicy, nad której wejściem napisane było „Bar”. Za rogiem ukazała im się krata, za którą siedział kolejny Powinnościowiec, tym razem w~kominiarce. Kiwnął stalkerom głową i~zapraszającym gestem pokazał, w~którą stronę winni się udać.
\pp
Zeszli jeszcze kilkanaście stopni, skręcając dwa razy. Na końcu odbili w~lewo, a~ich oczom ukazała się całkiem przyjemna, jak na tak surowe warunki, speluna. Muzyka, telewizor, jakaś kuchenka. W~tle było widać dym, który roznosząc się po pomieszczeniu, poza siwą mgiełką roztaczał woń smażonego mięsa. Krótko mówiąc --~bar był miejscem, w~którym można było dobrze zjeść, dobrze się napić, wyspać, schronić przed emisją, pogadać ze stalkerami, zaciągnąć do Powinności, dostać jakieś ciekawe zadanie czy wreszcie, naprawić lub wymienić wyposażenie i~fanty. A ceny za artefakty były dość wysokie.
\pp
Ola zaciekawiło coś jeszcze. Mężczyzna stojący za ladą. Niskiego wzrostu, z mocno zaokrąglonym brzuchem i~bujnym, sumiastym wąsem zasłaniającym górną wargę. Wania! Stalker podszedł do Barmana, wyciągnął doń rękę i~uśmiechnął się lekko.
\sx \textit{Prywiet!} Wania, pamiętasz?
\xx Skąd znasz moje imię, kim jesteś? Odczep się, bo wezwę Powinność, nie znam cię!~--~oburzył się.
\xx Czekaj. --~Olo ściszył głos. --~Wiozłeś mnie wtedy z Polski, w~połowie września. Nazywam się Dawid Kart.
\xx Aaa! Pamiętam! --~uradował się Wania. --~Ty już tutaj? Jak mogłeś tak szybko?
\xx Aj, długa historia. Skroili nas i~przyszliśmy uzupełnić wyposażenie.
\xx Jasne, jasne, macie coś ciekawego?
\qd
\hspace{16.4em} Olo obrócił się bokiem do lady i~tylko uchylił kieszeń. Nie znał obyczajów tego miejsca, to też nie był pewien, czy ktoś nie zechce sobie przywłaszczyć jego znalezisk.
\sd
\xx \textit{Job twoju mać!} --~wrzasnął wąsaty gość. --~Chodźcie na zaplecze.
\qd
\hspace{30em}Skinął ręką innemu członkowi Powinności, by ten przepuścił stalkerów. Mężczyzna cofnął się w~głąb pomieszczenia, jednak jego wyraz twarzy nie uległ zmianie. Nawet okiem nie mrugnął. Witek wiedział, że Powinnościowcy w~większości grają takich twardzieli, prywatnie to całkiem spoko goście, jednak dla Ola ta sytuacja była kompletnie nowa.
\pp
Przeszli kilka metrów, tam zaczekali na Wanię. Poprowadził ich do jakiegoś ciemnego pomieszczenia, zaprosił do środka, a~sam wszedł ostatni, zamykając drzwi. Po omacku znalazł lampkę i~ją zapalił. Następnie kazał samotnikom wyjąć towar. Ci zaczęli grzebać po kieszeniach. Zaraz potem na stole znalazły się dwa Śluzy, jeden Ślimak i~ta dziwna, niespotykana tafla. Na jej widok Handlarz wybałuszył oczy.
\sx Gdzie to znaleźliście?
\xx W~Starym Erneście. Ale nie pytaj, co tam robiliśmy, to długa historia.
\xx Jasne. Co do artefaktów\3k za Śluz daję z reguły siedemset, osiemset rubli. Wam, w~drodze wyjątku, wydam tysiąc za sztukę. No to dwa kafle macie. Za tego Ślimaka\3k \xx zatrzymał się na chwilę i~zmrużył oczy. --~Za niego dostaniecie kolejne dwa tysiące. Ale co do tej tafli\3k wydaje mi się, że to Błona, ale ręki uciąć nie dam. Skontaktuję się z naukowcami. Wybieracie się gdzieś?
\xx Chyba nie, musimy się wylizać po ostatniej przygodzie --~odparł Olo.
\xx Bardzo dobrze. Pokręćcie się trochę po obozie. Kasę wydać na miejscu, czy na razie się wstrzymacie?
\xx Zabieraj Ślimaka i~Śluz, daj nam broń, amunicję i~trochę pestek.
\xx Jasna sprawa. Gdyby wszyscy przynosili takie fanty jak wy\3k Olo, ty wiesz, jak to jest na zewnątrz. Taki Ślimak to też dwa tysiące. Z tym, że waluta zmienia się na dolary. Jeżdżę co kilka miesięcy i~opycham te dziwactwa.
\xx O, właśnie. Kiedy teraz jedziesz na zewnątrz?
\xx A nie wiem, gdzieś\3k może w~marcu dopiero. Zimą strach jechać, wpadnę w~poślizg, wpieprzę się w~jakąś Karuzelę i~tyle mnie będą widzieć.
\xx Aha, może się z tobą zabiorę, jak się uda. --~oznajmił Olo.
\xx Jest tu może jakiś medyk w~okolicy? --~wtrącił się milczący do tej pory Witek.
\xx Jest, w~barakach Powinności. Zaraz dam im znać, że do nich idziesz. A, w~ogóle, jestem Wania --~powiedział i~wyciągnął rękę.
\xx Witek.
\qd
\hspace{4.5em}Opuścili po kilku minutach pomieszczenie. Dawid został w~barze, zamówił sobie coś do jedzenia. Chciał zapoznać się ze stalkerami i~trochę odpocząć. Jego kolega natomiast poszedł do bazy Powinności, opatrzyć swoje rany. W~pewnej chwili, do stolika, przy którym siedział Kart, dosiadł się mężczyzna z czarną apaszką zasłaniającą prawie całą twarz i~kapturem naciągniętym na czoło. Nie przedstawił się, ani nic, od razu zaczął rozmowę.\looseness-1
\sx Słyszałem, że uciekliście z bazy wojskowej. Zabiliście pijawkę i~burera w~Starym Erneście, co nie?
\xx No\3k tak. --~zawahał się Olo. --~Skąd wiesz?
\xx Wiem dużo. Awdan jest kretem. Robi na dwa fronty. To wszystko przez niego.
\xx Co?! --~Dawid z wrażenia aż upuścił sztućce.
\xx Przykro mi z powodu Andreja. --~powiedział beznamiętnie mężczyzna i~wstał od stolika.
\qd \mm
Zniknął tak szybko jak się pojawił, zostawiając Ola w~kompletnym szoku. Dopiero po chwili do świadomości przywrócił go głos dobiegający zza lady. Spojrzał w~tamtym kierunku. Wania machał do niego. Kart wstał od stolika i~podszedł do handlarza.
\sx Pojawi się tu jutro naukowiec z ośrodka nad Jantarem. Jest tylko jeden szkopuł. Będziesz musiał pójść do ich bazy i~eskortować tego typka do nas.\\
Po wszystkim, odstawisz go z powrotem.
\xx Jakieś trudności?
\xx Nie, prawie żadnych. Pójdziesz starym kanałem ściekowym, który prowadzi stąd aż do samiutkiego jeziora. Jakieś trzysta metrów, od wyjścia do bunkra jajogłowych możesz spotkać paru zombie i~snorków. Jak szybko przebiegniecie, to nawet was nie zauważą.
\xx Hm, to wszystko? Nie może być takie proste. --~dziwił się Olo.
\xx No\3k właściwie to nie jest. --~Wania nabrał powietrza. \xx Od kilku dni prowadzimy z zielonymi spekulacje na temat emisji. Nie było żadnej od ponad tygodnia, powinna się na dniach pojawić. Poza tym, w~tej okolicy dzieje się coś bardzo dziwnego z umysłami stalkerów. Jak myślisz, dlaczego wszyscy samotnicy znikają w~niewyjaśnionych okolicznościach po wyprawie nad Jantar, a~potem są identyfikowani jako bezmózgie żywe trupy? Takie rzeczy dzieją się tylko tam i~w pobliżu czerwonego lasu pod samą elektrownią.
\xx Nie mam pojęcia. Ale mogę pójść po tego twojego doktorka. Samemu?
\xx Jak chcesz. Możesz iść z tym swoim Witkiem, chociaż chyba lepiej by było, gdyby został parę dni w~obozie i~wypoczął. Jest też jeszcze jeden typek, który od kilku dni nie ma nic do roboty i~błąka się, szukając banalnych artefaktów. Ale jest dobrym stalkerem. Dawno temu służył w~Specnazie.
\xx Dobra, to daj mi tego gościa.
\xx Okej, jutro o siódmej rano ruszacie. Poszlibyście od razu, ale tamten gdzieś polazł, może z godzinę przed waszym przybyciem. Wysłał mi tylko wiadomość, że będzie jutro koło dziesiątej, bo ma do załatwienia sprawę w~barze. To ściągnę go wcześniej.
\xx A, przy okazji, --~Olo rozejrzał się po sali. --~Nie wiesz kim był ten gość, który się do mnie dosiadł, zanim mnie zawołałeś?
\xx Nie, przykro mi, nic nie zauważyłem.
\xx No dobra, to jutro przyjdę. Przygotujesz mi jeszcze jakąś pukawkę? Nawet na wypożyczenie.
\xx Jasne, jutro rano coś ci znajdę. Jak dobrze wypełnisz tą robotę, to cena tego twojego artefaktu trochę podskoczy, a~i~może dostaniesz jakąś zabawkę na własność. Oczywiście, z kompletem amunicji, w~prezencie od przyjaciela. --~Wania puścił do niego oko.
\xx Ta, przyjaciela\3k --~zaśmiał się Olo, chwycił butelkę z piwem i~wrócił do stolika.
\qm
Siedział tak, sącząc brązowy napój przez kilka minut, gdy nagle złowieszczo rozkrzyczały się PDA wszystkich zgromadzonych. Stało się to niemal jednocześnie. Na kilka sekund pomieszczenie ogarnął przeraźliwy pisk tych małych urządzeń. Właśnie do stalkerów dotarło ostrzeżenie o nadchodzącej emisji. Barman niemal automatycznie nacisnął jakiś przycisk pod ladą i~udał się na zaplecze, w~sobie tylko znanym kierunku. Inni zeszli przez otwarty właz w~podłodze jeszcze niżej, do czegoś w~rodzaju schronu. Nie było to wielkie pomieszczenie, ale dobrze chroniło przed zwarciem. Upchnęło się w~nim jakieś piętnaście osób, w~tym Olo. Kątem oka widział, że w~barze tłoczy się coraz więcej stalkerów. Głowa bolała go coraz bardziej, niskie buczenie połączone z niemal ultradźwiękowym piskiem rozwiercało jego czaszkę, wzrok rozmywał się. Całym budynkiem trzęsło.
\pp
Emisja nie trwała długo, może z dwie minuty. Jednak dochodzenie do siebie po czymś takim trwa zazwyczaj kilka kwadransów, czasem ten czas wydłuża się nawet do kilku godzin. Wszyscy, którzy zdążyli znaleźć sobie schronienie w~barze powoli wychodzili na zewnątrz. Pierwsi zniknęli ci, którzy wpadli tu w~ostatniej chwili. Olo na swoją kolej musiał czekać kolejne parę minut. Gdy wreszcie udało mu się wypełznąć z tego bunkra, wrócił do swego stolika by dopić piwo. Jednak nie smakowało ono tak jak wcześniej. Było ciepłe, rozgazowane, z wyraźnym metalicznym posmakiem zostającym w~ustach. Zdegustowany, wylał trunek do jakiegoś kanału biegnącego wzdłuż ściany. Wtedy w~drzwiach pojawił się miejscowy bukmacher, Arni.
\sx Dziś wielka walka! Iwan Groźny przeciwko Saszce Sępowi! Nie można tego przegapić! Dzisiaj w~arenie, o 16.30! Wielka walka! Iwan Groźny kontra Saszka Sęp! Zakłady przyjmujemy tylko do 16! Wielka walka, dziś w~arenie! --~darł się, a~słychać go było chyba aż w~Kordonie. Stalkerzy ignorowali jego głos, to nie była dla nich żadna nowość. Ktoś tam rzucił parę drobnych na Iwana, ale tym sposobem nie dało się w~Zonie dorobić fortuny.
\qd
\hspace{12.4em}Mimo to, Olo zainteresował się walkami na arenie. Skorzystał z chwili ciszy i~podszedł do barmana.
\sx Wania, \textit{skaży ty mnie,} o co chodzi z tymi zakładami?
\xx Czasem Powinność dorwie jakichś łebków, czy to z Wolności, czy bandytów, czy nawet samotników, którzy coś tam przeskrobali i~prowadzi ich na arenę, ku uciesze tłumu. Tam stawiani są do pojedynku z innymi jeńcami. Ten, który wygrywa zyskuje wolność, pod warunkiem, że znów nic sobie nie przeskrobie. Czasem ktoś sam się zgłasza do walki. Ot tak, wyżyć się na jakichś kotach. Można wygrać parę groszy, ale można też przegrać życie.
\xx Rozumiem. A ci dwaj, co o nich gadał ten bukmacher, kto to?
\xx Iwan Groźny to jeden z najpotężniejszych bandytów w~całej Strefie. Nie bez powodu nazywa się Groźnym. Chyba podpadł swoim dawnym ziomkom, skoro dał się złapać. Po nim i~jego bandzie spodziewałem się raczej krwawej walki na śmierć i~życie, w~ostateczności samobójstwo, ale na pewno nie więzienie.
\xx A ten drugi?
\xx Saszka? --~upewnił się Wania. --~Saszka chodzi po Strefie w~tę i~na zad, doczepiając się do różnych grupek. Łazi i~nic nie robi, tylko zbiera fanty, których inni nie dadzą rady unieść, albo które im się nie spodobają. Oferuje niby pomoc w~zbieraniu łupów, a~potem spieprza ile sił w~nogach. Czeka kilka dni i~przychodzi to opchnąć. Taki pasożyt. Nie raz i~nie dwa już z tego powodu dostał dobrze w~mordę, nawet ja mu kiedyś przyfasoliłem.
\xx Aha. Ciekawy typ. A jak się dostać na arenę? Chyba sobie to obejrzę.
\xx No to tak. --~zamyślił się na chwilę barman. --~Wychodzisz z baru i~idziesz pod ten pierwszy transparent, z napisem „Bar 100 Radów”. Na wprost masz duży, ceglany budynek. Skręcasz w~prawo i~idziesz wzdłuż ściany. Za rogiem jest dobrze oznaczone wejście na arenę. Tylko nie daj się wrobić i~nie idź na ring, a~na trybuny.
\xx Dzięki, stary. --~Kart rzucił dziesięć rubli na ladę, za piwo i~obiad.
\xx Nie ma sprawy\3k A, Olo, śpiesz się, do walki zostało czterdzieści minut. Najwyższy czas, żeby zająć miejsce siedzące.
\qd \mm
Stalker poszedł według zaleceń barmana. Trafił bez problemu. Arni, tak jak zakładał Wania, próbował go namówić na udział w~walce, ale Dawid skutecznie się wymigał. Zajął dobre miejsce, niemal dokładnie po środku hali, skąd obie postacie było widać jak na dłoni.
\pp
Wnętrze, w~którym odbywały się walki nie było zbyt duże. Miało na oko czterdzieści metrów długości i~około piętnastu szerokości. W~środku poustawiane były kontenery, skrzynie, beczki, stare samochody, części maszyn i~inny złom. Na obu końcach dało się dostrzec małą kratkę, za którą paliło się światło. To były wejścia dla walczących. Ponad ich głowami wisiało kilka gondoli, które pełniły rolę trybun. Walki z reguły zbierały wielu kibiców, toteż trudno było znaleźć dobre miejsce do obserwowania. Nie inaczej było tym razem. Na szczęście, Olo pojawił się odpowiednio wcześnie.
\pp
W pewnej chwili przez stary, wojskowy megafon rozległ się głos:
\sx \textit{Wnimanije, wnimanije!} Za chwilę na arenę wkroczą dwaj przestępcy, którzy stoczą ze sobą walkę na śmierć i~życie. Zwycięzca zyskuje wolność! Z lewej strony do boju staje Iwan Groźny, jeden z najstraszniejszych bandytów, którzy chodzą po Zonie! Żebyś zdechł, mendo! --~nieoczekiwanie odezwał się Arni do przestępcy. --~Mnie też kiedyś ograbił! --~wyjaśnił.
\qd
\hspace{10em}W tej chwili z trybun rozległo się buczenie. Bukmacher dał się tłumowi wykrzyczeć i~kontynuował przedstawianie zawodników.
\sx Z Iwanem zmierzy się Saszka Sęp! Wszyscy znamy tego frajera! Plącze się za nami i~kradnie fanty.
\xx Zdechnij, cioto! Obaj zdechniecie! --~wrzasnął tym razem ktoś z trybun.
\xx Stawiam piwo dla tego pana! Mocne słowa, kolego. Niech się rozpocznie! --~rzekł Arni i~z charakterystycznym piskiem wyłączył mikrofon.
\qd
\hspace{24em}Kraty odsunęły się i~na arenę wyskoczyło dwóch gości. Jeden w~całkiem dobrej kurtce, czymś w~rodzaju glanów i~w brudnych dżinsach. Ściskał w~rękach obrzyna, a~z kieszeni wystawał mu nóż. To był Iwan. Saszka natomiast miał bladożółtą kurtkę i~szedł powoli, w~rękach ściskając dwa pistolety PMm. Stalkerzy dopingowali ich z góry, jedni pomagali odnaleźć przeciwnika, inni celowo wprowadzali w~błąd. Dla nich to było jak walki gladiatorów w~starożytnym Rzymie.\looseness+1
\pp
Pierwsze starcie. Iwan wychyla się z za kontenera i~oddaje jeden strzał. Saszka w~tej samej chwili jednym susem dopada do ściany, następnie wystawia tylko dłoń i~pruje na oślep cały magazynek. Stalker to był z niego marny. Dlatego kradł fanty i~łaził za ludźmi, bo gdyby chodził sam, to dostałby najwyżej kulkę w~dupę.
\pp
W tym czasie gość z obrzynem zachodzi powoli za kontener z drugiej strony, wyciąga rękę z bronią w~kierunku Saszy, który wciąż strzela na oślep i~zastyga w~bezruchu. Potem lekko tyka przeciwnika lufą w~ramię. Gdy ten się odwraca, Iwan oddaje drugi strzał, po którym zawartość czaszki samotnika ląduje na ścianie za nimi.

Z trybun rozlega się głośne buczenie, bo to jednak bandyta uzyska wolność. Widzowie mają pełne prawo wyrazić swoje niezadowolenie. Dwóch Powinnościowców w~tej chwili podbiega do Iwana, rzucają go na ziemię, związują ręce, zasłaniają oczy i~wyprowadzają. Chwilę potem na arenę wpuszcza się psy, by z resztek Saszy zrobiły sobie małą ucztę.%\looseness+1

\mm Stalkerzy powoli rozeszli się. Olo wrócił do baru i~choć walka trwała może trzy minuty, na zegarku dochodziła już siedemnasta. Zamówił kolejne piwo, ziemniaki ze smażoną kiełbasą i~usiadł przy stoliku. Nie miał ochoty nic więcej dziś robić. Zjadł, odniósł talerz, wrócił na miejsce i~otworzył piwo. Sączył powoli, słuchając radia. Przesiedział tak jeszcze z dwie godziny. Był wyczerpany, spytał barmana, gdzie może się przekimać. Ten odpowiedział, że po uiszczeniu opłaty u Powinnościowca siedzącego przy schodach można dostać łóżko lub śpiwór, w~zależności od kwoty, jaką chce się przeznaczyć na nocleg.
\pp
Olo niezwłocznie udał się do stróża, dał mu piętnaście rubli i~poszedł się położyć. Zasnął bardzo szybko i~spał mocnym snem. Był zbyt wyczerpany, by zapamiętać mary, więc uważał, że tej nocy żadnych snów nie miał.
\section*{NAUKOWIEC}
\mm Wstał i~sennym, niedobudzonym krokiem ruszył do baru. Spostrzegł za ladą Wanię. Podszedł i~zaspanym głosem przywitał się. Zamienił z handlarzem kilka słów, gdy w~drzwiach pojawił się jakiś znajomy mężczyzna. Swe kroki również skierował prosto do barmana
\sx Gdzie jest ten gość, o którym mi mówiłeś? --~zagaił, a~Olo próbował przypomnieć sobie, skąd go zna.
\xx O, tu stoi. Poznajcie się, To jest Olo, a~to Janka Tytan.
\xx To my się już znamy, w~takim razie. --~zaśmiał się Janka.
\xx Tak, a~skąd?
\xx Spotkaliśmy się wczoraj, kilometr stąd, jak błądziliśmy w~poszukiwaniu obozu.~--~wtrącił się do dyskusji Olo.
\xx No, to bardzo dobrze. Janka, ty już parę razy chodziłeś po jajogłowych do ich bazy, znasz drogę. Poprowadzisz kumpla.
\xx Spoko. A co ja będę z tego miał?
\xx A to już ustalaj z Olem. --~wzruszył ramionami Wania.
\xx Dobra. --~ziewnął Kart. --~Pójdę się przewietrzyć, oprzytomnieję i~możemy ruszać. Masz dla mnie jakąś broń?
\xx Jasne, na zapleczu czeka na Ciebie IL 86 z zapasem amunicji. --~odezwał się wąsaty.
\qm
Olo wyszedł na zewnątrz i~pospacerował trochę po obozie. Była siódma rano. Niebo było zasnute chmurami, które jednak powoli ustępowały. Ziewnął, przeciągnął się i~wrócił do baru. Wypił kawę, którą w~gratisie przygotował mu Wania i~wszedł z Janką na zaplecze. Tam zabrał karabinek i~pestki. Wziął też plecak, do którego zapakował artefakty, manierkę z wodą i~jakąś flaszkę. Barman pokazał im drewniane schody prowadzące w~dół. Stalkerzy udali się tam, następnie przez lekko zakamuflowane wejście skierowali się do tunelu. Szli szybko i~pewnie. Janka zapewniał, że można tutaj trafić najwyżej na pijawkę, anomalie zaczynają się dopiero za czwartym czy nawet piątym rozgałęzieniem i~widać je jak na dłoni. Olo zaufał swojemu kompanowi.
\sx Słuchaj, Barman mówił coś o snorkach, co to właściwie jest? --~spytał w~pewnej chwili.
\xx Snork? Snork to jedna z najgorszych mutacji w~Zonie. Mówi się, że to żołnierze strzegący Strefy, którzy pod wpływem emisji i~licznych zmian w~organizmie zostali przeobrażeni w~bestie. Wyglądają, jak ludzie, z tym, że poruszają się na wszystkich czterech kończynach. Na twarzach z reguły mają maski gazowe, które zrastają się ze skórą tak, że nie da się ich zdjąć. Stwory, które masek nie mają, według opowieści, mają wielkie kły zamiast zębów. Skóra na plecach tych mutantów jest rozdarta na całej długości i~ukazuje wzmocnioną budowę kręgosłupa. Nogi również mają umięśnione bardziej niż u zwykłego człowieka. Potrzebne im to do ataku. Skaczą na wielkie odległości, czasem ponad dziesięć metrów, by rzucić się na swoją ofiarę, rozszarpać ją na kawałki, a~następnie pożreć.
\xx Ja pierdzielę\3k --~wyszeptał Olo.
\xx Co? Straszne? Straszne, kolego, to jest to, gdy taki cham wyskoczy na ciebie z jakiegoś korytarza, a~nie sam wygląd. Idzie się w~kombinezon zesrać ze strachu. Uwierz mi, wiem co mówię\3k
\qd
\hspace{14.5em}Stalkerzy szli dalej, choć po trzecim rozgałęzieniu nieco zwolnili. Tutaj zaczynało się robić niebezpiecznie. Pod nogami plątały się mutanty nazwane chomikami, które bardziej przypominały skrzyżowanie wiewiórki z wielkimi uszami i~bobra. Pojedynczo uciekały widząc coś większego od siebie, ale gdy natrafiło się na większe stado, te chochliki Strefy, jak je również nazywano, potrafiły przegryźć się przez kombinezon i~nieźle poharatać nogi. Zwykle dwa mocne kopnięcia wystarczyły, by zabić bądź unieszkodliwić stworka, ale czasem trzeba było otworzyć ogień.

Doszli właśnie do czwartego rozgałęzienia, gdy nagle Janka krzyknął „Stój!”. Kart posłusznie zastygł w~bezruchu.
\sx Spójrz, tam. --~wskazał tunel przed sobą.
\xx Kurde, dzięki stary. --~Olo klepnął kolegę w~ramię.
\qd
\hspace{23.6em}Przed nimi, z górnej części betonowej konstrukcji wystawały jakieś patyki, sięgające do ziemi. Dawid zrozumiał, że to spalony puch. Odwinął rękawy, naciągnął na dłonie jakieś rękawiczki, a~na głowę włożył kaptur. Położył się twarzą do ziemi i, niemal szorując nią po betonie, próbował przedostać się pod rośliną. W~tym momencie do ich uszu dotarł ryk, jakby lwa. Jego źródło miało miejsce w~rozgałęzieniu. Kart skamieniał, będąc dokładnie pod anomalią. Jego kompan złapał go za nogi i~przyciągnął do siebie. Olo podniósł się, włączył latarkę i~spojrzał do pobocznego tunelu. Widział tam jakiś ruch, więc wycelował broń. Czworonożna postać poruszała się powoli w~ich kierunku. Janka rzucił w~nią granatem, po czym popchnął Ola w~kierunku wyjścia. Stalkerzy zasłonili twarze przebiegając przez puch i~biegli, nie zważając uwagi na szalejący wykrywacz anomalii, wszyty gdzieś w~kombinezon Tytana. Co chwila uderzały ich fale gorąca powodowane spalaczami, czy ściągało ich na bok, gdy mijali kolejne Wiry. Po
kilku chwilach coś
za nimi wybuchło, a~ze ścian tunelu zaczął się sypać beton. Samotnicy zatrzymali się u jego wylotu.
\sx Cholera, chyba się zawalił\3k --~Janka Tytan ocierał pot z czoła.
\xx Jak to? I jak my teraz wrócimy z tym naukowcem, geniuszu? --~spytał Olo, patrząc podejrzliwie.
\xx Normalnie, przez Rostok. Tylko\3k --~podrapał się po głowie. --~Tylko tam aż roi się od mutantów i~najemników.
\xx Aha\3k Dobrze wiedzieć. Dzięki stary. --~odpowiedział z przekąsem.
\xx Wolałeś zobaczyć snorka pierwszy i~ostatni raz w~życiu?
\qd
\hspace{26.2em}Olo nie odpowiedział. Odwrócił wzrok i~zaczął się rozglądać. Jakieś trzysta metrów przed nimi, wśród pojedynczych, martwych drzew wyrastał bunkier naukowców otoczony blaszanym ogrodzeniem. Za nim było widać małe bagno, z którego również strzelały w~niebo przerażające konary. Na prawo stała wielka fabryka, teraz opustoszała od wielu lat. Chodzą słuchy, że w~jej podziemiach zbudowane jest tajne wojskowe laboratorium, w~którym badane są oddziaływania fal radiowych na mózg. To tylko domysły, ale jakoś trzeba było wyjaśnić, dlaczego stalkerzy, którzy zapuszczą się w~te tereny bez ochrony zamieniają się w~bezmózgie żywe trupy. Nikt jeszcze nie odważył się zejść pod fundamenty kompleksu.
\sx Widzisz to? --~Janka wystawił dłoń w~kierunku bunkra.
\xx Widzę.
\xx Ruszamy na trzy, sprintem. Ty pierwszy, ja mam wykrywacz. Jak w~coś wpadniesz, to cię wyciągnę. Zrozumiano?
\xx Tak. --~skinął głową Olo.
\xx Uwaga, uwaga, idzie eskorta po naukowca, bądźcie gotowi za trzydzieści sekund.~--~powiedział Janka do urządzenia, które zaczęło wydawać jakieś niezrozumiałe dźwięki.~--~Powtarzam, za trzydzieści sekund otwórzcie bunkier.
\qm
Złapał broń w~pozycji do biegu i~przykucnął.
\sx Raz, dwa, trzy! Leć! --~krzyknął do Ola, a~po chwili ruszył za nim.
\qd
\hspace{30.5em}Biegli, tak jak obliczył, pół minuty. Wokół nich rozlegały się jęki i~ryki, w~tej okolicy roiło się od zombie i~snorków. Przebiegli za ogrodzenie i~w tej chwili otworzyły się stalowe wrota bazy. Wpadli do środka, po czym Janka wcisnął przycisk, zamykający właz. Stalkerzy zostali spryskani jakimś preparatem, po czym otworzyły się drugie drzwi. Wewnątrz czekał już na nich mężczyzna odziany w~zielony kombinezon. Przywitał się z nimi.
\sx Jestem doktor Sacharow. --~powiedział. --~Widzę, że się panowie zmęczyliście. Może odpoczniecie chwilę?
\xx Bardzo chętnie. --~Janka przysiadł na ławce i~wyjął z plecaka napój energetyzujący.
\xx Proszę mi opowiedzieć o tym miejscu, doktorze. --~zagaił Olo.
\xx Zostaliśmy tu wysłani przez rząd Ukrainy, by badać mutacje znajdujące się w~tej okolicy. Jest ze mną jeszcze doktor Krugłow, ale obecnie sprawdza promieniowanie po drugiej stronie naszego ośrodka. Musimy zaczekać, aż powróci. Ten budynek nie może stać pusty. Przynajmniej jeden z nas musi sprawdzać na bieżąco odczyty.
\xx Proszę powiedzieć, co się właściwie stało z Jantarem?
\xx Ścieki wylewane przez pobliską fabrykę stopniowo zanieczyszczały jezioro. Pływanie nie było tu co prawda zakazane, ale kąpiący się wchodzili do wody na własne ryzyko. Jeszcze w~latach dziewięćdziesiątych dokonywaliśmy obserwacji ryb. Wszystko zmieniło się po 2007 roku.
\xx Po tym wybuchu, którego epicentrum, ani dokładnego powodu, nikt do tej pory nie potrafi określić? --~wtrącił się Olo.
\xx Tak, dokładnie. Nagle to miejsce zamieniło się w~odrażające, zakwaszone bagno. O żołnierzach pilnujących fabryki\3k Cóż, przynajmniej w~papierach jest to fabryka, ja myślę, że podziemia skrywają o wiele więcej. --~zadumał się naukowiec. --~O czym to ja\3k A, tak, wojsko sprowadzone tu do ochrony obszaru zniknęło, wyparowało, ślad po nich zaginął. Pewnie dalej żyją, jako mutanty.
\xx Snorki?
\xx Nie wiem, jak wy, stalkerzy, to określacie. Chodzi o czworonożne bestie, które w~pozycji wyprostowanej nawet przypominają ludzi. Maski gazowe, kły, rozdarta skóra na plecach\3k
\xx Tak, dla nas to snorki. --~potwierdził Janka.
\xx Niech będzie. Dodatkowo te zombie\3k Ten fenomen bardzo mnie ciekawi. Państwo jednak odmawia dofinansowania, byśmy mogli zbadali teren fabryki. Powinność zjawia się tu rzadko, jak chcą nam odsprzedać artefakty, zawsze to my musimy stawić się w~ich bazie. Wolnościowców nie ma co pytać, walą do nas jak do kaczek. Myśleliśmy z doktorem Krugłowem, by zatrudnić do tego celu jakiegoś samotnika, ale każdy, kto się tu pojawia, albo zamienia się w~zombie, albo widzi co się dzieje i~ucieka ile sił w~nogach.
\xx Nie dziwne, to nie jest zbyt gościnne miejsce.
\qd
\hspace{21.4em}Nagle drzwi do bunkra otworzyły się i~pojawił się w~nich jakiś mężczyzna. Zdyszany, w~kombinezonie naukowca. To właśnie był doktor Krugłow. Jego ubiór miał w~kilku miejscach dziury. Wyjaśnił, że gdy sprawdzał promieniowanie, zza budowli wypełzł zombie i~zaczął strzelać. Naukowiec ani myślał się bronić, dał nogę jak tylko poczuł uderzenie w~miednicy. Tytanowa płytka zatrzymała kulę, ale siła, z jaką ta się w~nią wbiła, była i~tak odczuwalna.
\sx Dasz sobie radę, kolego? --~spytał Sacharow. --~Ja idę do Baru, kupić trochę próbek.
\xx Jasne. Tylko wróć niedługo.
\xx Najpóźniej jutro będę, o ile nie pojawią się żadne komplikacje.
\qd
\hspace{28.5em}Naukowcy skinęli sobie głowami, po czym drzwi bunkra otworzyły się. Faceci biegli prosto do drogi prowadzącej w~pobliże starej stacji kolejowej Rostok, zwanej przez stalkerów dziczą. Teren ten miał taką nazwę, ponieważ mało kto o zdrowych zmysłach tam zaglądał. Przy drodze prowadzącej do baru koczowali bandyci, których Powinność nie wpuszczała na jego teren, nieco głębiej obóz stworzyli najemnicy, a~dalej\3k dalej była już tylko Strefa. Anomalie, radiacja, mutanty. W~starych wagonach nocowały zmutowane psy, w~magazynach roiło się od pijawek, a~podziemnego parkingu podobno strzegł kontroler, jednak nikt nie zamierzał za żadne skarby tego sprawdzać. Budowniczych Rostoku zaskoczyła katastrofa, porzucili niedobudowany biurowiec. Właściwie, stał tam tylko trzypiętrowy szkielet, na kondygnacjach którego kolejni najemnicy mieli swoją bazę. Fundamenty zaś zamieszkiwane były przez psy.\looseness-1
\section*{WIZYTA W~DZICZY}
\mm Stalkerzy nie biegli już po mokradłach, udało im się wyzwolić z grozy Jantaru, choć Olo w~pewnej chwili usłyszał świst kuli tuż ponad jego głową. Poruszali się teraz asfaltową, popękaną drogą. Dotarli do tunelu nazwanego spalonym. Ochrzczony został tym imieniem, ponieważ w~jego wnętrzu wytworzyło się w~tajemniczych okolicznościach mnóstwo anomalii typu spalacz. Dla stalkera z dobrym kombinezonem to był raj na ziemi. Wystarczyło przycupnąć i~czekać, aż z którejś formacji wypadnie artefakt. A nie były one złe. Tyle tylko, że w~kilka chwil z nikąd mogłyby tu ściągnąć tłumy zombie, albo najemników. No i~nie było zabezpieczenia przed emisją.\looseness-1
\sx No, proszę panów, tutaj zaczyna się dzicz. Głowy nisko i~nastawiamy uszy. Zło może czaić się za każdym rogiem.~--~powiedział Janka.
\xx Chłopaki, nie dajcie mnie zabić. Pierwszy raz tu jestem, a~stalker ze mnie marny.~--~wyjęczał przerażony Sacharow.
\xx Jasna sprawa, doktorku. Tylko głowa nisko.~--~pacnął go w~osłonę hełmu Olo.
\qd \mm
Naukowiec dostał od stalkera pistolet i~dwa magazynki do niego. Ruszyli przez tunel. Janka rzucał śrubki, szukając anomalii, doktor Sacharow używał jakiegoś detektora, który chyba nie był zbyt dobrze skalibrowany. Gdyby nie ręka idącego na końcu Karta szarpiąca go co chwilę do tyłu, spaliłby się tam żywcem dobre pięć razy. W~pewnej chwili, w~czasie tego bezkresnego kluczenia z jednej z uaktywnionych anomalii wprost na ręce Tytana wskoczyła jakaś okrągła rzecz. Ceniony, choć dość powszechny artefakt o nazwie Kula Ognista. Wyglądał jak niemal idealna sfera, zrobiona z lawy, jednak w~środku było widać niewielki, zielononiebieski płomyczek. Pomimo, że powietrze wokół falowało, sam obiekt był zimny jak lód. Przyczepiony w~odpowiednim miejscu wzmacniał ochronę kombinezonu przed ogniem. Stalker wrzucił go do kieszeni na piersi i~ruszył dalej, jak gdyby nigdy nic. W~końcu wydostali się z tego piekła, spoceni i~umęczeni, na zimny beton. Przechodzili przez tunel niecałe dziesięć minut, ale w~czasie ich tańca ze
spalaczami niebo zdążyło zasnuć się ciemnymi chmurami, z których po chwili zaczął padać deszcz.

Mężczyźni usiedli i~zdjęli swoje hełmy, obmywając spieczone twarze w~strugach wody. Opad był właściwie rzęsistą mżawką, ale po przeprawie przez spalony tunel nawet przeciekająca rynna była dla stalkerów wybawieniem.
\pp
Mężczyźni siedzieli w~milczeniu, oddychając ciężko. Po paru minutach, gdy pod ścianą wyrosła mała kałuża, Janka wstał i~zakomenderował, by pozostali zbierali się w~dalszą drogę. Chwilkę po tym wszyscy byli już na nogach. Szli przy samej elewacji, trzymając głowy tak nisko, jak tylko się dało. Dochodziła godzina szesnasta, słońce chyliło się ku zachodowi. Panowała lekka szarówka, pogłębiana dodatkowo przez chmury i~wodę zebraną na płytach wykonanych z betonu.
\pp
Stalkerzy wyszli za róg, a~ich oczom ukazał się stary żuraw, wykorzystywany do budowy biurowca, który miał stanąć w~tym miejscu w~roku 1987. Wewnątrz gmachu, na drugiej kondygnacji dało się zaobserwować wesołe języki ognia, rzucające na ścianę cień jakichś postaci. Według Janki, to byli najemnicy. Wszystko wskazywało na to, że ma rację. Jakieś szelesty, odgłosy broni dobiegające z góry. Tamci musieli być bardzo dobrze uzbrojeni. Eskorta doktora Sacharowa miała ciężki orzech do zgryzienia. Potrzebne im było teraz coś, co odwróci uwagę najemników, a~jednocześnie pozwoli przedostać się do torów kolejowych, skąd do baru było tylko dwieście metrów.
\sx Wiem!~--~szepnął entuzjastycznie Janka.
\xx Dawaj.~--~popatrzył w~jego kierunku Olo.
\xx Przeczołgamy się koło nich. Jest dość ciemno, jeśli będziemy ostrożni, to nas nie zauważą.~--~gestykulował Tytan.
\xx A mutanty?
\xx Psy? Przeczołgamy się tam, za płotem, nie usłyszą.
\xx Ty jesteś w~Zonie najdłużej. Prowadź.
\xx Dobra. Poruszamy się w~odstępie dziesięciu metrów od siebie. Jak coś się dzieje, to leżycie i~czekacie na mój ruch. Jak przylezie w~okolicę jakaś pijawka, to głowy w~piach, ci z budynku zaraz ją zastrzelą. Anomalie omijamy szerokim łukiem, poruszając się za mną jak po sznurku, na artefakty nie zwracamy uwagi.
\xx A co?...~--~przerwał mu Olo, ale nie dokończył, bo Janka mówił dalej.
\xx Jeśli nas odkryją, to walka nie ma sensu. Spieprzamy ile się da, uważając, żeby w~nic się nie wpakować. Korzystamy, panowie, z osłon.~--~odczekał chwilę.~--~To co, gotowi? Ruszamy!
\qd \mm Delikatnie położył się na ziemi i~powoli, ostrożnie przekładając każdą kończynę zaczął się czołgać do przodu. Przedostał się za płot i~machnął ręką do Sacharowa. Ten powtórzył jego manewr. Na końcu na ziemię położył się Olo. Poruszali się płynnie, jeden za drugim. Jednak dobre pomysły mają to do siebie, że często zawodzą. Wystarczyły niespełna trzy minuty. Ktoś w~budynku wychylił się na papierosa i~zobaczył w~trawie nienaturalnie poruszającą się butlę, a~pod nią cały kombinezon. Za chwilę szczęknął zamek karabinu, rozległo się gromkie \textit{,,Agoń!'' }i~powietrze zaczęły przecinać kule.
\pp
Janka zerwał się na równe nogi i~biegł, strzelając na oślep w~kierunku biurowca. Olo doczołgał się do przerażonego naukowca i~przykrył go swoim własnym ciałem. Odpowiadając na ogień najemników krzyknął doktorowi: ,,spierdalaj, bo sam cię zatłukę!''. Ten wstał i~pobiegł za Tytanem. Olo strzelał dalej, w~ferworze walki nie dostrzegł, że Sacharow jest już bezpieczny. W~pewnej chwili zabrakło mu amunicji. Rzucił karabinek IL 86 o ziemię, zerwał się do biegu i~ruszył w~kierunku kolegów. Dogonił ich. Biegli teraz we trójkę, omijając kontenery, taranując psy i~przeskakując przez płoty. Wbiegli na rampę załadunkową, a~licznik Geigera w~kieszeni Sacharowa zaczął trzeszczeć. Nie przejęli się tym. Uciekali od wściekłych, uzbrojonych po zęby typów, którzy gotowi byli zabić bez żadnych wyrzutów sumienia. W~tej chwili to było gorsze, niż promieniowanie. Dotarli do starego garażu. Wpadli do środka i~stanęli jak wryci. W~kanałach do przeglądu pojazdów bulgotały Galarety, a~przy wraku ciężarówki słyszeli jakieś buczenie,
podobne do tego, jakie wydaje Wir. Jednak byli pod dachem, w~niemal kompletnych ciemnościach, pościg nie mógł do nich dotrzeć.
\pp
Dysząc i~starając się nie zbliżyć do anomalii, nasłuchiwali otoczenia. Po paru chwilach, obok garażu zaczęli przebiegać jacyś ludzie. Najpierw jeden, drugi, potem trzeci. W~końcu szybkie kroki zaczęły być tak liczne, że ze słuchu nie dało się wychwycić ilości osób. Janka wychylił się ostrożnie zza rogu i~ujrzał coś, czego się nie spodziewał. To nie byli żadni najemnicy. Przed sobą widział bandytów uciekających w~stronę baru. Powodem ich rajdu jednak nie były niebezpieczeństwa Zony. Tytan wydedukował, że oni też podeszli za blisko najemników. W~istocie. Za grupą biegł mały oddział odzianych w~niebieskoszare, szeleszczące mundury mężczyzn. Po parunastu sekundach kroki ucichły i~zrobiło się dziwnie spokojnie.\looseness-1
\pp
Zaraz jednak rozległy się strzały. Uciekinierzy wpadli w~zasadzkę.
\sx No, trochę się powybijają i~te mendy niebieskie przejdą do gmachu, po opatrunki. Posiedzimy z godzinkę lub dwie i~będzie można spokojnie wrócić do baru.~--~powiedział Tytan ni to do siebie, ni to do współtowarzyszy.
\qd
\hspace{31.4em}Cóż innego mogli w~takim wypadku zrobić? Znaleźli sobie skrawki względnie bezpiecznej przestrzeni i~położyli się na zimnym, wilgotnym betonie. Siedzieli tam, popijając wódkę na zbicie promieniowania, czekając, aż minie wszelkie zagrożenie. Po kilku minutach odgłosy walki ucichły. Nastała cisza, wyłączając monotonne uderzanie kropel o metalowe konstrukcje i~jęki rannych bandytów. Stalkerzy postanowili jednak poczekać jeszcze chwilę. Jak się przekonali, to nie był najlepszy pomysł. Spokój został zmącony przez inne odgłosy. Gorsze, niż można sobie wyobrazić. Jakieś niezidentyfikowane ryki, chrząkanie, nieludzkie nawoływanie. Odgłosy Strefy. Olo ostrożnie wystawił głowę zza osłony. Dostrzegł po drugiej stronie torów kolejowych trzy snorki omijające anomalie. Wziął głęboki oddech i~zwrócił się do swoich towarzyszy:
\sx Mamy do wyboru: rajd między anomaliami, albo ciche przejście pomiędzy nimi. Osobiście wolę to pierwsze. Do hali jest stąd jakieś czterdzieści, pięćdziesiąt metrów. Co myślisz, Janka?
\xx Przejdź sam i~oznacz nam drogę, przebiegniemy.
\xx Tak sądzisz? Dobra, osłaniaj mnie.
\qd
\hspace{17em}Stalkerzy skinęli sobie głowami i~Olo ostrożnie wyszedł z kryjówki. Szedł powoli, rozważnie stawiając kroki i~nasłuchując dźwięków Dziczy. Kładł małą, zabarwioną na żółto śrubkę co trzy kroki. Dotarł do anomalii „elektro”. Ominął ją z lewej strony w~odległości trzech metrów od krawędzi jej rażenia. Po chwili dobrnął do hali. Ostrożnie zajrzał do środka. Było pusto. Jedynie w~kącie po przeciwnej stronie leżało rozkładające się ciało. Z tej odległości Olo widział tylko małą dziurkę w~czole. Była czarna, w~przeciwieństwie do lekko brązowej skóry głowy.
\sx Postrzelony\3k --~odetchnął.
\qd
\hspace{13.4em}Wychylił się z budynku i~spojrzał w~stronę garażu. Naukowiec wraz ze swoją eskortą byli gotowi do drogi. Wzrok Karta spotkał się ze spojrzeniem Tytana. Skinęli głowami, a~Janka zaczął mówić do Sacharowa, gestykulując przy tym żywo. Chwilę potem klepnął go w~ramię, a~ten wyrwał do przodu biegnąc wzdłuż ścieżki wyznaczonej przez śrubki. Po tym, gdy naukowiec dotarł do czwartej śrubki, jego śladem ruszył także stalker. Dobiegnięcie do hali zajęło im raptem dziesięć sekund. Cała trójka weszła po drabinie na piętro skąd rozciągał się widok na Rostok. Snorki dalej plątały się wśród anomalii. Było ciemno, więc ruch przy torach dało się dostrzec jedynie dzięki słabemu światłu bijącemu od formacji „elektro”. Stalkerzy włączyli latarki i~po chwili zeszli na dół.
\sx Do baru sto pięćdziesiąt. Nie widziałem nikogo z tej strony. Uważajcie tylko na „Wiry”, a~zaraz dojdziemy do Handlarza i~pójdziemy lulu.~--~zagadał Tytan.
\xx Ja to wcześniej będę musiał się jeszcze wysrać.~--~roześmiał się Olo.
\xx Oj, racja. Chyba ci w~tym pomogę.~--~odpowiedział Janka.
\qd
\hspace{26.9em}Byli wycieńczeni przeprawą przez Dzicz. Chcieli usiąść w~jakimś spokojnym miejscu i~oddać swe ciała Morfeuszowi. Wiedzieli jednak, że muszą jak najszybciej doprowadzić Sacharowa do Baru. Zeszli na dół i~ruszyli w~kierunku „Bramy”, jak stalkerzy nazywali miejsce, gdzie kończy się teren baru i~zaczyna Strefa. Po paru chwilach dotarli do niej, a~ich oczom ukazał się pierwszy posterunek Powinności.
\sx Stójcie!
\xx Spokojnie, Barski, prowadzimy Sacharowa do Barmana.
\xx Dobrze, dobrze. Potem zgłoście się do generała, zapiszemy sobie, Tytan, że byliście w~Dziczy. Generał Woronin będzie chciał wiedzieć, co tam robiliście.
\xx Dobra, dobra, nie spinaj się tak.
\qd
\hspace{15.7em}Janka klepnął Powinnościowca w~ramię i~trójka minęła posterunek. Przeszli oznaczonymi ścieżkami do wejścia do piwnicy. Schodząc powoli w~dół, dotarli wreszcie do sali ze stołami.
\sx Olo, Tytan, Sacharow?!~--~zdziwił się Wania.
\xx No, my.
\xx Chodźcie, wpuść ich.~--~rozkazał Powinnościowcowi stojącemu przy wejściu na zaplecze.
\qm
Mężczyźni zeszli do pokoju, w~którym odbyli spotkanie z samego rana. Tym razem jednak, dzięki obecności Sacharowa, mogli dokładniej przyjrzeć się dziwnemu artefaktowi, który Olo znalazł z Witkiem w~Starym Erneście.

W czasie, w~którym naukowiec sprawdzał właściwości tworu, handlarz wyciągnął Karta na pogawędkę, na korytarz. Odeszli kawałek od pomieszczenia, w~którym byli pozostali i~zaczęli rozmowę.
\sx Słuchaj, mam do ciebie interes.~--~zagaił Wania.
\xx \textit{Szto wam nużno, Iwan}?
\xx Pojawił się tu ostatnio jakiś szemrany typ. W~Kordonie nikt o nim nie słyszał, na wysypisku niby dwie osoby. Gość pojawił się z nikąd z całkiem dobrym sprzętem i~dziwnie się zachowuje. Znajomy widział go raz w~akcji, chłop jest obcykany z bronią, powalił chmarę psów jednym magazynkiem, po czym bez emocji wyciągnął fajka i~poszedł przed siebie. Trochę nienaturalne zachowanie, jak na świeżaka.
\xx \textit{Sołdat?}
\xx Nie jestem pewien.
\xx Dobra, a~co ja mam z tym wspólnego?~--~zmarszczył brwi Olo.
\xx Byłeś trepem, znasz ich. Powęszysz trochę wokół niego, powiesz mi, jakie licho go tu sprowadziło, ewentualnie skasujesz, jeśli zajdzie taka potrzeba. Dostaniesz gotówkę i~może jakieś świecidełka. Co ty na to?
\xx Liczę na zaliczkę\3k --~Kart podrapał się po brodzie.
\xx Ile?~--~spojrzał podejrzliwie handlarz.
\xx Broń i~amunicja.
\xx Bladź! Gdzie IL 86?
\xx Skończyły mi się pestki, zostawiłem go w~dziczy.
\xx Cholera!~--~zamyślił się.~--~Czekaj, czekaj. Jak to w~dziczy?! Co wyście tam robili?
\xx No\3k spotkaliśmy snorka w~tunelach. Janka rzucił w~niego granatem\3k
\xx I?!
\xx Chyba się zawalił\3k
\qd
\hspace{10.4em}Stos inwektyw jaki poleciał w~kierunku Karta był niewyobrażalny. Natychmiast z pokoju wychyliły się głowy Sacharowa i~Tytana, nawet Powinnościowiec stojący na ochronie obrócił się ze zdziwieniem. Wania to jednak w~gruncie rzeczy rozsądny typ.\looseness-1
\sx Dobra\3k Spokojnie\3k Podaruję Ci ten tunel. Ale tylko dlatego, że mam na tapecie tego żołnierzyka. Gdyby nie to, to\3k~--~urwał.
\xx Jasne\3k
\xx Wracajmy, zobaczymy do czego doszedł ten doktorek.
\qd
\hspace{25.3em}Sacharow stał przed stolikiem, na którym leżał artefakt i~coś notował, przerywając co chwila pisanie, by dotknąć tworu jakimś kablem. Tytan w~tym czasie siedział w~kącie i~palił papierosa. Po chwili przypatrywania się pracy naukowca, Olo w~końcu zagaił.
\sx No i~co, doktorze, coś z tego będzie? Czy to tylko bezwartościowa błyskotka?
\xx Będzie. I to sporo. Spotkałem się z czymś takim tylko dwa razy. Proszę mi powiedzieć, czy zaobserwowaliście jakieś szczególne działanie tego przedmiotu?
\xx Właściwie, to tak. Wyciągnęliśmy go z kałuży kwasu, tak zwanej Galarety. Mój towarzysz, Witek, poparzył sobie przy tym rękę. Jednak chwilę po tym, gdy złapał artefakt, jego dłoń natychmiast się zagoiła.
\xx Hmm\3k --~zamyślił się naukowiec.~--~Tak, to typowe przy tego rodzaju odkryciach. Przynajmniej według definicji. Jednak kompletnie nie pasuje mi jego przewodność. Jest zbyt duża. Będę musiał zbadać obiekt w~laboratorium.~--~powiedział, po czym schował artefakt do jakiejś foliowej torebki.
\xx Hola, hola! Doktorze! Nie robiliśmy tego wszystkiego w~ramach wolontariatu.~--~oburzył się Olo.
\xx Ależ naturalnie. Iwan, proszę, wypłać mu dziesięć tysięcy rubli.
\xx Ni chuja! Aż tyle?! Za co?!~--~wściekł się handlarz.
\xx Opanuj się, przyjacielu. Sam artefakt wart jest z siedem tysięcy. My, jako naukowcy, zawsze dopłacamy bonus za to, że obiekt nie idzie w~obieg, a~wspomaga rozwój techniki i~medycyny. Poza tym, ten człowiek uratował mi tam, w~Rostoku, skórę. No i~wreszcie, sypię mu trochę z dobrego serca. Nie widzisz, że jest kompletnie goły?
\xx Ale\3k --~zająknął się Wania.
\xx Powoli. Później się z tobą rozliczę.
\xx \textit{Eh, bladź}. Niech będzie. Chodź, Olo.
\qd
\hspace{17.6em}Mężczyźni pożegnali się. Kart wyszedł z handlarzem do sali ze stołami, zaraz po nich pojawili się tam również Sacharow i~Tytan.
\sx Janka!~--~krzyknął Olo.
\xx Co jest?
\xx Trzymaj, za wsparcie.~--~przekazał mu dwa tysiące rubli.
\xx Dzięki. Trzym się. Niech Cię Zona pochłonie!~--~przybili sobie piątki i~Tytan udał się w~sobie tylko znanym kierunku.
\qd
\hspace{15.5em}W tej chwili Kart spostrzegł w~kącie sali znajomą twarz. Podszedł do stolika.
\sx Cześć, Witek, jak żyjesz? Pocerowali cię trochę?
\xx Ta, jakoś się trzymam, a~co u ciebie? Gdzie byłeś, jak cię nie było?
\xx Złożyłem wizytę naukowcom w~bunkrze w~Jantarze.
\xx Pieprzysz! I co tam było?
\xx Trochę zombie, potem najemnicy w~Rostoku\3k \x Kart opowiadał o tym tak, jakby mówił, co zjadł dziś na obiad.
\xx A, no wiesz, te niby tajne labo pod fabryką, to prawda?~--~Witek zniżył ton do poziomu konspiracyjnego szeptu.
\xx A skąd ja mam to wiedzieć? Sacharow coś tam majaczył, ale go nie słuchałem. A, przy okazji, trzymaj, to za tą Błonę, którą znaleźliśmy w~Erneście.
\xx Cztery i~pół? Nipłocha.
\xx Dostałem dziesięć, ale dwa musiałem oddać Tytanowi. To ten, co nam wskazywał drogę, jak wyszliśmy z tunelu.
\xx Wiem, pamiętam go. Czemu bierzesz sobie mniej?
\xx Dostałem fuchę u Wani.
\xx Jaką znowu fuchę?
\xx Mam pokręcić się trochę za jednym typkiem i~wywiedzieć się kto on, skąd on i~po co tu jest. Nie mam siły tłumaczyć. Idę spać.
\xx Spoko, na razie!
\qd
\hspace{8.0em}Koledzy pożegnali się, po czym Olo ruszył w~kierunku kwater sypialnych.
\section*{ZAGADKA}
\mm Wstawszy rano, Kart poczuł lekki ucisk w~żołądku. Głód. Ubrał się i~skierował do głównej sali. Stanął przy ladzie i~czekał na barmana. Po chwili do kantorka wszedł wąsaty mężczyzna.
\sx \textit{Prywiet}, Wania!
\xx Cześć, Olo. Co tak stoisz z samego rana jak chuj po przebudzeniu?
\xx Na żarcie czekam, ot co! Co mi polecisz?
\xx Jakąś zupkę, czy wolisz coś mięsnego?
\xx Dawaj mięcho.
\xx Dobra, zrobię ci jakiegoś kotlecika.
\xx Ile?
\xx Dwadzieścia rubli.
\xx Trzymaj.~--~Olo przekazał pieniądze i~zasiadł przy wolnym stoliku.
\qd
\hspace{30.5em}Bar powoli zapełniał się stalkerami. Każdy coś zamawiał, Wania nie dawał sobie rady z szykowaniem posiłków, ale tak było co ranek, więc nikt nie miał mu tego za złe. Czasem ktoś dostał nie to, co sobie wybrał, czasem danie zdążyło wystygnąć, ale dla głodnego stalkera liczyło się tylko to, że wreszcie coś zje, a~nie, czy dostanie to, co zamówił, albo czy będzie to dobrze, czy źle przyprawione.
\pp
Olo siedział w~milczeniu i~przyglądał się pojawiającym się ludziom. W~pewnej chwili jego uwagę przykuł wysoki mężczyzna, z krótkimi, ciemnymi włosami, w~nowym kombinezonie, z karabinkiem AKS74/U przewieszonym przez ramię. Człowiek ten usiadł przy stoliku w~kącie, tam, gdzie wczoraj siedział Witek. W~tej chwili jednak Olo musiał przerwać obserwację. Został bowiem zawołany przez Wanię po odbiór swojego dania.\looseness-1
\sx Widzisz tego gościa we „Wschodzie Słońca”?~--~zagaił barman.
\xx Ta, zwrócił moją uwagę. Fajny ma kombinezon.
\xx Nowiutki, jeszcze bez zadrapań. To ten, o którym ci mówiłem.
\xx Śledzić go?
\xx Na razie z nim pogadaj. Może uda ci się dowiedzieć dokąd idzie. Potem idź za nim.
\xx Dobra, dobra, coś się wymyśli.
\qd
\hspace{15em}Olo wziął swój talerz z czymś co wyglądało jak kurczak i~usiadł koło tamtego mężczyzny.
\sx Cześć, wolne?~--~spytał z uśmiechem na twarzy.
\xx Yyy\3k tak, tak, proszę.~--~zająknął się facet.
\xx Co się tak stresujesz? Nowy jesteś?~--~zaśmiał się Kart.
\xx Właściwie, to tak. Przybyłem tu niedawno.
\xx Tak szybko dorobiłeś się „Wschodu Słońca”? To drogi kombinezon. Musiało ci się nieźle pofarcić, co?
\xx Tak. Miałem szczęście.
\xx A w~ogóle, co tak na pusto siedzisz?~--~rozejrzał się Olo.~--~Żadnej flaszki, żarcia, co ty tu w~ogóle robisz?
\xx Już jadłem\3k Yyy\3k --~zaciął się, marszcząc brwi.
\xx Olo, mów mi Olo.~--~wyciągnął rękę.~--~A ty?
\xx Diego.
\xx Hmm, ciekawe. Skąd taki pseudonim?
\xx Jeszcze z dziesięciolatki. Od nazwiska.
\xx Jasne.~--~Kart schylił się pod ławę i~wyjął półlitrową, szklaną butelkę.~--~Chodź, napijemy się.
\xx Nie, nie, dzięki. Ja nie piję.
\xx O nie, nowy jesteś, to nie wiesz. Wóda zbija promieniowanie, bez niej długo tu nie pociągniesz. Leki są drogie.
\xx Skoro tak twierdzisz\3k --~niepewnie wyciągnął rękę.
\qd
\hspace{24.1em}Wypili kieliszek za udane łowy i~kontynuowali rozmowę. Siedzieli w~barze dość długo. Prawie wszyscy opuścili pomieszczenie, a~oni wciąż gadali i~pili wódkę. Pod koniec butelki Olowi udało się wyciągnąć, w~jakim kierunku zmierza Diego. Okazało się, że musi dostać się na Zaton, przechodząc przez wysypisko. Celu podróży jednak nie wyjawił.
\pp
Kart posiedział chwilę, po czym powiedział, że musi się przewietrzyć. Złapał plecak i~wyszedł. Udał się do jednego z garaży, posiedzieć z wolnymi stalkerami i~wytrzeźwieć nieco. Po jakiejś godzinie wrócił do handlarza po obiecaną broń. Dostał najzwyklejszego kałasznikowa i~trochę pestek do niego. Kupił też wódkę i~jedzenie.
Spakował się niechlujnie, rozlokował zakupy po plecaku i~kieszeniach i~wyruszył na południe, w~kierunku złomowiska pojazdów.
\section*{DOTYK ŚMIERCI}
\mm Była najpóźniej ósma rano, kiedy Kart opuścił granice obozu. Minął posterunek Powinności, chwilę potem mostek przerzucony ponad zasiekami. Dojrzał też legowiska zmutowanych psów wylegujących się w~porannym słońcu. Było ciepło. Wystarczająco ciepło by Olo zarzucił swoją kurtkę na ramię i~szedł w~samym swetrze. Pod kurtką skrył się AK47, więc wydawało się, że jest całkowicie bezbronny.
\pp
Dotarł po chwili do kolejnego posterunku. Tutaj znajdowała się granica obszarów. Między dwoma wzgórzami stało trochę blachy i~brama „pożyczona” pewnie z jakiejś fabryki. Po drugiej stronie stał jeszcze stary wagon i~jakieś betonowe kloce. Olo przeszedł na drugą stronę, po czym został zatrzymany przez gościa w~czarnym kombinezonie.
\sx Dokąd to, stalkerze?~--~zapytał złowrogo, ale Kart wiedział, że chłopcy z Powinności lubią zgrywać bohaterów.
\xx Dobra, nie wygłupiaj się. Idę do hali pogadać ze\3k
\xx Cholera, wielkie stado dzików biegnie tu od doliny mroku! Sasza, ratuj! kurwa, co robić?!~--~nagle, krzycząc w~niebogłosy, pojawił się jakiś młodzik.
\xx Uspokój się i~łap za broń. Rozwalimy je. Stalkerze, pomożesz?
\xx Niech będzie. Ale uzupełnicie potem moją amunicję.
\xx Dobra. Rusz się!
\qd
\hspace{9em}Olo rzucił swoją kurtkę na ziemię i~ustawił się za betonowym elementem. Czekał, z celownikiem przyłożonym do oka. W~pewnej chwili zza wzgórza wypadło dość pokaźne stado bestii. Nie były one łatwymi przeciwnikami. W~starciu, gdy na jednego człowieka przypadały trzy lub więcej, zwykle jedynym sposobem zachowania życia była ucieczka. Stworzenia te ważyły do dwustu kilogramów i~osiągały wysokość w~kłębie do półtora metra. Stąpały po ziemi wielkimi kopytami. Potrafiły też szarżować, a~z potężnymi kłami wyrastającymi z czaszki umiały narobić szkód. Zabić je było dość trudno. Czasem dwa pociski śrutowe wystrzelone z bliskiej odległości nie wystarczyły, by te bestie unieszkodliwić.

\mm Tymczasem jednak chmara zbliżała się coraz bardziej. W~pewnej chwili Powinnościowiec otworzył ogień. Zaraz potem dołączyli też pozostali. Mutanty zaraz zastrzelono, choć dwa z nich padły dosłownie na wyciągnięcie ręki.
\sx Dobra, tyle ze mnie. Pomogłem wam. Teraz wy pomóżcie mi.
\xx Hola, hola! Myślisz, że sobie postrzelasz i~coś za to dostaniesz? Nie ma tak łatwo.
\xx Słuchaj, kolego. Nie radzę ci zmieniać warunków umowy, bo może się to dla was nieciekawie skończyć. Przyjmiecie po kulce, a~kto poświadczy, że ja tu w~ogóle byłem?
\xx Patrz, żebyśmy my z tobą nie skończyli! Rozwalimy cię, nawet nie zauważysz kiedy.
\xx A chuj wam w~lufy. Nie będę się kłócił o dwa magazynki do AK.
\xx Masz rację. Wynoś się stąd, stalkerze!
\qd
\hspace{18.4em}Olo wziął swoją kurtkę i~ruszył w~stronę hali naprawczej, nucąc coś bezładnie. Wiedział, że mógłby ich pozabijać bez mrugnięcia okiem, ale byłoby to wbrew wszystkim jego poglądom. Poza tym, gdyby nie ten posterunek, to bar „100 Radów” nie byłby bezpieczny. Czasem trzeba się dobrze zastanowić nad konsekwencjami swoich działań. Olo, jako były zawodowy żołnierz wiedział o tym doskonale.
\pp
Szedł powoli, rozglądając się uważnie w~poszukiwaniu zagrożeń, ale też podziwiając piękno ukraińskich lasów. Ten widok złotoczerwonych liści osadzonych na brązowych pniach oraz rozrzuconych bezładnie na ziemi wokół nich zawsze przyprawiał go o zachwyt. Chciał pójść między drzewa, usiąść na polanie i~rozkoszować się śpiewem ptaków. Wiedział jednak, że to niemożliwe. Pomijając to, że w~Strefie próżno było szukać jakiejkolwiek „normalnej” zwierzyny, w~lasach aż roiło się od tworów Zony, mutantów i~anomalii. Artefaktów natomiast było relatywnie mało. Jeśli ktoś chciał popełnić samobójstwo, wystarczyło wejść bez broni do lasu. Nawet unikając pułapek grawitacyjnych można było nieopatrznie się w~coś wpakować.
\pp
Kart jednak nie myślał o skończeniu ze sobą. Miał zadanie do wykonania. Przeszedł obok hałdy napromieniowanych śmieci, a~jego oczom ukazał się duży, ceglany budynek. Za czasów ZSRR była tu lokomotywownia, jedna z kilku na obszarze wokół elektrowni. Teraz jednak służyła stalkerom jako noclegownia, schronienie przed deszczem i~przed emisją. Najczęściej przebywali tu wolni zbieracze, którym udało wydostać się z Kordonu. Jednak cały czas borykali się z bandytami, koczującymi w~zawalonym tunelu nieopodal. Zbójom bardzo podobał się budynek i~chcieli go sobie przywłaszczyć. Raz im się to nawet udało, ale niemal natychmiast wpadł tam uzbrojony oddział Powinności. Poskutkowało, na jakieś dwa, może trzy tygodnie. Wykurzyli bandytów z tego terytorium. Niedobitki przegoniono aż do kordonu. Niestety, gnidy nie poddają się tak łatwo. Wrócili już po kilku tygodniach. Teraz na szczęście panował tam względny spokój.
\pp
Olo dotarł do starej bramy. Na placu walały się jeszcze jakieś narzędzia, stały przerdzewiałe samochody pozostawione przez pracowników. Było widać, że halę przed prawie trzydziestu laty opuszczono bardzo szybko. Teraz jednak nigdy nie było tam pusto. Zawsze ktoś siedział przy ognisku wewnątrz budynku. Nie inaczej było i~tym razem. Olo dostrzegł czerwoną łunę przebijającą się przez uchylone wrota. Wszedł powoli do środka i~usłyszał dźwięk gitar i~śpiew. Stalkerzy. Jest bezpiecznie. Zawołał, by nie potraktowano go jak wroga i~spokojnie wychylił się zza rogu. Jego oczom ukazała się stara lokomotywa i~kilka wagonów stojących tu od pamiętnego 1986 roku. Pomiędzy nimi wesoło migotały ogniste języki. Wokół beczki, z której wydobywał się płomień siedziało trzech mężczyzn. Dwóch z nich było w~zwykłych kurtkach, trzeci miał na sobie lekko podrapany, zielony kombinezon. Kart podszedł do nich i~przywitał się. Następnie usiadł przy ognisku i~czekał na Diega. W~jego głowie rodził się plan, jak nie zdemaskować swojego
zamiaru. Jednocześnie, siedząc, przysłuchiwał się rozmowom stalkerów.
\pp
Zawsze mieli do powiedzenia coś ciekawego. Przy ognisku można było dowiedzieć się kilku ciekawych plotek z życia Zony. W~obozach, na opuszczonych stacjach benzynowych, wewnątrz przystanków autobusowych i~wszędzie tam, gdzie przebywali wolni, przyjacielsko nastawieni stalkerzy, dało się zwyczajnie usiąść i~posłuchać. Gdy w~okolicy nie było żadnego zagrożenia, panowała tam istna sielanka. Śpiewy, wódka, gry, dowcipy i~inne typowo towarzyskie zachowania. Kart nie przywiązywał uwagi do tego, z kim przebywa, o ile ten ktoś nie miał zamiaru zrobić mu nic złego. Stalkerzy w~lokomotywowni niemal nie spostrzegli, gdy się do nich dosiadł. Akurat jeden z nich opowiadał dowcip:
\sx Słuchajcie! Spotyka się trzech facetów. Niemiec, Francuz i~Rosjanin. No i~tak gadają, gadają, to o tym, o tamtym, popijają piwko. W~pewnym momencie rozmowa zeszła na samochody. Zaczęli się kłócić, kto produkuje najlepsze. No, to Niemiec, jak to Niemiec, mówi: „Mercedes!”. Francuz wybiera: „Hmm\3k Peugeot\3k czy Renault\3k Renault!”, a~Rosjanin po chwili zastanowienia wypala: „Kamaz!”. Tamci dwaj zdziwieni, pytają, czemu akurat Kamaz, a~Rusek z uśmiechem na ustach: „A ktoś z was kiedykolwiek widział serwis Kamaza?”
\qd
\hspace{20.4em}Stalkerzy wybuchli śmiechem. Zaraz potem z wagonu wychyliła się twarz. Zaspana, jak po trzech dniach picia. Nieogolona, brudna, z polepionymi kłakami rozrzuconymi bezładnie na czole i~podkrążonymi oczami. Facet krzyknął tylko:
\sx  \textit{Blad'!} Dajcie się wyspać!
\qd
\hspace{12.8em}Któryś odpowiedział, by się nie denerwował. Tamten schował się w~wagonie.
\sx Chłopaki, kto to?~--~odezwał się Olo.
\xx To Kuba Białorus. Siedzi w~Zonie od roku. Kiedyś \textit{bandzit}. Po tym, jak Powinność pogoniła go aż do obozu Żory w~Kordonie zmienił się. Rzucił to gówno i~został normalnym stalkerem, jak my wszyscy.
\xx On zawsze tak wyglądał?
\xx Nie, dopiero niedawno do nas wrócił. Te skurwole trzymały go w~piwnicy przez tydzień, jak się dowiedziały, że kiedyś był jednym z nich. Przyszedł tu, poprosił o jakieś żarcie. Kurwa, jak on wyglądał! Poobijany, brudny, wychudzony. Normalnie tragedia. Od razu daliśmy mu jeść i~coś mocniejszego. Wczoraj znalazł jakąś błyskotkę, to ją sprzedał jakiemuś kotu za dużą kasę i~nakupował wódy. Ale było chlanie!
\xx A wiadomo, kto go tak urządził? Kapuś, czy go poznali?
\xx Słyszałem, na parkingu, że wkręcił go ktoś obozu kotów. Jakoś\3k na „A”? Awadan, Awatar, chuj go wie.
\xx Awdan?!~--~nachylił się Olo.
\xx Dokładnie! Znasz go?
\xx Jeśli to prawda, to mnie chyba też wkopał. Chuj.
\xx Pierdolisz! A co zrobił?
\xx Aj, długa historia\3k~--~Olo wyjął z plecaka termos z herbatą.~--~Co się z nim dzieje? Widziałeś go ostatnio?
\xx Nie, ale ponoć dalej siedzi w~wiosce kotów.
\xx Jeśli to prawda, to trzeba się dowiedzieć, po co wsypuje i~komu dokładnie\3k
\qd
Stalkerzy siedzieli od tej pory w~ciszy. Nagle minął im cały humor. Olo wyjął PDA i~napisał Wani, że chyba musi przerwać zadanie. Handlarz zezłościł się, ale pozwolił mu iść, pod warunkiem, że znajdzie na swoje miejsce kogoś innego, „równie kompetentnego”, kto pójdzie za Diegiem. Kart spytał więc nowych znajomych. Zapowiedział dobre pieniądze, za samo siedzenie na ogonie „pewnego typka”. Ten, z którym rozmawiał, powiedział, że on z wysypiska się nie rusza, ale drugi, Jurij Owad, zgodził się. Kart wysłał informację barmanowi. Nie poszedł jednak od razu do Kordonu. Pozostał w~hali.
\pp
Była może jedenasta. Słońce grzało coraz bardziej. W~pewnym momencie, z lewej strony coś skrzypnęło. Stalkerzy, jak na zawołanie chwycili za broń i~wycelowali w~kierunku wrót. Po chwili pojawił się w~nich facet w~kombinezonie. Podniósł ręce do góry i~powoli zdjął kaptur. Potem maskę przeciwgazową. To był Diego. Podszedł powoli do ognia.
\sx Mogę z wami posiedzieć?~--~zapytał.
\xx \textit{Kanieczno}, siadaj.~--~odpowiedział Jurij.
\xx Dziękuję.
\xx Cześć, Diego!~--~podniósł wzrok Olo.
\xx O, witaj. Co tu robisz?
\xx Idę do Kordonu. A ty?
\xx Ja na Zaton.
\xx Tędy? Bandyci, Powinność, wojsko. Jesteś pewien?
\xx Dam sobie radę.
\qd
\hspace{8.6em}Kart wyjął z kieszeni kartkę i~coś na niej zapisał. Podał Jurijowi. Ten przeczytał, zgniótł papier i~wrzucił w~ogień. Następnie skinął Dawidowi głową.
\pp
Humor powoli powracał. Mężczyźni znów opowiadali między sobą legendy i~żarty. Zaraz po południu Diego wstał, rzucił, że idzie dalej, złapał za broń i~wyszedł z hali.
\sx To ten?~--~zapytał Jurij.
\xx Tak. Poczekaj z dziesięć minut i~ruszaj. Potem idź od razu do Wani, w~„Stu radach”.
\xx \textit{Paniatno}.
\qd
\hspace{6em}Po paru chwilach, złowrogo zapiszczały PDA całej trójki. Zbliżała się kolejna emisja. Oni byli bezpieczni, ale Diego nawet nie wiedział, czym jest ta anomalia pogodowa Strefy. Kart wymamrotał jakieś przekleństwo, wstał, splunął na ziemię i~oznajmił, że idzie po gościa. Musiał dowiedzieć się kim jest, skąd jest i~czego chce. Jak to mówią, po trupach do celu.
\pp
W lokomotywowni zostawił karabin, wziął jedynie małego PMm i~kilka magazynków, tak na zaś. Wybiegł wzdłuż torów i~pośpiesznie, zdając się na doświadczenie, zaczął lawirować między anomaliami. Nie było czasu na bawienie się w~dostrajanie wykrywacza i~rzucanie śrubkami. Po prostu biegł, szukając śladów w~trawie i~unikając ciemnych plam na ziemi.
\pp
Dotarł właśnie do drogi, gdy usłyszał głuche uderzenie. Pozostały mu góra trzy minuty na znalezienie Diega i~ukrycie się. Wbiegł na wzgórze. Dostrzegł w~oddali znajomą postać i~próbował doń krzyknąć. Jednak w~tej samej chwili zerwał się wicher.
\sx Kurwa!~--~zawył rozpaczliwie.
\qd
\hspace{14.4em}Biegł dalej, prosto do celu. Przez łzawiące od podmuchów wiatru oczy widział, jak mężczyzna, którego goni łapie się za głowę i~upada. Po chwili podniósł się i, na klęczkach, z wielkim trudem, próbował ruszyć w~jakąkolwiek stronę. Olo dopadł stalkera w~momencie, gdy chmury zaczynały „eksplodować”.
\sx Co to jest?!~--~spytał Diego głosem pełnym trwogi.
\xx Nie gadaj, tylko biegnij!~--~Kart złapał go za ramię.~--~Do samochodu!
\qd
\hspace{31.68em}Przy drodze prowadzącej do Instytutu Agroprom stała porzucona wojskowa Wołga. Była kompletnie przerdzewiała, jednak dawała większą szansę na przeżycie niż pozostawanie na otwartej przestrzeni. Olo zaczął właśnie odczuwać skutki emisji. Uczucie puchnącej głowy, rozdwajanie się obrazu przed oczyma, czy w~końcu pulsujący, silny ból wewnątrz oczodołów. Mężczyźni dopadli do auta, ale nie wskoczyli do środka. Postarali się wpełznąć pod podwozie, odgradzając się od szalejącej anomalii jak tylko się dało.
\pp
Olo starał się złapać instynktownie za głowę, próbując powstrzymać ból. Sapał przy tym ciężko i~boleśnie, zmęczony szaleńczym sprintem. Jego kompan miotał się w~tym czasie jakby w~konwulsjach, jednak z powodu maski gazowej, jaką miał na twarzy, dało się usłyszeć, jak oddycha nie mniej ciężko niż on sam. Po chwili ziemia zaczęła lekko drżeć. Jednolity, odgłos gorączkowego nabierania powietrza zakłócały teraz co jakiś czas jęki. Samochód, pod którym leżeli stalkerzy, przesuwał się wraz z ruchami gruntu, a~elementy jego podwozia wbijały się w~kombinezony i~ciała mężczyzn powodując silny ból.
\pp
Emisja właśnie zbliżała się do maksimum. Ziemia trzęsła się coraz mocniej, ból wewnątrz czaszki zaczynał być nie do zniesienia. Nagle pojawił się bardzo dziwny, niewytłumaczalny odgłos, jakby nie mający źródła. Zaczynał się od cichego pomruku, a~kończył na wysokim, niemal ultradźwiękowym pisku rozwiercającym głowę od środka. Dźwięk ten obejmował cały obszar zwarcia, wraz ze wszystkimi istotami żywymi znajdującymi się w~jego zasięgu.
\pp
Kilka sekund później nagle wszystko ustało. Wiatr osłabł, tajemniczy dźwięk powoli zanikał, grzmoty ucichły. Mężczyźni jednak nie ruszali się spod samochodu. Leżeli bezwładnie, oddychając ciężko. Ich sponiewierane, obolałe ciała nie miały dość siły, by się wydostać. Olo z trudem otworzył oczy. Wydawało mu się, że jego powieki ważą kilka kilogramów. W~szczelinie, między podwoziem, a~starym, spękanym asfaltem nie dostrzegł absolutnie niczego. Dopiero po paru chwilach jego oczom zaczęły ukazywać się poszczególne elementy krajobrazu~--~najpierw nawierzchnia drogi, przy której porzucono samochód. Spoczywało na niej kilka martwych wron. Po drugiej stronie szosy wyrastały młode brzózki, zza których złowrogo patrzyły wysokie sosny i~świerki. Świat nabierał kolorów, stawał się też coraz jaśniejszy. Stalker z trudem przesunął rękę po zimnym betonie i~otarł pot z czoła. Był potwornie zmęczony. Przypomniał sobie o uratowanym koledze.
\sx Die\3k Die\3k Diego. Ży\3k żyjesz?~--~zapytał słabym szeptem.
\xx Mhm\3k \x ten mruknął jeszcze boleśniej.
\qd
\hspace{19.4em}Dawid zbierał w~sobie siły, by wydostać się z tej żelaznej osłony. Coś go jednak podkusiło, by spojrzeć jeszcze raz w~kierunku gaju po drugiej stronie drogi. Spostrzegł, że drzewa nienaturalnie wyginają się i~jakby rozmazują. Skierował oczy w~dół. Martwy ptak przesuwał się bezwładnie w~jego kierunku, ciągnięty jakąś tajemniczą siłą. Olo wytężył słuch tak bardzo, jak tylko pozwalało mu w~tej sytuacji jego ciało. Usłyszał niskie, ciche buczenie. „To nie może być prawda!”~--~pomyślał. Rzeczywistość jednak nie kłamała. Oto przed nimi, w~odległości może metra od samochodu pojawiła się anomalia grawitacyjna. Stalker wolał się jednak upewnić. Ptak, z każdą sekundą zbliżał się coraz bardziej. Nabierał też przy tym szybkości. W~pewnym momencie zniknął, w~gwałtownym obłoku, któremu towarzyszył syk powietrza. Nic z niego nie zostało. Ani jedna kość, żadna kropla krwi. Wchłonęła go Zona.
\sx Kurwa.~--~szepnął.~--~Diego, ocknij się!~--~szturchnął kompana w~ramię.
\xx Ee\3k Nie mam siły.~--~syknął z bólu.
\xx Rusz się i~sprawdź, czy można się stąd wydostać!
\xx Co? Jak niby?~--~pytał słabym głosem.
\xx Obserwuj. Widzisz coś nienaturalnego?
\xx Nie\3k Wszystko w~porządku, a~co?
\xx To dobrze. Z tej strony mamy anomalię.
\xx Co to znaczy?~--~zapytał Diego.
\xx To znaczy, że mamy przesrane! Wyłaź stąd, do cholery.~--~Olowi puszczały nerwy.
\qm
Diego z całej siły próbował wydostać się spod samochodu, jednak nie miał dość siły i~miejsca, by przecisnąć się pod podwoziem. Postanowił oczekiwać pomocy. Było na tyle ciasno, że nie dał rady sięgnąć nawet po swoje PDA.
\pp
Po kilku godzinach bezowocnego leżenia i~oczekiwania stalkerzy dostrzegli na horyzoncie grupę kilkunastu ubranych na czarno mężczyzn. Szli oni w~kierunku hali, ustawieni w~szyk bojowy. Kilku na czele oświetlało drogę latarkami, gdyż słońce własnie znikało za widnokręgiem. W~pewnej chwili snop światła przemknął tuż przed samochodem. Grupa przystanęła na chwilę. Po krótkiej naradzie dwóch z nich wyciągnęło pistolety, Olo usłyszał charakterystyczny szczęk towarzyszący odciągnięciu zamka. Postanowił udawać martwego, szepnął też to samo swojemu koledze.
\pp
Tajemniczy mężczyźni ruszyli w~stronę wraku. Zbliżyli się na około metr, przykucnęli. Olo słyszał, jak konserwa będąca w~kieszeni płaszcza jednego z nich ociera się o materiał przy każdym ruchu. Leżał twarzą do ziemi, z zamkniętymi oczami, a~mimo to niemal czuł na sobie światło latarki, którą przybysze oświetlali „ciała”.
\pp
Usłyszał dźwięk odciąganego kurka. Strzał. Światlo latarki przy kasku gaśnie. Mężczyzna upada, po kolejnym strzale jego koleżka również leży bez ducha. Rozpętała się strzelanina. Olo wydedukował, że bandyci chcieli o zmierzchu zakraść się do hali i~przepędzić z niej stalkerów, którzy jednak dowiedzieli się o tych planach i~postanowili się na nich zaczaić.
\pp
Modlił się, jak nigdy bał się śmierci. W~wyobraźni ujrzał zmarłą żonę i~córkę. Widział, jak do nich dołącza. To pobudziło go do działania. Nie mógł tu umrzeć. Nie teraz. Nie jak pies. Próbował się wygrzebać, małymi, milimetrowymi ruchami wydostać spod tej żelaznej trumny.
\pp
Był już w~połowie na zewnątrz, gdy usłyszał niedaleko głuchy, metaliczny huk. Jakby na ziemię upadła jakaś puszka wypełniona kamieniami. Obejrzał się w~lewo. Po drugiej stronie drogi toczył się granat. Wiedział, co to oznacza. Zdążył tylko wtulić twarz w~ramiona i~przylec do zimnego asfaltu.
\pp
Wybuchowi towarzyszył huk, jaki towarzyszy wypadkom samochodowym przy dużej prędkości. A z drzew opadały powoli złoto-brązowe liście, skąpane w~czerwonym blasku zachodzącego słońca.
\begin{center}
KONIEC
\end{center}
\end{document}