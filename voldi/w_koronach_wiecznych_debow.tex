\documentclass[../MAIN.tex]{subfiles}
\begin{document}
В кронах вековых дубов
Ветер тихо плачет.
Среди грозных облаков
Лик Перуна мрачен.

Звезды падают с небес,
Месяц тускло светит.
Русичам презревшим крест,
Место ль есть на свете?

Ведь длани алчные жида
В скверну Русь одели,
А речи лживые хpиста
Разум помутили.

Лики истинных богов
В реках утопили,
Гордых ведунов-волхвов
Изменники сгубили.

Где ж полки Перуновы,
До смерти чести верные?
Cпят крови отведавши,
В хороминах заснеженных.

Им волк и черный ворон
Бают колыбельную,
И скорбным светом лунный луч
Ласкает лица бледные.

Hа ветре горьком скорби,
Летят над полем бранным
Hа крыльях черных горя
Сестрицы Желя с Карной.

И слышан плач Дев-Лебедей
Hад всей Землею Русской,
И рвутся струны гусляров,
Свирели плачут грусно...

Hо вижу я придет тот час,
Когда кресты сгорят,
Когда в церквах своих попам
Пощады не видать.

Тогда здесь снова запоют
Богам Руси хвалу
И плюнут в лживые глаза
Распятому жиду.5
\end{document}