\documentclass[../MAIN.tex]{subfiles}
% \twocolumn
\begin{document}
\ro{CZĘŚĆ I}
% 
\sx Bazyl, udało się! To tu, dotarliśmy!~--~Wykrzyknął Andriej, wyczołgując się z tunelu ukrytego w starej toalecie publicznej.
\xx Już? Chryste, jak dobrze. Jestem taki zmęczony.~--~Odparł towarzysz, padając na popękaną posadzkę.
\xx Podnieś się. Zaczyna się robić ciemno, trzeba znaleźć jakieś bezpieczne schronienie.
\qd
Andriej zgasił swoją latarkę, po czym pchnął lekko zbutwiałe, drewniane drzwi. Te jednak nie ustąpiły. Zawiasy zapiekły się przez prawie trzydzieści lat tak bardzo, że trzeba było użyć naprawdę dużej siły, by je poruszyć. Spróbował ponownie. Silniej. Metal skrzypnął, drzwi uchyliły się, a pomieszczenie zalało ciepłe, czerwone światło. Słońce chyliło się już ku zachodowi.
\sx
Pamiętaj, zachowujemy się tak cicho, jak tylko można. Nie mam ochoty oddawać krwi jakiejś pierdolonej pijawce ani zostać nadzianym na pal przez Monolit!~--~szepnął Andriej.
\xx A myślisz, że ja mam?! Po tym, co dziś przeszedłem chcę tylko w spokoju zasnąć.
\xx To rozumiem. Idziemy!
\xx Czekaj!~--~Bazyl szarpnął kompana za ramię.~--~Przecież mamy mapę, może sprawdzimy chociaż, gdzie jesteśmy?
\xx Niegłupie. Dawaj tą mapę, Koczowniku.~--~odpowiedział Andriej, cały czas szepcząc.
\qm
Bazyl sięgnął do kieszeni w spodniach i wyciągnął złożony w kostkę kawałek papieru. Drugi mężczyzna wyrwał mu ją z ręki po czym pospiesznie rozwinął. Przykucnął i włączył swoją czołówkę. Mocny strumień światła padł na papier, przy okazji oświetlając toaletę. Andriej spojrzał na nagłówek: \textit{„Karta Prip’yat, 1985 rik. Wykorystowujte tilki armija UNR”}, co oznaczało: „Mapa Prypeci, rok 1985. Wyłącznie do użytku Armii Czynnej Ukraińskiej Republiki Ludowej”. Schematyczna mapa pokazywała wszystkie budynki: bloki mieszkalne, szpitale, szkoły, naniesiono na nią lokalizację urzędu miasta, basenu, a nawet sklepu ogrodniczego. Zawarto w niej również informacje tajne dla ludności cywilnej takie jak schrony, podziemne magazyny, tunele, w tym ten prowadzący do fabryki Jupiter, czy drogi ewakuacyjne. Teraz jednak te dane nie były potrzebne. Andriej znalazł punkt, w którym się znajdowali. Określony został „Wyjściem awaryjnym z tunelu Prypeć-1”.
\sx
Sto, może sto trzydzieści metrów za nami mamy szkołę średnią. Tam przenocujemy. Chowaj mapę. Przyda się jeszcze.
\xx Dobra, gotowy do drogi? Idź pierwszy, Włóczęgo.
\qd
Andriej zgasił latarkę. Wyprostował się i podszedł do drzwi. Wychylił z nich samą głowę, tylko nasłuchując. W oczy rzuciła mu się znajoma sylwetka diabelskiego młyna, teraz okraszona rudymi promieniami słońca. Pływała ona spokojnie w błyszczących na zielonkawy kolor liściach, tak jakby nigdy się tu nic nie wydarzyło, jakby nie było to miejsce jednego z największych dramatów XX wieku. Cisza panująca w Prypeci aż dzwoniła w uszach.

Andriej otrząsnął się z tej zadumy i skinął ręką, dając znak Bazylowi, by ten ruszył za nim. Wyszli z toalety. Obeszli ją wzdłuż ściany po czym, powoli przedzierając się przez krzaki, skierowali się do betonowego gmachu przed nimi. Cały park był zarośnięty rozmaitymi krzewami, trawami, a gdzieniegdzie ponad zielony mur przebijały się jakby wieże, wzniesione z młodych brzóz i dębów.

Mężczyźni byli obyci z przedzieraniem się przez chaszcze. Nie bez powodu jednego nazwano Włóczęgą, a drugiego Koczownikiem. Andriej znany był z tego, że potrafił przedostać się do niemal każdego miejsca, nie poruszał się chyba tylko wtedy, kiedy spał. Bazyl podobnie. Jak wskazuje jego przydomek~--~nigdzie~nie zostawał w tym samym miejscu dłużej niż dwa, góra trzy, dni.

Przedarcie się przez małą dżunglę nie trwało długo. W ciągu trzech minut znaleźli się przy ścianie budynku, do którego zmierzali. Pozostało im jedynie odnalezienie wejścia do gmachu i wyszukanie jakiegoś cichego miejsca, w którym mogli przenocować.

Okrążyli budynek, do środka dostali się drzwiami prowadzącymi na salę gimnastyczną. Były zablokowane, toteż Bazyl posłużył się solidnym kopnięciem, by ustąpiły. Narobił przy tym nieco hałasu, więc mężczyźni wbiegli do hali, po czym postarali się zatrzeć za sobą ślady. Zwrócili uwagę na korytarz łączący salę z resztą szkoły. Ostrożnie, unikając sprzętu sportowego rozrzuconego bezładnie pod nogami, ruszyli w jego stronę.
\sx
Tam powinna być jakaś szatnia, albo chociaż kanciapa dla nauczycieli.~--~rzucił Bazyl.
\qm
Weszli po spróchniałych stopniach na jakiś tandetny parkiet. W korytarzu panował półmrok, na prawym jego końcu widać było promienie słońca, przebijające się przez zakurzone okna i stare framugi. Tam był główny hol szkoły. Z lewej strony mieściły się szatnie. Stalkerzy ruszyli w ich stronę, nie zapalając jednak latarek w obawie przed zwróceniem na siebie uwagi.

Jedne z drzwi były uchylone. Weszli do pomieszczenia. Na obu ścianach zamocowane były rzędy wieszaków. Na niektórych z nich do dzisiaj wisiały przegnite koszulki, spodenki czy lniane torby na strój gimnastyczny. Na jednym z wieszaków Andriej zauważył nawet skórzaną torebkę, pewnie jednej z uczennic. Wokół walało się też kilka masek gazowych. Zapewne ogłoszenie o ewakuacji dotarło do pracowników szkoły w trakcie lekcji. Ciekawe, co teraz dzieje się z tymi ludźmi. Żyją? Jeśli, tak, to gdzie i w jakich warunkach? Andriej nie mógł znać odpowiedzi na to pytanie. Pewnie nikt, poza samymi przesiedlonymi, nie zna na nie odpowiedzi.

Położyli się pod ścianą, przy drzwiach, by słyszeć każdy ewentualny szmer. Nie od dziś wiadomo, jak bardzo trzeba uważać, nocując w nieznanych miejscach. Tym bardziej, jeśli takie miejsca wybiera się w zamkniętej strefie, tak blisko elektrowni.
\sx
No, całkiem przytulnie. Hotel Mir to to nie jest, ale spać się da.~--~Zagaił Bazyl.
\xx Masz rację, spać się da.~--~Odparł półprzytomnie Andriej.
\xx Mogę cię o coś spytać?
\xx Pewnie, pytaj śmiało.
\xx Dlaczego właściwie tu jesteśmy?~--~Spytał Koczownik, było jednak słychać, że niełatwo przeszło mu to przez gardło.
\xx No tak. Najwyższy czas, byś się wreszcie dowiedział\3k~--~Zaczął Andriej.~--~Ja\3k Ja tu kiedyś mieszkałem, przed katastrofą.
\xx Co?! Znamy się od studiów, czemu wcześniej mi o ty\3k
\xx Nie tak głośno. Chcę dożyć jutra.~--~Przerwał Bazylowi.~--~Nie mówiłem o tym wcześniej, bo znając ciebie, zbytnio byś się napalił. A ja potrzebowałem cierpliwości i czasu.%\looseness=-1
\xx Mamy teraz po trzydzieści dwa lata\3k~--~Zastanowił się.~--~Ile miałeś, kiedy to się stało?%\looseness=-1
\xx Sześć, może siedem, nie pamiętam. Jednak ja\3k~--~Andriej przerywał. Mówienie nie przychodziło mu zbyt łatwo. Było widać ból w jego oczach.~--~Przenieśli nas do Doniecka. Tam kazali zapomnieć o wszystkim co się wydarzyło, zacząć nowe życie. Dopiero dziesięć lat później, grzebiąc w starych albumach zrozumiałem, czego byłem świadkiem. Zamarzyłem odwiedzić rodzinny dom. Dom, w którym się urodziłem. Miejsce, z którego pochodzę. Rozumiesz? W tym roku minęło dwadzieścia pięć lat\3k Do tego, na studiach w Kijowie, trafiłeś się ty, pamiętasz? Zafascynowany Czarnobylem i Zoną.
\xx Znasz adres?~--~Nieśmiało odezwał się Bazyl.
\xx Tak. Przyjaźni Narodów 13/73. Naprzeciwko szpitala.~--~Odpowiedział Andriej obojętnie, patrząc w przestrzeń przed sobą.~--~Chodźmy spać.
\xx Chodźmy, jutro ciężki dzień. Śpij spokojnie.
\qd
\ro{CZĘŚĆ II}
Obudzili się o świcie. Poranek nie był tak ładny, jak zmierzch poprzedniego dnia. Budził raczej nastrój niepokoju i w pewnym sensie grozy. Andriej wstał, przejechał ręką po twarzy po czym zabrał się za zwijanie śpiwora. Bazyl jeszcze dosypiał. W pewnej chwili Włóczęga zauważył małe okienko w ścianie stojącej naprzeciw niego. Nie dostrzegł go poprzedniego dnia. Podszedł i oglądał przez nie mgłę, która rozlewała się wokół szkoły jak mleko, otulając wszystko dookoła gęstym, szarym puchem.
\sx
Niedobrze\3k~--~powiedział sam do siebie.~--~Widoczności\3k góra na pięć metrów.
\qm
Po chwili ciszy, mrucząc pod nosem, dodał:
\sx Cholera.
\qd
Obrócił się i wrócił do ściany przy drzwiach. Usiadł przy towarzyszu, a następnie lekko trącił go łokciem. Bazyl natychmiast się obudził. Andriej dał mu trochę czasu na oprzytomnienie, poczekał, aż spakuje wszystko, z wyjątkiem jedzenia. Sam sięgnął do plecaka i wyciągnął bochenek chleba oraz butelkę wódki. Koczownik zaś przygotował wodę, konserwę i jakiś owoc.

Nim zaczęli jeść, odmówili krótką modlitwę, dziękując Bogu za dotarcie do Prypeci, za znalezienie schronienia i przeżycie nocy, a także prosząc Go o pomoc w drodze do celu podróży i o szczęśliwy powrót.
\sx Amen.
\xx Amen.
\xx No dobrze, panie kolego. Polejcie, za przeżytą noc.~--~Zawołał cicho Bazyl.
\xx Już się robi.
\qd
Andriej sięgnął po flaszkę, uderzył trzy razy w dno butelki,
potem obrócił ją właściwie i trzy razy „spoliczkował” szyjkę.
To był taki ich żartobliwy zwyczaj ,,wyganiania diabła''.
Następnie odkręcił zakrętkę i nalał po równo do kieliszków,
wykonanych z rozciętych puszek. Mężczyźni podnieśli naczynia, a
po chwili wlewali sobie ognisty płyn do gardeł.

Śniadanie jedli takie jak co dzień. Dwie kanapki z konserwą, popijane wodą, a na koniec jakiś owoc, na pół. Ponoć dawało im to siły na większość dnia.

Po kilkunastu minutach milczenia w czasie spożywania posiłków
stalkerzy zaczęli szykować się do wyruszenia w dalszą drogę.
Spakowali butelkę wódki, pozostałą wodę i chleb. Wszystko
szczelnie owinęli i wrzucili do plecaków, które następnie
solidnie zamknęli. Pomogli je sobie założyć na plecy. W sumie,
z całym osprzętem taszczyli do piętnastu, dwudziestu kilo. A
dodając 	kombinezon, broń i amunicję wychodzi całkiem
spory ładunek.

Wyszli z szatni. Nie skierowali się jednak na salę gimnastyczną. Postanowili, że opuszczą szkołę głównym holem. Ruszyli przed siebie, posuwając się przez korytarz bardzo powoli, ostrożnie stawiając każdy krok. Rozglądali się bardzo uważnie, z kolbą karabinu opartą o ramię zaglądając do każdego zaciemnionego miejsca. Kto wie, czy budynek nie był miejscem spotkań Monolitu, lub skupiskiem anomalii? Śmierć czyhała wszędzie, dosłownie wszędzie.

Dotarli do wyjścia, które niemal niczym nie różniło się od korytarza. Ten sam tandetny parkiet, te same obdrapane ściany, ten sam sufit z kasetonami pomalowanymi na brązowo, te same ławki porozrzucane po całym korytarzu. Inny był tylko przewrócony stolik leżący pod ścianą, budka woźnego i nieco szersza niż zwykle odnoga, prowadząca do wyjścia. Inne było też ciało, w stanie zaawansowanego rozkładu, leżące gdzieś pod ścianą. Obok leżał jakiś połyskujący metal, może broń denata. Mężczyźni nie zwrócili na niego większej uwagi, skręcili w prawo chcąc opuścić budynek. Wyszli przez duże, przeszklone drzwi. Były otwarte. Przykucnęli przy murku, rozglądając się za jakimkolwiek zagrożeniem, jednak mgła otulająca martwe miasto była taka, jak przypuszczał Andriej~--~widoczności na pięć metrów.

Nie byli pewni co do godziny, o której się obudzili. Może to była piąta rano, może szósta, a może nawet i ósma. Przez tą mgłę nie można było niczego wywnioskować. Była tak gęsta, że nie przepuszczała nawet słońca.
\sx
Nie ma wyjścia. Ruszamy.~--~Powiedział Andriej.
\xx Pamiętasz drogę?
\xx Tak. Już teraz tak.
\qd
Przez chaszcze zaczęli przedzierać się w kierunku szpitala. Z tej strony było ich jakby mniej, ale mgła i tak skutecznie utrudniała poruszanie się. Po kilkunastu sekundach dotarli do budynku domu kultury „Energetyk”. Idąc wzdłuż ściany znaleźli się przed gmachem, na betonowym placu, pełnym porzuconych, napromieniowanych samochodów. Stały Łady, Ziły, Kamazy i inne dzieła radzieckiej myśli technicznej.
\sx
Teraz uważaj.~--~Szepnął Andriej towarzyszowi.~--~Na „trzy” biegniemy w tym kierunku.~--~Wskazał ręką południowy wschód.~--~Przeskakujemy plac, musisz uważać na anomalie. Pewnie jest ich tu sporo. Dobiegamy do ulicy i wpadamy w osiedle. O ile mnie pamięć nie myli, tam jest jakiś sklep. Będzie można się w nim skryć, jeśli nikogo nie będzie w środku. Jasne?
\xx Jak słońce.
\xx Słońca to dziś akurat nie uświadczysz. Gotowy? Raz\3k dwa\3k trzy!
\qd
Mężczyźni puścili się pędem po przekątnej pla\-cu. Mijali samochody, kolejne młode drzewka, przebijające się przez asfalt. Bazylowi wydało się w pewnej chwili, jakby błysnął mu przed oczami jakiś artefakt, jednak nie miał czasu, by się przyjrzeć. Przebiegali właśnie przez ulicę, kiedy powietrze przeszył huk wystrzału. Obaj w jednej chwili, jakby na rozkaz, przyśpieszyli. Przeskoczyli przez pas zieleni, potem przez chodnik i wpadli do klatki dziesięciopiętrowego bloku mieszkalnego. Jednego z wielu w Prypeci.
\sx
Cholera, słyszałeś to?~--~zapytał Bazyl zdyszany.
\xx Trudno było nie słyszeć.
\xx Myślisz, że nas odkryli?
\xx Kto?
\xx No\3k Monolit.
\xx A, no tak.~--~Zamyślił się Andriej.~--~Nie, myślę, że nie. Podejrzewam, że strzelali do mutantów. Z której strony było słychać strzał?
\xx Jakoś w prawo od nas.
\xx Czyli z zachodu. Jesteśmy bezpieczni. Odsapnij. Już blisko. Moje mieszkanie jest po drugiej stronie osiedla.
\xx Jesteś pewien?
\xx A udowodnić ci to na mapie?
\xx Okej, okej\3k
\xx Z tym blokiem też mam miłe wspomnienia. Tu mieszkał Dima, mój najlepszy kolega. Był trochę starszy, ale bardzo się lubiliśmy.
\xx To ciekawe.
\xx Ciekawe\3k~--~Powtórzył Andriej, po czym obaj mężczyźni zamilkli.
\qd
\ro{CZĘŚĆ III}
\mm Parę chwil później znów byli gotowi przedzierać się w stronę szpitala. Zanim jednak wyszli z bloku upewnili się, czy jest to bezpieczne. Tym razem Bazyl pełnił rolę zwiadowcy. Wychylił się przez metalowe, zardzewiałe drzwi i słuchał. Słuchał milczenia zamkniętego od dwudziestu pięciu lat w tych drewnianych oknach, przerdzewiałych samochodach, popękanych ulicach. Cisza była wręcz ogłuszająca. Nigdy dotąd nie spotkał się z podobnym uczuciem. W mieście duchów panował spokój, od którego aż dzwoniło w uszach. Żadnych dźwięków, ani odgłosów. Nie szeleścił żaden liść, nie trzeszczała żadna gałąź poruszana wiatrem, nie było słychać ptaków czy owadów. Nie było słychać nic. Po prostu ogłuszająca, martwa cisza.
\sx
Bezpiecznie, chodźmy. Prowadź.~--~Szepnął po chwili Andriejowi.
\qd
Wyszli z klatki na chodnik, przez który przebiegali kilka sekund wcześniej i skręcili w prawo. Poruszali się wzdłuż ściany dziesięciopiętrowca przez jakąś minutę, ostrożnie stawiając każdy krok. Miasto wydawało się martwe, tak jak większość miejsc w Zonie, ale naprawdę życie w nim wręcz tętniło. Dziesiątki, jak nie setki, Monolitian, religijnych popaprańców wyznających jakąś kosmiczną skałę ukrytą w bloku czwartego reaktora, gotowych zabić każdego, kto nie mówi ich językiem, nie ma tych samych wartości ani, krótko mówiąc, wypranego mózgu; niezliczona ilość mutantów, dzieci Zony; wojsko, pojawiające się tu raz na jakiś czas, i wiele innych rzeczy, o których istnieniu mogli nawet nie wiedzieć.

Mężczyźni doszli do końca budynku. Ponownie skręcili w prawo, tym razem w alejkę przecinającą osiedle. Do celu pozostało im jakieś dwieście metrów. Ten etap wydawał się najłatwiejszy. Dzień, w świetle którego można było zawczasu dostrzec niebezpieczeństwa, gęsta mgła otulająca okolicę, zapewniająca w pewnym sensie niewidzialność, cisza, w której da się usłyszeć każdy szmer. Prypeć tego dnia była jakby utopią, sprawiała wrażenie niezwykle pustej i łaskawej dla podróżników. Mimo to, mężczyźni ani na chwilę nie tracili koncentracji.
\sx
Zobacz. Moja podstawówka\3k~--~Andriej wskazał palcem budynek stojący z lewej strony.~--~Za nią jest alejka, która prowadzi prosto do bloku.
\qd\mm
Osiedle nie było tak zarośnięte drzewami jak okolice parku, czy domu kultury, choć dzieliło je raptem pół kilometra. Owszem, wokół pełno było młodych brzóz obrastających budynki, jednak aleje przecinające ten teren wciąż były alejami wyłożonymi twardą, betonową płytą, a nie niebezpieczną dżunglą.

Po jakichś trzech minutach niemal dotarli do celu. Z blokowiska wyszli wprost na chodnik przy ulicy Przyjaźni Narodów. Z prawej strony mieli blok numer 13.
\sx
Pierwsza klatka. Wchodzimy.~--~Drżącym już głosem szepnął Andriej do kolegi.
\qm
Drzwi były nieco ponad poziomem chodnika, musieli pokonać kilka stopni, by tam dotrzeć.

W końcu znaleźli się w środku. Włóczęga podszedł do wiszącej na ścianie tablicy ze spisem mieszkańców, zdjął rękawiczkę i przejechał po niej ręką. Tak, jak robią to dzieci zaglądające do sklepu z cukierkami przez szybę. Zatrzymał się przy numerze 73.
\sx Gorbunow Irina\3k~--~Szepnął.~--~Babcia\3k \qd
 Po chwili milczenia dodał:
\sx Chodźmy na górę.
\qd
Odszedł od tablicy, włączył swoją latarkę czołową i ruszył powoli schodami. Miał do pokonania cztery piętra. Cała drogę bił się z myślami, chciał się nawet wycofać, uciekać stamtąd jak najprędzej, jednak nie zrobił żadnej z tych rzeczy. Po prostu szedł. Mijał mieszkania kolejnych sąsiadów: państwa Inumienko, starszego małżeństwa, które zawsze wszczynało awantury o to, że telewizor za głośno, że dzieci za głośno, że samochód źle postawiony i o wiele innych mniejszych lub większych, głupot. Minął dom pani Prokopienko, której mąż, pan Iwan, był strażakiem, jednym z pierwszych, którzy dotarli do elektrowni po wybuchu.

Wreszcie dotarł na czwarte piętro, do mieszkania numer 73. Jego mieszkania. Wahał się jakiś czas, czy wejść do środka, czy stać na klatce i wpatrywać się w drzwi. Po chwili jednak pchnął je. Okazało się, że były zdjęte z zawiasów. Pchnięte po prostu przewróciły się na szafę stojącą prawie naprzeciw nich. Andriej przekroczył próg. Zrzucił na ziemię plecak, odłożył broń, odpiął od paska nóż, który niedbale rzucił gdzieś w kąt i smętnie powlókł stopami do salonu.

Dobrze pamiętał ten czarny regał z bordowymi zdobieniami, który kiedyś omal się na niego nie przewrócił. Wciąż stał na nim telewizor. Ten sam, przez który tak często płakał, kiedy nie mógł oglądać bajek. Pamiętał rozkładany stół, na którym do tej pory, mimo upływu dwudziestu pięciu lat, stał szklany flakon na kwiaty i popielniczka. W kącie, przy wyjściu na balkon, leżała przewrócona donica, w której kiedyś stał jakiś duży kwiat. Na podłodze walało się mnóstwo książek, ubrań i zabawek.%\looseness=-1

Andriej zawrócił i smętnie, przesuwając rękę po ścianie, skierował się do sypialni. Bazyl, stojący cały czas w progu mieszkania, zauważył w jego oczach łzy. Opuścił wzrok.
\sx
Wejdź, przyjacielu\3k~--~Szepnął Andriej.
\qd
Koczownik posłusznie przestąpił próg. Stanął przy wybitym oknie wychodzącym na ulicę Przyjaźni Narodów i wpatrywał się w gmach szpitala, znajdującego się po jej przeciwnej stronie. Mgła opadała, coraz więcej słońca przebijało się przez chmury, których ilość też jakby się zmniejszała.

Andriej wrócił do salonu i siadł pod ścianą wśród sterty zakurzonych papierów i ubrań. Skulił się, ukrył twarz w dłoniach i po prostu się rozpłakał. Bazyl wciąż milczał. Byli przyjaciółmi, jednak ten podjął decyzję, by nie wspierać kolegi w obecnej sytuacji. Andriej Włóczęga musiał poradzić sobie z nią samodzielnie.%\looseness-1

Mijały kolejne, ciągnące się w nieskończoność, minuty. Prypeć budziła się do życia. Mgła zniknęła, chmury się rozwiały, świeciło piękne, wiosenne słońce, miło ogrzewające twarz. W dole dało się słyszeć szczekanie psów, czasem nawet można było zauważyć jakieś stado biegnące w stronę zagajnika znajdującego się kilkaset metrów dalej.Tylko mieszkanie numer 73 zamarło. Andriej wprawdzie już się opanował, jednak wciąż siedział w milczeniu, tępo wpatrując się w ścianę naprzeciw. Bazyl zaś stał nieruchomo przy oknie. Trwało to dla niego wieczność, ale nie widział innej możliwości zachowania się w tej chwili. Do tej pory żaden jeszcze nic nie powiedział.

Było to bardzo nietypowe. Bazylowi trudno wytrzymać kilka minut bez odzywania się. Jednak od wkroczenia do Prypeci jakby coś się zmieniło Spochmurniał nieco, przybladł, oczy straciły część blasku i zapału, jakim emanowały zawsze w czasie studiów. Zachowywał się zupełnie inaczej, niż zwykle. Może robił to dla przyjaciela, a może powód jego zmiany był kompletnie inny? Chyba nawet on sam tego nie wiedział\3k

Andriej powoli otrząsał się z szoku. Zaczął wodzić wzrokiem po ścianach, szukać znajomych okładek na podłodze. Nadal jednak żaden z nich nie odważył się odezwać, choć czasu minęło bardzo dużo.
\sx{Monolit}.~--~Szepnął Bazyl po kolejnych kilku minutach dzwoniącej w uszach ciszy.
\xx Co?! Gdzie?!~--~Andriej wprawdzie wciąż siedział, ale już wyprostowany i gotowy do działania.
\xx {Monolit}.~--~Koczownik obrócił się w stronę towarzysza i sięgnął do kieszeni.
\xx Gdzie Monolit, co się dzieje?~--~W jego głosie było słychać wyraźne podenerwowanie.
\xx {Monolit}.~--~Bazyl wyjął z kieszeni pistolet i przeładował go.
\xx Stary, co z tobą?! Co się, do kurwy nędzy, dzieje?!
\xx {Monolit}.~--~Szepnął Bazyl i wycelował broń w kierunku Andrieja.
\qd
\sw[2em] STRZAŁ \qw
\end{document}
