\documentclass[../MAIN.tex]{subfiles}
\begin{document}
\sw[14em]
W koronach wiecznych dębów\\
Wiatr cicho łka\\
Pośród groźnych obłoków\\
Ponure oblicze Peruna
\qw
Kirył wyszedł ze Skadowska tuż przed świtem. Dzień zapowiadał się z tych dżdżystych – niebo już poprzedniego wieczora zasnuło się ciemnymi, gęstymi chmurami, z których nieustannie siąpił deszcz. Trawy kołysały się na wietrze, a krople wody z nieustającym szumem pokrywały każdy skrawek powierzchni. Czasem zza kłębów na nieboskłonie przesączał się słaby blask księżyca. Cel był jeden: Żelazny Las. Konkretnie: wartościowa rzecz, którą miał ze sobą pewien poległy tam stalker. Wartościowa dla Sowy, jednego z miejscowych handlarzy, rzecz jasna.

O Żelaznym Lesie krążyło wiele legend. Niektórzy mówili o tym, że w rzekomych podziemiach kompleksu mieszka sam diabeł. Inni opowiadali o hordach zombie żałośnie krążących wokół. Ktoś zawodził na temat lewitujących w powietrzu beczek i wszelkiego innego zgromadzonego śmiecia. Z daleka miejsce wydawało się jak każde inne w Zonie. Ponury, opuszczony kompleks, stacja transformatorowa – prawdopodobnie stąd wzięła się nazwa. Skupione na niewielkiej przestrzeni metalowe, wysokie na trzydzieści metrów słupy wysokiego napięcia faktycznie wyglądały jak niewielki zagajnik. Przeciągnięte między nimi lub zerwane i zrzucone na ziemię linie przesyłowe kojarzyły się z żałosnymi gałęziami martwych drzew. Raczej mało kto miał odwagę, by się tam zapuszczać. O miejscu głośno zrobiło się dopiero, gdy między słupami zawisł jeden z wojskowych śmigłowców. Legendy o Żelaznym Lesie namnożyły się za sprawą ciekawskich stalkerów żądnych łupów ze zniszczonej maszyny. Wielu z nich zaginęło, próbując odkryć tajemnice katastrofy, tudzież 
samego miejsca.
\sw[15em]
Gwiazdy padają z nieba\\
Miesiąc słabo lśni\\
Rusicze w pogardzie krzyża\\
Jest dla nich miejsce na świecie?
\qw
Ale robota przecież sama się nie zrobi. Pieniądze nie spadną magicznie spośród chmur. Droga na miejsce, choć dosyć długa, okazała się zadziwiająco lekka. Kirył wybrał porę, w której stalkerzy jeszcze śpią, a żerujące nocą mutanty wracają nasycone do kryjówek. Na miejscu był po niespełna godzinie.

Już z pewnej odległości dostrzegł padający na ściany budynków administracji blask bijący od kilku elektrycznych anomalii zgromadzonych pod urządzeniami przesyłowymi. Jasnoniebieska łuna napawała pewnego rodzaju strachem, ale w obecnej sytuacji była wręcz sojusznikiem – gdyby na terenie kompleksu czaiło się coś niedobrego, Kiryłowi pewnie udałoby się dostrzec rzucane przez zło cienie. Zatem nie przejął się nimi zanadto – wejść na plac stacji było kilka. Może anomalie sprawiłyby mały kłopot, gdyby trzeba było się nagle ewakuować, ale plan zakładał spokojne podejście, znalezienie towaru i powrót na Skadowsk. Stalkerzy mieli niepisaną umowę – każde nietypowe zjawisko miało być zgłaszane od razu, po powrocie do bazy. Kilku z nich było w okolicy Żelaznego Lasu poprzedniego dnia i niczego odbiegającego od normy nie zaobserwowali. Ot – opuszczone miejsce, jakich wiele w Zonie. Przygnębiające, ale nie zawierające w sobie wielkiego zagrożenia. Pozornie.
\sw
Przez ręce chciwego żyda\\
W Ruś plugastwo wniesione\\
Kłamliwą mową Chrysta\\
Umysły otumanione\\
\qw
Na teren kompleksu dotarł nieco okrężną drogą – między lasem kryjącym tajemniczy Sosnodąb a starą stacją benzynową. Z jednej strony miał watahę psów, z drugiej niewielkie stadko mięsaków. Obie „bandy” na szczęście pogrążone były we śnie. Stalker zszedł z mokrego, spękanego asfaltu do niewielkiej dolinki porośniętej drzewami i obrzuconej – niby ogromny plac zabaw – wielkimi głazami, pamiętającymi pewnie ostatnie zlodowacenie. W półmroku omal nie stratował czyjejś mogiły. Dosłownie w ostatniej chwili ominął drewniany krzyż. Stary. Omszały i popękany. Pewnie jeszcze sprzed katastrofy. Na świeżych grobach zwykle wieszało się maski przeciwgazowe nieszczęśników i jakieś ich atrybuty. Tu ich brakowało. Zresztą, po dwa tysiące szóstym raczej nikt nie miał siły, czasu i materiałów, by zbijać solidny krucyfiks, najpewniej z podkładów kolejowych. Posiadał nie tylko trzy belki poprzeczne mające – według prawosławnej tradycji – symbolizować kolejno filakterium zawierające królewski tytuł Chrystusa, belkę, do której 
przybito jego ręce i umieszczoną pod kątem podpórkę dla nóg, ale i zakończenia ramion ornamentami w kształcie liści koniczyny – ukazanie jedności Chrystusa i Trójcy Świętej. Zdecydowanie ten krzyż pochodził jeszcze sprzed Wielkiej Emisji. Kirył zatrzymał się przy nim, obejrzał dokładnie z każdej strony, pociągnął lasem, splunął pod symbol i obróciwszy się na pięcie skierował w kierunku Żelaznego Lasu. Po chwili znalazł się na wzniesieniu, na którym wiła się jedna z wielu dojazdówek, jakimi przeorany był cały Zaton.

Przeskakując nad kałużami wkroczył na niewielką zatoczkę – parking dla pracowników ośrodka. Między przeżartym przez rdzę zdewastowanym ziłem a prowizoryczną barierą z kawałka blachy opartej o drewniany stelaż dostrzegł niewielki krąg z kamieni. Pozostałości po ognisku. Na oko sądząc – co najmniej sprzed tygodnia. Możliwe, że było to dzieło najemników, którzy mieli nieopodal swoją placówkę.
\sw[16em]
Oblicza bogów prawdziwych\\
W rzekach utopili\\
Dumnych czarowników i magów\\
Zdrajcy zniszczyli\\
\qw
Doskoczył bezszelestnie do betonowego muru okalającego teren stacji. W tym miejscu brakowało jednego segmentu, więc mógł bez przeszkód wejść na plac ze słupami. Wyjrzał zza ogrodzenia i zlustrował okolicę. Poza deszczem padającym z szumem na konstrukcje otaczała go martwa cisza.

Wkroczył do Lasu. Stąpał ostrożnie, przytłoczony gąszczem metalowych konstrukcji. Po drugiej stronie placu, zawieszony na pogiętych wspornikach jakieś siedem metrów nad ziemią, majaczył wojskowy Mi-24. Księżyc po raz kolejny wyłonił się zza chmur zalewając okolicę mdłym światłem. Pod maszyną Kirył dostrzegł rozrzucone bezładnie tłumoki. Najpewniej ciała nieszczęśników, którzy posadzili tam śmigłowiec. Jeśli od czegoś miał zacząć poszukiwania tego wartościowego przedmiotu, to tylko od resztek pilotów. Ostrożnie zbliżył się do nich, omijając wszystkie wydające odgłosy śmieci. Na betonowej wylewce walało się mnóstwo skrzyń, puszek, kawałków mebli, metalowych tek i innych typowo przemysłowych odpadków. Czasem między nogami błyskały błękitne iskierki – na wilgotnej powierzchni elektry miały większy niż zazwyczaj zasięg, ale mimo wszystko w miejscu, w którym się znajdował raczej nie groziło mu porażenie prądem o natężeniu rzędu kilkuset tysięcy amperów.
\sw
Gdzież ołtarz Perunowy\\
By uczcić wiernych śmierć?\\
Rozkosz spijania krwi\\
W horominach zaśniezonych
\qw
Zbliżył się do leżących na betonie ciał. Dzieci Zony zdążyły się już nimi zająć – z kombinezonów zostały strzępy, a flaki trupów walały się wszędzie dookoła. Potwierdza się, że w przyrodzie nic nie ginie, a nawet jeśli zginie to po to, by ktoś inny mógł żyć. Zniesmaczony tym widokiem Kirył odszedł w kierunku płotu odgradzającego teren kompleksu. Wydawało mu się, jakby u wejścia do budki – z której być może prowadzono jakieś eksperymenty – leżało kolejne ciało.

Przeszedł pod wiszącym śmigłowcem, który dokładnie w tej chwili zaskamlał cicho, poruszony zapewne jesiennym wiatrem. Kilkunastotonowa maszyna utrzymująca się zaledwie na cienkich śmigłach wirnika nie mogła wisieć stabilnie. Pewnie mocniejsza wichura i Mi-24 spadłby z hukiem na ziemię. Kilka chwil później Kirył wzdrygnął się, po plecach przebiegł mu zimny dreszcz. Poczuł, jak pod grubymi rękawami kombinezonu tworzy się gęsia skórka. Sięgnął lewą ręką po latarkę zaczepioną przy pasie. Rozejrzał się, czy nadal jest na terenie kompleksu kompletnie sam. Pusto.
\sw[12em]
Im wilk i czarny kruk\\
Śpiewają kołysanki\\
Żałobne światło księżyca\\
Pieści blade twarze
\qw
Skierował urządzenie w kierunku budki. Czarne okna zionęły przeraźliwą pustką, tuż pod framugą, w której kiedyś zapewne były drzwi spoczywało ciało. Już chciał puścić w tamto miejsce snop światła, choćby na chwilę. W momencie, kiedy jego kciuk spoczął na przełączniku, w oddali przeraźliwie zaskrzeczał kruk. Raz, a potem drugi. Kirył zerknął w górę, na jeden ze słupów wysokiego napięcia. Na tle jasnoszarych, mglistych chmur dostrzegł ciemny kształt ptaszyska, które wpatrywało się w niego mądrym, pełnym zrozumienia i żalu wzrokiem. Po chwili kontaktu wzrokowego zwierzę uniosło głowę i jeszcze raz kracząc, odleciało.

Przeczuwając zagrożenie Kirył machinalnie zaczepił latarkę o pas i pewniej chwycił – nomen omen – postarmiejnego AKS-74U. Kierując lufę w kierunku jam po oknach zrobił krok w przód. Tuż za jego plecami coś zahuczało straszliwie. Wyrobiony instynkt podpowiedział ucieczkę. Ruszył pędem w kierunku zionącej pustką framugi i po kilku susach wpadł do pomieszczenia. Chwilę potem o zewnętrzną ścianę trzasnął wielki arkusz grubej blachy. ku*wa – powiedział sam do siebie. Teraz i tak wszyscy w promieniu kilkuset metrów wiedzieli, że w Żelaznym Lesie coś się dzieje.
\sw[15em]
W powietrzu czuć smutku gorycz\\
Unoszą się nad polem bitwy\\
Na skrzydłach czarnej rozpaczy\\
Siostry Żela i Karna
\qw
W przypływie złości wyszarpał latarkę. Zaczep ułamał się i spadł z cichym łoskotem na podłogę. Dłoń zapiekła – rozciął skórę o złamany plastik. Z wściekłością pchnął przełącznik, a z końca urządzenia wystrzelił ostry snop. Budynek sprawiał upiorne, klaustrofobiczne wrażenie. Z małego przedsionka prowadziły dwie odnogi – jedna do pomieszczenia, w którym stało biurko i regał na dokumenty, druga do schodów prowadzących w dół. Wszedł do pierwszego. Omiótł je światłem latarki, ale jedyne co tam znalazł, to kilka pustych puszek po konserwach. Skierował się do klatki schodowej.

Stary, pofałdowany parkiet zaskrzypiał z boleścią. Zaraz potem z podziemi coś zawyło. Czyżby legendy o diable czającym się pod kompleksem nie były tylko wyssaną z palca gadaniną? Zawrócił. Cokolwiek czaiło się na dole, raczej nie było nastawione na towarzyskie spotkanie. Kirył chciał wyjrzeć przez pustą framugę okna, ale gdy tylko do niego podszedł, poczuł na plecach ostre mrowienie. Sekundę potem do pomieszczenia wleciało kilka drewnianych ochłapów.
\sw[14em]
Słychać płacz Diabli-Łabędzi\\
Nad całą Rosyjską Ziemią\\
Rwą się kartki psałterzy\\
Flety z żałością brzmią\\
\qw
Kirył zawrócił do przedsionka. Wielka, metalowa płyta skutecznie zablokowała wyjście. Na szczęście uderzenie wepchnęło trupa spoczywającego pod framugą do pomieszczenia. Skierował na niego światło latarki. Ciało wyglądało na świeże. Kirył odstawił karabin i przykucnął. Zaczął szperać po kieszeniach, próbował wymacać cokolwiek ukrytego, co nieszczęśnik mógł mieć przy sobie.

W końcu wyczuł niewielkie pudełko, w wewnętrznej kieszeni skórzanej kurtki. Szarpnął za materiał i dosięgnął palcami zimny, metalowy przedmiot. Wyciągnął go. Był to niewielki pojemnik, zamykany małym kluczykiem. Kirył zaczął szperać po wszystkich kieszeniach, niestety nie znalazł niczego, czym mógłby odblokować zamek. Mimo to zabrał znalezisko ze sobą. Umieścił je w zamykanej na suwak kieszeni na udzie.
\sw[17em]
Ale przyjdzie w końcu ten czas\\
Gdy krzyże zgoreją\\
Gdy w cerkwiach swoim kapłanom\\
Nikt miłosierdzia nie da
\qw
Teraz pozostało już tylko wyrwanie się z tej nory, powrót na Skadowsk i rozliczenie się z Sową. Wstał i obejrzał dokładnie arkusz blachy. Ważył na oko z sześćdziesiąt kilogramów, do ściany przylegał prawie równolegle. Wystarczyło trochę siły i na pewno przeszkodę dałoby się usunąć z drogi. Wyjście z budynku było z innej strony, niż okna. Tamtędy wolał się nie wydostawać – przed oczami znów mignęły mu drewniane szczapy. Zamknął je na moment i potrząsnął głową.

Wyłączył latarkę i schował ją do głębokiej kieszeni z lewej strony kombinezonu. Stanął przy zaporze i zaparł się barkiem. Przyłożył do zimnej powierzchni dłonie i pchnął całym ciałem, ze wszystkich sił. Arkusz zajęczał i delikatnie drgnął, a ze ściany posypało się trochę tynku. Po kilku chwilach przepychanki w końcu ustąpił i z przeraźliwym łoskotem runął na ziemię. Kirył odetchnął głęboko, złapał karabin i wyszedł na zewnątrz. Już świtało. Zerknął w górę – niebo jaśniało i przechodziło z ciemnogranatowego w szarość. Dostrzegł, że budynek jest otoczony betonowym murem i tylko w jednym miejscu można z niego wyjść na główny plac kompleksu. Działy się tu zdecydowanie dziwne rzeczy. Chciał już wrócić na Skadowsk, do ciepła, gwaru i spokoju. Z podziemi znowu dało się słyszeć przeraźliwe wycie. I znów przed oczami pojawiły się latające resztki desek.
\sw
A wówczas zapieją jeszcze raz\\
Pochwałę bogom Rusi\\
Oplują tym lżywe ślepia\\
Rozpiętego na drzewie żyda\\
\qw
Zdecydowanym krokiem ruszył do wyjścia. Wtulił się w krawędź ogrodzenia i wyjrzał ostrożnie na plac. Teraz Żelazny Las wyglądał jeszcze bardziej upiornie, ale w tej upiorności był jakiś tajemniczy majestat. Wiszący Mi-24 kołysał się z pojękiwaniem, pod nim spoczywały resztki pilotów, wszędzie wokół mnóstwo śmieci. Gdzieś w oddali pobrzękiwały elektry, jakby czerpały z czegoś energię. Tylko z czego?

Rozejrzał się jeszcze raz i opracował drogę ucieczki. Wycie z podziemi zdawało się narastać i przemawiać w jakimś nieznanym, bełkotliwym języku. Kirył odetchnął głęboko, chwycił pewnie karabin, przymknął oczy, wypuścił całe powietrze i puścił się pędem pomiędzy żelaznymi słupami energetycznymi.

Zza jednego z wielkich transformatorów wysunęła się błękitna kula przypominająca piorun kulisty. Przyspieszył. Do zewnętrznego ogrodzenia, zatoczki i bezpiecznej dolinki dzieliło go raptem czterdzieści, może pięćdziesiąt metrów. Martwa cisza zmieniła się w przeraźliwe buczenie, odgłosy z głębi budynku zwielokrotniły się i nawarstwiły. Śmieci rozrzucone na placu zadrżały i powoli uniosły się ku niebu. Zaczął strzelać na oślep do zbliżającego się pioruna. Potknął się o skrzynię i omal nie upadł. Zwolnił na chwilę, ale za to złapał równowagę. Był już tuż-tuż wyłomu w murze.

Dostrzegł zbliżający się przedmiot. Nie zdążył go uniknąć. Ogromny, stary, drewniany krzyż spadł na niego jak grom i przygwoździł do ziemi. Wielki, wystający z drewna gwóźdź wbił się prosto w środek czoła, zagłębiając się głęboko w czaszce. Stalker upadł i bezwładnie sunął po betonie jeszcze dobry metr. Karabin z łoskotem spadł trochę dalej.

Wycie z podziemi ucichło. Brzęczenie ucichło. Uniesione w górę przedmioty z hukiem opadły na wilgotny beton. Gdy wszystko ucichło, dało się słyszeć kruka, skrzeczącego gdzieś cicho i żałośnie.

Ponure oblicze Peruna znów skryło się pośród żelaznych drzew.
\end{document}