\documentclass[../MAIN.tex]{subfiles}
\begin{document}
  \ro{Wieść}
%
Wskazówki kieszonkowego chronometru ustawiły się
prostopadle demonstrując tym samym kwadrans po dwunastej.
Trwającą dotychczas ciszę zakłócił zgiełk i hałas zasuwanych
krzeseł. Spiker przez megafon donośnie ogłosił przerwę w pracy.
Giennadij schował podarowany niegdyś przez ojca zegarek do
jednej z kieszeni uniformu i wstając zakomunikował woźnemu
koniec pełnionej funkcji. Zanim ruszył do wyjścia złączył plik
kartek i starannie umieścił go w szufladzie, po czym delikatnym
ruchem ręki umieścił kluczyk w zamku, mechanizm matowo
szczęknął tym samym zabezpieczając wartościowe dokumenty przed
wścibskim okiem osób trzecich. Giennadij zasunął fotel i
skierował się do wyjścia.
%
\sx Jak panu czas mija?~--~odezwał się woźny bawiąc się
kluczami.
\xx Wyjątkowo szybko, dzisiaj akurat mało roboty i można trochę
poleniuchować za biurkiem~--~odpowiedział Giennadij wyciągając
z torby śniadanie.
\xx Widzi Pan, ja nawet nie mogę kopyta wyłożyć na taboret, bo
zaraz przyleci kierownik i opieprzy. Ostatnio to mi krzyżówki
zabrał dziadyga jeden, was to tam nie pilnują, bo robota inna,
ale mnie od razu za uszy ciągną.
\xx A ja panu powiem, że papierkowa robota też za ciekawa nie
jest w dodatku muszę zajmować się obserwacją i pomiarem próbek.
Czasami zamienił bym się z kimś pracą, tak dla odmiany~--~
uśmiechnął się Giennadij i pozdrawiając kolegę udał się do
jadalni.
\qd
W ciągu paru minut salę wypełnił cały personel, panujący gwar
nosił się po ścianach tworząc przenikliwe echo. Giennadij idąc
korytarzem co rusz witał pozostałą personę, która z przejęciem
pałaszowała wymyślne dania sporządzone przez siebie lub
rodzinę. Pracownik postanowił przysiąść się do kolegów z kadry.
%
\sx Cześć Gieno, co tam Ci dziś żonka sprezentowała, moja się
pofatygowała i jabłecznik upiekła~--~rzucił starszy przyjaciel,
na, którego wszyscy wołali Wuser.
\xx Kanapki z szynką i ogórkiem, nic szczególnego~--~pochwalił
się Giennadij rozwijając papier z pieczywa.
\xx No widzisz, trzeba się postarać w nocy, by rano dostać taki
specjał~--~zaśmiał się Wuser odsłaniając zęby pełne ciasta.
\qd
Wszyscy obok zrobili to samo, o mało się nie krztusząc.
Brechtające gęby nagle umilkły, gdy obiekt żartów wyciągnął zza
pazuchy buteleczkę. Wszyscy zrobili konspiracyjne miny i po
chwili ponownie zarechotali strojąc przy tym komiczne miny.
Wszyscy z aprobatą przyjęli pomysł strzelenia przysłowiowego
kielicha i potajemnie wyciągnęli w stronę Giennadija kubeczki z
termosów. Kieliszków nikt nie nosił z obawy przed kontrolą,
samo noszenie napojów alkoholowych do pracy mogło przynieść za
sobą poważne konsekwencję. Jednak koneksje z pracownikami
wyższego szczebla dawały gwarancję na przymrużanie oka w
przypadku sytuacji łamiących regulamin, a takowe układy
posiadał Giennadij.
%
\sx Może kawałek ciasta? Śmiało częstować się~--~oznajmił
Wuser, wpychając do ust kolejną porcję jabłecznika.
Zgromadzeni przy stole przyjęli podarek w ciszy delektując się
słodkim deserem.
\xx Słyszeliście o pewnej rekrutacji, którą mają tutaj wdrożyć.
Ponoć tajne ugrupowanie zbiera ludzi do bliżej nieokreślonych
prac związanych z badaniem promieniowania w Czarnobylu. Wiadomo
mi, że inicjatorem tego przedsięwzięcia ma być niejaki Piotr
Kaługin. Słyszeliście o takim?~--~zapytał Wuser robiąc przy tym
kwaśną minę.
\xx Nie znam gościa, pewnie jakiś świr. Ciekaw jestem co oni
chcą badać w tym wyludnionym mieście zawalonym kupą stali i
innymi rupieciami~--~wtrącił inny kolega z personelu.
\xx Mi to wygląda na jakiś zakulisowy program. Przecież u nas w
Prypeci prowadzimy diagnozy nad oddziaływaniem promieniowania
na tutejszą florę i faunę, więc dlaczego mieliby to samo robić
w Czarnobylu. Może środowisko inne i badania ciekawsze. Jeśli
mnie wybiorą to z chęcią zmienię otoczenie, może lepiej płacić
będą~--~dodał i uśmiechnął się półgębkiem młody Patejuk
\xx A ty Giennadij co o tym sądzisz?~--~rzucił ponownie Wuser
ignorując ostatnią wypowiedź.
\xx Ciekaw jestem co z tego wyniknie. Zastanawiam się, gdzie
oni prowadzaliby te doświadczenia, bo chyba nie na wolnym
powietrzu. Może w planach mają budowę kompleksu badawczego,
tylko że postawienie takiego ośrodka to nie bagatela. Prace
budowlane zajęłyby co najmniej dwa lata.
\xx Według mnie wszystko jest już przygotowane, skoro
ewentualna rekrutacja ma się niebawem odbyć. W ogóle skąd wiesz
o tej całej akcji?~--~spytał Patejuk.
\qd
Wuser otworzył lekko usta, lecz nic z nich nie wydobył. Wziął
do ręki kubek z kawą i zakręcił, nim mieszając resztki czarnego
naparu. Zwięzłe milczenie przerwała fonia dzwoniącej aparatury
głosząc tym samym koniec przerwy.
\sx Kierownik wspominał, ale jak, by co gębą na kłódkę. Nie
warto rozpowiadać, bo się rozejdzie po całym zakładzie i się
zamęt zrobi. To do jutra!~--~zakomunikował Wuser i odszedł od
stołu zabierając torbę.
\qd
Patejuk z resztą kadrowiczów postąpił tak samo.

Przy stole został tylko Giennadij, w tle cichły ostatnie
rozmowy i odgłosy kroków. Salę ogarnęła osobliwa atmosfera.
Strzeliste białe ściany piętrzyły się przed wzrokiem
osamotnionego pracownika, a silny blask jarzeniówek potęgował
dziwną aurę. Pomieszczenie wyglądało teraz jak wielka
poczekalnia, a osoba siedząca w centrum była jak ostatnia dusza
na ziemi będąca na łasce kapryśnego stwórcy wahającego się
przed wyborem~--~ukarać czy ocalić przed wiecznym potępieniem.
Zamyślony Giennadij wyrwał się z letargu poruszony jakąś
niewidzialną siłą. Zdając sobie sprawę, że przerwa dawno minęła
pośpiesznie zerwał się na nogi i w żwawo ruszył do wyjścia. W
jego umyśle zagnieździła się nowina na temat rekrutacji i stała
się bodźcem do dalszych przemyśleń o ewentualnej zmianie pracy.
Zegar wskazywał godzinę dwunastą trzydzieści.
%
  \ro{Gość}
%
Za oknem panowała okropna szaruga, intensywny wiatr szarpał
nagie gałęzie i w wirującym rytmie miotał zżółkłymi liśćmi.
Niebo okryło się ciemną kłębiastą szatą nie pozwalając rzucić
jakiegokolwiek promyku na jesienną ziemię. Marna pogoda i
ponura sceneria z betonu odbierała wszystkim witalności i
wprawiała w przygnębienie. Gęsty nastrój za ścianami ośrodka
badawczego nie przeszkadzał jednak nikomu w codziennej pracy.
Giennadij w swoim gabinecie analizował rośliny poddane
radiacji, próbki różnej wielkości miały swoją unikalną
sygnaturę i oddzielną charakterystykę w notatkach. Egzemplarze
dostarczano z Czerwonego Lasu, gdzie flora przeszła największą
mutację.

Doświadczony naukowiec wyciągnął zza ucha długopis i zaczął
kreślić wymyślne symbole w ramkach podręcznego notesu, żmudną
pracę zakłóciło pukanie do drzwi.
%
\sx Witaj, zajmę Ci tylko chwilę. Mam nadzieje, że nie
przeszkadzam. Dzisiaj około piętnastej przyjdą ci od
rekrutacji. Powiedziałem kierownictwu, żeby nowo przybyli
goście rozpatrzyli twoją kandydaturę. Oczywiście żadnych
oficjalnych wniosków o posady nie było, ale jak, by, co to się
piszesz, no to trzymaj się~--~Wuser kiwnął głową i zamknął
drzwi za sobą.
\qd
Giennadij nie zdążył nawet podziękować. Ucieszył się z faktu,
że może zostać przeniesiony i poszerzyć swoje horyzonty, ale
jednocześnie dziwił się czemu Wuser tak bardzo troszczy się o
jego zawód. Nie marnując czasu wrócił do poprzedniego zajęcia.
W jednym z protokołów napisał:

\textsl{"Próbka numer [3] Populus Nigra wyraźnie wykazuje
wzmożone
zdolności regeneracyjne. Ociosany pień po, zaledwie dwóch
dniach wraca do pierwotnej postaci;słabo widoczne ślady
okaleczenia rośliny. Komórki kambium zmieniły swoją wewnętrzną
strukturę w wyniku przyjęcia wysokiej dawki promieniowania.
Przyspieszona ekwipotencjalność oznacza brak zagrożenia przed
czynnikami patogennymi. Liście Populus Nigra zawierają
stosunkowo duże ilości niezidentyfikowanej do końca substancji,
przypuszczalnie mogą to być neurotoksyny; planowane
doświadczenie na szczurach. Zaobserwowano niską wodochłonność.
Dalsze eksperymenty wykażą czy badana Populus Nigra może zostać
uznana jako nowy gatunek."}

Przelany na arkusze papieru wynik doświadczeń został zapisany
również w formie nagrania. Giennadij wszelki rezultat swojej
pracy przechowywał w postaci cyfrowej, lubił mieć wszystko w
jednym miejscu na małym mobilnym dysku. Młody naukowiec usiadł
na obitym skórą fotelu robiąc sobie krótką przerwę od zajęć.
Ziewnął przeciągle drapiąc się za uchem. Wyciągnął z kieszeni
swój poczciwy zegarek i określił czas dzielący go od rzekomej
wizyty ludzi od rekrutacji. Pozostały dwa kwadranse, bezczynne
oczekiwanie sprawi, że czas na złość mijać będzie wolniej.
Inercji w pracy Giennadij unikał jak ognia, więc postanowił
zająć się czymkolwiek, padło na czyszczenie sprzętu
laboratoryjnego.

Po, zaledwie kilku minutach wszystkie probówki o różnorakiej
wielkości i wymyślnych kształtach błyszczały pod ostrym
światłem lamp żarowych. Giennadij mył właśnie ręce, gdy do
drzwi zapukała tajemnicza osoba.
%
\sx Witam. Nazywam się Stiepan Sewienkov. Pan nie musi się
przedstawiać. Zapewne, domyślacie z jakiej racji się tu
zjawiłem. Pozwolicie, że opuścimy gabinet i udamy się do
specjalnego pokoju, gdzie omówimy ważne sprawy~--~oznajmił i
gestem grzecznościowym wskazał drzwi wyjściowe~--~Dokonania
zrobiły na nas duże wrażenie i nie ukrywam, że pański biogram
jest gwarantem nowej posady.\\
Giennadij nieco speszony milczał, ale uważnie słuchał
hipnotyzującego głosu, na potwierdzenie słów kiwał tylko głową.
\xx Procedury wymuszają na mnie zadanie pytania, które zapewni
dalsze postępowanie wobec badacza. Podjęcie nowej pracy będzie
się wiązało z poufnością. Badania przeprowadzane w Strefie są
ściśle tajne i tylko kilkanaście osób wie o ich działaniu,
akredytując przedstawione warunki automatycznie wyraża pan
zgodę na trzymanie w tajemnicy wyników pracy. Więc jak? Odmowa
sprawi, że zostawię pana tu i teraz i dalsze informacje nie
zostaną udzielone.~--~rzucił gość robiąc pełne powagi
spojrzenie.
Odbiorca zamyślił się na chwilę. Atmosfera zgęstniała na tyle,
by poczuli to obaj panowie. Tajemnicza postać nieco teatralnie
zdjęła okulary i odsapnęła aluzyjnie wymuszając na badaczu
podjęcie ostatecznej decyzji. Giennadij czuł na ciele dreszcz i
nietypowe podniecenie przed czymś nowym, nieznanym. Sytuacja
nie była do końca jasna, akceptacja warunków postawionych przez
Sewienkowa wiązała się z pewnymi ograniczeniami. Odmowa nie
wchodziła w grę, Giennadij był za bardzo podekscytowany i
ciekawy projektu, którego miał być świadkiem, a jednocześnie
ciężko mu było wyjść z klinczu i po prostu odpowiedzieć z
aprobatą i rozwiać wszelkie wątpliwości. Wyraźnie czuł jak
chłodne krople potu przemieszczają się po jego ciele, wzrok
cierpliwego przybysza zdawał się oddziaływać coraz mocniej.
Każda ulatniająca się sekunda sprawiała, że w oczach przyszłego
pracodawcy naukowiec wyrastał na błazna i niezdecydowanego
dzieciaka, który nie wie o jakim smaku lizaka chce dostać.
Giennadij musiał podjąć finalną decyzję.
\xx Przyjmuję propozycję~--~wyrzucił w końcu młody badacz.
\xx Doskonale lepiej zastanowić się kilka razy niż pochopnie
coś postanowić.~--~odrzekł powściągliwie i ruszył dalej
korytarzem.
Słowa o lekko sarkastycznym akcencie nie uderzyły w Giennadija.
Wnikliwość narastała z każdym pojedynczym cyknięciem sekundnika
kieszonkowego zegarku.
\xx Zapali pan?~--~wyrwał nagle gość.
\xx Nie palę~--~odparł standardowo Giennadij.
Stiepan za pomocą połyskującej zapalniczki benzynowej zajął
papierosa i zaciągnął się ochoczo. Od razu dało się poznać, że
Sewienkov jest silnie uzależniony, przysłowiowy dymek sprawiał
mu wiele radości i nie dało się tego ukryć.
\xx Proszę tutaj. Niech pan usiądzie.
\qd
Drzwi za nimi trzasnęły i wszelki hałas dobiegający z korytarza
umilkł. Silnie oświetlony gabinet został już wcześniej
przyrządzony. Na środku salki znajdował się stolik
rozdzielający dwa eleganckie fotele. Dwaj panowie zasiedli przy
drewnianej ławie i przystąpili do kontynuowania kontraktu.
%
\sx Otrzyma pan teraz kilka formularzy, które wystarczy
skwitować podpisem. Streszczę większość stronic, by
zaoszczędzić nam czasu. Przypominam, że podjął pan ostateczną
decyzję, bez odwołania. Wszelkie próby złamania zaakceptowanych
przez pana zasad poniosą za sobą poważne konsekwen...Zresztą
dorzeczy, darujmy sobie te formułki. Poza panem zostały wybrane
także dwie inne osoby, nazwisk zdradzać nie będę, wszystko w
swoim czasie. Machnie pan autograf w kilku miejscach i tym
samym stanie się pan jednym z elementów naszej układanki.
Projekt nosi nazwę "S-01" i jego celem jest stworzenie
"unikalnego środka" chroniącego przed bardzo wysokimi dawkami
promieniowania. Eksperymenty są niebezpieczne i dlatego
potrzebujemy wykwalifikowanych pracowników z doświadczeniem.
Tam nie będzie pan obserwował roślinek. Projekt "S-01" to
zupełnie inny kaliber. Więcej dowie się pan w najbliższym
czasie.
\xx Od, kiedy zaczynam i co z moją aktualną posadą?~--~spytał
Giennadij podpisując ostatni dokument
\xx Jutro o tej samej godzinie zostanie pan przetransportowany
na nowe miejsce pracy. Spotkamy się i omówimy jeszcze kilka
drobnych spraw. Owoce pańskiej pracy zostaną na miejscu.
Jeszcze jakieś pytania?~--~Stiepan wstał dogasił peta i
oczekując odpowiedzi zaczął pakować pliki do aktówki.
\xx To chyba wszystko. I dziękuje~--~Giennadij także wstał
wystawiając dłoń w geście pożegnania.
\xx To ja dziękuje~--~gość uścisnął dłoń i otwierając na oścież
drzwi wpuścił do gabinetu nieco korytarzowej wrzawy i świeżego
powietrza.
\qd
Młody naukowiec został sam, ponownie, z kolejną serią
dręczących go pytań.

\end{document}