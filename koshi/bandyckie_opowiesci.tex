\documentclass[../MAIN.tex]{subfiles}
\begin{document}
\ro{CZĘŚĆ I}
Krople deszczu bębniły miarowo po pokrytym blachą dachu tworząc monotonną
melodię, która w~duecie ze skrzypiącą, stalową, konstrukcją budynku tworzyła upiorną pieśń podsycaną raz po raz skowytem wiatru buszującego w~wybitych
oknach. Diabelska piosenka, która nieodmiennie towarzyszyła nadciągającej emisji, nie była w~stanie zagłuszyć piękna utworu sączącego się z~jednego z~pomieszczeń na piętrze. Melodia spływała po schodach w~dół rozpływając się pomiędzy siedzącymi przy ogniskach ludźmi.
\\
Bandycka brać, mordercy, złodzieje i~kryminaliści, którzy uciekli do Zony przed
prawem, by szukać lepszego jutra. Nikt z~siedzących w~starej, opuszczonej hali,
nie przejmował się nadchodzącym dniem. Ważne było tu i~teraz.
\\
Był jednak człowiek, który czuwał nad wszystkim, a każdy nadciągający dzień
spędzał mu sen z~oczu. Stanowił serce i~mózg organizacji skupiającej zebranych
tu ludzi, planował akcję, rządził, dzielił łupy i~rozdawał karty.
\\
Człowiek ten siedział w~jednym z~pomieszczeń na piętrze. W dyskretnym mroku,
przecinanym tylko migoczącą na stole świecą, delektował się grą na fortepianie popijając przemyconego do Strefy Cristobala. Ubrany w~czarny, skórzany płaszcz z wyhaftowanym na plecach chińskim, czerwonym, smokiem słuchał dźwięków sączących się z~ocalałego Bechsteina, którego parę lat temu przywieziono z~jednego z~domów w~Prypeci.
\\
Wystrój pomieszczenia stanowiła dodatkowo brązowa, skórzana sofa,
należąca kiedyś do jakiegoś prominenta lub dyrektora, tego samego, z~którego
domu Bandyci wywieźli fortepian kilkadziesiąt dni po ewakuacji Prypeci. Dziś
sofa lata świetności miała już za sobą. Stała pod ścianą na betonowych pustakach
zastępujących coś, co kiedyś było misternie rzeźbionymi, drewnianymi nogami
\\
W pokoju stały też dwa krzesła, metalowa szafa na akta, leżący w~kącie materac
oraz ogromny, stalowy sejf, który górował swym cielskiem nad resztą mebli. Na
jednej ze ścian wisiał na kołku Vintorez, na drugiej, naprzeciwko broni,ogromna
mapa okolic Czarnobyla, w~którą powbijano dziesiątki kolorowych pinezek i
szpilek z~chorągiewkami.

Wieko fortepianu opadło wprawione w~ruch delikatną kobiecą dłonią.
Galina
skończyła grać i~obróciła się w~stronę siedzącego na sofie mężczyzny. Dima był
jej mentorem, przyjacielem, kochankiem oraz jedną z~najpotężniejszych osób w
Zonie~--~szefem Bandytów.
Zaopiekował się nią, kiedy uciekła z~domu przed ojcem pijakiem i~trafiła do Zony
z oddalonej o setki kilometrów Moskwy. Uratował jej też życie, kiedy w~Kordonie
trzech Stalkerów próbowało ją zgwałcić. Dwóch, Dima zatłukł na miejscu w~ataku
furii. Nie zdążyli nawet wyciągnąć broni. Trzeciego, z~przestrzelonymi kolanami i~dziurą w~brzuchu, zostawił, aby mógł się wykrwawić. To, co się tam wydarzyło, Galina pamięta do dziś. Wydarzenia tamtego dnia znów
wróciły pobudzone melancholijną muzyką.
\sx Wracaj skąd przybyłaś~--~powiedział Dima podając jej plecak, kiedy
było już
po wszystkim.~--~To nie miejsce dla ciebie. Nie mam pojęcia co tu robisz, ale
nie wróżę ci tutaj świetlanej przyszłości.
\xx  Nie mogę\3k~--~odpowiedziała dziewczyna łkając.~--~Nie do tej świni! Znów
się będzie do mnie dobierać\3k A~jeśli nawet się
zdecyduję wrócić\3k to co będę robić? Żebrać nie zamierzam, a dawać dupy na
ulicę nie pójdę. Mam swoją godność! Zostaję tutaj i~szlus!
\qm
To, co usłyszał, zszokowało go. Nie takiej odpowiedzi się spodziewał. Zresztą
nie liczy na żadną odpowiedź. No może tylko na coś w~stylu: „Dziękuje za
uratowanie mi życia” lub coś podobnego. Myślał, że wyjdzie od tak z~poczuciem
dobrze wykonanego obowiązku, a dziewczyna weźmie plecak i~wróci do domu czy tam
gdziekolwiek, skąd przybyła. Obojętnie gdzie, byle zniknęła i~przestała być
problemem. Mylił się. Skonsternowany zatrzymał się w~drzwiach.
%
\sx A~co tu niby masz zamiar robić!? Umiesz strzelać? Zabiłaś kogoś, ukradłaś
coś? Nie masz pojęcia czym jest Zona i~jakie zasady w~niej panują. Zginiesz
prędzej czy później, jest to jasne jak słońce. A~skoro jest to jasne to nie wiem
po co marnuję swój czas tłumacząc ci to. Weź swoje łachy i~idź stąd. Tak będzie
dla ciebie najlepiej!
\xx  Poczekaj!~--~rzuciła do odwracającego się mężczyzny Galina.~--~Nauczysz mnie
strzelać i~tych wszystkich rzeczy, no wiesz\3k co są
potrzebne, żeby przetrwać tutaj\3k mam trzy tysiące rubli \x
powiedziała wyjmując z~plecaka garść pomiętych banknotów. Może to starczy na
początek?
\xx  Chyba cię pogięło!~--~odpowiedział Dima.~--~Nie będę cię niczego uczył. A~te
trzy kafle, które trzymasz w~ręku, już
wykorzystałaś. Jakbyś zapomniała, właśnie uratowałem twój tyłek.
\xx  Masz!~--~rzuciła w~jego stronę pieniądze zdenerwowana Galina.Weź je sobie!
Zarobiłeś, są twoje! Nie chcesz mnie uczyć to nie. Wracam do
tej bazy, co byłam tam wcześniej. Może tam mi pomogą\3k
\xx  Taaa, pewnie. Tak się akurat składa, że Strefa pełna jest różnej maści
szumowin i~kryminalistów. Przyciąga ich jak magnes. Ci też są nie lepsi.
Wskazał na martwych mężczyzn.
\xx  Stalkerzy, \textit{job twoju mać}.~--~Splunął w~kierunku dwóch ciał leżących
na środku budynku.~--~Niektórzy z~nich nie są gorsi od naszych~--~dodał po
chwili.
\xx  Nie trafiłaś zbyt dobrze dziecinko, to nie miejsce dla dam.~--~Zaśmiał się
Dima.~--~Taki cukiereczek w~Zonie jak ty, to jak gwiazdka z~nieba dla żyjących
tu skurwysynów.~--~Nie ci, to inni spróbują się do ciebie dobrać. Myślisz, że
Stalkerzy są tacy prawi? Właśnie miałaś przykład, jacy są szlachetni. No już,
bierz plecak i~wypierdalaj stąd!~--~warknął Dima do zdumionej dziewczyny.~--~Nie
będę cię niańczył. Nie mam czasu!
\xx  Nigdzie nie pójdę!~--~rzuciła hardo dziewczyna.~--~Nie skończę jak szmata i nie pozwolę, żeby jego łapska znów mnie obmacywały!
Idę z~tobą.
\xx  Taka jesteś cwana!~--~powiedział zdenerwowany Dima.~--~To go zastrzel! \x
Wskazał jęczącego w~kącie budynku mężczyznę.~--~Przecież to dla ciebie pryszcz!
To będzie twój pierwszy test~--~dodał po chwili podając jej starego Makarowa
należącego do jednego z~zabitych.
\xx  Zdasz, idziesz ze mną. Nie zdasz, zabierasz się stąd i~więcej cię na oczy nie
widzę. Idę na szluga, masz pięć minut~--~rzucił bandyta i~wyszedł z~budynku.
\qm
Minęło pięć minut odkąd Dima skończył rozmawiać z~nieznajomą. Bandyta dopalił
Marlboro po czym rzucił go w~krzaki i~już miał ruszać w~kierunku wysypiska,
kiedy z~budynku wyszła Galina. Roztrzęsioną ręką podała mu pistolet, lecz nic
nie powiedziała. Nie słyszał wystrzału, ilości naboi w~komorze się zgadzała,
więc nie było mowy o pomyłce. Nie zabiła go. Dokładnie tak, jak myślał. Była za miękka, by odebrać komuś życie.
%
\sx Oblałaś.~--~powiedział, kiedy odbierał z~jej rąk broń.~--~Mówiłem ci, to nie
jest miejsce dla ciebie. Rozstajemy się zgodnie z~umową. Żegnaj.
\xx  On\3k on nie ży\3k~--~wymamrotała coś dziewczyna łapiąc go za ramię,
kiedy odchodził.
\xx  O co ci chodzi do cholery?~--~spytał zdenerwowany Dima odwracając się.~--~Mów
o co biega, bo nie mam czasu!
\xx  Nie żyje\3k ~--~wymamrotała dziewczyna.~--~Zabiłam go.
\xx  A~niby kurwa jak?~--~spytał zdziwiony mężczyzna.~--~Nie słyszałem ani jednego
wystrzału.
\qm
Nie usłyszał odpowiedzi. Kobieta sta\-ła przed budynkiem trzęsąc się w~szoku.
Spytał ją jeszcze o coś, ale już nic nie odpowiedziała. Wszedł więc do budynku,
by sprawdzić, co się stało z~rannym mężczyzną.
Leżał na wznak na stercie drewna
z podciętym gardłem. Charczał jeszcze coś w~agonii, próbując coś powiedzieć, ale krew zalała mu już widocznie struny głosowe przez co gulgał jak indor, niż mówił.
Dima kompletnie go zignorował, nie wart był nawet, by patrzeć jak umiera.
Mężczyzna omiótł pomieszczenie wzrokiem natrafiając na leżący w~kącie
zakrwawiony nóż. Było to narzędzie, którym dziewczyna skróciła nędzne życie
leżącego obok nikczemnika.
\sd
\xx  Miała pistolet, a poderżnęła mu gardło nożem\3k Po co?~--~zapytał sam siebie zdumiony bandyta po czym opuścił
pomieszczenie.
\qd
Wychodząc z~budynku dostrzegł śmierć lubieżnie obejmującą
konającego, która przytuliła się do niego głaszcząc go czule kościstą ręką po
włosach.
\sx
Zgadza się, nie żyje~--~powiedział do Galiny, która dalej stała przed
budynkiem patrząc nieobecnym wzrokiem w~jakiś niewidoczny punkt na horyzoncie.
Tylko zastanawia mnie, czemu nie strzeliłaś temu pacanowi w~łeb, jak Bóg
przykazał, tylko poderżnęłaś mu gardło?
\xx  Szkoda było amunicji na takiego śmiecia\3k~--~odpowiedziała beznamiętnie
dziewczyna.
\xx  Ach tak\3k~--~skwitował Dima.~--~Jakaś ty rezolutna\3k Może coś jeszcze
jednak z~ciebie będzie. Dobra, dość
gadania~--~dodał po chwili.~--~Bierz graty i~spadamy stąd. Wracamy do obozu.
Makarowa weź sobie, to i~tak
stary lump.
\qm
Dziewczyna tylko skinęła głową na znak zgody, ale nic nie powiedziała. Wciąż
była oszołomiona tym, co się przed chwilą stało. Kiedy Dima ruszył w~stronę
bazy, podniosła plecak i~podążyła za nim niczym cień.
Minęły dwie godziny, odkąd opuścili miejsce , gdzie się poznali. Dziewczyna
wykonywała wszystkie polecenia bandyty, jakby była w~transie, ale nic nie
mówiła. Za odpowiedź służyło jej kiwnięcie głowy w~pionie lub poziomie w
zależności, czy chodziło jej o TAK lub NIE.
Dimie pasował taki układ. Zero zbędnych pytań, tłumaczenia. Sam nie wiedział, po
co prowadził ją do obozu bandytów. Była tam narażona na zdecydowanie większe
niebezpieczeństwo, niż w~osadzie, gdzie zatrzymywali się nowo przybyli do Zony.
Ale słowo się rzekło, test zdała. Poza tym zaimponowała mu. Co innego zabić w
walce, a co innego zabić bezbronnego podcinając mu gardło.
\sd \xx  Ma dziewczyna jaja~--~pomyślał.
\qd
Pożyjemy, zobaczymy, co z~tego wyniknie. Dywagacje mężczyzny przerwało
pytanie.
\sx Striełok\3k Znasz go?
\qm
Dima obrócił się i~spojrzał na dziewczynę. Ledwo stała na nogach trzymając w
ręce plecak, którego spód nieomal przetarł się od ciągnięcia go po asfalcie i
ziemi.
\sx Alleluja!~--~krzyknął Dima.~--~Ty mówisz\3k Super! Striełok\3k Taaa,
znałem go. Każdy w~Zonie zna jego historię. I co z~tego?
\xx  Słyszałam, że tu gdzieś jest coś, co spełnia marzenia\3k
\xx  Nieee, kurwa nie!!! Kolejna, która uwierzyła w~te bzdury! Kto ci naopowiadał
takich bajek???
\xx  W metrze był taki jeden\3k co mówił\3k
\xx  Eeee\3k Daj spokój\3k~--~powiedział Dima i~machnął zrezygnowany ręką.
Powiem w~skrócie, bo nie chcę mi się dużo gadać. Było tak:
\qd
\textsl{Była awaria w~elektrowni, i~wszystkich ewakuowali, a tereny dookoła jej
zostały
skażone. Ale katastrofa i~cała ta radiacja coś popieprzyły i~zaczęły się tutaj
dziać dziwne rzeczy. Ja tu przybyłem gdzieś na początku lat dziewięćdziesiątych,
jak promieniowanie zmalało na tyle, że można było zwiedzać okolicę Czarnobyla.
To były czasy, kiedy nie było podziału na Bandytów i~Stalkerów. W Zonie
wylądowali ci, którzy chcieli poszukać swojego szczęścia, uciec przed kimś,
czymś, rozpocząć nowe życie, zniknąć. Było nas niewielu, ale dość szybko
odkryliśmy, że to, co sprawiło, że miejsce to stało się takie porąbane,
spowodowało, że można było nieźle zarobić. Artefakty i~ich szukanie w~anomaliach
to była żyła złota. Niektórzy się naprawdę nieźle obłowili i~wyjechali stąd
bogaci, inni zostali, a jeszcze inni\3k no cóż\3k zginęli szukając
szczęścia. Wieści o czarnobylskim Klondike szybko rozniosły się. Do Zony zaczęli
zjeżdżać różni ludzie skuszeni wizją szybkich pieniędzy Większość z~nich to byli
kryminaliści, bandyci, mordercy i~inne szumowiny, którym nie chciało się robić,
a chcieli się wzbogacić kosztem innych. Mordowali, grabili, kradli artefakty. W
skutek tego powstały dwa obozy, które zaczęły ze sobą walczyć o wpływy, tereny i
kontrolę nad tym zapomnianym przez Boga miejscem. Wtedy Striełok zdecydował się
pójść ze znajomymi w~głąb Zony, tam, gdzie nikt nie odważył się wcześniej wybrać
w samą paszczę diabła, do elektrowni\3k Po wizycie twierdził, że znalazł
tam maszynę, która spełnia życzenia, ale sądząc po stanie w~jakim wrócił, myślę,
że radiacja i~promieniowanie psioniczne zdewastowało mu tak mózg i~cały
organizm, że dostał urojeń i~omamów. Spełniacz życzeń to była jego obsesja.
Ciągle o nim gadał, kombinował jak tam wrócić, aż wreszcie poszedł. Sam\3k I
przepadł jak kamień w~wodę. Słuch po nim zaginął.
Wniosek jest prosty~--~nie ma
tam żadnej maszyny spełniającej życzenia, tylko ogromna radiacja i~jakieś inna
siła zwana psioniką, o której praktycznie nic nie wiemy. Nikt się w~okolice
elektrowni nie zapuszcza, bo kończy momentalnie jak warzywo, czyli żywy trup. I
Striełok pewnie też tak skończył, albo zabiło go promieniowanie. Oto cała
historia. Historia faceta, któremu po prostu usmażyło mózg i~sfiksował. Chcesz
coś jeszcze wiedzieć kochana?}
\sx
Czyli nie ma żadnej maszyny, co spełnia życzenia???~--~spytała zdumiona
Galina
\xx  Nie. Nie ma, nie było i~nie będzie. Marzenia może sobie spełnić, jak
znajdziesz trochę artefaktów i~je sprzedasz z~zyskiem, albo popracujesz dla
którejś z~frakcji, ale raczej wątpię, czy odłożysz na tyle, żeby ustawić się do
końca życia. Tylko nielicznym się udaje.
Teraz nie tak łatwo się wzbogacić co
kiedyś\3k Do tego to pieprzone wojsko, które wszystko chce kontrolować\3k
\xx  To po co ja tu przyjechałam? Nie będę szukać jakiś arte~--~czegoś tam\3k i
nie będę dla nikogo pracować, bo co niby miałabym robić? Dziwką nie zostanę!
\xx  Nie taki rodzaj pracy miałem na myśli~--~zaśmiał się Dima.~--~Co nie znaczy,
że nie zrobiłabyś na tym kariery. Ile ty w~ogóle masz lat dziewczyno?
\xx  Siedemnaście.~--~odpowiedziała Galina.
\xx  No tak\3k To nie dziwię się, że łyknęłaś taki bałach. Do rodzinki pewnie
nie wrócisz, spełniacza nie ma\3k to co ty chcesz tu robić? Pomijam, że dwie
godziny temu poderżnęłaś kolesiowi gardło i~byłaś świadkiem dwóch morderstw.
Normalka. Dzień jak co dzień.
\xx  No właśnie nie mam pojęcia. Do Moskwy nie pojadę. Mam co prawda ciotkę
mieszkającą niedaleko stolicy, mogłabym u niej zamieszkać, ale prędzej, czy
później mój ojczulek zbok~--~milicjant mnie wytropi i~raczej będę go musiała
zabić, niż wrócę do domu. Na ulicę nie pójdę, a w~sumie to nic też konkretnego
nie umiem. Co tu robić???
\xx  No to masz przechlapane. A~gotować chociaż umiesz?
\xx  Tak, całkiem nieźle.~--~Uśmiechnęła się dziewczyna.
\xx  Dobra, powiem chłopakom, że znalazłem nowego kucharza. Ostatniego zeżarła
chimera, a Baleron tak marnie gotuje, że ostatnio pół załogi dostało sraki po
jego genialnym obiedzie. Jak się sprawdzisz, to zostaniesz, a jak nie to trzeba
będzie wysłać kogoś do Moskwy, żeby zlikwidował tą świnię twojego tatuśka, a ty
wrócisz później do matki jako skruszona córka i~wszyscy będą szczęśliwi. Amen!
podsumował Dima.~--~Dobra idziemy do obozu, to już niedaleko.
\qd
  \ro{CZĘŚĆ II}
Chmury dryfowały po niebie przypominając strzępy jakiejś gigantycznej waty
cukrowej, któ\-rą ktoś porozrywał i~niedbale rozrzucił po firmamencie. Leniwie
sunęły po nieboskłonie niczym parowóz, który próbuje pokonać ostatnie
wzniesienie resztką sił.
Wiatr nieśmiało szumiał pośród konarów nagich drzew
przewalając po ziemi stosy mieniących się jesienią liści. Kiedyś stanowiące
okrycie rozłożystych dębów, rozcapierzonych grabów, czy smukłych topól, dziś
tańczyły razem w~powietrzu niewidzialny taniec, aby osiąść wreszcie na
zachlapanych błotem drogach, porzuconych, rdzewiejących samochodach czy brudnych
reliktach komuny, jakimi były różnego rodzaju żelbetowe prefabrykaty, które
nigdzie nie miał już być wbudowane.
Jedną z~takich dróg maszerowało dwoje ludzi.
Kobieta i~mężczyzna. Zmierzali w~stronę obozu Bandytów, co nie umknęło też
uwadze wartowników pilnujących obozu. Szybko rozpoznano, że jedną z~osób jest
bandyta~--~Dima Tatarenko. Drugiej osoby nie zidentyfikowano. Stwierdzono tylko,
iż jest to kobieta i~to w~dodatku dość młoda. Wieść o tym rozprzestrzeniła się
po obozie niczym błyskawica. Kiedy Dima wraz z~towarzyszką dochodzili do obozu,
huczało tam już jak w~ulu. Otwarto bramę, za którą ustawił się komitet powitalny
złożony z~kilkudziesięciu Bandytów. Dima, widząc jak huczne przyjęcie chcą mu
zgotować koledzy, odwrócił się do Galiny i~rzekł:
\sx
Nie odpowiadasz na żadne pytania i~zaczepki, nie zatrzymujesz się i~nie
gapisz się na nikogo! Podążasz za mną krok w~krok i~nie oddalasz się dalej, niż
na pięć metrów. Czy to jasne?~--~spytał retorycznie.
\xx  Tak.~--~odpowiedziała cicho dziewczyna.
\xx  To dobrze. Zaoszczędzi to nam kłopotów. Idziemy.
\qm
Wraz z~malejący dystansem, który dzielił ich od obozu, hałas dochodzący z~bazy
wzmagał się. Mijali właśnie miejsce zwane Golgotą, kiedy z~jednego z~wbitych w
ziemię drewnianych krzyży doleciał ich cichy szept:
\sx
Dimaaa, pomóż, błagam cię\3k
\qm
Tatarenko przystanął i~spojrzał w~kierunku, z~którego doleciał głos. Na jednym z
krzyży wisiał jeden z~Bandytów zwany przez kolegów Kłamcą. Pozostałe dwa krzyże,
które stały obok, były puste. Kłamca jeszcze raz się wysilił i~zachrypiał:
\sx Pomóżżż\3k Zastrzel mnie\3k
\xx  Doigrałeś się Iwan, doigrałeś.~--~skwitował Dima ignorując jego rzężenie. \xx
Wiedziałem, że prędzej czy później cię złapią. Może i~bym ci pomógł, ale powiedz
przyjacielu, gdzie jest mój pistolet, który zginął mi dwa tygodnie temu? Nie
widziałeś go czasem?
\xx  Dima, to nie byłem ja. Przysięgam\3k Nie miałem z~tym nic\3k
\qm
Zrezygnowany Dima machnął ręką, po czym powiedział:
\sx Wiesz, jest takie stwierdzenie, które idealnie pasuje do twojej osoby. Brzmi
tak: Kłamca zawsze będzie kłamcą. Nawet, jakby cię żywcem obdzierali ze skóry,
nie przyznałbyś się\3k
\xx  Ja naprawdę go nie ukradłem\3k~--~tłumaczył się mężczyzna.
\xx  Ehhh\3k Chodź Galina, idziemy. Szkoda marnować czas na rozmowę z~nim. Choćby
miał umierać jeszcze trzy dni to i~tak nie powie, co się stało z~moim gnatem.
Zajebał go jak nic. Dlatego dostał ksywę Kłamca. Bo wiecznie kłamał, kłamie i
kłamać będzie dopóki nie umrze. Módl się, może szybko przyjdzie Emisja i~skróci
twój nędzny los.\x powiedział Bandyta w~stronę konającego, po czym ruszył wraz
z dziewczyną w~stronę obozu.
\qm
Kłamca jęczał coś jeszcze, gdy oddalali się, ale Dima nie słuchał, co mężczyzna
miał do powiedzenia. Był to stek wyzwisk, błagań i~gróźb, które nie robiły na
Bandycie żadnego wrażenia. Werdykt Piłata był niepodważalny. Kłamca został
skazany na śmierć, bo okradał swych towarzyszy i~nic nie mogło tego zmienić.
\sd \xx
Niech się cieszy, że szef nie kazał go nabić na pal~--~pomyślał Dima wkraczając do obozu.
\qm
Obóz przywitał ich szarością. Metalowe konstrukcje chłostane wiatrem i~kwaśnymi
deszczami rdzewiały czekając z~utęsknieniem na konserwację, która nigdy nie
miała nastąpić. Żelbetowe konstrukcje obnażały swoje stalowe żyły i~arterie w
wymuszonym przez mróz ekshibicjonizmie. Każdej zimy proces zamrażania i
rozmrażania coraz bardziej dewastował betonowe molochy powodując odpadanie z
nich kolejnych fragmentów betonu. Pokryte mchem, który rozprzestrzeniał się po
nich niczym liszaj, umierały powoli niszczone siłami natury.
Dima wraz z
dziewczyną minęli właśnie dawny budynek portierni, którego ściany pokrywały
niegdyś piękne płytki elewacyjne~--~jedne z~najlepszych, jakie swojego czasu
można było dostać na Ukrainie. Dziś płytki leżały potłuczone wokół budynku
porozrzucane niczym liście opadające z~drzew jesienną porą. Przez puste otwory w
budynku przewalał się wiatr.
Bandyta, i~uczepiona rękaw jego płaszcza
dziewczyna, minęli na wpół zdewastowany budynek i~skierowali swoje kroki w
stronę dwóch budowli połączonych ze sobą wiatą. Tłum podążył za nimi drąc się i
wyjąc. Raz po raz z~stłoczonej za nimi ciżby dolatywało:
\sx Dima, ale żeś se towar znalazł!!! Daj wypróbować!~--~darł się jeden z
Bandytów.
\xx  Maleńka, przyjdź do mnie wieczorem! Pokażę ci mojego rumaka!~--~krzyczał
inny.
\qd
\sx  Idioci~--~pomyślał Tatarenko wchodząc pod wiatę. Spojrzał na Galinę. Dziewczyna była blada, a w~jej oczach czaił się strach.
\qd \sx  Mam trochę fantów po tych zabitych Stalkerach.~--~powiedział Dima, aby
rozładować napięcie i~odwrócić uwagę dziewczyny od wrzeszczącej hołoty. \x Trzeba by to opchnąć, bo raczej mi się to nie przyda. Są tu dwa bary, w
których
można sprzedać taki szmelc.\x Wskazał ręką w~kierunki drzwi do budynków, które
stały naprzeciwko siebie po obu stronach wiaty. Knajpa na lewo zwie się "Pod
zamulonym sterem". Ogólnie mówiąc jest to mordownia, gdzie serwuje się podły
alkohol. Lądują tam zazwyczaj ci, którzy nie mają kasy, albo których wyrzucono z
drugiego baru. Prowadzi ją gostek zwany Kapitanem. Taki tam złodziejaszek i
oszust, służył kiedyś w~marynarce, ale go wywalili dyscyplinarnie za kradzieże.
Drugi bar jest nieco bardziej ekskluzywny. Zwie się "Pod wulgarną małpą" i
prowadzi go facio zwany Małpą. Małpa nie sprzedaje trefnego alkoholu, ma
normalną klientele, można też u niego coś zjeść. Jest tam taki mały grill. No i
podest. Czasami ktoś tam występuje nawet na nim. Ma natomiast ceny dwa razy
wyższe\3k no i~robi się agresywny, jak stwierdzi, że ktoś jest już za bardzo
pijany. Najpierw prosi o kulturalnie wyjście. Jeśli klient nie posłucha rady
staje się mniej miły. Otwiera jego głową drzwi, czasem zdarzy się, że i~okno,
jeśli do wyjścia ma za daleko\3k
\xx  Kapitan to rozumiem, to jakiś stopień wojskowy. Pewnie był kapitanem\3k \x
przerwała mu Galina.~--~Ale czemu ten drugi nazywa się Małpa?
\xx  Małpa jest obrośnięty na całym ciele. Ma włosy dosłownie wszędzie. Na
plecach, na klacie, na rękach. Kłęby włosów. Wygląda jak jakiś zwierz.
Mieszkańców poznasz później.~--~Uciął dyskusje Dima.~--~Teraz trzeba opchnąć
graty, a potem idziemy do Piłata omówić twoją sprawę.
\xx  W porządku. Ty tu rządzisz.~--~odparła dziewczyna.
\xx  Nie inaczej.~--~skwitował Bandyta.~--~Idziemy do Małpy. Musisz coś zjeść,
marnie wyglądasz.
\qm
Knajpa była pusta. Jedynie za kontuarem na końcu sali siedział właściciel wsparty
na łokciu o blat lady. Z obojętną miną przyglądał się, jak Dima lawirował
pomiędzy poprzewracanymi krzesłami, oponami i~wielkimi bębnami po kablach
telefonicznych, które zastępowały stoliki. Za Dimą podążała dziewczyna. Małpa
zlustrował dziewczynę. Długie, blond włosy, twarz anioła, usta koloru jarzębiny.
Pod obcisłą bluzką kryło się coś, co mogło śmiało konkurować z~najwyższymi
szczytami w~Himalajach. Kiedy dziewczyna wyszła z~za jednego z~bębnów
odsłaniając swe długie, zgrabne, nogi chronione jedynie krótką spódniczką, Małpa
aż jęknął z~wrażenia.
\sx No Dima\3k Słyszałem, że sprowadziłeś dziewczynę do obozu, ale nie
podejrzewałem, że jest tak olśniewająco piękna\3k Skąd ją wytrzasnąłeś???
\xx  To długa historia. Nie mam czasu ci wszystkiego tłumaczyć. Muszę zaraz
pogadać z~Piłatem. Mam tu też parę rzeczy, których chciał bym się pozbyć.
Rzucisz okiem?
\xx  Naprawdę nic mi nie powiesz na jej temat?
\xx  Nie. Potem ci wszystko wyjaśnię. Teraz nie mam czasu. Zaraz się tu wszyscy
zlecą.
\xx  Daj, co tam masz.~--~powiedział zrezygnowany barman.
\qm
Bandyta wyłożył na stół rzeczy zabitych w~Kordonie mężczyzn. Były wśród nich dwa
dozymetry, kompas, wielofunkcyjny scyzoryk, pistolet Fort-12 oraz kilkanaście
sztuk nabojów kalibru 9 mm. Małpa obejrzał fanty i~pokręcił głową.
\sx Kurwa, ale szajs\3k Nie było nic lepszego? To wygląda jakbyś ograbił jakiś
frajerów. Sam złom\3k
\xx  Istotnie tak było. Jakby to było coś wartościowszego to wiesz, że udałbym się
z tym do Miazgi. On dobrze płaci za wartościowe rzeczy.
\xx  A~co, ja nie płacę dobrze?~--~odezwał się obrażony Małpa.
\xx  Tak. Dajesz naprawdę wysokie ceny.~--~zadrwił Dima wskazując za plecami Małpy
stoły pełne przeróżnych dupereli.
\xx  Dam ci za to czterysta rubli. Pasuje?
\xx  Może być. Ale dorzuć to tego jeszcze setkę wódki i~szaszłyka. Mała jest chyba
głodna.
\xx  Setkę i~szaszłyk\3k Niech pomyślę\3k Daj jeszcze jakiegoś fanta, żebym nie
był stratny.
\xx  Dobra. Masz książkę.~--~powiedział Dima wyciągając z~kieszeni płaszcza
niewielkie zawiniątko.
\xx  A~na chuj mi książka Dima? Widziałeś, żebym coś, kiedyś czytał?
\xx  To jakiś pieprzony poradnik o przetrwaniu. Znalazłem go w~plecaku jednego z
tych idiotów. Myślę, że wart jest szaszłyka. Miałem poczytać, co za bzdury tam
wypisują\3k
\xx  W porządku. Daj to dziadostwo. Jesteśmy kwita. Może będzie na to kupiec.
\qm Bandyta popchnął książkę po ladzie w~stronę Małpy. Ten odwrócił się w~stronę
stojącego w~kącie grilla, po czym podał Galinie dymiący kijek z~nabitymi nań
mięsami. Dziewczyna wzięła go z~rąk barmana, przystawiła do nosa i~powąchała.
Zapach okazał się jej nieznany.
\sx
Co to za mięso?
\xx  Dziczyzna.~--~odpowiedział barman lakonicznie.
\xx  Dziwnie pachnie\3k
\xx  Nie marudź, jedz.~--~odezwał się Dima, chociaż wiedział, że może jedną
czwartą zawartości dania stanowił dzik.~--~Mam jeszcze z~Małpą coś do obgadania,
ale to już moje prywatne sprawy. Zajmij się mięsem i~nie obracaj się. Cały obóz
gapi się na twój hmmm\3k Kapujesz?
\xx  Myślę, że tak.~--~stwierdziła dziewczyna.~--~Jeść i~nie obracać się. Proste.
\xx  Dokładnie. A~teraz pozwól, że omówię pewne kwestie z~barmanem.~--~powiedział
Tatarenko i~obrócił się w~stronę Małpy.
\qd
Galina nie słyszała dalszej części
rozmowy. Mężczyźni rozmawiali szeptem. Zajęła się jedzeniem skrupulatnie
obgryzając mięso z~patyka. Kiedy patyk był już prawie ogołocony ze swojej
zawartości, Dima kończył właśnie rozmowę z~Małpą. Chwilę później Dima odezwał
się, kiedy poustalał już wszystko z~obrośniętym mężczyzną.
\sd \xx
Małpa dał mi klucze na zaplecze. Wymkniemy się tylnym wejściem. Chyba
rozumiesz, że wyjście frontowymi drzwiami nie skończy się dobrze? Nie będę się
bił o twoją cnotę z~całym obozem\3k
\xx  Uhmm\3k~--~powiedziała dziewczyna połykając ostatni kęs szaszłyka.
\xx  To posłuchaj uważnie, co robimy. Za parę sekund przejdę za ladę i~ty zrobisz
to samo. Tam, w~głębi pomieszczenia, są drzwi. Pobiegniemy tam, a potem przez
zaplecze wydostaniemy się na zewnątrz. A~potem walimy prosto do Piłata. Wszystko
jasne?
\xx  Tak.
\xx  W taki razie wiejemy!~--~krzyknął Dima omijając kontuar.
\qm
Chwilę później Bandyta i~dziewczyna byli już przy drzwiach prowadzących na
zaplecze. Podniecony nagłą ucieczką tłum ruszył za nimi próbując dostać się za
bar. Tratując krzesła i~stoliki natarli w~stronę lady, za którą stał Małpa.
\sd \xx
Wypierdalać z~mojego baru!!!~--~ryknął barman.~--~Nie będziecie mi tu
demolować knajpy!
\xx  No co ty Małpa\3k~--~krzyknął ktoś.
\xx  My tylko chcieliśmy\3k~--~odezwał się ktoś inny.
\xx  Chuj mnie obchodzi co chcieliście.~--~powiedział barman wyciągając spod stołu
obrzyna.~--~To mój bar i~obowiązują tu moje zasady. Jedna z~nich mówi, że w
barze ma być cicho i~kulturalnie. A~ja tu nie widzę kultury do kurwy nędzy tylko
drącą ryja hołotę, która myśli w~tym momencie fiutami, a nie głową!
Coś się nie podoba???\x spytał barman zaczepnym głosem słysząc pomruki niezadowolenia z
głębi sali.%\looseness-1
\xx  Dobra, dobra wychodzimy.~--~rzucił ktoś z~tłumu.
\xx  Spadajmy zanim Małpie
znów odbije szajba.~--~zawtórował mu drugi.
\xx  To jazda stąd!~--~ryknął na odchodne barman.~--~Przyjdźcie, jak już wam
sperma przestanie bić do głowy. Wypad!
\qm
Tłum wycofał się na zewnątrz lokalu. Z wkurzonym Małpą nie było żartów. Potrafił
być nieobliczalny, jeśli się zdenerwował. Po Dimie i~Galinie nie pozostał ślad.
Bandyci obeszli budynek, w~którym znajdował się bar, ale nie natknęli się na
parę. Sprawcy zamieszania zniknęli.
\ro{CZĘŚĆ III}
\mm Podczas, gdy Galina rozmyślała o swoim pierwszym dniu w~Strefie, zza jednej z
nóg Bechsteina wygramolił się niewielki pająk. Szybko ruszył poprzez bezkres
popękanej, betonowej posadzki, w~stronę dwóch wyrastających z~podłoża
gigantycznych filarów, które obrał za cel swojej podróży. Chwilę później wspiął
się po jednym z~słupów, którym okazała się jedna z~nóg Dimy, i~wylądował na
jakimś miękkim, nieznanym liściu. Świetne miejsce pomyślał rozpoczynając tkać
sieć pomiędzy dwoma połami płaszcza. Pająk systematycznie i~mozolnie konstruował
przyszłą pułapkę niepomny istnienia Bandyty. Nic dziwnego, organizował pierwszą
w życiu zasadzkę w~swym krótkim, zaledwie kilku minutowym, życiu. Kiedy dzieło
było już praktycznie ukończone, na oparciu zniszczonej sofy ukazał się drugi,
znacznie większy pająk. Przyglądał się chwile, jak jego młodszy kolega pracuje,
po czym podszedł bliżej i~powiedział.
\sx
Daremna twoja praca przyjacielu.\\ Niezbyt szczęśliwe miejsce wybrałeś na
zastawienie sieci\3k
\xx  Dlaczego?~--~odparł zaskoczony tymi słowami budowniczy.
\xx  Olbrzym zaraz się poruszy i~porozrywa sieć.~--~oznajmił większy pająk.
\xx  Olbrzym? A~któż to taki???
\xx  Ahhh\3k Widzę, że jesteś bardzo młody i~nie poznałeś jeszcze ludzi. Kim są
ludzie\3k Dobre pytanie. Długo by tłumaczyć\3k Powiedzmy, że są to ogromne
istoty, które uwielbiają na nas polować.\\ Tyle ci powinno wystarczyć na początek,
reszty dowiesz się sam.%\looseness-1
\xx  Jak to polują??? Żywią się nami?~--~spytał zdumiony mały pająk.
\xx  Hahhha.~--~zaśmiał się drugi z~pająków.~--~Nic z~tych rzeczy. Wystarczy im,
że walną nas gazetą lub strącą ze stołu pstryknięciem palców. Nie stanowimy dla
nich żadnego zagrożenia i~nie jesteśmy im do niczego potrzebni.Popatrz w~górę, a przekonasz się.
\qd
Mniejszy pająk spojrzał we wskazanym kierunku za radą kolegi. Ujrzał oblicze jakiejś ogromnej
istoty. Olbrzym spoczywał na sofie w~bezruchu z~na pół przymkniętymi powiekami.
\sx
Chodź, pokażę ci, gdzie można rozstawić sieć. Znam dobre miejsce całkiem
niedaleko.\x zaindagował większy osobnik.~--~Można tam złapać niezłą muchę czy
ćmę. Taką jak ta.~--~Wskazał fruwającego dookoła świecy owada. Ćma igrała z~losem
co rusz to podfruwając, to odlatując od wabiącego ją źródła światła. Ciekawość
wzięła jednak górę, ćma zbliżyła się zbyt blisko ognia. Sekundę później skrzydła
owada stały się wspomnieniem. Owad runął w~dół niczym Dedal lądując w
miniaturowym morzu, wypełniającym brzegi kieliszka, z~którego Bandyta sączył
wino.
\xx  Tak szkoda.~--~podsumował większy pająk.~--~Chodźmy nim olbrzym się zbudzi!
ponaglał kolegę.
\qm
Mniejszy pająk posłuchał. Ruszył za swoim nauczycielem. Chwilę później Dima
sięgnął po kieliszek. Nitki pajęczyny pękły niczym bańka mydlana. Po sieci nie
pozostał ślad.
\sx
Kurwa mać!~--~powiedział Bandyta wylewając resztkę Cristobala do stojącego
pod stołem wiadra.~--~Pieprzone robactwo, wszędzie go pełno. Chyba musimy znowu
pobielić ściany wapnem\3k Jak sądzisz?~--~zwrócił się Dima do siedzącej przy
fortepianie Galiny.
\xx  Mam nadzieję, że uda im się.~--~odpowiedziała wyrwana z~zamyślenia kobieta.
\xx  Komu? Malarzom?~--~spytał Dima zaskoczony odpowiedzią.~--~Co za problem
pobiałkować ściany. Nawet największe cymbały spośród naszych nie są w~stanie
tego spierdolić.
\xx  Ach\3k~--~westchnęła kobieta.~--~Zamyśliłam się. Muzyka mistrza zawsze
wprawia mnie w~melancholijny nastrój. Myślałam o pierwszy dniu w~Zonie. O co
pytałeś?
\xx  Już o nic. Trzeba odświeżyć pokój, bo szwenda się tu masa robactwa. Wyłażą
dosłownie z~każdej dziury i~szczeliny. Jeden taki chuj spróbował nawet utkać
pajęczynę pomiędzy połami mojego płaszcza, ale gdzieś się ulotnił.~--~Zaśmiał
się Bandyta.~Bardzo ładna melodia, tylko strasznie smutna. Wagner?
\xx  Oj Dima. Słabą masz pamieć. Wagnera puszczałam ci przed szturmem na bazę
Powinności. Z adaptera. Do dziś liżą rany po tym ataku. Świetnie go
poprowadziłeś.
\xx  Taaa, stare dzieje.~--~rzucił krótko Tatarenko.~--~Ale wracając do rozmowy, o
kim wspominałaś wcześniej?
\xx  Wysłaliśmy do Czarnego Bagna oddział badawczy. Pamiętasz?
\xx  Jakoś słabo kojarzę.~--~stwierdził Dima.~--~Tyle się ostatnio działo, że
gdzieś mi to umknęło z~pamięci.
\xx  Miał wrócić cztery dni temu. Nie martwiłabym się tak bardzo, gdyby to był
dzień, dwa\3k Coś się musiało wydarzyć. Nikt do tej pory nie wrócił\3k
\xx  Jak dawno temu wyruszyli?
\xx  Jakieś dwa tygodnie temu. Puściłam za nimi Ducha, żeby trzymał rękę na
pulsie.
\xx  Są jakieś wieści od niego?
\xx  Kompletna cisza\3k Nie odezwał się.~--~powiedziała cicho Galina.
\xx  Trzeba tam kogoś wysłać. Za dużo zainwestowaliśmy w~tą operację. Dwa
kombinezony ochronne to kupa szmalu nie licząc broni i~osprzętu. Jutro poślemy
tam tropicieli.
\xx  Dobrze. Zorganizuje wyprawę, a ty tymczasem odpręż się. Doleję ci wina. \x
powiedziała Galina rozlewając resztę zawartości butelki do dwóch stojących na
stole kieliszków.
\qd \mm
Kobieta usiadła obok Dimy na miękkiej sofie i~przytuliła się do Bandyty
przykładając głowę do jego płaszcza. Płaszcza, który pachniał wonią śmierci, dymem papierosowym, tanią wódką i prochem.
\sx Wrócą.~--~Dima pogładził Galinę po włosach.~--~Musi się im udać. Dostali
najlepszy sprzęt.
\xx  Tak sądzisz?~--~spytała dziewczyna obejmując Bandytę.
\xx  Jestem tego pewien.~--~odparł Tatarenko.~--~Okolice Czarnego Bagna są prawie
wymarłe. Wątpię, by kogoś tam spotkali. Mutanty też się tam nie kręcą. Zbyt
łatwo wpadają w~te cholerne doły pełne czarnej mazi i~nie mogą potem z~nich
wyleźć. To je odstrasza. Nic, tylko szukać artefaktów i~modlić się, byśmy
znaleźli to, co szukamy. A~jeśli już to znajdziemy, to opuścimy wreszcie to
przeklęte miejsce!~--~Oczy Dimy zalśniły w~mroku niczym dwa diamenty.
\qm
Intymny nastrój panujący w~pokoju zmąciło pukanie do drzwi. Chwilę później, za
pozwoleniem Dimy, do pomieszczenia wszedł Bandyta zwany Bestią~--~jeden z
członków straży przybocznej Tatarenki strzegący drzwi do jego prywatnej kwatery.
\sx
Wróciła wyprawa z~Czarnych Bagien.~--~rzucił lakonicznie.
\xx  Jaki stan?~--~spytał Tatarenko.
\xx  Przeżył tylko Sztylet. Reszta nie żyje, napadły na nich pijawki koło starego
młyna w~drodze powrotnej. Wykończyły wszystkich. Kombinezony i~broń też
przepadły, bo Sztylet nie miał jak ich zabrać. Ledwo co uciekł. Klapa na całego.
\qm
Dima zasępił się. Wyprawa poniosła klęskę. Stracił ludzi i~sprzęt, co
zwiastowało kłopoty. Wielu czekało, aby powinęła mu się noga i~popełnił błąd.
Zdenerwowany wstał z~kanapy i~ruszył do drzwi. Kieliszki z~winem stały obok
siebie nie tknięte. Aura romantyzmu zniknęła, zastąpiła je szara rzeczywistość.
Bandyta wyszedł z~pomieszczenia i~zszedł po popękanych schodach do hali
fabrycznej. Na środku stał Sztylet. Wyglądał, jakby zaraz miał wyzionąć ducha.
Jego wygląd utwierdził Dimę, że Bestia miał rację. Wszystko poszło nie tak, jak
trzeba. Tatarenko podszedł bliżej. Tłum rozstąpił się wpuszczając go do środka
okręgu, który utworzył się dookoła przbyłego bandyty.
\sx Melduj.~--~powiedział Bandyta do słaniającego się na nogach mężczyzny.
\xx  Dotarliśmy do bagna bez problemów. W anomaliach nic nie znaleźliśmy, ale
kombinezony sprawiły się bardzo dobrze. W końcu uznaliśmy, że pora wracać, bo nic
nie znajdziemy. Nie mieliśmy farta.~--~wysapał Sztylet.~--~Wszystko szło dobrze
dopóki nie dotarliśmy do starego młyna\3k
\xx  Słyszałem. Pijawki was zaatakowały. Podobno zabiły resztę\3k Ile ich było?
\xx  Nie pamiętam, dwie, albo trzy. Wszystko się tak nagle potoczyło\3k
\xx  I nie byliście się w~stanie przed nimi obronić? Przecież było was tam
dwunastu. Nie mów mi, że rozwaliły was jakieś durne dwie pijawy. Nie wysłałem
was tam z~obrzynami, mieliście najlepszy sprzęt.
\xx  Zaskoczyły nas. Maczugę i~Waleta zabiły od razu, nie zdążyli nawet wyjąć
broni. Wywiązała się strzelanina, ale były cholernie szybkie. Zabiły Śledzia, a
Wasyla omal nie rozcięły na pół. Wykrwawił się biedaczek chwilę później. Potem z
młyna wypadły jeszcze dwie maszkary zwabione walką i~zapachem krwi.
\xx  A~Talib? Miał przecież RPK. Powinien rozsmarować to cholerstwo po ziemi!!!\x
powiedział wściekły Dima.
\xx  Zdążył zastrzelić jedną, ale druga nieomal urwała mu głowę. Nie mieliśmy
żadnych szans szefie\3k Wyrżnęły wszystkich w~pień. Tam musiało być ich gniazdo,
w tym starym młynie\3k
\xx  Dziwne\3k~--~powiedział zrezygnowany Dima.~--~Nigdy nie było tam mutantów.
Rozumiem, że sprzęt został tam na miejscu.
\xx  Tak. Nic nie zabrałem, bo nie miałem jak. Ledwo stamtąd spierdoliłem. Jeśli
go ktoś nie zabrał, to pewnie tam leży.
\xx Powiedz mi Sztylet, jak to się stało, że wszyscy nie żyją, a tobie się udało
zwiać? Bardzo mnie to zastanawia.
\xx  W trakcie walki jedna z~pijawek rzuciła się na mnie. Drasnęła mnie, wpadłem w
krzaki. Karabin mi gdzieś upadł, nie mogłem go znaleźć. Myślałem, że zaraz mnie
wykończy, ale rzuciła się na kogoś innego. Chwilę potem pijawki zabiły
wszystkich chłopaków. Nie mieli szans. W tym młynie było ich chyba całe stado.
Zacząłem się czołgać w~stronę Kordonu. Pijawki nie zwracały na mnie uwagi. Może
mnie nie widziały, nie wyczuły, nie wiem. Zaczęły konsumować zwłoki. Odczołgałem
się na bezpieczną odległość i~uciekłem stamtąd. Ledwo tu dotarłem, bo pistolet
też mi musiał gdzieś wypaść, jak czołgałem się przez te zarośla.
No cóż. Nie mam podstaw, by ci nie wierzyć. Nigdy mnie nie zawiodłeś. Szkoda
sprzętu, ale mówi się trudno. Nie opłaca się tam wracać, już za duże straty
ponieśliśmy. Może wytrujemy to dziadostwo, jak załatwię Sarin od Wojska, ale nie
jest to naszym priorytetem. Są ważniejsze sprawy, niż użeranie się z~pijawkami.~--~podsumował Tatarenko.
\xx  Szefie, a co z~zapłatą?~--~spytał zmęczony Sztylet.
\xx  Achh\3k Zapłata\3k Anomalie sprawdziliście, czyli kontrakt wykonaliście. Nie
twoja wina, że straciliśmy sprzęt. Stało się, jak się stało\3k Dostaniesz forsę
za wszystkich chłopaków, bo taka była umowa. Sześćdziesiąt patoli. Całkiem
sporo\3k Jesteś bogaty Sztylet, uśmiechnij się.~--~powiedział Dima.~--~Możesz
kupić sobie niezłego gnata lub pancerz, pobalować u Małpy lub Kapitana z
miesiąc, może nawet wybrać się gdzieś na dziewczynki\3k Przyjdź do mnie po
forsę, jak się już ogarniesz.~--~zakończył Dima.
\xx  Brawo!!! Co za piękna historia!~--~powiedział Duch klaszcząc w~dłonie.
Tłum rozstąpił się i~do wnętrza zgromadzonego wobec Dimy, Galiny i~Sztyleta,
okręgu wszedł jeden z~Bandytów niosąc w~ręce jutowy worek.
\xx  Byłbyś Sztylet
niezłym bajkopisarzem. Nie myślałeś o wydaniu książki?~--~dodał po chwili.
\xx  O czym ty Duch gadasz???~--~spytał zdziwiony Sztylet.
\xx  O czym ja gadam? Może byś opowiedział kolegą, co wydarzyło się koło młyna?\\
Tylko tą prawdziwą historię, a nie bajeczkę, którą przed chwilą zaserwowałeś
wszystkim.
\xx  Jak to?~--~Sztylet zbladł.~--~Było tak, jak mówię. Maczugę zabiła pijawka,
Waleta zresztą też\3k
\xx  Dość!~--~przerwał mu Duch wyciągając coś z~worka.
\qm
Powiedziawszy to uniósł przedmiot do góry. Była to ludzka głowa kogoś, kogo
Bandyci zwali Waletem.
\sx  Ludzka głowa, pchii.~--~wydął wargi Sztylet. I po co ją tu przyniosłeś? O
czym to ma świadczyć? Że Walet nie żyje? Skoro tak, to mam rację!
\xx  Tak, Walet nie żyje. Tylko ciekawe, dlaczego zamiast śladów macek, Walet ma
dziurę w~głowie po strzale z~pistoletu. Nie wiem, od kiedy pijawki umieją strzelać\3k
\qm
W sali zapadła niezręczna cisza. Sztylet wbił wzrok w~podłogę. Został
zdemaskowany.
\ro{CZĘŚĆ IV}
\mm Podczas, gdy Sztylet tępo wpatrywał się w~podłogę, do Dimy podszedł Duch i~przekazał mu na ucho jakieś informację. Po chwili rozpłynął się w~tłumie wraz z~niezręcznym pakunkiem, jakim była głowa Waleta. Nie był już potrzebny, wykonał swoje zadanie, a to, co miało nastąpić w~ciągu kolejnych minut, pozostało w~gestii jego szefa. Dima również nie miał zamiaru tracić reszty dnia na dywagacjach nad losem zdrajcy, ale po skonfrontowaniu tego, co powiedział mu Duch, a tym, co przed chwilą usłyszał od Sztyleta, stwierdził, że coś śmierdzi w~całej tej sprawie. Postanowił się tego dowiedzieć, zanim wyda winowajcę w~ręce plutonu egzekucyjnego.
\sx Coś mi kurwa nie pasuje w~całej tej historii\3k Powiedziałeś, że kombinezony były w~porządku, ale Duch nie odnalazł ich. Możesz powiedzieć, co się z~nimi stało? Zajebałeś~je?
\xx A~jakie to ma znaczenie? I tak za parę minut będę martwy. Kombinezony zostały na Bagnach. Chcecie, to sobie je odszukajcie. Przy okazji znajdziecie martwego Kościeja i~Magika. Pewnie Duch ci przekazał, że nie mógł odnaleźć ich ciał.
\xx Zgadza się. I mam wrażenie, że masz z~tym dużo wspólnego. Gadaj bo zmienię zdanie i~każę cię sprowadzić do lochów pod halą. Chyba nie chcesz tego?
\xx Prowadź!~--~powiedział hardo Sztylet.~--~Chuj mi zależy. I tak jestem martwy!
\xx Ehhh\3k~--~pokręcił głową zrezygnowany Tatarenko.~--~W ten sposób do niczego nie dojdziemy. Poza tym idzie emisja i~nie mam ochoty bawić się dzisiaj w~kata. Mam lepszy pomysł. Powiesz, czemu zabiłeś Waleta, albo każę zaprowadzić cię pod wioskę Stalkerów i~przywiązać do pierwszego lepszego drzewa. Na pewno szybko znajdziesz tam nowych kolegów. Zwłaszcza wśród tych, których skroiłeś, lub którym zabiłeś przyjaciół. Myślę, że moje zabawy z~obcęgami i~rozżarzonymi węglami będą niczym wobec tego, co cię tam czeka~--~powiedział Dima drwiącym głosem.~--~To jak? Idziemy na mały spacerek do Kordonu?
\xx Dima powiedz mi, co byś zrobił, gdyby ktoś chciał cię wydymać?~--~zapytał Sztylet.
\xx Zrobiłbym mu z~dupy jesień średniowiecza. Nie jestem dziwką, żeby ktoś mnie mógł bezkarnie dymać. Proste.
\xx Otóż to! Walet był sukinsynem, który nie tylko chciał wydymać ciebie, ale także mnie, członków wyprawy oraz wszystkich tu zgromadzonych. Dlatego dostał w~czapę. Tak, strzeliłem mu prosto w~łeb! Ale czemu to zrobiłem, nikt nie zapytał\3k
\xx W takim razie oświeć nas. Jesteśmy ciekawi, jakież to ciekawe wydarzenia na Bagnach ominęły nas.
\xx Dajcie wódki! Muszę się napić\3k
\xx Galina, idź na górę i~przynieś dwie flaszki~--~powiedział Dima do stojącej obok kobiety.
\xx Niech stracę. To niewielka cena za poznanie prawdy. I niech ktoś skombinuje mu krzesło. Czuję, że trochę potrwa, zanim się wszystkiego dowiemy.
\qm
Chwilę później Sztylet siedział na środku sali z~butelką alkoholu w~ręce. Nerwowo palił szluga zaciągając się nim łapczywie. Nic dziwnego, ostatni raz w~swoim życiu czuł w~płucach gryzący dym papierosowy. Papieros topniał z~każdym pociągnięciem, niczym świeca wystawiona na mękę płomienia. Gdzieś w~górze ktoś cicho szeptał. Może to Charon, który zapraszał go do swojej łodzi, a może to tylko ptaki gnieżdżące się na dachu starej hali oznajmiały sobie, że nadchodzi skąpany w~purpurze władca zony? Czy miało to jakieś znaczenie? Wątpił\3k Za parę, może paręnaście minut, jeśli przeciągnie historię, ujrzy swoje przeznaczenie. Kotły z~wrzącą smołą i~olejem, maszyny do łamania kołem i~zastępy czerwonych pomiotów lubieżnie torturujące nieszczęśników w~kakofonii wrzasków i~jęków. Piekło miało się stać jego drugim domem i~wiedział o tym. Nie na darmo ochrzczono go w~stalkerskim światku Sztylet. Czas umierać~--~pomyślał opróżniając w~dwóch łykach kolorowe szkło Kozaka. Alkohol rozluźnił zmęczone wyprawą
mięśnie, sprawił,
że czas stanął w~miejscu. Porcja płynu z~drugiej butelki spowodowała, że język rozwiązał się i~bandyta zaczął opowieść:\looseness+1

\dd
\textsl{Podróż na Bagna zajęła nam trzy dni. Dni, które minęły bezproblemowo, bo nie napotkaliśmy żadnych trudności po drodze. W radosnej atmosferze Magik i~Kościej rozpoczęli przeszukiwanie Czarnego Bagna z~nadzieją na znalezienie poszukiwanego artefaktu, ale efekt był mizerny. Cieszyło nas, że kombinezony sprawowały się świetnie. Bez problemu można było w~nich wchodzić w~Spalacze, Elektro i~inne anomalie. Chroniły też dobrze przed promieniowaniem, wskutek czego udało się nam eksplorować większą część bagien. Niestety, bezskutecznie. Nic nie znaleźliśmy. Na koniec drugiego dnia doszło do kłótni między Waletem, Maczugą, Magikiem i~Kościejem. Grozili sobie nawet bronią i~nieomal się pobili. Wtedy jeszcze nie wiedziałem, o co poszło. Następnego dnia Magik i~Kościej poszli ponownie przeczesywać anomalie. Ale nie wrócili na czas. Coś było nie tak\3k Wreszcie, po prawie dwóch godzinach, zjawili się. Słaniali się na nogach, co chwilę któryś wymiotował pianą. Zdjęliśmy im hełm i~podaliśmy cały zapas leków, jakie
mieliśmy.
Większość z~nich to były jakieś witaminy i~środki przeciwbólowe, które miał jeden z~chłopaków. Leków antyradiacyjnych mieliśmy tylko trzy porcje. Czyli tyle, co nic\3k Zaaplikowaliśmy im, co było, ale po to tylko chyba, żeby zabić sumienie, bo medykamenty, które posiadaliśmy, nie mogły im pomóc. Nie było dla nich ratunku, otrzymali zbyt dużą dawkę radiacji\3k Kościej zmarł po jakiś pięciu minutach po podaniu leków. Walet wraz z~Maczugą stwierdzili, że kombinezony musiały być trefne i~poszli zakopać gdzieś Kościeja, żeby, jak to powiedzieli, nie świecił nam pod nosem. Ja zostałem z~Magikiem, bo reszta postawiła na nim krzyżyk i~poszła pić. Stan Magika był fatalny. Bełkotał coś o jakimś kamieniu i~bramach czasu. Uznałem to za majaczenia konającego, więc nic się nie odzywałem. W międzyczasie wrócił Walet z~kompanem, ale byli jacyś wkurwieni. Zaraz przyszli do mnie i~pytali, czy Magik mówił coś ważnego i~czy czasem czegoś nie dał mi. Powiedziałem, że\3k}
%
\sx Sztylet, streszczaj się!~--~powiedział rozdrażniony Dima.~--~Nie mamy całego dnia na twoje opowiastki.
Emisja idzie do kurwy nędzy! Tak że do rzeczy! Pij tą flaszkę i~kończ spowiedź.
\xx Powiedziałem, że nic mi nie dał i~nie mówił nic sensownego\3k~--~podjął opowiadanie bandyta.\x Mogę kontynuować?
\xx Tak, tylko mniej pierdolenia Sztylet, a więcej faktów.
\xx Dobra. Dalej było tak:
\qm
\textsl{
Magik wytrzymał jeszcze godzinę, potem wykitował. Przed śmiercią dał mi jakąś mapę z~dziwnymi znakami, ale Bóg wie, co wskazywała. Wspominał też o filtrach, ale ciężko go było zrozumieć. Chwilę potem zjawił się Walet z~Maczugą i~znów truli mi dupę, czy Magik przed śmiercią mi coś dał lub czy coś mówił. Oczywiście skłamałem. Wkurwieni poszli zakopać go gdzieś na bagnach. Nie było ich z~dwie godziny. Wrócili jeszcze bardziej rozdrażnieni i~przestali z~kimkolwiek rozmawiać. Siedzieli z~daleka od wszystkich i~debatowali. Koło dwunastej w~nocy coś mnie tknęło. Dodałem dwa plus dwa i~nagle dostałem olśnienia. Wymknąłem się z~obozowiska i~poszedłem poszukać ciał Magika i~Kościeja. Znalazłem ich z~jakieś pięćset metrów od obozu\3k Nie zakopanych\3k Kombinezony mieli pocięte na wióry. Wtedy, w~mojej głowie, jak wielki neon, rozbłysnął napis: \mbox{FILTRY}. Sprawdziłem je. Bardzo zmyślnie ktoś je uszkodził. Na tyle mało, aby Magik i~Kościej mogli wrócić z~bagien, ale na tyle też skutecznie, żeby nie przeżyli tej
wyprawy. A
kto mógł to zrobić? Patrząc na pocięte kombinezony odpowiedź była tylko jedna: uczynni grabarze.
}
\sx Sugerujesz, że Walet z~koleżką popsuli celowo filtry?
\xx Nie sugeruję. Ja to wiem! Kombinezony działały przez dwa dni dobrze, a potem się popsuły? Idiotyzm\3k
\xx No dobra, a gdzie mapa?
\xx Do mapy dojdziemy~--~powiedział Sztylet pociągając łyka Kozaka.~--~Dajcie mi skończyć.
\xx Kończ w~takim razie, bo mało czasu zostało.
\qm
\textsl{No więc, wróciłem do obozu przed świtem. Wiedziałem, że mamy w~szeregach zdrajców, ale co mogłem zrobić. Otwarta konfrontacja skończyła by się masakrą. Postanowiłem udawać, że o niczym nie wiem i~niczego się nie domyślam. Ale Walet i~Maczuga zorientowali się, że coś wiem. Opuściliśmy obozowisko rano i~udaliśmy się do przepompowni, którą wyznaczyliśmy na następne miejsce odpoczynku. Całą drogę Walet i~Maczuga obserwowali mnie. Pod wieczór wiedziałem, że konfrontacja jest nieunikniona. Ale noc minęła spokojnie, choć nie przespałem ani minuty. Kolejnym miejscem odpoczynku miał być Stary Młyn. Wtedy byłem już pewny, że wszystko rozegra się w~nocy podczas postoju. Następnego dnia mieliśmy wejść do Kordonu. Jeśli do czegoś miało dojść, to tylko na Bagnach. I tak się stało, jak myślałem. Upilnowali mnie. Zasnąłem na dwadzieścia minut\3k Obudził mnie metaliczny posmak broni w~ustach i~szyderczy uśmiech Waleta, który wykrzywiał gębę niczym klaun cyrkowy}.%\looseness-1
%
\sx No, to jesteśmy w~domu~--~przerwał Dima.~--~Walet chciał ci zajebać mapę, a ty go odpaliłeś. Przy okazji Maczugę też usztywniłeś. Potem wpadły pijawki, zatłukły resztę, a ty spierdoliłeś z~mapą. Wróciłeś do obozu i~sprzedałeś nam historyjkę o stadzie pijawek, które was zaatakowało, żeby zgarnąć szmal i~przy okazji coś, co skitrał Magik. Sprytnie. Gdzie mapa? Duch jej nie znalazł.
\xx A~skąd mam wiedzieć? Uciekłem stamtąd, jak pojawiła się pijawka. Była tylko jedna, ale zdążyła na tyle skomplikować sytuację, że wyciągnąłem schowany na plecach pistolet i~wpakowałem Waletowi kulkę między oczy. Resztą zajęła się maszkara. Zabiła Maczugę, a potem pozostałych chłopaków. Walet zabrał im broń, jak spali. Zostały im tylko noże. Nie mieli szans\3k
\xx Czyli zostawiłeś bezbronnych kolegów w~gównie, a sam zwiałeś. Pięknie!
\xx Kolegów??? Powiedzieli, że mają gdzieś moje porachunki z~Waletem i~Maczugą i, że nie będą nadstawiać dupy dla jakiegoś papierka. Nie uwierzyli też, że zepsute kombinezony, to była ich sprawka. Byliśmy już prawie w~Kordonie, każdy myślał o forsie, jaka na niego czekała, a nie o komplikowaniu sobie życia.
\xx Ehhh, kurwa\3k Powinniście się na Bagnach pozabijać i~miałbym z~wami święty spokój\x podsumował Tatarenko.\x Co jeden z~was, to lepszy.
\qm
Po krótkiej chwili Dima podwinął rękaw swojego płaszcza
i~wyciągnął rękę. Na jej wewnętrznej stronie widniał
wytatuowany napis:

\centerline{\bf\em AVE CAESAR MORITURI TE
SALUTANT}
\sx
Czytaj!~--~powiedział stanowczym głosem szef Bandytów.
\xx Witaj Cezarze, idący na śmierć pozdrawiają cię.
\xx Wiesz, co to znaczy?
\xx Nie rób sobie jaj Dima. Mam taki sam napis. Jak każdy tutaj.
\xx Wiesz, czemu tatuujemy go każdemu, kto do nas dołącza? Bo stanowi to esencję naszego kodeksu. A~mówi on o tym, że nie zdradza i~nie oszukuje się kolegów. Dlatego każdy, ale to każdy z~nas, ma ten jebany napis! Zostawiłeś ich na pewną śmierć i~chciałeś nas wyrolować! Rozumiem też, że mapy nie masz, i~nie wiesz, gdzie jest.
\xx Możesz mnie torturować. I tak ci nie powiem, gdzie jest mapa, bo nie mam pojęcia, co się z~nią stało. Jeśli Duch jej nie znalazł, to widocznie ktoś w~międzyczasie musiał obrabował trupa Waleta.
\xx Może tak, a może nie. Nie będę wnikał. Mapa nie jest mi potrzebna do szczęścia. Kombinezony zostały zniszczone, broń zniknęła, a poszukiwanych artefaktów nie znaleźliście. Porażka na całej linii\3k Warto było?~--~dodał po chwili Dima.
\xx Nie wiem\3k~--~powiedział Sztylet dopijając Kozaka. \xx Nie miałem wyboru. Zostałem wplątany w~intrygę, w~której nie chciałem brać udziału. Walet chciał mnie zabić, dlatego zginął. A~chłopaki\3k Cóż, powiedziałem im o całej sytuacje, ale olali mnie. Może im Walet coś obiecał, nie wiem. A~może nie chcieli kłopotów, ciężko stwierdzić. Sami są sobie winni. Walet też ich wystawił, zabierając im broń, ale skąd mieli wiedzieć, że tak się sytuacja skomplikuje.
\xx Nie mogłeś powiedzieć od razu, co się wydarzyło na Bagnach? Miałbyś chociaż rozprawę.
\xx I co by mi to dało?~--~spytał drwiąco Sztylet.~--~W najlepszym wypadku ugrałbym konfiskatę mienia i~deportację z~Zony, ale podejrzewam, że nie zdążyłbym z~niej wyjechać. Walet i~Maczuga mieli dużo kolegów\3k Pewnie zginąłbym, zanim dojechałbym do Kordonu.
\xx Być może, ale ocaliłbyś życie. Okłamując nas wybrałeś śmierć.
\xx W taki razie kończmy ten cyrk. Zwołaj pluton egzekucyjny i~miejmy to już za sobą~--~powiedział Sztylet wstając z~krzesła.~--~Jako skazany na śmierć mam prawo do jednego życzenia zgodnie z~kodeksem Bandytów. Czyż nie, Dima?
\xx Owszem. Pod warunkiem, że będzie to coś rozsądnego\3k
\xx W takim razie proszę o komisyjne otworzenie po śmierci mojej skrytki i~sprzedanie całej jej zawartości po aktualnych cenach handlowych. Zebrane w~ten sposób pieniądze przekażcie na konto mojego syna, Igora. Leży na oddziale onkologii szpitala Św. Łukasza w~Mińsku~--~dokończył ze łzami w~oczach Sztylet.
\qm
Dima podrapał się po głowie. Takiego zakończenia nie spodziewał się. O różne rzeczy proszono przed śmiercią, ale to życzenie zaskoczyło go. Zasadniczo nie miał nic przeciwko temu. Rzeczy nie były kradzione. Sztylet sam na nie zarobił swoją ciężką pracą. Postanowił jeszcze o coś spytać.
\sx
Nie mówiłeś, że masz syna\3k Podejrzewam, że przez to skomplikowałeś tak sobie życie. Zgadza się?
\xx Tak. Nie miałem wyboru. Trzeba było szybko działać. Cały plan zrodził się właściwie w~momencie, kiedy zaatakowała nas pijawka. Postawiłem wszystko na jedną kartę. Widocznie na nie właściwą. Skąd miałem wiedzieć, że w~talii, którą grasz, jest joker, który zwie się Duch?
\xx Nie wiedziałeś. Zresztą ja też nie. Galina go wysłała. Wyroku nie cofnę, oszukałeś nas. Natomiast cały twój sprzęt zostanie sprzedany zgodnie z~twoim życzeniem. Co prawda niechętnie, ale zgodnie z~kodeksem Bandytów, skazuję cię na śmierć przez rozstrzelanie za oszustwo i~nie udzielenie pomocy kolegom. Straże! Przygotować miejsce stracenia!
\qm
Sztylet usiadł na krześle. Pusta butelka po alkoholu wypadła mu z~ręki roztrzaskując się o szarość podłogi. Emocje opadły, zgromadzony dookoła widowiska tłum rozszedł się w~harmidrze komentarzy i~spekulacji. Zamroczony wódką zamyślił się. Przed oczami ujrzał wychudzoną twarz Igora walczącego każdego dnia o życie. Jak w~mantrze, powtarzał cały czas:
\sd
\xx Igorku, starałem się jak mogłem, ale wszystko spieprzyłem. Wybacz mi\3k
\qd
Lecz syn nie reagował. Z każdą chwilą stawał się coraz mniej widoczny niczym znikający duch. Po kilku chwilach rozpłynął się całkowicie w~powietrzu, a przed oczami Sztyleta pozostało tylko puste łóżko.
\sx
Nie!!!~--~krzyknął zrozpaczony mężczyzna.~--~Nie odchodź!!!~--~błagał Bandyta.
\xx Ogarnij się Sztylet~--~powiedział Bestia łapiąc go obiema rękami za głowę.~--~Czas na ciebie. Może jednak dziś nie umrzesz\3k Chociaż wątpię, czy nowa kara będzie mniej bolesna\3k Galina ubłagała Dimę, żeby odwołał pluton egzekucyjny. Wstawaj, idziemy do starych zbiorników wyrównawczych. Tam dowiesz się wszystkiego.
\qm
Pięć minut później Sztylet stał na krawędzi głębokiego na sześć metrów żelbetowego basenu. Dookoła zbiornika piętrzył się zgromadzony tłum, który nie bardzo wiedział, czego oczekiwać po zaistniałej sytuacji. Chwilę później nadszedł Dima w~towarzystwie Galiny. Ruchem ręki uciszył zebranych i~rzekł:
\sx
Za namową mojej kobiety postanowiłem dać ci szansę~--~powiedział do stojącego nad przepaścią Sztyleta. Galina poradziła mi, abym zastosował dawno zapomniane prawo, a mianowicie Kodeks Hammurabiego. Sądzę, że lepsze to, niż śmierć przez rozstrzelanie. Wiesz, o co w~nim chodzi?
\xx Mniej więcej\3k Oko za oko, ząb za ząb. Coś w~tym stylu\3k
\xx Dokładnie. W rurze doprowadzającej wodę do zbiorników mieszka pijawka. Wpadła do basenów parę dni temu podczas nocnej wizyty. Zabić jej nie ma jak, bo nie chce cholera wyjść na zewnątrz. Uciekła do kanałów i~żywi się tam szczurami. I będzie sobie tak wegetować, aż znajdzie sposób, żeby stamtąd zwiać. A~wtedy\3k Wtedy będziemy mieli przesrane.
\xx Rozumiem, że mam tam zejść i~ją zabić? Chyba ocipieliście!
\xx Dokładnie. Masz nóż przy sobie, czyli to samo, co mieli chłopaki, kiedy zaatakowała ich pijawka. Myślę, że to sprawiedliwe. Oko za oko, ząb za ząb\3k
\xx Skurwiel z~ciebie Dima~--~wymamrotał z~posępną miną Sztylet.~--~A jeśli się nie zgodzę?~--~dodał po chwili.
\xx Chyba nie masz wyboru.~--~Przewrócił oczami szef Bandytów i~wskazał majaczącą się w~oddali ścianę, pod którą dokonywano egzekucji. \xx Żeby cię zmotywować dam ci te sześćdziesiąt tysięcy rubli jeśli zabijesz pijawkę. Sam zawieziesz forsę do Mińska. O ile uda ci się przeżyć. Jeśli nie, zawiezie ją ktoś inny.
\qm
Sztylet nie odezwał się nic. Sprawa była przesądzoną. Obrócił się w~stronę basenu i~zaczął schodzić w~dół po zardzewiałej drabince. Tymczasem na górze wrzało. Bandyta zwany Dolarem zbierał zakłady, kto zwycięży. Za pijawkę płacił dwa do jednego. Za Sztyleta dwanaście. Na tyle wycenił szansę przeżycia swojego kompana. Chwilę później skazaniec znalazł się na dnie zbiornika. Ku ogólnemu zdziwieniu nie ruszył jednak w~stronę kanału wlotowego, gdzie mieszkał mutant, lecz stał oparty plecami o drabinkę i~czekał. Minuty mijały, a tłum darł się z~niezadowolenia. Pijawka również nie wykazywała chęci opuszczenia swojego nowego legowiska.
\sx
Co to kurwa szachy, czy co!? Niech że ruszy dupsko i~idzie ją zabić! Zaraz będzie Emisja, a ten ciul stoi pod drabiną i~czeka na Bóg wie co~--~zaklął zdenerwowany Dima.
\xx Zrobiłbym tak samo~--~odparł stojący obok Duch.
\xx Nie rozumiem\3k
\xx Widzisz Dima, tam w~kanale, Sztylet nie ma żadnych szans. Dlatego czeka, aż potwór wyjdzie na zewnątrz. I doczeka się, bo pijawka musi zaatakować. Zbyt długo żywiła się szczurami, żeby przeżyć kolejne dni. Jeśli za długo będzie zwlekać, to zamiast obiadu dostanie popcorn. Emisja idzie\3k
\xx Sprytnie~--~przytaknął Dima.~--~Zatem czekamy.
\qm
I istotnie, po chwili, z~głębi kanału doprowadzającego wodę do zbiorników, doleciał przerażający ryk. Parę sekund później oczom zgromadzonych nad basenem ukazał się pijawka. Powoli zbliżała się w~stronę drabiny. Sztylet również zauważył ją i~przygotował się na atak. Dima obserwował wszystko z~góry. Dokładnie widział, jak mutant znika, by uderzyć znienacka. Parę sekund później Sztylet obficie krwawił. Odparł atak, ale łapa pijawki rozorała mu paskudnie ramię. Nim ochłonął, nastąpił drugi, który zdewastował mu dłoń. Sztylet spojrzał bezradnie na drabinkę. Z takimi ranami nie miał żadnych szans na powrót na górę. Pozostało walczyć.
\sx
Tylko jak?~--~pytał sam siebie oglądając poranione kończyny.\x Ledwo trzymam ten pieprzony nóż. Następny atak mnie zabije\3k Myśl!~--~powtarzał sobie.~--~Musi być jakiś sposób, żeby ją zabić.
\qm
Podczas, gdy pijawka szykowała się do ostatniego ataku, Sztylet ruszył w~stronę wielkiego zaworu wystającego ze ściany dzielącej basen. Pokrętło to opuszczało i~podnosiło grodź, która rozdzielała rezerwuar na pół. Kiedy znalazł się przy zaworze, zacisnął zęby i~szarpnął kołem. Ani drgnęło. Słyszał jak bestia przedziera się przez sterty mebli i~elektro śmieci, lecz nic nie mógł zrobić. Cholerny zawór zardzewiał do reszty. Zdesperowany odrzucił nóź i~wziął z~posadzki kawałek metalowej rury. Używając jej jako dźwigni opuszczał stopniowo grodź, lecz tempo, z~jakim to robił, przypominało ruchy ślimaka. Tłum zamarł. Obserwował w~milczeniu wysiłki byłego towarzysza, lecz nikt nie wiedział, po co Sztylet opuszcza stalową płytę. Zasuwa była jakieś pół metra nad ziemią, kiedy pijawka przedarła się przez ostatnie zwały złomu. Sztylet obrócił się. Dziesięć metrów dzieliło go od zipiącej kreatury. Naparł z~całej sił na rurę. Zawór jęknął i~zrobił pół obrotu, które kosztowały go życie. Grodź z~łoskotem opadła na dół
zagłuszając częściowo jęki syreny oznajmiającej Emisję. Parę chwil później zwierzę szybko wyssało krew z~rannego bandyty i~ruszyło w~stronę swojego lokum w~kanale, by schronić się przed nadciągającą zagładą. Lecz drogę odwrotu blokowało masywne cielsko stalowej płyty, które tak mozolnie opuszczał jej niedawny posiłek. Została uwięziona.

\dd
Na krawędzi zbiornika stał Dima. Samotnie obserwował jak monstrum pozbawiało życia jednego z~jego najlepszych ludzi. Jednak nie kłamał~--~pomyślał szef Bandytów. Ostatni raz spojrzał na wymięte implozją ciało Sztyleta, po czym przeniósł wzrok na pijawkę. Bezradnie tłukła łapami o pokrytą strupami rdzy stal. Mutant wyczuł, że ktoś go obserwuje. Wściekły spojrzał w~górę.
\sd
\xx Szach i~mat!~--~rzucił Tatarenko do uwięzionego zwierzęcia, po czym pobiegł przez pokryty błotem dziedziniec w~stronę najbliższych zabudowań.
\qm
Ścigany okrutnym rykiem konającej pijawki schronił się w~mordowni zwanej Zamulonym Sterem. Pół minuty później nastąpiła Emisja\3k
\begin{center}
KONIEC
\end{center}
\end{document}