\documentclass[../MAIN.tex]{subfiles}
\begin{document}
Trzecia wieczorem. Piszę samotnie w ciszy rozdartej blaskiem
świecy. W pokoju pachnie cynamonem i czarnym, jak noc
atramentem, który nie chce sklejać się w żaden sensowny tekst.
\\
Słodki zapach łechta mile zmysł powonienia i sprawia, że czuje
się głodny na myśl o plackach, które kilka godzin temu jadłem.
Po kolacji pozostał tylko ślad w formie porosłych tłuszczem
talerzy i butelki czerwonego wina~--~sprawcy plam na ostatnich
kartkach pergaminu, jakie posiadam. Papier i stół pokryty
kroplami taniego bożole wygląda niczym sceneria do kiepskiego
filmu grozy, ale nie dbam o to. Posprzątam \mbox{później}.

Najważniejsze, by przyszła Erato. Tylko na nią czekam. Tylko
jej dziś pragnę. Niestety, Erato zawiodła mnie. Zamiast niej
pojawił się na parapecie spasiony kot barwy węgla, który
miauczeniem zaczął domagać się, a bym go wpuścił do środka.
Parszywe bydlę~--~pomyślałem. Będzie mi koncerty pod oknem
urządzał. Spróbowałem zignorować nocnego gościa, lecz zwierze
było metodyczne. Niczym zacięta płyta gramofonu powtarzało:
miau! kręcąc się po parapecie. Zirytowany otwarłem okno i
spojrzałem w ciemność. Chłostane wiatrem wierzby pochylały się
na boki i garbiły pod strugami wody cieknącej z nieba. Chodź
przyjacielu~--~powiedziałem, wpuszczając kota do pokoju.
Zwierzęcia nie trzeba było zapraszać, szybko wślizgnęło się do
pomieszczenia. Zamknąłem delikatnie butwiejące okiennice bojąc
się, by nie rozsypały się w pył, po czym wróciłem do pisania
zostawiając gościa własnemu losowi. Minuty mijały, a wena nie
przychodziła. W pół do czwartej byłem przekonany, że mój list
miłosny nie zostanie ukończony dzisiejszej nocy. Nic, będę się
musiał wstrzymać jeszcze trochę ze swoimi
amorami~--~pomyślałem, zwijając w rulon płachty pergaminu.
Schowałem papier, tusz i gęsie pióro do szuflady, zgasiłem
świecę, po czym ruszyłem po omacku w stronę sprężynowego łóżka.
Chwilę później ległem skonany na wygniecionym materacu. Obok
mnie spał futrzany osobnik, o którym kompletnie zapomniałem.
Ciekawe, jak cię zwą koleżko?~--~zamyśliłem się i pogłaskałem
kota. Ku mojemu zdziwieniu, pod palcami, wyczułem coś twardego.
Obroża? Dziwne, nie zauważyłem jej wcześniej\3k Wyciągnąłem z
kieszeni spodni PDA i skierowałem emitowany przez urządzenie
snop światła w stronę zwierzęcia. Istotnie, kot miał obrożę.
Ba, była do niej nawet dołączona przywieszka. Poświeciłem
bliżej. Na metalowej blaszce widniał napis. Zbliżyłem
urządzenie na tyle blisko, by móc odczytać inskrypcję, i
przeliterowałem:

M~--~E~--~F~--~I~--~S~--~T~--~O~--~T~--~E~--~L~--~E~--~S

W gardle zaschło mi z wrażenia, a żołądek zaczęła wypełniać
wielka gula, która rosła z zatrważającą szybkością. Poczułem
się tak, jakby ktoś uderzył mnie w głowę gigantycznych
rozmiarów młotem. Wróciły wspomnienia sprzed dwóch lat. Przed
oczami, jak w kalejdoskopie, przesuwały się kolejne obrazy z
przeszłości: ogromne granity, rozbite PDA, wykrzywione
szatańskim grymasem twarze i światła Hangaru w oddali~--~tak
bliskie, a zarazem tak dalekie.

Tej nocy Zona oddychała ciężko i niespokojnie. Śniły się jej
koszmary, w których królowały demony, a my postanowiliśmy ją
obudzić wkraczając w jej najbardziej intymne tereny.
Zapłaciliśmy za to najwyższą cenę\3k Czerwony Las nas
styranizował i zdeptał. Wycieńczeni, ścigani przez nocne sługi
zła, uciekaliśmy na oślep, aż natrafiliśmy na zwałowisko
kamieni, które miało być naszym schronieniem. Dla Witka okazało
się grobem, nie przeżył tej nocy\3k Zmarł wewnątrz niewielkiej
komnaty, która utworzyła się pod spiętrzonymi głazami. To pod
nie wczołgaliśmy się próbując ocalić życie. W objęciach skał
postanowiliśmy doczekać dnia. Szybko okazało się jednak, że
Witek odniósł poważniejsze rany, niż nam się na początku
wydawało. Obandażowałem zranione miejsca, podałem leki, lecz
nic nie mogło zatrzymać infekcji. Nie mieliśmy odtrutki na
wprowadzony do organizmu wirus. Nikt jej nie miał. Wątpię, czy
nawet istniała. Poczułem się bezsilny. Postanowiliśmy wezwać
pomoc przez PDA. Bezskutecznie. Moje urządzenie gdzieś
zniknęło. Aparat kolegi, choć funkcjonował, był bezużyteczny.
Potrzaskany, ledwo reagował na bodźce i co chwilę zawieszał się
lub gubił obraz. Witek zamachnął się i rzucił nim ze złością o
ścianę. Siedzieliśmy tak po ciemku w milczeniu, zrezygnowani,
zmarznięci i pokonani. Za kilka godzin miał się rozegrać
dramat, o którym nawet nie chcieliśmy~myśleć.

\emph{„Niech fantastycznie lutnia nastrojona\\
Wtóruje myśli posępnej i ciemnej;\\
Bom oto wstąpił w grób Agamemnona,\\
I siedzę cichy w kopule podziemnej,\\
Co krwią Atrydów zwalana okrutną.\\
Serce zasnęło, lecz śni.~--~Jak mi smutno!”}

Powiedział Tokarczuk, przerywając posępną ciszę. Uśmiechnąłem
się i spojrzałem w stro\-nę niewielkiego otworu pomiędzy
kamieniami, przez który wpadał do wnętrza jaskini blask
księżyca.

\sx Niezwykle trafne określenie sytuacji~--~skomentowałem
dobrze
mi znany fragment.~--~Szkoda, że cytujesz go w tak fatalnym dla
nas położeniu.
\xx Poezja nie wybiera czasu, ani miejsca. Po prostu jest,
towarzyszy nam każdego dnia. Pamiętasz, jak się poznaliśmy?
Jakimi słowami mnie przywitałeś?
\xx Pamiętam~--~odpowiedziałem,~przypominając sobie wiosenny
wieczór 1988 roku.
\qd
\vspace{-.6em}
W tle Wawel, przed nami rozpływające się w
tafli Wisły, niczym impresjonistyczne malunki, światła lamp
biegnących wzdłuż Konopnickiej. I my: ja, Witek, Marzena i
Basia. Dwie pary zakochanych, które obrały sobie za miejsce
romantycznej randki tą samą ławkę w parku pod siedzibą królów.
Po krótkim zapoznaniu piliśmy wspólnie bułgarską Sofię i
dyskutowaliśmy nad sensem życia wpatrując się w migoczącą
złotem wodę. Marzena i Basia dawno nas zostawiły, ale przyjaźń
z Witkiem przetrwała. Razem zdawaliśmy na Uniwersytet
Jagielloński, razem zarywaliśmy noce i szliśmy na wykłady
pijani jazzem, który łapczywie chłonęliśmy w Jamie Michalika.
Wspólnie czytaliśmy Grę w klasy Cortazara, wspólnie omawialiśmy
nieskończoną ilość razy Fausta. Wspólnie wyjechaliśmy też do
Czarnobyla po zakończeniu studiów. I to był chyba największy
błąd. Gdyby nasze drogi rozeszły się, może Witek by żył do tej
pory. Może\3k

By zapobiec transformacji mój przyjaciel strzelił sobie w głowę
o trzeciej nad ranem. Pamiętam tą scenę do dziś. Zmarł z
uśmiechem na ustach wypowiadając przed śmiercią swe ulubione
słowa: \emph{"Trwaj chwilo, jesteś piękna"} . Takiego właśnie
zapamiętałem Witka. Uśmiechniętego chłopaka, który nigdy się
nie poddawał i zawsze czerpał z życia garściami.

Retrospekcja skończyła się niczym taśma filmowa zerwana nagłym
pociągnięciem ze szpuli. Zaspanym wzrokiem omiotłem pokój.
Czarnego jak węgiel kota nigdzie nie było. Przede mną stała
Anuszka, moja miłość, i budziła mnie.
%
\sx Wstawaj Bogdan. Walerian cię wzywa. Ma dla ciebie jakieś
nowe
zadanie.
\xx Pakuj się, wyjeżdżamy stąd! Śnił mi się
Witek~--~powiedziałem
roztrzęsiony.
\xx Ale gdzie chcesz jechać?~--~zapytała zdumiona kobieta.
\xx Nie ważne. Byle dalej od tego przeklętego miejsca. Tu czeka
na nas tylko śmierć\3k
\qd
Pół godziny później mijaliśmy ostatnie posterunki w Kordonie
skryci pod plandeką wojskowego samochodu. Obejmując czulę
Anuszkę byłem pewien, że sen, który mi się przyśnił, był
znakiem od Witka. Parę dni potem dowiedziałem się, że wioska, w
której mieszkałem, została zrównana z ziemią. Odwiedzili ją
Królowie Nocy\3k
\end{document}