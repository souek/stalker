\documentclass[../MAIN.tex]{subfiles}
\begin{document}
Woron potknął się i upadł. Po tysiąckroć spalony pył, uniósł się dookoła jego ciała ze spragnionej ziemi, tworząc czarne kłęby. Momentalnie eksplodowała fala gorąca, rozpalając atmosferę i plując ogniem na wszystkie strony. Ciemny, ciasny zaułek między starym murem a zrujnowaną ścianą budynku gospodarczego rozbłysnął nieoczekiwaną czerwienią, której to sam widok powodował ból w oczach. Natychmiast zahuczało od masy płomieni wyrzucanej w chłodne, nocne powietrze. A spośród tego wszystkiego przeraźliwy krzyk i syk palonego kombinezonu.

Zaobserwowana przez nas regularność z jaką anomalia nasilała i łagodziła swą aktywność musiała być błędna. Tyle czasu zmarnowaliśmy na czekanie, aż owe zjawisko wytraci swą moc. Byłem już po drugiej stronie, ale Woron wpadł w sam środek tego piekła, w którym sam Szatan nie powstydziłby się zamieszkać. Czułem to ciepło. Poszarpany, stary Ekolog stanowił tylko namiastkę ochrony, jaką pierwotnie zapewniał. Przez rozszczelnione otwory wdawał się straszliwy żar, który palił skórę i powoli rozlewał się po całym wnętrzu. Gotowałem się. Pot z twarzy i wydychanego powietrza w okamgnieniu zamienił się w parę, która pokryła cała szybkę. Piekącymi oczyma widziałem wielkie słupy ognia materializujące się metr nad ziemią, a strzelające wysoko w górę niczym sztuczne ognie, a tuż pod nimi majaczące zarysy Worona, który czołgał się w moją stronę.

Niewiele myśląc, rzuciłem się prosto pod żywioł. Dałem nura między płomieniami, byle tylko dosięgnąć Worona.

Potworny ból.

Czułem jak moja skóra, marszczy się i rozrywa się atakowana wrzącym powietrzem. Mój plecak stanął w płomieniach, podobnie jak inne łatwopalne elementy jakie miałem na sobie. Czułem swąd spalenizny~--~mojego własnego ciała, które topi się i krzyczy. A potem twardy upadek na ziemię tuż pod szalejącą pożogą. Tu było epicentrum~--~największy żar niczym we wnętrzu ogniska. Czułem się jak kiełbasa, którą ktoś włożył w ów żar, który wyciska z niej soki, sprawia, że pęka i marszczy się.

Tak, to idealne porównanie.

Kątem oka dostrzegłem rękę Worona. Dalej nie wytrzymałem i zamknąłem oczy. Zdałem się na refleks. Brakowało mi prawej rękawicy, to też z całej siły wyciągnąłem lewą rękę i dosięgłem dłoni towarzysza. Uścisnął mnie mocno. Widać, że nie zamierzał się poddać – to dodało mi sił. Szarpnąłem najmocniej jak umiałem, potem znowu. Szybko wyciągnąłem go spod szalejącej anomalii i rzuciłem pod mur. Póki jeszcze miałem siły, zdjąłem płonący plecak i cisnąłem go daleko od siebie.

Powietrze dookoła nas było gorące, ale w porównaniu z żarem, z którego uciekliśmy, było niczym powiew bryzy. Upadłem tuż koło Worona. Cały czas krzyczał. Żal mi było chłopaka. Znałem go ledwo dwa miesiące, ale w Zonie to wystarczy, by nabrać zaufania. Nigdy nie przypuszczałem, że skończy się to w ten sposób. Uwięzieni w martwej strefie między gołymi ścianami, a rozszalałą anomalią, której nijak nie dało się już wyłączyć. Gdzieś nad nami unosiły się grawitacyjne. Widać było jak wzbudzony pył faluje nienaturalnie pod ich wpływem. Natomiast na drugim końcu ciasnego przesmyku ukrywał się nasz cel, schowany gdzieś między rozrzuconą w bezładzie stertą cegieł.
\sd
\xx Ojciec! Ojciec! – wołał mnie. Tak mnie tu zwali, choć księdzem nie byłem, ani też dzieci nie miałem.
\xx Dasz radę? – krzyknąłem.
\qd
Nie odpowiadał. Nie jęczał już, tylko leżał spokojnie, zupełnie bezwładny. Jego kombinezon był równie podniszczony co mój, a żar wywarł w nim dodatkowe zniszczenia. Czuć było jaki jest nagrzany, a roztopione wnętrze rozlewa się po ciele Worona, klejąc się do skóry. – Nie poddawaj się. Jesteśmy już blisko.

Obróciłem go na plecy. Z szokiem stwierdziłem, iż czarna od ognia szybka hełmu była pęknięta.
\sd
\xx Jasna cholera, nie umieraj mi tu tylko!
\qd
Chwyciłem za zaczepy jego hełmu. Nie było syknięcia, ani niczego~--~zupełnie się rozhermetyzował. Zdjąłem go powoli, uważając na głowę towarzysza. Widok przyprawił mnie o mdłości. Twarz była cała we krwi, wydzierającej się spod czarnej, popękanej skóry. Trząsł się, z trudem otwierał usta. Czułem okropny swąd palonych włosów, które zupełnie zniknęły z jego głowy. Woron, cały czas mocno zaciskał oczy, lecz po chwili próśb otworzył je. Były czerwone, ale reagowały na bodźce, a wiec jeszcze mnie widział. Coś jednak było nie tak. Znikł gdzieś blask w jego oczach, patrzył na mnie żałośnie, a każde mrugnięcie sprawiało mu ból.\looseness-1

Gdybym tylko mógł mu teraz jakoś pomóc. Wszystko co miałem spłonęło razem z plecakiem. Wszelkie leki, maści\3k
\sd
\xx Nie uda się nam\3k~--~wybełkotał, dławiąc się i plując krwią.
\xx Nie pierdol, kurwa. Jesteśmy już tak blisko. Sam przecież tak srałeś się tym artefaktem. Jebane ruble. Jebany kryształ! Spodobały ci się tak bajki o jego mocy, to teraz masz!
\xx Spierdoliłem. Wybacz.
\xx Ech\3k. Ale, kurwa~--~spojrzałem na buchające za nami płomienie. – Jesteśmy w dupie. I jak się teraz stąd wydostaniemy?
\xx Nie w\3k Nie wiem.
\qd
Spoglądałem na płomienie. Woron też zwrócił w tamtą stronę głowę. A może sama mu opadła? Niekończąca się fontanna ognia wznosiła się wysoko, rozświetlając okolicę niczym latarnia.
\sd
\xx Nie zgaśnie – wymamrotał.
\xx Że co?
\xx Nie zgaśnie. Spłoniemy tu.
\xx Co ty gadasz? – Chciałem go pokrzepić, ale chyba rozumiałem, co miał na myśli. Aktywował anomalię, wyjątkowo silą w dodatku. Spalacz będzie płonął godzinami, tak jak wtedy, gdy ostatnim razem aktywowaliśmy go dla testów. Spełniły się więc nasze najgorsze obawy. Zostaliśmy tu uwięzieni.
\xx To koniec, stary. Po\3k po nas. kurwa.
\xx Nie łam się, wyciągnę nas stąd jakoś. Tylko mi tu, kurwa, nie umieraj.
\qd
\mm Złapałem go i usadziłem w pozycji siedzącej, by nie krztusił się własną krwią. Podniósł, trzęsącą się rękę i wytarł sobie usta. „Może jednak nie jest z nim tak źle”, pomyślałem z nadzieją. Mimo to sam zaczynałem wątpić w powodzenie naszego zadania. Ruszyłbym dalej sam, ale przecież nie zostawię go tutaj. Ja nie z tych. Nie bez powodu noszę ten pseudonim. Żaden kryształ nie jest wart więcej niż prawdziwy stalker. Weszliśmy tu we dwóch i we dwóch stąd wyjdziemy. Zawsze jest jakaś nadzieja. Zawsze jest jakieś wyjście.
Tylko gdzie?

Wstałem. Spalone ciało, sprawiało mi straszy ból, a przecież byłem w anomalii zaledwie przez kilka sekund. Mogłem sobie tylko wyobrażać jakie katusze przezywa teraz Woron. Zacząłem rozglądać się za czymkolwiek, co mogłoby pomóc przyjacielowi.
\sd
\xx Robi się go\3k gorąco.
\qd
Miał rację. Anomalia bez przerwy nagrzewała powietrze. Gotowałem się niemalże jak wtedy, gdy skoczyłem do jej wnętrza. Szybko odciągnąłem przyjaciela kilka metrów dalej. Bałem się jednak iść głębiej. To był jakiś rodzaj symbiontu, Bóg jeden wie jakie anomalie czają się dalej w głębi zaułka.
\sd
\xx Ojciec, ja nie chcę umierać – jęczał Woron, dysząc ciężko.
\xx Spokojnie, nie umrzesz.
\qd
Naturalnie, starałem się go pocieszyć. Ale jak? Mam kłamać? Co tu wiele mówić, sytuacja była przejebana. Jeśli nie uciekniemy stąd w ciągu kilku minut, ugotujemy się żywcem. Zrobiłem kilka kroków dalej, zbliżając się do stery cegieł. Poczułem ciepło. Przed nami była kolejna anomalia ogniowa, czająca się gdzieś tam ukryciu. Stanąłem jak wryty w obawie, by i jej nie aktywować. Sięgnąłem po Echo, ale najwyraźniej aparatura została uszkodzona. Najwyraźniej jak już się ma pierdolić, to wszystko na raz.
 Wróciłem do Worona. Było z nim coraz marniej. Długo nie pociągnie bez opieki, to pewne. A ja nie wiem jak mogę go uratować.
\sd
\xx Nie dam rady\3k~--~zakasłał.~--~Ojciec. Mój kombinezon to wióry. To pył i\3k i wióry. Ty\3k Twój chyba jest ok?
\xx To już żadna ochrona. Na wiele się nie przyda.
\xx Ojciec\3k Ojciec\3k
\qd
Przerażało mnie to. Czułem, że sam zaczynam panikować. O co mu chodziło? „Ojciec, ojciec”. Twarz mu drżała, z całych sił mrużył oczy. Czułem się niepewnie. Czego ode mnie oczekiwał? Przecież nie teleportuje go stąd. Niech się w końcu zamknie!
\sd
\xx Nie pozwolisz mi umrzeć? Nie zostawisz mnie tu?
\qd
Strasznie ciężko przychodziło mu dobieranie słow. Wyczuwałem w tym strach, niepewność. Nigdy wcześniej nie czułem tego w jego głosie. Zawsze był żwawy, pewny siebie. To było straszne. Jak gdyby to nie był on, a ktoś inny. Ktoś\3k obcy.
\sd
\xx Nie zostawię. Czemu w ogóle męczysz mnie o to?
\xx Ty masz sprawny kombinezon. Nie chcę tu umrzeć. Nie\3k chcę.
\qd
Czemu on to ciągle powtarzał? Kim on był, ten poparzony człowiek? Czy ja go znałem? Czy to nadal był ten sam Woron?
\sd
\xx Kurwa, no! – Nie wytrzymałem.~--~Czy ja się stad gdzieś ruszam? Na litość Boską, zamknij się wreszcie i pozwól mi myśleć.
\qd
Uspokoił się trochę. Obrócił głowę w bok. I dobrze! Sięgnąłem po swego Mossberga 590. Nie miałem zamiaru słuchać tego dalej. A jęcz sobie do woli! Jęcz, płacz. Tylko pozwól mi działać albo za chwilę z nas obu zostaną tylko prochy!

Strzeliłem w mur przede mną. Śrut rozbił się o twardą cegłę, wyrzucając w powietrze masę odłamków i pyłu. Strzeliłem ponownie. I jeszcze raz.

Znienacka wielki płomień buchnął mi prosto w twarz. Upadłem na plecy i zobaczyłem jak zza muru wystrzeliwuje w górę kolejny słup ognia, wyzierając swe ogniste języki przez dziurę, jaką zrobiłem w murze. Broń upadła mi gdzieś pod murem. Szybko odczołgałem się do tyłu, z obawy iż w każdej chwili ściana może się rozpaść.

Usłyszałem za sobą dźwięk repetowanej broni. Odwróciłem się gwałtownie. To Woron celował ze swego Walthera prosto w moją głowę. W drugiej ręce trzymał w gotowości nóż.
Obleciał mnie zimny pot. Zastygłem w bezruchu. To Woron? To naprawdę on?
\sd
\xx Co ty, kurwa, wyprawiasz?! Nie celuj we mnie, kretynie!
\xx Oddaj mi kombinezon – mówił przez zaciśnięte zęby. Policzki mu się trzęsły, ręka drżała, ale lata wyszkolenia bojowego, sprawiły, że mimo to trzymał broń pewnie.
\xx Oszalałeś?!
\xx Oddaj mi kombinezon. Tylko je\3k eden stąd wyjdzie. Rozumiesz\3k Ja albo ty. Wybacz, stary.
\xx Woron\3k
\xx Wybacz\3k
Piorun przeszył mnie od głowy w dół. Woron zmrużył oczy i nacisnął za spust.
\vspace{.6em}
\xx Klik!
\vspace{.6em}
\xx Klik!
\vspace{.6em}
\xx Klik!
\vspace{.6em}
\qd
To był jak trans, z którego nagle się ocknąłem. Pistolet wypadł mu z ręki i uderzył o suchą ziemię.
\sx Woron\3k
\xx Nie! – krzyknął, po czym chwycił za nóż i wbił go sobie w pierś.
\qd
Trwało to ułamek sekundy. Nim zdążyłem zareagować, jego ciało wzdrygnęło się, zatrzęsło nim, po czym znieruchomiał.
Powoli osunął się na bok.

Usiadłem. Zapanowała dla mnie cisza. To nic, że huk rozdzierających powietrze fontann ognia nieustannie uderzał w moje uszy. Czułem się jak gdyby ktoś mi strzelił koło głowy. Myśli krzątały się gdzieś bez ładu, a ja nie wiedziałem co począć.
  „Woron, czemu to zrobiłeś?”, powtarzałem sobie. „Wiedziałeś, że tylkojedna osoba jest stanie się stąd wydostać. Po prostu wiedziałeś to. Ale czemu tak?”.

Trwałem tak jeszcze przez chwilę, nie zważając na szalejące dookoła piekło. Straszne myśli przychodziły mi do głowy, a ja nie chciałem ich słuchać. A jednak. Czy to prawda? Czy on właśnie ocalił mi życie?
„Tylko jeden stąd wyjdzie”. To jego słowa. Tak, to jednak był on. Cały on. Wiedział czego chce. Tylko Woron zdolny był dostrzec to, co ja próbowałem zatuszować. Czemu chciałem zaprzeczyć.
Nadzieja. Czy tchórzostwem jest trwanie w nadziei w obliczu zagłady? A może głupotą? Czy byłbym w stanie zachować się jak on? Poświęcić, życie swoje albo jego. Ileż odmian ma chęć przetrwania. Ileż ścieżek można w niej obrać. Jakże wybrać właściwą?
Zachowałem się jak tchórz. Chciałem ocalić nas obu, on chciał tylko jednego z nas. Mimo to czułem, że postąpił właściwie.

Teraz, właśnie teraz wiedziałem, że mam jeszcze szansę.

Wstałem. Spojrzałem wprost w ciemność, w której czaiła się ostateczna nagroda. To po co tu przybyliśmy. Rzecz, z którą wielu stalkerów oddało już życie, a teraz jeszcze Woron. Przedmiot wielkiej mocy, którego pochodzenia nikt nie jest w stanie jak dotąd wyjaśnić. Musiał tam być. Inaczej nie miałoby to sensu.

Wziąłem głęboki wdech. Teraz albo nigdy.

Wbiegłem na cegły. Było ciepło. Coraz cieplej. Nie zważałem na to. Szybko wdrapałem się na szczyt. Blask ognia za plecami był wystarczająco jasny. Spojrzałem w dół.

Był tam. Ciężko było go dostrzec, ale udało mi się. Przezroczysty kryształowy twór, migocący blaskiem szalejącego dookoła ognia. Był gigantyczny, może wielkości mojej głowy. Nigdy wcześniej nie widziałem takiego wielkiego. Leżał swobodnie, niewinnie niczym porzucona zabawka. A jednak czułem jego moc, wiedziałem, że oto mam przed sobą przedmiot nie z tej ziemi.

Panował piekielny żar. Poparzona skóra protestowała, zmuszała mnie do ucieczki przed czającą się tu anomalią. Złapałem za jedną z cegieł i rzuciłem nią w dół, w stronę artefaktu. Nagle, wielki słup ognia wystrzelił w niebo. Potworny gorąc owiał moje ciało. Zasłoniłem ramieniem oczy, cofnąłem się o krok. Ta fontanna ognia byłą wyższa niż wszystkie inne. Strzelając w górę po jakichś pięciu metrach, wyginała się w przedziwne kształty pod wpływem znajdującej się tam anomalii grawitacyjnej. Rozlewała się na wszystkie strony, tworząc w powietrzu spirale i serpentyny żywego ognia. A na dole, wśród szalejących płomieni tańczył muskany płomieniami Kryształ.

Musiałem to zrobić teraz. Nie było już wyjścia. Skuliłem się i zrobiłem krok w przód po kruchych cegłach. Przez porysowaną szybkę hełmu widziałem go wśród płomieni.

Kolejny krok w dół

I kolejny.

Nagle, jedna z cegłówek obsunęła mi się spod moich stóp. Sparaliżował mnie strach. Upadłem na plecy i poczułem, że staczam się prosto do szalejących płomieni. Chciałem się czegoś złapać, ale wzbijałem tylko pył i strącałem kolejne cegły. Ogień był coraz bliżej. Huk płomieni ogłuszał mnie, że nie słyszałem nawet własnego krzyku.

I znowu przerażający żar.

Spadłem na sam dół. Uderzyłem brzuchem o twardą ziemię, z lewą ręką pod sobą.. Przede mną rodziło się piekło. Nie mogłem patrzeć. Paliłem się, gotowałem. Czułem jak kombinezon kurczy się na mnie, a materiał zaczyna się wypalać. Rozgrzane powietrze wniknęło do środka, owiało mi ciało, parząc i wysuszając je. Byłem pewien, że umrę.

 I wtedy mrugnął mi przed oczyma delikatny blask Kryształu. Odruchowo wyciągnąłem gołą dłoń w stronę ognia i poczułem to.

 Był chłodny. Trzymałem rękę na Krysztale w samym sercu ognia, a mimo to czułem chłód. To uczucie rozlało się po moim ciele. Znikł gdzieś ból oparzeń, przestałem odczuwać żar. Języki ognia prześlizgiwały mi się między palcami, muskając je swym ciepłym dotykiem. Włosy na dłoni momentalnie uległy spaleniu, ale skóra pozostawała nietknięta.

 Powoli wyciągnąłem artefakt ze środka. W jego wnętrzu widziałem szalejące głęboko płomienie. A więc, to było to? Ta moc, o której mówił Woron? Ciężko opisać jak się wtedy czułem. To uczucie było zbyt nierealne. Nie jestem bogiem, a mimo to żywioł nie czynił mi krzywdy. Niesamowite wrażenie niczym sen. Tyle, że materialne i prawdziwe.

 Oniemiały z wrażenia wstałem i szybko wygramoliłem się na górę. Powoli zacząłem iść w stronę ściany ognia, która uprzednio o mały włos nie zabiła mojego towarzysza. Teraz ten leżał martwy koło mnie, gdy go mijałem. Ale ja patrzyłem się w kryształ.

 Nie mogłem w to uwierzyć. Tyle zachodu po ten kawałek kryształu? Czy to naprawdę było tego warte? Życie moje i Worona postawione nad przepaścią. Nie miałem dużo czasu. Wiedziałem, że z każdą chwilą artefakt wydziela kolejne dawki zabójczego promieniowania. Mimo to nie śpieszyłem się.

 Powoli wszedłem w zamykający wyjście z zaułka wir ognia, który był teraz dla mnie niczym powiew ciepłego wiatru. Choć na tę chwilę posiadłem moc, o jakiej wcześniej mogłem marzyć, wiedziałem że uczucie to jest daremne, chwilowe. Ale nie to było najgorsze. Wiedziałem, że na zewnątrz nikt na mnie nie czeka, tylko mroźna pustka Zony. Lasy pełne mutantów, bandyci i anomalie. Gdzieś dalej na wschód Bar, a w nim nagroda.

 Piętnaście tysięcy rubli.

 Właśnie tyle dostanę za dostarczenie tego artefaktu do barmana. Taka była cena życia Worona. Tyle zapłacą mi za ów boską moc, jaką im dostarczę. Starczy to ledwo na leczenie oparzeń i nowy kombinezon. I tyle, nic więcej. Nie ma sprawiedliwości w Zonie.

 Proszę, nie pytajcie mnie, czy było warto. Nigdy nie pytajcie, żadnego stalkera czy było warto, bo nie ma ceny rublach czy innej walucie, która zrekompensuje straty jakie każdego dnia ponosimy. Dzień za dniem. I tak w kółko.
\center
 Takie to już życie w Zonie.
\end{document}