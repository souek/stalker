\documentclass[../MAIN.tex]{subfiles}
\begin{document}
%
\ro{7.30}
%
Z dokładnością jak w~szwajcarskim zegarku codziennie budzę się
o tej godzinie. Nie wiem, czy to odgłosy życia Zony zmieniają
ton w~taki sposób, że mój organizm to wyczuwa i~woła na
pobudkę, czy po prostu zegar biologiczny ustalił sobie takie, a
nie inne działanie. Zresztą, to nieistotne.

Na dworze delikatna, poranna mgła, najpewniej zniknie w~ciągu
maksymalnie dwóch godzin~--~zapowiada się piękny dzień, na ile
dzień może być piękny w~takim miejscu. Nie czuję żadnych
zapachów, co oznacza że moja specjalna maść idealnie spełniła
swoje zadanie, a~także nic podejrzanego nie kręci się w
pobliżu. Dźwięki? Delikatny szum pobliskich drzew i~kapanie
ostatnich kropel wody po deszczowej nocy.

Po tych wstępnych ocenach otoczenia dochodzę do wniosku, że
mogę bezpiecznie wyjść ze śpiwora. Cały cykl czynności łącznie
z zapakowaniem się do drogi zajmuje mi nie więcej jak trzy
minuty. To jednak może być za wolno, na przyszłość muszę nad
tym popracować.

Szybko przeglądam stan ekwipunku~--~,,pionówka'' jak zawsze
idealna, beretta będzie niedługo wymagała naprawy gwintu
tłumika, ale póki co da radę, nóż wciąż ostry jak brzytwa,
a~skórzany płaszcz i~kamizelka zdjęta z~najemnika, mimo paru
nowych naszywek, jeszcze przez jakiś czas będą spełniać swoje
zadanie.

Postanawiam ruszyć się z~mojej kryjówki~--~wyglądam przez
zabrudzone upływem czasu i~działaniem natury okno, szukając
śladów podejrzanej aktywności.

Spokój i~cisza, można wychodzić.

Opuszczam drabinę ze strychu, schodzę w~dół wywołując
trzeszczenie próchniejącego drewna przy każdym kroku i~staję na
drewnianej podłodze, wzniecając drobinki kurzu. Otwieram
skobelek na spróchniałych drzwiach~--~totalnie bezcelowe
zabezpieczenie~--~i wychodzę na zewnątrz. W moje nozdrza z~całą
swą mocą uderza zapach Zony~--~specyficzny, słodko-słony smród
stęchlizny i~rozkładających się ciał wymieszany z~aromatem
napromieniowanych traw i~pączkujących drzew. Wychodzę spomiędzy
otaczających moją kryjówkę chaszczy i~trzcin, czujnie
rozglądając się dokoła.

Cisza.

Wypracowanym przez lata życia w~Zonie szóstym zmysłem wyczuwam,
że coś musi być nie w~porządku~--~w tym piekle praktycznie
nigdy
nie jest aż tak cicho, chyba że\3k

Jak na znak mojej myśli, niebo zaczyna przybierać
krwistoczerwoną barwę i~zachodzić gęstymi chmurami, a~do głowy
wpycha się przeciągłe buczenie. Emisja! Wygląda na to, ze jedna
z~mocniejszych~--~mój mały domek na pewno nie da mi teraz
niezbędnego schronienia. Staję jak wryty na moment, w~głowie
obliczając odległości do paru innych kryjówek.

Jaskinia pod nasypem kolejowym! Spinam mięśnie do biegu i
sprintem puszczam się przed siebie, nie zważając na nogi
wpadające w~błoto czy wysokie trawy smagające mnie po twarzy.
Biegnę skupiony tylko na dotarciu do schronienia, zanim będzie
za późno. Nogi zaczynają ciążyć mi coraz bardziej, oddech
zaczynam słyszeć jakbym trzymał głowę w~wielkim słoiku -
dociera do mnie coraz wyraźniej świadomość, że mogę nie dobiec
na czas. Wokoło mnie powietrze gęstnieje, kolor nieba zmienia
odcień na coraz ciemniejszy, słyszę coraz bliżej wyładowania
uderzające w~ziemię\3k Jeszcze tylko parędziesiąt metrów\3k
Przebiegam pod zwalonymi częściowo przęsłami mostu, czując w
ustach coraz intensywniejszy smak krwi i~w~końcu rzucam się do
jaskini i~wbiegam w~głąb niej.

Jestem wyczerpany, ale nie czuję
już tego tępego buczenia w~głowie, jak gdyby mój czerep miał
eksplodować. Przed oczami zaczynają biegać mi mroczki: po
części od wysiłku, po części od długiego wystawienia na
działanie emisji, w~związku z~czym daję radę tylko ułożyć się
na podłożu, by nie poobijać się przy upadku i~tracę
przytomność\3k
%
\ro{8.42}
%
Kiedyś kupiłem u Sidorowicza zegarek na rękę z~funkcją alarmu.
W zasadzie budzik był i~jest mi całkowicie zbędny, jednak
zawsze przydaje się możliwość sprawdzenia aktualnej godziny.
Choćby teraz: dowiedziałem się ile czasu byłem nieprzytomny.

Po otwarciu oczu i~zobaczeniu godziny sprawdzam stan swojego
wyposażenia jak i~organizmu. Oprócz błocka na całym płaszczu i
lekkiego bólu głowy wszystko wydaje się być w~porządku.
Wychodzę ostrożnie przed jaskinię i~nasłuchuję dłuższą chwilę,
by wyczuć nieproszonych gości. Nic podejrzanego, dobrze.
Postanawiam wspiąć się na przęsła częściowo zawalonego mostu,
aby stamtąd rozejrzeć się wokoło za innymi stalkerami –
rannymi, lub może raczej martwymi~--~albo pobliską zmutowaną
zwierzyną. W oddali zauważam bawiące się stadko mięsaczy~--~to
dość smutne, a~zarazem zabawne i~ironiczne, że istoty gotowe
wygryźć ci trzewia są też w~stanie bezinteresownie bawić się,
kulając swoje obłe cielska po ziemi.
Chociaż\3k
Nachodzi mnie
nagła myśl, że przecież my, ludzie, jesteśmy dokładnie tacy
sami~--~potrafimy rano mordować bez mrugnięcia okiem, a
wieczorem bawić się w~najlepsze. Kontynuuję rozglądanie się i
kilkadziesiąt metrów od jaskini widzę nieruchomy kształt,
częściowo zanurzony w~błocku. Postanawiam to
sprawdzić~--~zejście z~mostu zajmuje mi parę chwil, ponowne
odnalezienie
ciała, tym razem z~perspektywy naziemnej, zabiera mi kolejnych
parę minut. W końcu staję przy zwłokach.

Stalker. Patrząc na kolory, wolnościowiec. Bied\-ny dureń, nie
zdążył dobiec na czas.

Trup leży twarzą do dołu, więc najpierw obszukuję jego plecak.
Dwie konserwy, pół czerstwego już chleba, jedna stłuczona
flaszka, zapasowe filtry do maski przeciwgazowej i~duża paczka
zapałek, pełna do połowy. Do tego pudełko tabletek na
nadciśnienie, zestaw medyczny pierwszej pomocy i~parę paczek
bandaży. Jest coś jeszcze\3k Pocztówka? Nie, zdjęcie. Młody
mężczyzna z~kobietą w~podobnym wieku, oboje uśmiechnięci.
Siostra, żona? Z tyłu jest coś napisane\3k „Choć tak daleko, to
jednak blisko. Lena”. Cokolwiek to znaczy. Chowam fotografię do
kieszeni płaszcza, może się przydać. Przy okazji zabieram całą
zawartość plecaka nieszczęśnika i~obracam go na plecy. Na pasie
ma dwa magazynki 9x18 mm~--~jeden sześć naboi, drugi osiem.
Ładownice na kamizelce kryją trzy magazynki łukowe do SWD, z
czego dwa puste i~jeden z~czterema pociskami. Karabinu nie ma
nigdzie w~pobliżu, być może zgubił go podczas ucieczki, jednak
na biodrze truposza jest Makarov z~pełnym magazynkiem. Zabieram
go razem z~amunicją~--~w~ostateczności opchnę je jakiemuś kotu.
Jednak kiedy chcę się podnieść i~oddalić, przez ślady błota na
ciele zauważam ranę, która zwraca moją uwagę\3k

Podcięte gardło?

Wolnościowiec jest jeszcze lekko ciepły, więc musiał zostać
zaatakowany na chwilę przed emisją\3k Ale kto ryzykowałby
atakowanie kogoś nożem na moment przed uderzeniem? Coś zaczyna
mocno mi tu nie pasować~--~tak samo jak brak karabinu w~pobliżu
zwłok. Zamykam oczy, próbując skupić swój słuch i~węch w
poszukiwaniu czegoś lub kogoś podejrzanego w~pobliżu. Nadal
nic. Otwieram oczy i~patrzę dokoła, szukając śladów w~błocie,
poza tymi pozostawionymi przez denata przed śmiercią.

Pusto.

Emisja potrafi całkowicie przemeblować Zonę\3k Zastanawiam się
chwilę, kto nie bałby się zaatakować innego stalkera na sekundy
przed jej uderzeniem~--~w~pobliżu nie było żadnej innej
kryjówki, poza jaskinią, w~której kryłem się sam, więc ktoś w
ten sposób naraziłby się tylko na bolesną i~bezsensowną śmierć.

Dźwięk!

Blisko mnie, co najwyżej pięć metrów za moimi plecami słyszę
delikatny odgłos, wydawany zazwyczaj przez kogoś, kto usilnie
stara się poruszać jak najciszej. W zasadzie usłyszałem go
tylko dzięki wyczulonym przez lata życia w~Zonie zmysłom\3k Nie
daję po sobie poznać, że wiem o obecności napastnika.
Napastnika\3k Osoba z~wrogimi zamiarami strzelałaby od razu,
nie
bawiąc się w~żadne podchody, natomiast ktoś nastawiony
przyjacielsko od razu dałby znać słowem lub gestem o swojej
obecności. Nie jestem pewny tej sytuacji, dlatego czekam na jej
rozwój.

Klik!

Słyszę dźwięk odciąganego zamka broni~--~czyżby Colt 1911?~--~i
czuję lufę przyłożoną do mojej głowy. Cóż za beznadziejnie
filmowy gest\3k Chwilę potem słyszę słowa wypowiadane zza
maski,
także lekko zniekształcone w~swoim brzmieniu:

- Stalkerze, masz coś, co nale\3k

Napastnik nie zdąża dokończyć zdania, bo błyskawicznie się
odwracam, lewą ręką odtrącam broń, a~prawą zaciskam w~pięść i
wbijam w~żebra celu. Kiedy wróg składa się po ciosie, poprawiam
kopnięciem w~kolano i~uderzeniem płazem dłoni w~kark.
Adrenalina opada, podobnie jak mój niedoszły oprawca~--~teraz
mogę mu się przyjrzeć.

Monolit\3k??

Patrzę na zwijającego się z~bólu, półprzytomnego członka
Monolitu\3k Dlaczego nie zabił mnie od razu? Może to zdobyczny
kombinezon\3k Choć w~zasadzie wygląda jak spod igły~--~poza
paroma
przybrudzeniami. Wyciągam z~plecaka flaszkę, odkręcam ją i
ściągając mu z~twarzy maskę zaczynam wylewać na jego twarz,
żeby odwrócić jego uwagę od bólu i~skupić na mnie.

Jego\3k JEJ twarz!

Po opadnięciu kaptura i~zdjęciu gumowej maski przeciwgazowej
widzę przed sobą młodą kobietę z~jasnymi włosami, próbującą
gorączkowo pozbyć się alkoholu, który dostał się do jej ust i
oczu. Kucam przed nią, chwytam ją pod brodę i~patrząc w~oczy
mówię do niej chrapliwym głosem odwykłym od prowadzenia
dialogów:

- Gadaj.

Dziewczyna w~końcu daje radę zogniskować na mnie swój wzrok –
patrzy na mnie ze strachem w~oczach. Żaden z~doświadczonych
bojowników nie czuje strachu, ani żadnych innych uczuć, więc
ona musi być dopiero wdrażana. Otwiera usta i~bezgłośnie nimi
porusza~--~domyślam się, ze wypowiadała jakąś mantrę do
Monolitu\3k Po chwili odpowiada łamiącym się głosem:

- Zdjęcie\3k Proszę\3k To pamiątka, specjalnie wróciłam po
nie\3k

Domyślam się od razu o jakie zdjęcie jej chodzi i~dlaczego to
pamiątka~--~to właśnie ona była dziewczyną u boku młodzieńca na
fotografii. Nie mam pojęcia jak się tu znalazła i~skończyła w
taki sposób i~nie wiem, czy chciałbym wiedzieć. Domyślam się
tylko, że zapewne kazali jej zabić przyjaciela jako przysięgę
wierności, czy jakoś tak\3k Świry. A ona na pewno uciekła od
patrolu, bo chciała nie zapomnieć tego, którego musiała zabić –
pieprzony świat. Muszę tylko spytać o jeszcze jedną rzecz,
która właśnie wpadła mi do głowy.

- Ilu was jeszcze jest?

Dziewczyna jest na skraju załamania, może też dlatego, że
właśnie patrzy kątem oka na swojego zabitego przyjaciela, czy
kochanka\3k Po chwili odpowiada:

- Tylko ja i~mój mentor\3k To miał być test, ale ja go nie
zdałam, on musiał go\3k

Zanim wypowiada zdanie do końca, słyszę huk wystrzału z~SWD i
widzę, jak jej głowa odskakuje po trafieniu kulą 7.62 mm\3k Do
głowy wpada mi szalona myśl „oto zaginiony karabin!”~--~zanim
jej ciało dotyka ziemi, uskakuję w~bok, w~kępę największych
traw i~krzaków, żeby ukryć się przed wzrokiem strzelca. Ona
upadła w~prawo, dźwięk rozchodził się z~tej strony\3k Most!
Strzelec jest na moście! Zaczynam spoglądać spomiędzy chaszczy
– niewielki dystans, może 30 metrów, bez problemu podziurawię
suczego syna śrutem. Ściągam powoli z~pleców ‘pionówkę’,
wyskakuję z~krzaków, robię obrót po ziemi, słyszę kolejny
strzał, szczęśliwie chybiony, po czym wstaję i~niemal od razu
przykładam kolbę do policzka~--~oddaję dwa strzały, jeden po
drugim. Na tym dystansie śrut nie ma jeszcze dużego rozrzutu,
więc siła uderzenia powinna przynajmniej mocno oszołomić cel.
Patrzę, jak strzelec spada z~przęseł na dół. Łamię karabin,
wyrzucając puste loftki i~nabijam go ponownie. Zakładam broń na
plecy, wyciągam berettę i~biegnę w~stronę mostu. Mój cel leży
pod nim~--~jego twarz jest poharatana śrutem, tak samo jak
wierzchnia warstwa pancerza. Do tego szyja jest nienaturalnie
wykrzywiona~--~widać skręcił kark przy upadku. Szybko
przeglądam
jego ekwipunek, zabieram karabin i~wracam do ciała dziewczyny.
Nie ma dla niej szans~--~kula przeszła przez szyję. Kiedy ja
ostrzeliwałem się z~jej mentorem, ona zadusiła się własną
krwią, o ile nie umarła od razu. Sprawdzam, co może mieć przy
sobie przydatnego~--~zabieram Colta z~czterema zapasowymi,
pełnymi magazynkami i~jej PDA~--~może dowiem się czegoś
ciekawego. Zamykam jej oczy, oddając ostatnią przysługę, po
czym oddalam się nie oglądając się za siebie.
%
\ro{9.37}
%
Idę przed siebie, myśląc kto mógłby złamać zabezpieczenia PDA
dziewczyny~--~parę chwil wcześniej okazało się, że jest
chronione hasłem. Nie mogę przestać myśleć o tej sprawie i~nie
mam pojęcia dlaczego~--~przecież nic ona dla mnie nie znaczyła,
a jednak nie potrafię wyrzucić z~głowy obrazu jej śmierci.

Rugam sam siebie~--~co mi odbija, żeby na starość robić się
jakimś sentymentalistą? Do głowy powolutku i~nieśmiało
przesącza się myśl, że może w~ten sposób podświadomie chcę
odkupić grzechy przeszłości\3k?

NIE!

Przeszłość, to przeszłość~--~co się stało, już się nie
odstanie.
Może moje działania wynikają po prostu z~faktu, że to była
kobieta, czy raczej dziewczyna? Samica w~każdym razie, a~ich
wiele tu nie widać\3k W zasadzie przez wszystkie te lata
widziałem tu tylko jedną~--~udawała faceta, co nie było trudne
z
jej wyglądem i~budową ciała. Jednak szybko źle skończyła\3k W
końcu sama odebrała sobie życie~--~i~pomyśleć, że tu, gdzie
każdy z~nas powinien trzymać się razem i~pomagać nawzajem, inni
są gotowi do takiego okrucieństwa\3k

Ból?

No tak. Zaciskam ze złości pięści z~całej siły\3k Rozluźniam
je, zatrzymuję się na chwilę i~unoszę dłonie do
oczu~--~przelały
tyle krwi, a~teraz wespół z~umysłem zbiera im się na heroiczne
czyny i~ratowanie martwych dziewic od nieistniejących smoków\3k
Szkoda, że nie słuchają mnie samego. Z drugiej strony\3k Zawsze
ciekawość zżerała mnie od środka w~przypadku spraw niejasnych i
tajemniczych.

Koniec fruwania z~głową w~chmurach!

Skupiam się na powrót na trasie jaką sobie obrałem. To w~końcu
Zona, nie spacer po parku. Sprawdzam mapę w~moim PDA~--~do bazy
tych czubków z~Czystego Nieba jeszcze około kilometra.
Jajogłowi kretyni ze spluwami\3k Ten cały Liebiediew śmierdzi
mi na kilometr jakimiś tanimi sekretami~--~ciekaw jestem, na
ile
on sam okłamuje swoich ludzi i~na ile oni mają tego świadomość.
Jednak póki co, najbliższy gość, który zna się na grzebaniu w
elektronice, jest u nich~--~mam trochę rubli, powinno
wystarczyć
na opłacenie usługi. Do tego zabrałem karabin tego
wolnościowca, użyty przez monoliciarza~--~jest w~dobrym stanie,
więc może dostanę za niego dobrą sumkę. Albo wymienię za
informację, czy przysługę\3k

Pik!

Staję w~pół kroku. Rozglądam się wokoło w~poszukiwaniu
anomalii\3k W tej części bagien były same 'trampoliny', z~tego
co dobrze pamiętam\3k Dokładnie tak~--~pół metra na prawo ode
mnie zauważam specyficzny bąbel z~krążącymi wokoło kawałkami
liści i~ściółki, trochę dalej drugi i~trzeci. Dla pewności
wyciągam detektor, ale niestety jest jak myślałem~--~niczego
nie
zrodziły. Omijam ostrożnie anomalie i~ruszam w~dalszą podróż,
tym razem zachowując wzmożoną czujność.

Kontynuuję marsz~--~po dzisiejszej emisji Zona jest
zadziwiająco
pokojowa i~spokojna\3k Nie zawsze tak się zdarza. Jednak nagle
jak na komendę jakiś dźwięk dociera do moich uszu.

Krzyk..?

Kryję się za pobliskim, skarłowaciałym drzewem i~nasłuchuję.
Tak, krzyk~--~ktoś woła o litość. Czyżby znów bandyci czy inne
ścierwa? Głos zaczyna się zbliżać, czyżby ktoś się urwał i
ucieka? Patrzę ponad wysokimi trawami i~tatarakami, po chwili
zauważam przyczynę zamieszania. Jakiś kot biegnie panicznie
przed siebie, co drugi krok potykając się i~wpadając w~błoto, a
za nim truchta\3k Dzik? Sam, jeden dzik? Z zażenowaniem uderzam
się dłonią w~czoło, po czym ściągam z~pleców 'pionówkę' i~biorę
zwierzaka na cel. Panikarz przebiega kilka metrów przede mną,
nawet mnie nie zauważając. Nie wiem, czy to zasługa moich
zdolności wtapiania się w~otoczenie, czy po prostu jest tak
przerażony, że nie zwraca uwagi na nic innego, poza uciekaniem.
Chwilę po nim w~zasięgu strzału pojawia się sprawca tego
rabanu. Naciskam spust, czuję uderzenie kolby w~ramię, lufa
minimalnie podbija do góry~--~dostał. Podchodzę bliżej, żeby
sprawdzić, czy jeszcze żyje~--~dzik leży na boku, coraz słabiej
wierzgając po ziemi kopytami. W jego oczach widzę niemy wyrzut
– smutne spojrzenie umierającego, rzucające we mnie
oskarżeniem\3k Mutant, czy nie~--~nie powinien cierpieć agonii.
Wyciągam berettę i~oddaję dwa strzały w~jego łeb.

Po chwili pojawia się panikarz, zwabiony odgłosem strzałów.
Podchodzi niepewnie, jak gdyby bał się, że jego też zastrzelę.
Daję mu znak ręką, że może podejść. W międzyczasie zabieram się
za skórowanie zwierzyny. Paskudna sprawa nieść takie
niewygarbowane~--~nomen omen~--~świństwo w~plecaku, ale do
czasu
dojścia do bazy nie powinno się zepsuć. Dodatkowo, wycinam parę
kawałków schabu~--~moczone przez jakiś czas w~wódce miękną
i~nie
są aż tak radioaktywne\3k

-Ehem\3k Dzięki za pomoc, wiesz?

No tak, gość stoi tu dobre pięć minut, skoro zdążyłem już
oporządzić dzika. Wstaję i~odwracam się do niego. Odpowiadam
chrapliwym głosem:

-Nie ma sprawy. Broń?

Pokazuję głową na pustą kaburę na udzie chłopaka~--~teraz
dopiero miałem okazję mu się przyjrzeć\3k Młody szczawik,
najwyżej 20 lat na karku. Nic dziwnego, że piszczał jak
panienka\3k Co tu tak ciągnie tych młodocianych głupców? W
międzyczasie moje rozmyślania przerywa jego odpowiedź:

-Eee\3k Zgubiłem\3k Jak uciekałem\3k Leży pewnie gdzieś w
błocie i~już go nie znajdę\3k A dałem za niego całe 500
rubli\3k

Robi mi się żal tego idioty, więc ściągam plecak, wyciągam z
niego zabranego wcześniej Makarova i~dwa magazynki i~podaję
dzieciakowi. Mówię mu:

-Bierz i~spadaj. I nie daj się głupio zabić. Albo najlepiej\3k

-Co najlepiej?

-Najlepiej wynoś się z~Zony i~dziękuj, że uszedłeś z~życiem.
%
\ro{11.25}
%
-Stój! Ręce w~górę!

No tak. Baza jest już blisko, w~trzcinach i~krzakach siedzą
strażnicy\3k Ten jeden brzmi dość nerwowo~--~albo jest nowy,
albo stało się coś, co wywołało taką nerwówkę. Może jedno i
drugie. Unoszę powoli ręce w~górę i~odwracam się do tyłu.
Istotnie, obok pobliskiej kępy trzcin stoi młody stalker,
ubrany w~kombinezon pomalowany w~kretyńskie niebiesko-białe
kolory. Kamuflaż, cholera\3k Widzę, że wyciąga zza pasa opaskę
na oczy, jednak uprzedzam go i~mówię:

-Nie ma potrzeby, znam drogę. „Za trzecią sosną w~prawo,
przewodniku”, co?

Strażnik jest zdziwiony~--~nie zaskakuje mnie to~--~ale kiwa
głową i~pozwala mi iść dalej, samemu kryjąc się na powrót w
trzcinach. „Za trzecią sosną w~prawo, przewodniku”, to dość
oklepane hasło, jednak łowcy i~tropiciele odwiedzający bazę
jajogłowych nadal je stosują. Powoli i~nieśpiesznie kieruję się
do bazy. Zaczynam już słyszeć gwar rozmów i~inne hałasy wiążące
się z~większymi skupiskami ludzi. Wychodząc zza kolejnej
gęstwiny krzaków i~drzew w~końcu zauważam cel mojej wędrówki.
Grupka stalkerów w~mundurach Czystego Nieba, paru neutralnych
łowców czy przewodników. W okolicy głównego budynku kręci się
jeszcze paru jajogłowych. Jak to jest, że każdy z~nich ma
prawie łysy łeb? Mimo woli przyglądam się drewnianemu,
piętrowemu budynkowi, będącemu centrum dowodzenia~--~od
ostatniego razu kiedy tu byłem, zaczął się sypać jeszcze
bardziej. Dziury w~ścianach, powyrywane okna, brak sporej
części dachu, wszędzie pleśń\3k Wygląda coraz gorzej. Odnoszę
wrażenie, że wcale o niego nie dbają, mimo że to przecież jakby
ich dom. Załamuje mnie taka bezmyślność niektórych. No nic –
wzruszam ramionami i~ruszam przed siebie, stawiając kroki po
ułożonych w~ścieżki deskach. Prowizoryczne zabezpieczenie przed
wdepnięciem w~co głębsze kałuże błota\3k W końcu to bagna.
Kieruję swoje kroki do barmana~--~jego siedziba także się nie
zmieniła. Mała, obskurna budka z~wypolerowanym jak lustro
szynkiem, a~na przeciw kilka wysokich stolików zbitych z
butwiejących desek, czy starych palet. Przy niektórych stoją
pojedyncze osoby, lub małe grupki, część jest pusta. Podchodzę
do tak zwanego 'kierownika bałaganu' i~staję przy ladzie.
Ściągam plecak, kładę go na blat, rozpinam i~wyciągam wilgotną,
zwiniętą w~kostkę skórę ściągniętą z~dzika. Patrzę na barmana
unosząc lekko prawą brew~--~niemy odpowiednik pytania „ile i
czemu tak drogo?”. On z~kolei nawet nie drgnie, udając że „nic
tu przecież nie ma”. Jednak zanim nasz niemy pojedynek~--~nie
pierwszy zresztą w~ciągu ostatnich lat~--~nabiera tempa,
podchodzi do mnie jeden z~klientów i~pyta:

-Ile za taki dywan, bracie?

Przyglądam mu się chwilkę i~widzę prawdziwego weterana Zony –
zarówno wiekiem jak i~doświadczeniem. To ten typ człowieka,
który może sobie pozwolić na noszenie zwyczajnego skórzanego
płaszcza z~maską przeciwgazową przy pasie i~używanie normalnej
dwururki~--~zawsze będzie tak samo skuteczny, choćby biegał na
gatkach z~leszczynowym badylkiem. Odpowiadam mężczyźnie:

-Pięćset.

Przez chwilę mierzy mnie wzrokiem, po czym podchodzi bliżej
lady, bierze w~ręce skórę, rozwija ją, oceniając jakość i
stwierdza:

-Czterysta i~zjesz z~nami porządny obiad przy dobrej gorzałce.
Lowa robi zawsze niezłe żarcie przy ogniu, nie to co
niektórzy\3k

Zamierzona kpina zawisła w~powietrzu jak siekiera w~malutkim
pomieszczeniu bez wentylacji, wypełnionym po brzegi nałogowymi
palaczami\3k Barman jednak grymasi tylko chwilę jak
naburmuszone dziecko, macha ręką dając nam gest w~stylu „jak
się wam nie podoba, to idźta se gdzie indziej” i~powraca do
opierania się łokciami o szynk. Wzruszam ramionami i~wyciągam
dłoń do mojego kontrahenta. Ten ściska ją, po czym wyciąga
umówione czterysta rubli i~wręcza mi banknoty. Wskazuje głową
na ognisko niedaleko, przy którym siedzi już paru stalkerów i
mówi:

-Za jakieś dziesięć minut Lowa skończy pichcić. Chodź, od razu
z nami przycupniesz, to odkorkujemy flaszkę przed wyżerką –
żeby oblać interes.

Kiwam głową na znak zgody i~ruszam za mężczyzną. Po chwili
siadamy na pniakach wokół ogniska, patrząc na siebie, ale nic
nie mówiąc. Nieufność\3k Im dłużej jesteś w~Zonie, tym bardziej
wysysa ona z~ciebie ludzkie odruchy\3k Jak gdyby chciała się na
tobie zemścić, za twój pasożytniczy byt. Jak gdyby chciała cie
odmienić w~sposób inny, niż mutacja, pokazać swoją władzę\3k
Siedzimy tak parę dłuższych chwil, popijając z~odkorkowanej
przez starca flaszki, wsłuchując się w~trzaskające płomienie i
wąchając wdzierający się do nozdrzy smakowity zapach obiadu. W
końcu mężczyzna nazwany wcześniej Lowa ściąga garnek znad
paleniska, rozdaje nam blaszane miski i~nalewa każdemu porcję.
Uśmiecha się szeroko i~z~teatralnym ukłonem mówi:
%
\sx Smacznego, moja droga klientello!
\qd
Nie jest stąd, to pewne. Ma akcent jakby prędzej przywiało go
tu z~zachodniej Europy. W każdym razie, każdy z~nas zabiera się
za jedzenie. O ile wygląd potrawki pozostawia wiele do życzenia
-- ot, brązowa breja z~kawałkami mię\-sa~--~o~tyle smak
rekompensuje wszystkie braki. Nie mam pojęcia, skąd on wziął tu
warzywa, ale parokrotnie przegryzam coś, co chyba jest
papryką\3k Mięso też smakuje nieźle~--~nawet nie czuć
specyficznego, lekko metalicznego posmaku napromieniowanego
pożywienia. Hm, do tego jeszcze przyprawy\3k Ciekawe. Z
przyjemnością zjadam danie do końca, wyjadając resztki z
miseczki pajdą chleba. Po wszystkim wstaję, kiwam z~uznaniem
głową kucharzowi, żegnam się gestem z~resztą i~odchodzę. Teraz,
z pełnym brzuchem na pewno lepiej będzie się gadało z
jajogłowym od elektroniki. Wchodzę do głównego budynku, w~duchu
cicho licząc na to, że nie zawali mi się na głowę. Technika
znajduję tam, gdzie zwykle~--~i~jak zawsze siedzi z~nosem w
jakichś elektronicznych cudach, o których wiem tylko tyle, że w
Zonie często zawodzą. Podchodzę do niego i~wymownie chrząkam,
żeby zwrócić jego uwagę. Odwraca się w~moją stronę, poprawia na
nosie okulary i~pyta:

-Tak? W czym mogę pomóc, stalkerze? Wiesz, mam dość napięty
harmonogram\3k

Wyciągam PDA dziewczyny i~podaję mu je. Mówię:

-Potrzebuję złamać zabezpieczenia. Ile i~jak długo?

W tak zwanym 'normalnym świecie' każdy w~takiej sytuacji
patrzałby na mnie krzywo. Na pewno ukradł. Na pewno zabił
poprzedniego właściciela. Na pewno chce wykorzystać te dane,
żeby kogoś pogrążyć albo szantażować. Na pewno kłamie. Pod tym
względem Zona oferuje komfort~--~zapłacisz i~wiesz, że nie
przyjdą nawet do głowy takie myśli. Tutaj nie ma miejsca na
rozterki moralne i~wątpliwości. Zona odbarwia wszystko i
zostawia tylko czerń i~biel. Czy może raczej czerń i~ciemny
szary kolor\3k Profesorek po chwili dłubania przy urządzeniu
odpowiada:

-Hmmm\3k Wezmę dwa tysiące, za mniej się nie da. Myślę, że to
cudo zajmie mi z~pół godzinki, może całą. Przyjdź za godzinę,
dla pewności.

Kiwam głową na znak zrozumienia i~wychodzę z~budynku. Mam
godzinę czasu na zmarnowanie.
%
\ro{13.33}
%
Godzina. To tylko sześćdziesiąt minut. Minuta to tylko
sześćdziesiąt sekund. Tylko. A jednak w~ciągu tych zaledwie
sześćdziesięciu~--~czy to minut czy sekund~--~może zdarzyć się
wszystko. Możesz zmarnować swoje życie, możesz ocalić je komu
innemu, możesz całkowicie przetracić ten czas. Jak ja\3k

Wyrzucam te myśli z~głowy. Nie chcę teraz wracać do wspominków
z przeszłości. Wcale nie chcę do nich wracać. Tak się dzieje,
kiedy nie wiesz, co ze sobą zrobić przez godzinę. Zły na siebie
podnoszę się z~prowizorycznego siedziska na uboczu obozu i
wracam do technika~--~powinien już skończyć łamać
zabezpieczenia. Kiedy wchodzę, PDA leży na stoliku nieopodal
wejścia, a~specjalista znów dłubie coś przy sprzęcie i
elektronice. Odwraca się w~moją stronę i~mówi:

-No, poszło dość dobrze, zajęło mi to ze trzy kwadranse. Jak
mówiłem, dwa tysiące.

Zabieram urządzenie, odliczam banknoty i~kła\-dę je na tym
samym
stoliku, po czym wychodzę. Wypadałoby znaleźć dobre miejsce,
żeby siąść i~w~spokoju zobaczyć co za „tajemnice” skrywa PDA
dziewczyny. Naturalnie, nie chcę za swoimi plecami tabunu
podglądaczy i~łowców informacji, więc schodzę do 'pokoju dla
VIPów' w~kantynie, uprzednio płacąc barmanowi stawkę w
wysokości 50 rubli. Kiedyś pewnie była to mała piwniczka na
drewno czy jakieś zapasy. Teraz, z~racji wszechobecnej
stęchlizny, nie warto zostawiać tam czegokolwiek. Zona upomni
się o swoją dolę wilgocią i~pleśnią\3k Pod ścianą stoi tylko
jeden mały stolik i~trzy krzesła, po jednym z~każdej strony.
Siadam na jednym z~nich i~wyciągam urządzenie. Wciskam przycisk
„on” i~czekam, aż się uruchomi. Foldery\3k Zdjęcia,
pamiętnik\3k Cóż, niespecjalnie podoba mi się myśl, żeby czytać
czyiś pamiętniczek, ale z~drugiej strony jego właścicielce
raczej jest to teraz obojętne. Jednak póki co, dalej szukam
innych ciekawych plików. „Przysięga Monolitu”? Co za
pompatyczna nazwa\3k Z ciekawości zaglądam, żeby zobaczyć,
jakie banały im wciskają.

„O, Monolicie, nasz Panie,
Stajemy jako orszak na twe wezwanie!
Dodaj nam sił i~męstwa,
A nigdy nie spotka nas klęska\3k”

Bla, bla, bla, co za banał\3k Nie chce mi się czytać dalej tego
bełkotu~--~przelatuję tylko wzrokiem resztę tekstu i~biorę się
za inne pliki. Po chwili postanawiam wrócić do folderu ze
zdjęciami. Oho, jest parę fot z\3k Chyba to brat~--~ten sam
chłopak, który był na starej fotografii. Teraz wyraźnie widzę
podobieństwo między tą dwójką~--~ten sam nos, usta, podobne
oczy\3k Co mamy dalej? Krajobrazy, widoki\3k Dziewczyna
najwidoczniej lubiła fotografować wszystko dookoła~--~niektóre
ze zdjęć wyglądają rzeczywiście ładnie. Jednak mimo wszystko,
tu nie ma już nic więcej ciekawego, zobaczmy pamiętnik.
Pierwszy wpis\3k Tutaj. Hm.

„To będzie mój dziennik. Piszę go po to, żeby nie zapomnieć,
kim jestem, a~także żeby w~wypadku najgorszego ktoś mógł poznać
moją historię. Nazywam się Lena Sasza Łukin i~znalazłam się tu,
żeby odnaleźć mojego brata, Wladimira. Jeśli ktokolwiek czyta
ten wpis~--~prawie na pewno nie żyję i~nadal nie udało mi się
go
spotkać. Proszę, znalazco, odnajdź go i~oddaj mu moje PDA:

Cóż, w~sumie wyszło na to, że cała ta szopka psu na budę –
rodzeństwo nie żyje, więc nie mam nawet komu przekazać
urządzenia. No nic, może ten stary krętacz Sowa da parę rubli
za jakieś dane tu zawarte. Chcę właśnie zabrać się za
przeglądanie pozostałych wpisów, kiedy nagle słyszę krzyk na
zewnątrz:

-Widmoooo!!

Chwilkę potem pada strzał. Umysł w~ciągu ułamków sekund
analizuje sytuację~--~'widma' to elita Monolitu, snajperzy i
łowcy. Skoro tu są, to prawie na pewno szukają mnie. Jestem w
piwnicy, ale nie wiem jak długo będę tu bezpieczny. Strzał,
skąd padł strzał\3k Niech tylko gnojek drugi raz naciśnie
spust\3k Tak! Wystrzelił kolejny pocisk\3k Wschód, południe\3k?
Południowy wschód! Ściągam z~pleców zdobyczne SWD i~wybiegam z
piwniczki na powierzchnię, po czym rzucam się za pobliski murek
dla osłony. Rozglądam się szybko po obozie~--~strażnik z~dachu
leży martwy na blaszanym pokryciu, na pewno to on krzyczał. Na
środku obozu leży jeden z~kotów w~barwach Czystego Nieba –
dostał w~udo, próbuje się doczołgać do osłony. Niestety, po
chwili pada kolejny, trzeci już strzał~--~młodzik dostaje znów,
tym razem w~głowę. Wszyscy próbują znaleźć jakąś
osłonę\3kWyskakuję zza murku i~biegnę w~stronę niedalekiej
szopy, powinna zapewnić mi wystarczającą ochronę przed wzrokiem
strzelca. Po drodze snajper oddaje kolejny strzał, który mija
mnie dosłownie o włos, raniąc tylko lekko mój bark. Piecze jak
cholera, ale nie zatrzymuję się. Dobiegam do budynku, opieram
się o niego plecami i~sprawdzam stan ramienia~--~kula przeszła
przez miękkie, nic groźnego, jednak mimo wszystko paskudzę
krwią na ziemię. Wyciągam szybko z~jednej z~kieszeni płaszcza
bandaż i~wciskam go pod ubranie, przykładając opatrunek do
ramienia. Widmo strzela ponownie; widzę jak kolejny stalker
pada wyłuskany zza osłony. W obozie tworzy się niezłe
zamieszanie~--~niektórzy z~łowców strzelają na oślep w~miejsce,
gdzie powinien znajdować się monoliciarz, dając mi tym samym
szansę na względnie bezpieczne oflankowanie przeciwnika.
Niedaleko są wzgórza, na sto procent siedzi gdzieś tam\3k
Napastnik strzela ponownie~--~tym razem chyba pudłując –
pozwalając mi na odkrycie jego pozycji.

Zrywam się do biegu spod stodoły; pędzę przez krzaki z
nadzieją, że nie zdąży mnie zauważyć. Gałęzie uderzają mnie po
twarzy, bark nadal piecze i~boli, ale nie zwracam na to uwagi –
skupiam się na czym innym. Po paru chwilach szalonego pędu z
karabinem w~rękach dobiegam do przyczółku~--~kilka sporych
głazów porośniętych jakimś mchem. Stąd mam dobry widok na
wzgórze, z~którego strzelał. Układam lufę w~szczerbie jednego z
kamieni i~szukam wzrokiem mojego celu, lub kolejnego rozbłysku
z lufy\3k

Jest!

Pada kolejny strzał~--~piąty, szósty? Była chwila przerwy,
widocznie przeładowywał. Przyglądam się przez lunetę i~dopiero
zauważam sylwetkę wroga~--~dobrze się ukrył\3k Biorę wdech,
starając się uspokoić bijące szaleńczo serce, celuję po czym
naciskam spust. Za każdym razem odnoszę wrażenie, jakby świat
wokoło zwalniał, jak gdyby z~zapartym tchem patrzał na
wystrzeloną kulę, aby zobaczyć, czy dosięgnie celu\3k Kolba
uderza mnie lekko w~bark, wywołując kolejną falę bólu w
ranionym ramieniu, lufa odbija w~górę, pocisk trafia strzelca.
Zarzucam karabin na plecy, liczę do trzydziestu, po czym biorę
do rąk 'pionówkę' i~truchtem ruszam w~stronę zwłok, kryjąc się
pomiędzy krzakami. W ciągu minuty pokonuję te kilkaset metrów.
Tak jak myślałem, trup. Kucam przy zwłokach i~kładę strzelbę na
kolanach. Kula przeszła przez oba płuca~--~wszystko wokoło jest
zachlapane ciemną, tętniczą krwią. Paskudna śmierć. Przeszukuję
trupa~--~nie ma nic wartego uwagi, nawet kasy. Amunicja\3k
praktycznie wcale~--~tylko jeden łukowy magazynek do SWU z
pięcioma nabojami. Sam karabin ma cztery naboje, obok leży
pusta ładownica. Coś tu śmierdzi\3k Zaczynam nasłuchiwać
odgłosów otoczenia~--~niby wszystko jest w~porządku, ale
jednak\3k

Klik\3k!

Parędziesiąt, może paręnaście metrów dalej słyszę dźwięk
zwalnianego zamka kałasza. Myśli jak szalone zaczynają krążyć w
głowie i~układać się w~jedną całość, w~ciągu paru sekund
rozumiem wszystko\3k Prawie. Po prostu monolitowcy chcieli mnie
zabić w~bazie, albo ewentualnie z~niej wywabić~--~wiedzieli, że
jak zawsze wyrwę się do przodu. Wystawili jednego ze swoich
jako podpuchę i~czekali na rozwój sytuacji\3k W tym wypadku po
prostu aż podejdę do trupa. Ciekawe, ilu ich teraz tam jest?
Dwóch, trzech? I czy nie wystarczyło po prostu poczekać, aż sam
opuszczę bazę\3k? Najwidoczniej ruszyłem coś ważnego, skoro tak
im się śpieszy\3k W duchu przeklinam Zonę wraz z
przyległościami i~całym ścierwem jakie wydała. Po chwili
błyskawicznie wstaję, obracam się i~dwukrotnie naciskam spust
strzelby, celując instynktownie. Jeszce zanim cichną echa
'pieśni' mojej broni, rozlega się pierwszy strzał\3k
%
\ro{14.27}
%
Pamiętam, kiedy dostałem pierwszą kulkę~--~to było jakiś czas
przed Zoną. Dowódca mówił do nas wtedy: "Macie walczyć do
ostatniej kropli krwi! Choćbyście mogli tylko ruszyć palcem -
to wsadźcie go w~oko pierwszego wroga, który podejdzie!".
Oczywiście, co za śliczna teoria\3k Miałem za zadanie ochraniać
jakiegoś spasionego bogacza, już nawet nie pamiętam jak się
nazywał. Kiedy z~tłumu ludzi pod operą wynurzył się gość z
pistoletem w~ręce i~złożył się do strzału, osłoniłem grubasa
własnym ciałem. Oberwałem dwie kule w~kamizelkę~--~skończyło
się
na połamanych żebrach~--~i jedną w~okolice nerki, przeszła
przez
kevlar\3k Od razu odechciało mi się wszystkiego~--~czułem, jak
wraz z~krwią ucieka ze mnie cała wola walki i~życia. Miałem w
głębokiej dupie wszystko~--~nawet to, że bogacz zginął,
trafiony
kolejnymi kulami zamachowca, który na końcu sam popełnił
samobójstwo. Koniec końców, z~braku winnego, zrobiono kozła
ofiarnego ze mnie. Na szczęście po dwóch tygodniach pobytu w
szpitalu pozbierałem się na tyle, by uciec stamtąd i
wykorzystując kontakty w~podziemiu, przenieść się do Zony -
jedynego miejsca, gdzie nikogo się nie szuka. Po tym wszystkim
przysiągłem sobie, że nigdy ponownie się nie poddam, nie
pozwolę sobie na takie odrętwienie jak wtedy.

Do dziś.

Wiem tyle, że dostałem trzy kule, sądząc po terkocie broni,
5.45mm. Gdzieś pomiędzy sercem a~płucami, w~udo i~nad prawą
nerką. Raczej relatywnie niegroźne, inaczej już bym nie żył.
Mimo wszystko jednak nie mam siły się ruszyć. Umysł znów
zaczyna zwalniać, jest mu wszystko obojętne\3k Nie liczy się
to, że leżę z~twarzą w~błocie, że właśnie powoli wycieka ze
mnie życie i~wsiąka w~brudne, wilgotne podłoże. Jednak do głowy
uparcie dobija się jedna myśl~--~kto strzelił zaraz po mnie?
Przecież nie był to żaden z~monoliciarzy\3k Poza
tym~--~dlaczego
ja nadal żyję? Słyszę jeszcze tylko, jak ktoś do mnie podchodzi
i coś mówi~--~nie rozróżniam już słów; czuję, jak na umysł
opada
zasłona ciemności i~tracę świadomość\3k

Światło?

Kiedyś nasłuchałem się opowieści, żeby "nie iść w~stronę
światła". Jednak jak to zwykle bywa, po paru sekundach
dochodzenia do siebie okazuje się, że to po prostu lampa, którą
jakiś idiota wycelował prosto w~twoją twarz\3k Otwieram powoli
jedno oko, starając się patrzeć jakoś w~bok, żeby ta cholerna
żarówka tak nie przeszkadzała. Próbuję się poruszyć i~zalewa
mnie fala bólu~--~czuję się niemal, jakbym oberwał ponownie
trzema kulami. Po chwili wzrok przyzwyczaja się na tyle, że
mogę jako-tako rozejrzeć się po pomieszczeniu. Jakaś sala
operacyjna? Cóż, do połowy ścian widać zżółknięte, małe
kafelki, gdzieniegdzie popękane~--~tak samo prezentuje się
podłoga. Zmysły powoli zaczynają działać jak powinny~--~w
nozdrza uderza mnie specyficzny zapach środków do dezynfekcji.
Podnoszę lekko głowę i~stwierdzam, że leżę na łóżku, przykryty
czymś, co kiedyś było wojskowym śpiworem, a~teraz robi za
kołdrę. Po chwili z~pomieszczenia obok~--~być może korytarza -
słyszę parę głosów, po czym drzwi się otwierają i~wchodzi
mężczyzna ubrany w~biały kitel. Znaczy, na pewno biały w
zamierzeniu~--~teraz jest "przyozdobiony" plamami krwi i~innymi
płynami. Widząc, że już się obudziłem, podchodzi z~uśmiechem i
mówi:
%
\sx No, braciak! Szczęście macie, że żyjecie! Wlado was tu
targał
niezły kawał drogi!
\qd
Co za Wlado, do diabła? Mrużę znów oczy, dając mu znać, żeby
zrobił coś z~tą nieszczęsną lampą\3k Widząc to, uderza się
płaską dłonią w~czoło i~odsuwa stojak ze 100-watową żarówką
nieco na bok. Teraz mogę przyjrzeć się facetowi: jest dość
niski, w~średnim wieku, lekko łysawy na czubku głowy i~ma
końską twarz, która, nie wiem dlaczego, przywodzi mi na myśl
weterynarza. Próbuję zadać pytanie, ale kiedy chcę otworzyć
usta, dociera do mnie, że chce mi się pić~--~mój język jest
sztywny jak kołek. Doktorek widząc mój grymas, podchodzi do
stolika i~podstawia mi do ust kubek z~wodą. Biorę kilka
łapczywych łyków, po czym daję znać, że na razie wystarczy.
Ponownie próbuję zadać pytanie:

- Gdzie\3k jestem?

Mężczyzna lekko się uśmiecha, po czym odpowiada:

- Jak to gdzie? W lazarecie naszego ukochanego Baru!

No tak, Bar. Gdzie indziej może trafić ranny stalker.
Oczywiście biorąc pod uwagę, że nikt go w~międzyczasie nie
dobije i~nie okradnie. Zadaje kolejne pytanie:

- Kto\3k mnie przyniósł?

- Ano, jak wam mówiłem~--~Wlado. Fajny chłopak z~niego,
porządny. Nie zostawi stalkera w~potrzebie.

Jakoś nie chce mi się wierzyć, że przypadkowy stalker~--~nawet
czysty jak łza~--~targałby mnie taki kawał na plecach do Baru.
Poza tym~--~baza Czystego Nieba była paręset metrów dalej, nie
mógł mnie zaciągnąć po prostu tam? No dobra, tego dowiem się
najwyżej później. Teraz zadaję jeszcze jedno pytanie:

- Jak długo\3k leżę?

- Dobre trzy dni będzie. No, praktycznie ciągle nieprzytomni
byliście. Sporo krwi żeście stracili, trzy kule to nie w~kij
dmuchał. Dobra, muszę iść zając się lekami, jak coś, to
wołajcie.

Mężczyzna odwraca się, wychodzi i~zamyka drzwi, a~ja zaczynam
myśleć o tym, czego się dowiedziałem~--~ot, jakiś stalker
wyciągnął mnie rannego z~bagien, niósł na plecach dobre cztery
kilometry przez pieprzoną Zonę, po czym zostawił mnie tu pod
pełną opieką medyczną, nie przekazując żadnej informacji o
zapłacie czy czymś podobnym\3k Przywykłem już dość dawno do
myśli, że tutaj nawet kuli nie dostaje się za darmo~--~bo po
tym
zabiorą ci całą kasę jaką masz. Jednak nagle do głowy wpada mi
jedna uporczywa myśl, od dłuższego czasu krążąca w~mojej
podświadomości.

Wlado? Wladimir?

Czyżby\3k? W PDA dziewczyny było napisane, że tak miał na imię
jej brat. Być może błędnie założyłem, że martwy wolnościowiec
był jej zaginionym rodzeństwem\3k W sumie wysnułem ten wniosek
patrząc na zdjęcie, jakie przy sobie miał\3k Ale skąd w~takim
razie u niego ta fotografia? No tak, niektórzy~--~tacy jak Sowa
- są w~stanie dać paręset rubli za "artefakt" tego typu. Znając
"wolność", być może po prostu okradł poprzedniego
właściciela\3k

Nie, nie\3k

W myślach uderzam się w~czoło~--~chyba mam poważną gorączkę,
skoro snuję już tak daleko idące teorie\3k Odsuwam od siebie te
myśli i~próbuję zasnąć~--~szczęśliwie teraz bez lampy świecącej
w oczy. Potem przyjdzie czas na myślenie i~zadawanie pytań.
%
\ro{7 dni później, 12.38}
%
Nienawidzę tego\3k To uczucie, kiedy jesteś pomiędzy bytem, a
niebytem, otumaniony lekami, a~twoja podświadomość wyciąga z
twojego umysłu najgorsze koszmary i~wykorzystuje je przeciwko
tobie\3k Chcesz się obudzić, ale wiesz, że to niemożliwe~--~to
nie
jest zwykły sen. W ten sposób znów odczuwam wyrzuty dawno
zagłuszonego sumienia, poczucie winy ponownie pulsuje mi w
głowie, a~dłonie znów sprawiają wrażenie zachlapanych cudzą
krwią\3k Po co wymyślać piekło, skoro znajduje się ono w~głowie
każdego z~nas? Każdego, kogo skalała Zona?

Cztery dni.

Tyle czasu trwała ta mordęga. Kolejne dni, jakie spędziłem w
lazarecie~--~półprzytomny, majaczący w~malignie, podłączony do
eksperymentalnej aparatury leczniczej, wykorzystującej do
działania sprzężenia pomiędzy artefaktami.

Cztery dni.

Na szczęście już po wszystkim~--~od rana jestem już na nogach,
nie czując wcale nie tak dawnych ran. Magia Zony~--~potrafi
zabijać tak samo skutecznie, jak leczyć. Odebrałem już swój
ekwipunek od kwatermistrza, sprawdziłem go i~zauważyłem, że
zniknęła tylko jedna rzecz~--~PDA dziewczyny. Mogłem się tego
domyślić\3k

Teraz stoję w~piwnicy, w~barze „100 Radów” i~popijam herbatę z
prądem. To miejsce zawsze miało swój klimat sprzyjający
rozmyślaniom\3k A przez ostatnie dni wydarzyło się tyle, że
jest
o czym myśleć. Jednak z~zadumy nad ostatnimi zdarzeniami wyrywa
mnie mężczyzna, który podchodzi do mojego stolika i~bez słowa
siada obok. Typowy łowca~--~ma na sobie lekki kombinezon „Świt”
pomalowany w~maskujące barwy, nóż myśliwski przy pasie i~maskę
przeciwgazową na plecaku. Odzywam się pierwszy, domyślając się,
kim jest:

- Wlado, tak?

Mężczyzna kiwa głową, po czym wyciąga PDA~--~właśnie TO~--~i
kładzie je na stole. Wskazuje na nie palcem, po czym mówi:

- Opowiedz mi\3k Jak zginęła?

Chciałbym móc odpowiedzieć coś epickiego i~pompatycznego, coś w
stylu „Lepiej opowiem ci jak żyła” albo „Drogo sprzedała swoje
życie”\3k Chciałbym. Ale to Zona~--~tu nie ma wielkich czynów
ani
epickich opowieści. Tu jest tylko brudne przetrwanie tych,
którzy uciekli do tego miejsca przed doganiającą ich
przeszłością, przybyli w~poszukiwaniu lepszego życia czy z
jeszcze innych powodów, znanych tylko im samym. Spoglądam
mężczyźnie w~oczy, po czym odpowiadam:

- Szybko i~bezboleśnie. Możesz być pewien, że nie cierpiała i
że myślała o tobie do ostatniej chwili.

Widzę po jego spojrzeniu, że chciałby usłyszeć co
innego~--~coś,
co pozwoliłoby wyidealizować pamięć o siostrze, zapamiętać ją
po śmierci jako niemal herosa z~dawnych mitów\3k Niestety, nie
potrafię opowiadać zmyślonych historii z~przejęciem i
przekonaniem, jak czynią to niektórzy stalkerzy. Starzy bajarze
i bardowie, którzy zasiadają przy ogniskach i~wieczorami prawią
historie, które wywołują łzy, wprawiają w~euforię, czy
poprawiają nastrój nawet najbardziej załamanego człowieka.
Niestety, ja nie posiadam takiego talentu. Wladimir kiwa głową
na znak zrozumienia moich słów, po czym bierze z~powrotem PDA,
chowa je do plecaka i~wstaje. Chce odejść bez słowa, lecz po
chwili zatrzymuje się odwrócony plecami do mnie i~rzuca przez
ramię:

- Jesteśmy kwita, stalkerze. Ani ty nie jesteś mi nic winien,
ani ja tobie.

Przytakuję gestem, mimo, że on nie może tego widzieć. Po chwili
wychodzi i~znika z~mojego pola widzenia. Dopijam herbatę, płace
barmanowi, po czym wychodzę na zewnątrz, po drodze odbierając
broń od bramkarza. Teraz wypadałoby sprzedać sprzęt z~plecaka i
zaplanować następne kilka dni. Idę do budynku na północ, gdzie
mieści się mała zbrojownia i~skup złomu zarazem. Sprzedawca,
czy jak go nazwać, siedzi jak zawsze za ladą z~nogami na
pobliskim stole i~żuje tanie cygaro~--~nigdy nie widziałem,
żeby
je palił. Podchodzę bliżej, ściągam plecak, po czym wyciągam z
niego zdobycznego 1911 i~kładę na stół razem z~czterema
zapasowymi magazynkami. Ten bierze broń do ręki, ogląda ją ze
wszystkich stron, rozładowuje, sprawdza działanie mechanizmów,
po czym mówi:

- Dam ci pięćset~--~dawno nie miałem żadnej czterdziestki
piątki
na stanie. Magazynki wezmę po pięćdziesiąt za sztukę.

Kiwam głową i~wyciągam rękę po pieniądze. Zbrojmistrz odlicza
banknoty, po czym podaje mi je. Po chwili namysłu ściągam z
pleców SWD i~pytam:

- Ile za to?

Proces badania i~oględzin broni powtarza się, tym razem trwa
jednak dłużej. W końcu mężczyzna stwierdza:

- Też zadbany, szybko pójdzie\3k Dam ci cztery tysiączki, no
może
cztery i~dwieście.

Przytakuję, wyciągam pozostałe puste magazynki i~podaję mu je,
biorąc w~zamian kasę. Znów wychodzę na ‘świeże’ powietrze. No
tak, muszę jeszcze opłacić ‘abonament’ u barmana, całkowicie
zapomniałem\3k Pamiętam, jak na początku bulwersowałem się o
cotygodniowe opłaty za korzystanie z~noclegowni na terenie Baru
– dopiero po pewnym czasie zrozumiałem, że ma to sens. W końcu
nie każdy stalker sprzedawał artefakty i~inny towar barmanowi
czy jego handlarzom, w~związku z~czym nie byłoby z~czego
utrzymać tego miejsca. A nikt nie chciałby płacić za wszystko z
własnej kieszeni\3k Wracam znów do piwnicy „100 Radów”
i~ponownie
podchodzę do barmana. Wyciągam 300 rubli, kładę je na ladzie i
mówię:
%
\sx Za tydzień noclegów.\qd
%
Barman bierze kasę i~chowa ją do kasetki pod stołem, po czym
znika na chwilę, kucając i~szukając czegoś zapamiętale. Po
chwili wstaje ponownie i~wyciąga w~moją stronę poplamioną,
wielokrotnie używaną i~zżółkniętą kopertę i~mówi:

- Ten stalker z~którym gadałeś\3k Kazał ci to dać, wrócił
chwilę
po tym, jak wyszedłeś.

Wzruszam ramionami, biorę kopertę i~siadam niedaleko. Obracam
chwilę w~rękach sztywny od starości i~brudu kartonik, po czym
otwieram go.
%
\ro{7 dni później, 16.42}
%
Po otworzeniu koperty moim oczom ukazuje się stara fotografia,
przedstawiająca CEJ. Zdjęcie wygląda, jak gdyby miało je w
rękach na przestrzeni lat wiele osób~--~jest poplamione,
sztywne
od brudu oraz starości i~konkretnie wypłowiałe. Ciekawe kto i
kiedy w~ogóle je zrobił\3k Na odwrocie jest coś napisane: "Tu
znajdziesz to, czego szukasz, stalkerze". Co do\3k? To brzmi
jak stara mantra, powtarzana przez tych, którzy dotarli kiedyś
do centrum Zony i~byli w~stanie wrócić o własnych siłach. Każdy
jeden z~nich zachowywał się jak niespełna rozumu, ale ze
wszystkich ich opowieści można było wyciągnąć jedną wspólną
rzecz~--~gdzieś tam znajduje się tajemniczy Spełniacz Życzeń.
Artefakt, urządzenie, czy cokolwiek innego, pozwalające na
spełnienie dowolnego marzenia śmiałka, który tam dotrze.Nigdy
nie wierzyłem w~takie bzdury, ale kiedy słyszy się to tyle
razy, od tylu różnych osób\3k Może jednak jest w~tym chociaż
ziarno prawdy?

Jedno życzenie. Moje życzenie. Moje.

Spośród wszystkich błędów, jakie w~życiu popełniłem\3k Do
naprawienia jednego z~nich rzeczywiście byłaby potrzebna taka
pomoc. A ja byłbym w~stanie oddać za to co tylko mogę\3k Też
dlatego do tej pory comiesięcznie wysyłam przez Sida anonimowo
pewną kwotę na adres w~Moskwie. Marne zadośćuczynienie za dawne
grzechy. Niestety, pieniądze nie zwracają życia. Nie leczą ran
na sercu. Nie uspokajają myśli. Pieniądze pozwalają tylko zająć
umysł, aby nie skupiał się na tym, co nas gryzie i~męczy.

Pieniądze.

Zwykły papierek, któremu sami nadaliśmy wartość, wmawiając
innym, że tak będzie lepiej. Nie zwróciliśmy uwagi, że wraz z
tym zaczęliśmy tracić nasze człowieczeństwo. Może i~w~normalnym
świecie ktoś może zaprzeczać, że tak wcale nie jest, jednak w
Zonie\3k W Zonie z~czasem każdy staje się gorszy od
mutantów.One przynajmniej nie kryją wrogich intencji. Cholerny,
zdradziecki świat\3k

I na szczycie ja~--~pieprzony tchórz.

Gdybym nie próbował uciec tutaj, do tego piekła, może nadal by
żyła, a~ja nie walczyłbym z~wyrzutami sumienia? Może. Jednak
wtedy byłbym martwy i~miałbym wszystko głęboko gdzieś\3k Życie
za życie~--~tylko nikt nie pomyślał, żeby spytać ją o zdanie,
czy się na to zgadza. I co z~tego, że to nie ja pociągnąłem za
spust? Moje wybory bezpośrednio do tego doprowadziły\3k

Moja biedna Marina\3k

Rugam siebie w~myślach~--~że też zebrało mi się na wspominanie
przeszłości, znowu. Rozglądam się podejrzliwie po głównym
pomieszczeniu baru~--~wszyscy zajmują się swoimi sprawami:
oblewaniem udanych wypadów, zapijaniem smutków po tych
nieudanych, czy piciem ku pamięci poległych, dobrych stalkerów.
No tak, przecież póki co nikt nie umie czytać w~myślach\3k Choć
w sumie słyszałem już legendy o artefakcie, który miałby niby
na to pozwalać. No dobrze, ale nie popadajmy w~paranoję.
Ponownie zatapiam się w~odmętach mojego umysłu, dając się
ponieść nurtom wspomnień\3k Niestety, głownie złych. Zona jest
takim paskudnym miejscem, które potrafi w~człowieku stłamsić
wszystko, co dobre, a~uwydatnić to, co najgorsze.

Zona. Co w~niej takiego jest, że ludzie ciągną tu jak muchy do
otwartego słoika miodu wytarzanego w~gównie w~upalny dzień?
Przecież jest tyle innych możliwości zarobienia sporej kasy\3k
Bezpieczniejszych możliwości, dodam. Z drugiej strony~--~życie
tutaj, w~permanentnym uczuciu zagrożenia w~końcu zaczyna
uzależniać, jak narkotyk. Na początku pojawiasz się, bo
słyszałeś, że można tu nieźle zarobić. Potem próbujesz zdobyć
ruble na powrót do domu, kiedy nie dajesz już rady. W końcu
jednak przywykasz i~żyjesz tu tak długo, aż nie zginiesz w
anomalii, od szponów mutanta, czy zastrzelony przez wojskowych
albo innych stalkerów. Poza tym~--~Zona to tajemnica.Iluż to z
nas w~dzieciństwie marzyło, żeby znaleźć się tam, gdzie nikt
jeszcze nie postawił stopy, odkryć coś nieznanego, rozwiązać
jakąś wielką zagadkę\3k Tutaj to marzenie w~pewien sposób się
spełnia~--~tylko niestety za darmo dostajemy również
wszechobecną śmierć i~walkę o przeżycie każdego dnia. A mimo
wszystko tu jesteśmy\3k Dlaczego?

Bo Zona jest tajemnicza. Przyciąga.

Kryje jeszcze tyle sekretów, których nie wyjaśniliśmy, tyle
fenomenów, których nie daliśmy rady wytłumaczyć, tyle miejsc,
jakich nie odkryliśmy\3k A także ta fotografia~--~wbrew
wszystkiemu fascynująca swoją tajemniczością.

"Tu znajdziesz to, czego szukasz, stalkerze"

Spełniacz Życzeń\3k Logika podpowiada, ze to bzdura, ale
wyobraźnia wówczas nasuwa na myśl choćby artefakty czy anomalie
- ich istnienie też może się wydawać bzdurą. Dopóki człowiek
nie spotka ich osobiście\3k

"Logika może cię zaprowadzić od punktu A do punktu B, ale
wyobraźnia może zaprowadzić cię gdziekolwiek"

Przypominają mi się stare słowa, które nie pamiętam już kto
wypowiedział\3k Jednak w~warunkach Zony niezbędny jest też
instynkt, który pozwoli przetrwać tą drogę~--~nieważne czy
wespół z~logiką czy z~wyobraźnią~--~w jednym kawałku.

Instynkt.

To właśnie nim się kieruję przebywając w~Zonie. Nie
zdradzieckim sercem, czy toporną logiką. Instynktem. To on
pozwalał mi przetrwać wszystkie lata niebezpieczeństw. A teraz
mówi mi, żebym wyruszył do centrum. A ja znów go posłucham.

Wstaję od stołu, wychodzę z~piwnicy i~kieruję swoje kroki w
stronę budynku sypialnianego~--~muszę dobrze się wyspać przed
jutrzejszym dniem, czeka mnie długa droga.

\end{document}