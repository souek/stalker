\documentclass[../MAIN.tex]{subfiles}
\begin{document}
\textit{„(\3k) blaskiem najjaśniejszym z\3k Nie. To żem
wpisał ja już do kroniki faktów\3k Tylko gdzież ona może być\3k
?}

\di{M}{oże ktoś raczy kiedykolwiek przeczytać, o ile znajdzie.
Jeszcze raz\3k Ale po co jeszcze raz? Jeszcze raz. Tak. Raz
jeszcze\3k Jeszcze raz\3k Bredzę, nie – myślę, że bredzę, choć
nie bredzę. A może myśląc, że bredzę, tak naprawdę bredzę
próbując się pocieszyć i udając, że jestem świadomy swych
bredni? Bredzę w szarości wirującego firmamentu pokoju,
szarych ścian, szarodrewnianego brązu szarawodrewnianego stołu
i krzesła, szarodcieniu szarości sufitu. Ba! Szarobrednie
szarego człowieka, pulsującego widokiem pomieszczenia i tynku
mimowolnie szarosypiącego. Szarowirujący świat statycznie,
mozolnie jaśniejący blaskiem naj\3k Nie. To żem\3k
Szarobrednie\3k \footnote{Zapiski urwane z powodu
nieczytelnego pisma
(przyp. Autora)}}

\di{O}{lawarjenko Wladimir, lat 41, wojskowy w służbie
ukraińskiego
rządu do spraw katastrof ekologicznych i zabezpieczania rejonów
skażonych radioaktywnie. Ktokolwiek - Ty – tak, ty! Jeżeli to
teraz czytasz, to zapewne skąpany w białoszarych firmamentach
nieboskłon pozwala żyć. Nie żyje? Półkolisty białoszary nie
żyje? Tt\3kTo bardzo prawdopodobne. Uwzględniam to w swojej
kronice. Spójrz czytelniku. Uwzględniam twój potencjalny tok
myślenia. Zapominam o swoim? Nie. Skądże – mojego nie ma. Więc
jak myślę i to tworzę skoro nie ma? Iluzja Szarobredni
Wladimira Lawarjenko w którejże to myśli, że tworzy i myśli,
więc sądzi, że jest. Jest? Nie jest. Iluzja jak wspomniał.
Znalazłem dziś chleb i trochę wody, wprawdzie udusiwszy się
szaroradiacją\3k \footnote{Zapiski urwane – nieczytelne pismo
(przyp.
Autora).}
\\
Godzina 12:20. Znalazłem chleb i trochę wody,
możliwe, że są radioaktywne, nie wiem, licznik Geigera szlag
trafił. Muszę coś jeść. Wszystko jest mgłą\3k Jak przez mgłę.
Widziałem błysk, później mgła. Mgła jest jak sen. I nie wiem
czy śnię, czy żyję. To jakbym tracił kontrolę nad sobą, a
przeczytawszy co wcześniej napisałem. Ja\3k Ja tego nie
pamiętam, inaczej! Mgła – jak przez mgłę szarobredni,
iluzorycznej szarostrefy szarego Wladimira.}

\di{N}{ów to! Ty który to czytasz, żyjesz w iluzji własnego
istnienia\3k \footnote{Zapiski urwane bez powodu (przyp.
Autora) (\3k)}}

\di{O}{piszę dzień po dniu. Tak. Będę spisywał dni. Może
tak choć na
chwilę utrzymam się na pół jawie szarosnu. A może nawet już to
jest we mgle? Pozwól, że przeczytam poprzednie zdania.
Przeczytałeś? Jeśli tak, uderz pięścią w ścianę. Posypał się
sza\3k Nie! Tynk. Krew. Moja ręka boli. Ból, to chyba jeszcze
coś co jest mi bardzo ludzkie i niemgliste. A jeśli nawet i ból
jest już tylko iluzją szarobredni? Białoszary żywy nie daje
spokoju. Gdy wyjdę po cokolwiek jest jak wiatr. Ale nie taki
zwykły wiatr. Szarowiatr, niejako strugający wnętrze szarociała
na wylot. Pulsujący firmament białoskłonu\3k
M\3k Mo\3k Mon\3k Montera. Po co to napisałem? W sumie. Czemu
miałbym tego nie napisać, nieszarobredny czytelniku szarobredni
Wladimira Olawarjenko?}

\di{L}{atoszarość gorąca białoszarego żyjącego jest dziś
mocniejsza.
Wieje mocniej, mgła jest subtelnie, w magnificencji szarości
aktywniejsza\3k}

\di{I}{ nie wiem czy dalej przeżyję. Boję się wyjść, boję się
przed
szarowiatrem stanąć. Dmie dzień w dzień, noc w noc potężnie,
nieprzerwanie, otula strapionych\3k Czytelniku parsknąłem
śmiechem. Odnoszę wrażenie, że zaczynam postrzegać Go jako coś
pozytywnego. Ależ tak! Wszakże jak mogłem nie dostrzec tego
wcześniej, szarogłupi Wladimirze. Wychodzę.}

\di{T}{eraz już wiem. Barwnowiatr odmienił mnie, zmienił
mnie,
przemienił mnie. Transformował w byt lepszy, nieskończonej
magnificencji jego wszechistnienia. Nie czuję nic, nie widzę
nic. Działam myśląc, że widzę tęczotwór okiełznany. Bo widzę.
Czytelniku. To wcale nie jest takie złe. Jam tego nie widział
wcześniej. Wszak wszyscy mnie naokoło żyją w magnificencji
Jego. Mo – że widzę to teraz lepiej – no – nareszcie barwa
szarotworu wszechszarości zanikła – li – więcej nie będę\3k. –
t – tak\3k to już ustalone. Nie ma. Bo jest. Jest, bo nie ma.
Zatracony w firmamencie syntezy metafizycznej tego ciała. Jest.
Wyzwolony”.}





\footnote{(Dalej – przyp. Autora)} 17 października, godzina
7:38. Dziennik znaleziony przy martwym. Przyczyna zgonu? Do tej
pory nie znana. Psychoanaliza słów martwego każe wskazywać na
skrajną degradację psychoruchową, narastające zaburzenia
afektywne paranoidalne. Niewykluczone omamy schizofreniczne
obiektu. Jutro zostanie przeprowadzona ekspertyza patogenna z
autopsji by wykluczyć ewentualne wątpliwości.

18 października, godzina 7:00. Rozpoczynam ekshumację zwłok. Po
zakończonej analizie dopiszę dokładniejszy opis analityczny
problemu\3k.

Godzina 9:02. Dokonałem szczegółowej autopsji ciała zmarłego.
Standardowe cięcie w kształcie litery „Y” nie wykazało żadnych
zmian patogenicznych. Serce niepowiększone, płuca doskonale
zachowane, przełyk w stanie nienaruszonym, niewielka ilość
płynu i resztek pożywienia w żołądku. Brak uszkodzeń
mechanicznych, które mogłyby spowodować ewentualną śmierć.

Godzina 9:34. To niebywałe. Otwierając czaszkę obiektu
napotkałem coś\3k niezwykłego. Płat czołowy i skroniowy
zmarłego wykazuje\3k niespotykane dotychczas zmiany. To jakby
mózg martwego dosłownie zapadł się do siebie. Powód nieznany,
zbyt mało danych.

19 października, godzina\3k Nie wiem która jest godzina. Z
prosektorium dochodzą dziwne hałasy. Idę to sprawdzić.

Godzina 3:29. Hałasy ustały. Zapewne mało kto uwierzy czytając
ten dziennik, ale ciało zmarłego poddane autopsji\3k zniknęło.
Wybita szyba. Tak. Jest jeszcze wybita szyba. Cokolwiek dzieje
się w Strefie, to nie jest to nic dobrego.

Petriuga Oleg, koniec dziennika. Sprawa odroczona z braku
materiału badawczego.

23 października, godzina 8:30
Petriuga Oleg, prywatny dziennik. Popijam poranną kawę i wciąż
rozważam. Wszak – brak materiału badawczego nie pozbawia mnie
rozumu i możliwości dedukcji. Tak czy inaczej w całej swojej
pracy nigdy nie miałem do czynienia z tak przedziwnym
przypadkiem. Obiekt po szczegółowej autopsji opuścił o własnych
siłach prosektorium! Zdaję sobie sprawę z tego jakie mam
miejsce pracy. Toć Strefa nie jest obszarem szczególnie
atrakcyjnym jeśli mowa o bezpieczeństwie\3k Ale to?
Człowiekowi wpajają od małego – „Bój się żywego, martwy nic Ci
nie zrobi!”. I bądź teraz normalny na umyśle, szczególnie po
zmroku. Idziesz spać i nagle\3k Żywe trupy Cię otaczają, w
paranoję popaść można. Naprawdę, tylko oszaleć, a po prawdzie,
wykonując swoją poniekąd lubianą pracę nikt nie chciałby być
straszony. Tymczasem takie coś\3k Czasami człowiek poważnie
zaczyna gubić granicę między snem, a jawą.

Jawa\3k Tak. Wróć – sen, idę po jeszcze jedną kawę...

Przechodzi mi przez myśl, że Strefa ze wszystkich robi
obłąkańców. Wszak cała ta sprawa, przecież to nie tak, że to
było i minęło. Przeszłość dotyka każdego z nas, a Strefa się
jakby w tej przeszłości zatrzymała, ona niejako wybrała swoje
własne istnienie w sposób stricte metafizyczny. Aż mi się Hegel
przypomniał i jego czas absolutny... A może on miał rację?
Może niczym u Hegla, Strefa dopełniła swój czas metafizyczny,
na wskroś całej czasoprzestrzeni, zostawiając jedynie mentalną
ingerencję w ludzkie jestestwo? To implikuje jeden fakt – gdy
tu trafiasz, wszystko co było, a dotyczyło ciebie – wraca – i
boli. Ludzie dostają depresji, siedząc na posterunkach, płacząc
całymi dniami i sącząc butelkę, choć weń nic już nawet nie ma.
Biedacy, kończą jak ten tutaj w swoim pamiętniku. A później
opowiada się historie po posterunkach, ludzie srają potem w
gacie po nocach nie mogąc spać i prosząc o warunkowe
zwolnienie. Oczywiście go nie dostają, i\3k Strefa ich
pochłania - <<Barwnowiatr odmienia ich>>

Powiadam wtem – Im dłużej patrzycie w otchłań, tym dłużej ona
patrzy w wasze jestestwo. I żaden nie okazał się dość silny by
sprostać otchłani Strefy. Zastanawiał się ktoś dlaczego mamy
tylu psychiatrów w placówkach tego cholernego miejsca? Bo
ludzie szaleją – czego dowodzą tylko miejscowe historyjki jak
to jeden z drugim bez powodu począł tłuc głową o mur. I ja
wcale się nie dziwię, że rządowi stawiają taki popyt na lekarzy
psychiatrów w Strefie. Ktoś musi leczyć tych nieszczęśników, bo
wkrótce okaże się, że pół Strefy będzie biło głową o ścianę, a
drugie pół miało z tego powodu pełno w spodniach.

Dochodzi godzina 9:00\3k
Ale ja wprawdzie nie o tym. Problematycznym jest prywatny
dziennik zmarłego, którego badałem. Zapiski wskazują na głęboką
degradację neuropsychologiczną. Wstępna analiza pozwala
stwierdzić, iż zaburzenia afektywne dwubiegunowe, wymieszane ze
schizofrenią wyraźnie dały się we znaki obiektowi.
Zastanawiającymi są czynniki psychiczne, wszak psychosomatyczny
profil zmarłego (wstępny) każe twierdzić, że człowiek ten był w
stanie niczym po nieudanej lobotomii. W końcu jego postrzeganie
świata zostało zaburzone do tego stopnia, iż siatka
psychologiczna jego osobowości skorelowała się bezpośrednio ze
zdogmatyzowanym, zamkniętym myśleniem. Potwierdzała by to
autopsja mózgu obiektu.

W następnych dniach przeprowadzę szczegółową analizę, i
sporządzę konkretny profil psychologiczny obiektu. Ekspertyza
powinna trafić w odpowiednie ręce, toć to już nie pierwszy
przypadek, kiedy mamy problem chorych psychicznie, którym
zdarzało się umierać. Tylko, że nigdy do tej pory żaden z nich
nikomu nie uciekał, co tylko podwyższa rangę całej sprawy\3k

Godzina 9:37
Interesującym jest jednak sam fenomen ciała, które opuściło
budynek prosektorium (wskazują na to liczne ślady stóp obiektu
oraz pozostawione resztki krwi, której kod DNA zgadza się z
wcześniej sprecyzowanym DNA zmarłego). Fenomen tego zjawiska
pozostaje mi nieznany, tak jak sam jego mechanizm działania,
tym bardziej, że mamy do czynienia z ciałem w którym wszelkie
funkcje biologiczne po prostu ustały. Zatem jeżeli owe funkcje
były martwe, a same fale mózgowe ustały, to jakiś czynnik
zewnętrzny musiał sztucznie wywołać połączenia między
neuronami. Czyżby drogą emisji pól kwantowych, nieznanego mi
pochodzenia? Nie\3k to wszystko absurd. Posługując się metodą
naukową interesują mnie twarde dowody, pragmatyzm i skuteczność
działania, a nie dialektyczne, nie oparte o doświadczenie
dywagacje. Na nic mi tu teraz filozofia\3k Jeszcze tym bardziej
platońska metoda dialektyczna.

Dochodzi godzina 10:00\3k
Idę zebrać odczyty promieniowania z Kordonu-4\3k

Godzina 12:20\3k
Odczyty wstępne w normie. Podwyższone promieniowanie
powierzchniowe. Ostatnie deszcze zapewne nasiliły to zjawisko
poprzez przyspieszenie osiadania opadu radioaktywnego. A to i
tak nie jest jeszcze apogeum, bowiem południowy wiatr odciąża
tereny nieopodal Kordonu-4 z radioaktywnych pyłów z centrum
Strefy. Analiza warunków meteorologicznych wskazuje na
korzystniejsze warunki na najbliższe trzy dni. Jako, że
Kordon-4 jest obszarem jedynie zagrożonym promieniotwórczo, to
zważywszy na dobrą pogodę, przewiduję wyjazd na teren głębszych
kordonów celem badań zmutowanego ekosystemu Strefy. Pytanie czy
powinienem informować wojskowych o sprawie tego
zmartwychwstałego trupa\3k

Noc – godzina 1:41
Swoją drogą\3k Do tej pory, jak pracuję tu dobre 4 lata, nie
mogę przywyknąć. Po prostu nie mogę. Wyobrażacie sobie szum
drzew? Niczego niezwykłego weń nie ma. Lecz nie w Strefie.
Budzisz się, bowiem ona tego chce, przypomina się szumem drzew,
trzaskiem gałęzi. Najgorsze są momenty apogeum
promieniotwórczego, i wprawdzie powinienem wmontować nową szybę
i osłonę, bowiem lepiej by radioaktywny opad nie dostawał się
do mojej pracowni. I tu znów wracając do sprawy dziennika
zmarłego – niekiedy ciężko wytrzymać w samotności\3k Lecz gdy
przynajmniej masz gdzie wyjść, popatrzeć, zaczerpnąć powietrza,
to to jest jakoś inaczej. Ale te apogea\3k Nie wyjdziesz ze
schronu przez najbliższe cztery, może pięć dni. I nie ma
nikogo, kto pokwapiłby się przyjść. Łączność nie działa, nic
nie działa. Tylko wyjść na dwór w samych gaciach, spojrzeć w
niebo i dać się napromieniować. A później rzygać cięgiem, by na
końcu umrzeć w męczarniach i samotności. Strefa zawsze dzieliła
ludzi, czyniła ich samotnymi. To tak jakby niewidzialna siła
okalała każdy skrawek tej przeklętej ziemi, czyniąc z każdego
takiego drugiego Schopenhauera. I tylko pozostaje niczym w
filozofii tego nieszczęśnika, paść w swej nihilistycznej próżni
egzystencjalnej, bezwiednie gapić się w sufit\3k I tak umrzeć.

Strefa nie jest przerażającym miejscem. Ona nie wywołuje u
tych, którzy tu mieszkają żadnych uczuć. I to czyni zeń po
prostu otchłanią. Tu nie ma nic w człowieku – jest tylko
przeszłość. Wybudzasz się ze snu, zalany potem, na półjawie
widzisz rzeczy, których widzieć nie powinieneś. Lecz one są.
Odwracasz się za siebie, patrząc czy nikogo nie ma. I
faktycznie – jesteś sam jak palec. Ale i wtedy Strefa o tobie
nie zapomina. O nie! Ona przypomni Ci, że coś może być teraz
przed tobą i znów poczynasz się odwracać. Wtem nim się
obejrzysz, machasz na lewo i prawo głową, aż zapomnisz ze
zmęczenia po co to czynisz\3k Powiadam\3k Ludzie popadają w
solipsyzm, głupieją wiedząc, że ich własne zmysły ich zawodzą.
I ani Kartezjusz, ani ktokolwiek inny nie pomoże w wyjaśnieniu
tego fenomenu iluzji w jakie popadamy. My, pracownicy Strefy\3k

24 października, godzina 6:12
Dziś wyruszam w głąb Strefy. To prawdopodobnie ostatnia
ekspedycja badawcza w jakiej będę brał udział. Zbliża się zima,
a radioaktywny śnieg nie jest szczególnie obiecującą
perspektywą. Po prawdzie zastanawiam się, czy nie załapać się
na przepustkę na okres najgorszych warunków meteorologicznych.
Wierzcie lub nie, ale 4 lata pracy w Strefie potrafią nauczyć
człowieka automatyzmu działania. Po prostu idziesz, i robisz co
trzeba, piszesz co trzeba\3k Żyjesz jak trzeba. I nie czynisz
tego z powodu schematyzacji, której nie jesteś świadomy. Strefa
jest tym miejscem, gdzie musisz działać jak robot, według
schematu – inaczej umierasz. Możliwe, że to dlatego ludzie
tutaj przestają cokolwiek czuć. To miejsce jest całkowitą
odwrotnością tego, z czym mamy do czynienia w świecie
zewnętrznym. Czterysta kilometrów na zachód, w pierwszym
większym mieście ludzie starają się żyć niczym cynicy,
pozbywając się swych odczuć\3k Pod pretekstem zmniejszenia
bólu. I czynią to, stają się perfidni i zgorzkniali, zatapiają
się w swojej własnej pysze, arogancji i bluźnierstwu\3k A my?
My pracownicy Strefy? My nie czujemy nic. Można powiedzieć, że
apatia i anhedonia jest tutaj nazywana swoistym – „Zdrowiem
psychicznym”. Doprawdy, Freud by się uśmiał. Co jest zatem
poniżej tej\3k „Granicy”? Depresja. Lub schizofrenia. Albo
jedno i drugie. W Strefie nie jest człowiek w stanie pożałować,
że tu przybył. Bowiem gdy już trafia do dziczy, nagle okazuje
się, że gubi swoje człowieczeństwo. Oczywiście wszystko
trzymane jest pod ścisłą mistyfikacją. Wszak oficjalnie, Strefa
jest tylko zbiorem skażonych i zdegenerowanych genetycznie
elementów ekosystemu. Co zatem z nieoficjalną wersją? Cóż\3k
Jeśli by ulokować przedsionek piekła na Ziemi, to podejrzewam,
że byłby on zbliżony do Strefy.

Tutaj po prostu nie da się zapomnieć o tym, że w każdej chwili
można dostać chorą dawkę promieniowania, nie da się zapomnieć o
wiecznie nienasyconych, schorowanych dzieciach Strefy\3k
Czasami odnoszę wrażenie, że to miejsce można by porównać do
wiecznie niezadowolonej, okrutnej matki. Tak. Strefa jest jak
matka. Ona przygarnia każdego kto zamieszka na jej terenie, na
jej łonie, by później z premedytacją z człowieka uczynić
nie-człowieka. A wszystko w imię nowego logosu Strefy.

Godzina 7:10
Punktem kontrolnym będzie seria badań w kordonie-3 i
granicznych terenach kordonu-2. Jest jeszcze niejaki – Kordon-1
– niemniej, zapuszczają się do niego jedynie wojskowe
ekspedycje rządowe, i po prawdzie wcale się temu nie dziwię.
Bez wzmocnionego kombinezonu i porządnej porcji ołowiu nie
warto się tam wybierać. Dzieci Strefy nie są nam przyjazne\3k W
każdym razie ekspedycyjni rzekomo ruszają pod samą barierę
Kordonu-1 i tam składają te swoje wszystkie raporty i badania.
Próbują chyba jakoś zapanować nad całym tym bałaganem, tak, że
nikt z zewnątrz nie wie co jest za barierą. Zdjęcia satelitarne
nie pokazują niczego szczególnego, jednakże teren około 100
kilometrów kwadratowych jest szczelnie osłonięty. Widać rządowi
mieli w tym jakiś cel wykładając furmankę pieniędzy na
odgrodzenie całego obszaru. Także do dalszych terenów Strefy
nie wpuszczają. Nie wiem\3k Zapewne to promieniowanie, choć dla
mnie to jakaś grubsza afera. Rządowi coś tam kryją i nie chcą,
by tacy jak ja dobrali się do\3k No właśnie. Czego? Czasami
dostaję ekspertyzy naukowe, trafia do mnie kilka plików
kartotekowych od psychologów wojskowych. W jednej z nich była
pośrednio mowa o głębi Strefy. Zamieszczam tekst źródłowy:

\textit{„Ocet. Dużo octu\3k Na cóż to powiedziałem? Rzekł ja
rzeczy,
których rzec nie winien, choć je spisał na tej kartce. Jest w
swej magnificencji, pełny i niezmierzony. Ty który to czytasz –
illuminacja w sferze mistycznej – dostąpisz jej, bo Ona
\footnote{(przyp.
Autora – zapewne chodzi o samą Strefę)} daje
każdemu swą szansę.
Nie. Muszę się ocknąć. Do bariery jest kawałek drogi, a jam
poza drogą, bo drogą inną – Jej drogą – kroczę”.}

Tą krótką notkę z pamiętnika znaleziono przy ledwo żywym
wojskowym z jakiejś rządowej grupy ekspedycyjnej. Po co go tam
wysłali – nie wiem. Tak czy inaczej początkowo miałem do tej
ekspertyzy dość\3k sceptyczne podejście. Jednak po tym co się
wydarzyło w prosektorium cała ta sprawa przykuła moją uwagę.
Wnioskując po budowie logicznej zdań oraz samej treści,
człowiek ten był w bardzo podobnym stanie, co badany przeze
mnie obiekt. Dalsza część opracowania naukowego mówi pokrótce o
człowieku, który począł dobijać się około godziny 5:19 nad
ranem do bram bariery w Kordonie-1. Mężczyznę rzekomo zabrali.
Nawet próbowali poddać go leczeniu, ale ten zmarł po kilku
dniach. Opowiadał podobno jakieś bzdury, jęczał i wył po nocach
mówiąc, że coś jest w jego głowie, weń jestestwie, świadomości.
W bardziej „normalnych” epizodach uskarżał się na okresowy pisk
w uszach. Lecz finał znany – nieszczęśnik zmarł w
niewyjaśnionych okolicznościach. Akcja jego serca po prostu
stanęła, a on sam padł jak manekin na ziemię, wybałuszając swe
przekrwione oczy i tocząc zwymiotowaną pianę. Nie wiadomo, czy
ciało poddano autopsji, cała sprawa została jakby\3k
zatuszowana, trupa moment spalili, prochy rozsypali i tyle było
po nim. Zostały te zapiski. Jak na ekspedycję rządową coś
szybko pozbyli się wszelkich materiałów dowodowych, zostawiając
sobie tylko nieliczne świstki i dokumentacje. Pachnie
mistyfikacją\3k

Mówię Wam. Ktoś coś ukrywa, i wyraźnie nie chce, by wejście do
głębi Strefy było dostępne. Najgorsze, bo rządowi w ogóle nie
reagują na to co się dzieje w kordonach. Ludzie tutaj wariują,
a oni nic. To jakby wiedzieli, że tak ma się dziać i
dopuszczali to do swej myśli, zezwalając jednocześnie na ten
bałagan. Nie wiem tylko jak tamci przy barierze psychicznie
wytrzymują. Cóż\3k albo i nie wytrzymują\3k Zważywszy na fakt,
że ja tutaj, w swoim rewirze Kordonu-4 potrafię pół nocy nie
przespać. Krzyki, mnóstwo krzyków. Są noce, kiedy jest cisza
tak głucha, że człowiekowi ślina w gardle staje. Usypiasz, bo
cóż innego, w końcu jest noc, czyż nie? I wciąż ta
przerażająca cisza. Wtem budzisz się jeszcze w półśnie słysząc
mrożący krew w żyłach przeraźliwy, głuchy krzyk. My pracownicy
Strefy zrywamy się wtedy z łóżek, ryglujemy wszystkie zamki, a
w dzień następny, doprawiamy kolejny. Tutaj nie ma zdrowia
psychicznego. Każdy szelest, szmer, krzyk i ryknięcie jest
śmiechem samej Strefy, jej swoistą manifestacją.

7:52 – czas wypić poranną kawę\3k
\end{document}